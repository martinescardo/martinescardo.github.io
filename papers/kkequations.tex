\documentclass[10pt]{article}

\newcommand{\Omit}[1]{}

\usepackage{times}

\usepackage{amsmath,amssymb,stmaryrd,amsthm,amscd,alltt,url,times}
\usepackage[english]{babel}
\usepackage{ukdate}

\usepackage{a4}
\usepackage{url}
%\usepackage[colorlinks=true,hypertexnames=true,linkcolor=blue]{hyperref}

\newcommand{\cauchy}{\operatorname{Cauchy}}
\newcommand{\godel}[1]{\ulcorner #1 \urcorner}
\newcommand{\mycite}[1]{{[#1]}}

\usepackage{diagrams}
\newarrow{Equal} =====
\newarrow{Into} C--->
\newarrow{Onto} ----{>>}
\newarrow{Dashto} {}{dash}{}{dash}>
\newarrow{Dashonto} {}{dash}{}{dash}{>>}

\newcommand{\fullversion}[1]{}

\newcommand{\tail}{\operatorname{tl}} 
\newcommand{\myomega}{\omega}
\newcommand{\cons}{\cdot}%{\operatorname{:{\!}:}}
\newcommand{\berger}{\varepsilon^{\text{Berger}}}
\newcommand{\bergerp}{\varepsilon^{\text{Berger}'}}
\newcommand{\varepsilonprod}{\varepsilon^{\prod}}
\newcommand{\varepsilonprodp}{\varepsilon^{\prod'}}
\newcommand{\varepsilonprodpp}{\varepsilon^{\prod''}}
\newcommand{\eberger}{\exists^{\text{Berger}}}
\newcommand{\ebergerp}{\exists^{\text{Berger}'}}
\newcommand{\evarepsilonprod}{\exists^{\prod}}
\newcommand{\aberger}{\forall^{\text{Berger}}}
\newcommand{\abergerp}{\forall^{\text{Berger}'}}
\newcommand{\avarepsilonprod}{\forall^{\prod}}
\newcommand{\cat}{}%{\cdot}
\newcommand{\eqdef}{\mathrel{:=}}%{\stackrel{\text{def}}{=}}
\newcommand{\id}{\operatorname{id}}
\newcommand{\K}{\mathcal{K}}
\newcommand{\Z}{\mathbb{Z}}%{\mathcal{Z}}
\newtheorem{cor}{Corollary}[section]
\newtheorem{lemma}[cor]{Lemma}
\newtheorem{Def}[cor]{Definition}
\newtheorem{Alg}[cor]{Algorithm}
\newtheorem{Algs}[cor]{Algorithms}
\newtheorem{Hypothesis}[cor]{Hypothesis}
\newtheorem{Rem}[cor]{Remark}
\newtheorem{Rems}[cor]{Remarks}
\newtheorem{prop}[cor]{Proposition}
\newtheorem{Example}[cor]{Example}
\newtheorem{conjecture}[cor]{Conjecture}
\newtheorem{Experiment}[cor]{Experiment}
\newtheorem{Question}[cor]{Question}
\newtheorem{Examples}[cor]{Examples}
\newtheorem{theorem}[cor]{Theorem}
\newtheorem{Discussion}[cor]{Discussion}
\newenvironment{definition}{\begin{Def}\em}{\end{Def}}
\newenvironment{algorithm}{\begin{Alg}\em}{\end{Alg}}
\newenvironment{algorithms}{\begin{Algs}\em}{\end{Algs}}
\newenvironment{discussion}{\begin{Discussion}\em}{\end{Discussion}}
\newenvironment{hypothesis}{\begin{Hypothesis}\em}{\end{Hypothesis}}
\newenvironment{experiment}{\begin{Experiment}\em}{\end{Experiment}}
\newenvironment{rem}{\begin{Rem}\em}{\end{Rem}}
\newenvironment{rems}{\begin{Rems}\em}{\end{Rems}}
\newenvironment{question}{\begin{Question}\em}{\end{Question}}
\newenvironment{example}{\begin{Example}\em}{\end{Example}}
\newenvironment{examples}{\begin{Examples}\em}{\end{Examples}}
\newcommand{\myparagraph}{\paragraph}%[1]{\vspace{0.5ex}\noindent{\itshape\bfseries #1}}%{\vspace{-2ex}\paragraph}

\newcommand{\below}{\sqsubseteq}
\newcommand{\bnf}{\mathrel{::=}}
\newcommand{\lift}[1]{#1_{\bot}}
\newcommand{\N}{\mathbb{N}}
\newcommand{\R}{\mathbb{R}}
\newcommand{\Q}{\mathbb{Q}}
\newcommand{\I}{\mathbb{I}}
\newcommand{\Bool}{2}%{\mathbb{B}}
\newcommand{\Sierp}{\mathcal{S}}
\newcommand{\pN}{{\N_{\bot}}}
\newcommand{\pT}{\mathcal{T}}
\newcommand{\pU}{\mathcal{U}}
\newcommand{\pI}{\mathcal{I}}
\newcommand{\pM}{\mathcal{M}}
\newcommand{\pR}{\mathcal{R}}
\newcommand{\pB}{{2_{\bot}}}
\newcommand{\pBool}{\mathcal{B}}
\newcommand{\If}{\,\mathrel{\operatorname{if}}}
\newcommand{\Then}{\mathrel{\operatorname{then}}}
\newcommand{\Else}{\mathrel{\operatorname{else}}}
\newcommand{\True}{1}%{\operatorname{tt}}%{\operatorname{true}}
\newcommand{\False}{0}%{\operatorname{ff}}%{\operatorname{false}}
\newcommand{\domain}[1]{{\D_{#1}}}
\newcommand{\total}[1]{{\T_{#1}}}
\newcommand{\totaleq}[1]{\sim_{#1}}
\newcommand{\quo}[1]{\qq_{#1}}
\newcommand{\qq}{\rho}
\newcommand{\D}{D}
\newcommand{\E}{E}
\newcommand{\C}{C}
\newcommand{\T}{T}
\newcommand{\kk}[1]{{\C_{#1}}}
\newcommand{\Scott}{\operatorname{Scott}}
\newcommand{\tproduct}{\sigma \times \tau}
\newcommand{\tfunction}{\sigma \to \tau}
\newcommand{\rh}[1]{\rho_{#1}}
\newcommand{\comp}{\circ}
\newcommand{\siff}{\iff}%{\Leftrightarrow}
\newcommand{\Search}{\operatorname{\mathcal{S}}}
\newcommand{\RR}{\operatorname{\text{\boldmath{$R$}}}}
\newcommand{\XX}{\operatorname{\text{\boldmath{$X$}}}}
\newcommand{\YY}{\operatorname{\text{\boldmath{$Y$}}}}
\newcommand{\ZZ}{\operatorname{\text{\boldmath{$Z$}}}}
\newcommand{\dd}{\operatorname{\text{\boldmath{$d$}}}}
\newcommand{\g}{\operatorname{\text{\boldmath{$g$}}}}
\newcommand{\f}{\operatorname{\text{\boldmath{$f$}}}}
\newcommand{\x}{\operatorname{\text{\boldmath{$x$}}}}
\newcommand{\e}{\operatorname{\text{\boldmath{$e$}}}}
\newcommand{\sK}{\operatorname{\text{\boldmath{$K$}}}}
\newcommand{\sF}{\operatorname{\text{\boldmath{$F$}}}}
\newcommand{\sU}{\operatorname{\text{\boldmath{$U$}}}}
\newcommand{\sV}{\operatorname{\text{\boldmath{$V$}}}}
\newcommand{\sQ}{\operatorname{\text{\boldmath{$Q$}}}}
\newcommand{\sP}{\operatorname{\text{\boldmath{$P$}}}}
\newcommand{\sE}{\operatorname{\text{\boldmath{$E$}}}}

\begin{document}

\author{Mart\'\i n Escard\'o \\
{\normalsize School of Computer Science, University of Birmingham, UK}}

\title{Computability of continuous solutions of \\ higher-type equations}

\date{{\normalsize Work in progress, draft version of \today}}

\maketitle

\section{Introduction}


We investigate the following problem in higher-type computation: 
\begin{quote}
  Given computable functionals $f \colon X \to Y$ and $y \in X$,
  compute $x \in X$ such that $f(x)=y$, if such an $x$ exists.
\end{quote}
We show that if $x$ is unique and $X$ and $Y$ are retracts
of Kleene--Kreisel spaces with~$X$ exhaustible in the sense
of~\cite{escardo:exhaustible}, then $x$ is computable uniformly in
$f$, $y$ and~$\forall_X$. The computation of unique solutions of
equations of the form $g(x)=h(x)$ with $g,h \colon X \to Y$ is easily
reduced to the previous case, because there are group structures on
the ground types that can be lifted componentwise to product types and
pointwise to function types. And, by cartesian closedness, the case in
which $g$ and $h$ computably depend on parameters, and in which the
solution computably depends on the same parameters, is covered.
Moreover, because Kleene--Kreisel spaces are closed under finite
products and countable powers, this includes the solution of finite
and countably infinite systems of equations with functionals of
finitely many or countably infinitely many variables.

We also consider generalizations to computational metric spaces that
apply to computational analysis, where $f$ can be a functional and $x$
a function. And we develop examples of sets of analytic functions that
are exhaustible and can play the role of the space~$X$.

The nature of the problem forces us to consider partial functionals
defined on total data, because the unique-solution functional is
promised to be defined on $(f,y,\phi)$ iff there is a unique $x$ with
$f(x)=y$ and $\phi=\forall_X$,
and cannot fulfill a better promise (see~\cite{beeson} and
the discussion in Section~\ref{discussion}).  Similar headaches
involving totality and partiality took place in our previous work on
exhaustible sets.  Before developing the above results, we consider a
remedy in Section~\ref{remedy}. In Section~\ref{reformulation} we
reformulate the definitions and results of our previous work on
exhaustible sets in a technically and conceptually simpler way using
the set-up of Section~\ref{remedy}. And then finally from
Section~\ref{proceed} onwards we proceed to develop the results discussed
above.

\section{Prerequisites and disclaimer} \label{prerequisites}

For the moment, we assume the notation and terminology of our paper
paper ``Exhaustible sets in higher-type
computation''~\cite{escardo:exhaustible}.  At present, the exposition
lacks proper background and credits and citations, and is written for
readers who are conversant with a particular, but also general,
approach to computability. Later we'll also address a more general set
of readers. This is a preliminary report, with emphasis on the
technical results.  A more detailed version will be written later. 
% If you feel that you or somebody else should be given credit or
% priority to any result, please let me know that I'll be happy to
% rectify this in a future version.

\section{Discussion on uniqueness and exhaustibility} 
\label{discussion}

This section justifies the assumptions made in later sections, and can
be safely skipped.
%
Given Kleene--Kreisel spaces $X$ and $Y$,
a continuous function $f \colon X \to Y$ and $y \in Y$,
we wish to compute $x \in X$, uniformly in $f$ and $y$, such that
\[
f(x)=y.
\]
We discuss several cases for $X$ and $Y$ and explain why, in general,
further assumptions and data are required.

The simplest case is $X=Y=\N$, for which the algorithm
\[
\mu x. f(x)=y
\]
computes a solution iff a solution exists. This is subsumed by the
next case. 

Consider $X$ arbitrary and $Y=\N$. By the Kleene--Kreisel density
theorem, $X$ has a computable dense sequence $\delta \colon \N \to X$, and,
by continuity of~$f$ and discreteness of~$\N$, if the equation has a
solution, there is one of the form $x=\delta_n$ for some~$n$. Hence the
algorithm
\[
x = \delta_{\mu n. f(\delta_n)=y}
\]
computes a solution iff a solution exists. Moreover, in this
particular case it is semi-decidable whether a solution exists,
with the algorithm $\exists n. f(\delta_n)=y$.
% Notice also that, by
% discreteness of $\N$, whenever there is a solution, there are
% miuncountably many solutions unless $X=\N$, and so solutions are never
% unique in this case.

Now consider $X=2$ and $Y=\N^\N$. Then a function $f \colon X \to Y$
amounts to two functions $f_0,f_1 \colon \N \to \N$, and computing a
solution to the above equation amounts to finding $i \in 2$ such that
$f_i = y$, that is, $f_i(n)=y(n)$ for all $n \in \N$. In other words,
under the assumption that
\begin{quote}
  $f_0 = y$ or $f_1 = y$,
\end{quote}
we want to find $i$ such that $f_i = y$. If the only data supplied to
the desired algorithm are $f_0,f_1,y$, this is not possible, because
no finite amount of information about the data can determine that one
particular disjunct holds (a similar situation is worked out in detail
below). However, if we instead assume that
\begin{quote}
  one of $f_0 = y$ and $f_1 = y$ holds, but not both,
\end{quote}
then we can compute $i$ as follows:
\begin{quote}
  Find the least $n$ such that $f_0(n)\ne y(n)$ or $f_1(n) \ne y(n)$,
  and let $i$ be the unique number such that $f_i(n)=y(n)$.
\end{quote}
Thus, in general, it is not possible to compute solutions unless we
know that they are unique, and in this particular case one can compute
unique solutions. This kind of phenomenon is well known --- see
e.g.~\cite{beeson}.

Next consider $X=\N$ and $Y=\N^\N$, and assume that the equation
$f(x)=y$ has a unique solution. Now it is no longer possible to
compute it uniformly in $f$ and $y$. For suppose there existed a
computable partial functional $s \colon (\N^\N)^\N \times \N^\N \to
\N$, defined on some superset of 
\[
S=\{ (f,y) \mid \text{$f(x)=y$ has a unique solution $x_0$}\}, 
\]
such that $x_0=s(f,y)$ is the solution
for $f,y \in S$.  By continuity, for any $(f,y) \in S$
there is a number $n$ such that $s(f,y)=x_0=s(g,y)$ for every $g$ such
that $g(x)(i)=f(x)(i)$ for all $x < n$ and $i<n$.  W.l.o.g.\ we can
assume that $x_0 < n$ by replacing $n$ by $\max(n,x_0)+1$ if
necessary.  Choose~$g$ defined by
\[
g(x)(i) = 
\begin{cases}
f(x)(i) & \text{if $x<n$ and $i<n$,} \\
y(i) & \text{if $x=n$}, \\
y(i)+1 & \text{otherwise.}
\end{cases}
\]
By construction, $x=n$ is the unique solution of $g(x)=y$, and hence
$s(g,y)= n$, which contradicts $s(g,y)=x_0$, and concludes the proof
of the impossibility claim. However, if we know that there is a unique
solution in a finite set $K \subseteq \N$, then the solution can be
found uniformly in $f,y$ and a finite enumeration $e_0,\dots, e_{k-1}$
of~$K$, as follows:
\begin{quote}
  Find $n$ and $j<k$ such that the decidable conditions $\forall
  i<n.f(e_j)(i)=y(i)$ and $\forall l<k, l \ne j. \exists i<n.
  f(e_l)(i) \ne y(i)$ hold, and take $x=e_j$.
\end{quote}
This generalizes the situation $X=2$, and is a particular case of
Theorem~\ref{unique:solution} below, which shows that unique solutions
in exhaustible subsets $X$ of Kleene--Kreisel spaces are computable
uniformly in $f$, $y$ and~$\forall_X$.

\section{Computational spaces} \label{remedy}

In order to simplify the development, both technically and
conceptually, we work with a computational category
investigated by Bauer~\cite{bauer:thesis}, with continuous versions discussed
by Bauer, Birkedal and Scott~\cite{bauer:birkedal:scott}. This goes
back to a 1996 unpublished manuscript by Scott in which he introduced
equilogical spaces.  In summary, we work with the category
$\operatorname{Mod}(\operatorname{EffScott})$, where
$\operatorname{EffScott}$ is the cartesian closed category of
computable maps of effectively given Scott domains. We propose to
refer to the objects of this category as \emph{computational spaces}
and to its morphisms as \emph{computable functions}.  This general
terminology for the objects is justified by the fact that this
category incorporates, as discussed
in~\cite{bauer:birkedal:scott,MR1948051}, all known models of
effective computation, including effectively given domains,
Kleene--Kreisel spaces, Berger's domains with totality, Weihrauch's
TTE, Blanck's domain representations, Schr\"oder and Simpson's QCB
spaces. But this is only a justification for our terminology: our main
reason for considering this category is simplicity in the formulation
and proofs of theorems, rather than just generality.

\subsection{Definition of the category of computational spaces}

Explicitly, the category we are interested in is defined as follows:
\begin{enumerate}
\item A \emph{computational space}, or \emph{space} for short, is a list
  $X=(|X|,D_X, T_X,\rho_X)$ consisting of an underlying set~$|X|$,
  often denoted by $X$ by an abuse of notation, equipped with an
  effectively given domain $D_X$ together with a set $T_X \subseteq
  D_X$ and a surjection $\rho_X \colon T_X \to |X|$.  

  The structure of the underlying set $X$ is presented in diagrammatic
  form as
\begin{diagram}[small]
   D_X & \lInto & T_X & \rOnto^{\rho_{X}} & X.
\end{diagram}

The idea is that the domain $D_X$ collects all possible total and
partial data that a particular computational mechanism can produce,
the subset $T_X$ collects the data one is interested in for a
particular purpose, and the set $X$ identifies different pieces of
relevant data that one regards as equivalent for that
purpose.

\pagebreak[3]
\item A computable map $f \colon X \to Y$ is a function of the
  underlying sets that is tracked by some computable map $\godel{f} \colon
  D_X \to D_Y$, in the sense that the diagram
\begin{diagram}[small]
   D_X & \lInto & T_X & \rOnto^{\rho_{X}} & X \\
   \dTo^{\godel{f}} &  & \dDashto &  & \dTo_{f} \\
   D_Y & \lInto & T_Y & \rOnto^{\rho_{Y}} & Y
\end{diagram}
commutes for some function $T_X \to T_Y$, which then must be the
restriction of~$\godel{f}$. This is equivalent to saying that, for any $d
\in T_X$, we have that $\godel{f}(d) \in T_Y$ and
\[ f(\rho_X(d))=\rho_Y(\godel{f}(d)).\]

\item Identities and composition are inherited from the category of sets.
\end{enumerate}
Notice that, in this category, computable maps $f \colon X \to Y$ are
always defined, in the mathematical sense, for every $x \in X$.
Partiality, in the computational sense, if desired, is achieved by
allowing $Y$ to have undefined or partially defined elements. We
denote finite products and exponentials in this category by the usual
notations $X \times Y$ and~$Y^X$, and sometimes $(X \to Y)$. For their
construction, see~\cite{bauer:thesis,bauer:birkedal:scott}. We just
emphasize that the underlying set of $Y^X$ consists of the maps that
are tracked by continuous, rather than computable functions, despite
the fact that morphisms of the category are tracked by computable
functions.  Computable maps, and elements of exponentials, are
continuous with respect to the computational topology:
\begin{definition}
  We endow a computational space $X$ with the $\rho_X$-quotient topology of
  the relative Scott topology of $D_X$ on $T_X$, called the \emph{computational
    topology of $X$}. \qed
\end{definition}
 \pagebreak[3]

\begin{examples} \label{examples}
We agree that
\begin{enumerate}
\item An effectively given domain $D$ is regarded as the space
described by
\begin{diagram}[small]
   D & \lEqual & D & \rEqual & D.
\end{diagram}
Then a computable map $D \to E$ of effectively given domains is a
computable map in the usual sense. That is, this construction is a
full embedding of the category of effectively given domains into the
category of computational spaces. Moreover, this embedding preserves
the cartesian closed structure.
\item This applies, in particular, to the domains $D_\sigma$ 
of the simple type hierarchy.
\item A Kleene--Kreisel space $C_\sigma$
is regarded as the space described by
\begin{diagram}[small]
   D_\sigma & \lInto & T_\sigma & \rOnto^{\rho_\sigma} & C_\sigma.
\end{diagram}
Then again a computable map $C_\sigma \to C_\tau$ is a computable map
in the usual sense, and this construction is a full embedding of
the category of Kleene--Kreisel spaces with computable maps into the
category of computational spaces, with the embedding preserving
the cartesian closed structure~\cite{bauer:birkedal:scott}.

\item $\N$ denotes the Kleene--Kreisel space $C_{\iota}$,
\begin{diagram}[small]
   \pN & \lInto & \N & \rEqual & \N, 
\end{diagram}
which has the discrete topology. 

\item $2$ denotes the Kleene--Kreisel space $C_o$, with underlying set $\{0,1\}$,
\begin{diagram}[small]
   \pB & \lInto & 2 & \rEqual & 2,
\end{diagram}
which again has the discrete topology (rather than the Sierpinski
topology).

NB.\ Of course, it is not necessary to include the base type $o$
because $2$ also arises as a computable retract of the Kleene--Kreisel
space~$\N$. \qed
\end{enumerate}
\end{examples}

\pagebreak[3]
\subsection{Subspaces, images and quotients}

We'll often use the following constructions to build spaces:
\begin{definition} \label{subspace} \leavevmode
A subspace inclusion $S \to X$ is a computable map of the form:
\begin{diagram}[small]
   D_S & \lInto & T_S & \rOnto^{\rho_{S}} & S \\
   \dEqual &  & \dInto &  & \dInto \\
   D_X & \lInto & T_X & \rOnto^{\rho_{X}} & X.
\end{diagram}
Given a space $X$ and a subset $S$, the relative subspace structure on
$S$ is given by taking $D_S=D_X$, $T_S = \rho_X^{-1}(S)$ and $\rho_S$
the restriction of $\rho_X$.  \qed
\end{definition}
\begin{definition}
  The image of a computable map $f \colon X \to Y$ is the
  set-theoretical image $f[X]$ equipped with the subspace structure.
  \qed
\end{definition}

\begin{definition} \label{quotient} \leavevmode A quotient map $q
  \colon A \to X$ is a computable map of the form
 \begin{diagram}[small]
    D_A & \lInto & T_A    & \rOnto^{\rho_A} & A \\
    \dEqual    &          & \dEqual &         & \dOnto_{q} \\
    D_X & \lInto & T_X    & \rOnto^{\rho_{X}} & X.
 \end{diagram}
 Given a space $A$, a set $X$, and a surjection $q \colon A \to X$,
the quotient structure on~$X$ is then given by
 $D_X=D_A$, $T_X=T_A$, and $\rho_X = q \comp \rho_A$. \qed
\end{definition}
\pagebreak[3]
The following will play an important role:
\begin{examples} \label{real} \leavevmode
  \begin{enumerate}
  \item The sets $\Z$ of integers and $\Q$ of rational numbers are
    regarded as discrete spaces via standard codings based on natural
    numbers.
  \item Then $\Q^\N$ is also a space by cartesian closedness, and so
    is its subset $\cauchy(\Q)$ of Cauchy sequences $q \in
    \Q^N$ such that $|q_n - q_{n+1}|<2^{-n}$ with the subspace structure.
  \item The \emph{real line} in the category of computational spaces
    is the set $\R$ of real numbers with the quotient structure
    induced by the surjection $\lim \colon \cauchy(\Q) \to \R$ that
    takes a sequence to its limit.
  \item We endow $[0,\infty)$, $[0,1]$, $[-1,1]$ etc.\ with the
    relative subspace structure of~$\R$. \qed
  \end{enumerate}
\end{examples}
\noindent
This material can and should be put in a proper categorical setting,
but at the moment we are giving priority to applications.

\subsection{Representing spaces}

For certain applications, we consider the structure $T_X$ of a space
$X$ as a space on its own (an object of the category, denoted by
$\godel{X}$), and the structure $\rho_X \colon T_X \to X$ of $X$ as a
computable map (a morphism of the category, denoted by the same name
without danger of
ambiguity):
\begin{definition} \label{representation} For any space $X$,
  define the \emph{representing space} $\godel{X}$ by
  \[ 
  D_{\godel{X}}=D_X, \quad T_{\godel{X}}=|\godel{X}|=T_X, \quad
  \rho_{\godel{X}}=\id_{T_X}.
  \] 
  In diagrammatic form, $\godel{X}$ is
  the space
\begin{diagram}[small]
   D_X & \lInto & T_X & \rEqual & T_X,
\end{diagram}
which of course is also written as
\begin{diagram}[small]
  D_{\godel{X}} & \lInto & T_{\godel{X}} & \rOnto^{\rho_{\godel{X}}} & \godel{X},
\end{diagram}
using our general convention for denoting spaces. Then $\rho_X$ is
regarded as a quotient map $\godel{X} \to X$, called the
\emph{representation function}, as follows:
\begin{diagram}[small]
  D_X & \lInto & T_X & \rEqual & T_X \\
   \dEqual &  & \dEqual &  & \dOnto_{\rho_{X}} \\
   D_X & \lInto & T_X & \rOnto^{\rho_{X}} & X.
\end{diagram}
Because it is tracked by the identity, it is computable. \qed
\end{definition}
Notice that $\godel{X \times Y} = \godel{X} \times \godel{Y}$.  Of
course, the map $\godel{f}$ defined below is not uniquely determined
by~$f$, and hence there is no computable map $\godel{-} \colon Y^X \to
\godel{Y}^{\godel{X}}$.
\begin{lemma} \label{representation:function} \label{kk:exhaustible:image}
  For any computable function $f \colon X \to Y$ there is a computable function
  $\godel{f} \colon \godel{X} \to \godel{Y}$ such that $f \comp \rho_X
  = \rho_Y \comp \godel{f}$:
\begin{diagram}[small]
\godel{X} & \rTo^{\godel{f}} & \godel{Y} \\
\dOnto^{\rho_X} & & \dOnto_{\rho_Y} \\
X & \rTo^{f} & Y.
\end{diagram}
Moreover, if $f \colon X \to Y$ is a surjection, then so is any representative
$\godel{f}$.
\end{lemma}
\begin{proof}
  Take the restriction of some computable $\godel{f} \colon D_X \to
  D_Y$ that tracks $f \colon X \to Y$ to $T_X \to T_Y$.  This regarded
  as a map $\godel{f} \colon \godel{X} \to \godel{Y}$ is computable,
  because it is also tracked by $\godel{f} \colon D_X \to D_Y$.  If
  $f$ is a surjection, three arrows of the square, without counting
  $\godel{f}$, are surjections, and hence the remaining one must also
  be a surjection.
\end{proof}

\section{Exhaustible spaces} \label{reformulation}

In previous work we investigated exhaustible \emph{subsets} of effectively
given domains, with emphasis on exhaustible sets of total elements, or
more precisely entire sets~\cite{escardo:exhaustible}. Here we define
exhaustible \emph{spaces} and transfer results for them
from that work. 

\subsection{Exhaustibility and related notions}
\begin{definition} \label{new:exhaustible}
\leavevmode
  \begin{enumerate}
  \item A space $K$ is called \emph{exhaustible} if the universal
    quantification functional \[ \forall_K \colon 2^K \to 2\] defined
    by
   \[
   \forall_K(p)=1 \iff \text{$p(x)=1$ for all $x \in K$}
   \]
   is computable.
 \item It is called \emph{searchable} if there is a
   computable selection functional \[\varepsilon_K \colon 2^K \to K\] such
   that for all $p \in 2^K$, if there is $x \in K$ with $p(x)=1$ then
   $x=\varepsilon_K(p)$ is an example. 
 \item A set $F \subseteq X$ is \emph{decidable} if its characteristic
map $X \to 2$ is computable. \qed
 \end{enumerate}
\end{definition}
\noindent
Equivalently, $K$ is exhaustible iff the map 
$\exists_K \colon 2^K \to 2$ defined by
\[
\exists_K(p)=1 \iff \text{$p(x)=1$ for some $x \in K$}
\]
is computable, as $\exists_K$ and $\forall_K$ are inter-definable
using the De Morgan laws. 
If $K$ is searchable, then it is exhaustible,
because
\[ \exists_K(p)=p(\varepsilon_K(p)).\]
The empty set is exhaustible, but
not searchable, because there is no map $2^\emptyset \to \emptyset$.

\medskip
\pagebreak[3]
We regard a subset $K$ of a domain $D$ as a space by firstly regarding
$D$ as a space (Examples~\ref{examples}) and then endowing
$K$ with subspace structure (Definition~\ref{subspace}).
\begin{lemma} 
The following are equivalent for any subset $K$ of a domain $D$:
\begin{enumerate}
  \item $K$ is exhaustible as a set, in the sense of~\cite{escardo:exhaustible}.
  \item $K$ is exhaustible as a space, in the sense of
    Definition~\ref{new:exhaustible}. 
\end{enumerate}
The two statements remain equivalent with \emph{searchable} in place
of \emph{exhaustible}. 
\end{lemma}
\begin{proof}
We just observe that if a
map $\godel{p} \colon D \to 2_{\bot}$ tracks a map $p \colon K \to 2$,
then $\godel{p}$ is defined on the set $K$, in the terminology
of~\cite{escardo:exhaustible}. 
\end{proof}
\subsection{Exhaustible subspaces of Kleene--Kreisel spaces}

%The following results are imported to our setting from~\cite{escardo:exhaustible}.
\begin{lemma}\leavevmode
\begin{enumerate} 
\item \label{cantor} The Cantor space $2^\N$ is searchable.

\item \label{exhaustible} Any exhaustible subspace of a
  Kleene--Kreisel space is compact in the computational topology, and
  moreover, if it is non-empty, it is searchable, a computable retract
  of the Kleene--Kreisel space, and a computable image of the Cantor
  space.

\item \label{searchable:1} Searchable spaces are closed under
  computable images, finite intersections with decidable sets, and
  finite products.

\item \label{searchable}
Searchable spaces that share the same domain structure are closed
  under countable products.
\end{enumerate}
\end{lemma}
\begin{proof}
  (\ref{cantor}):~It is searchable as a subset of a domain, using
  Berger's algorithm~\cite{escardo:exhaustible}.
%
  (\ref{exhaustible}):~This is formulated and proved for entire sets
  in~\cite{escardo:exhaustible}.
%
  (\ref{searchable:1}):~The algorithms given
  in~\cite{escardo:exhaustible} apply more generally to computational
  spaces. 
%
  (\ref{searchable}):~Use the product algorithm
  of~\cite{escardo:exhaustible} to building the tracking map.
\end{proof}

It is well known that any Kleene--Kreisel space is a computable
retract of a Kleene--Kreisel space of the form~$\N^X$, and we use this
in our arguments of Section~\ref{equations} below, which then
automatically apply to $kk$-spaces:
\begin{definition}
  A \emph{$kk$-space} is a computable retract of a Kleene--Kreisel
  space. \qed
\end{definition}
\noindent
Because retracts compose, the $kk$-spaces are precisely the computable
retracts of the Kleene--Kreisel spaces of the form~$\N^X$.  By a
general and easy argument, $kk$-spaces are closed under the formation
of finite products and exponentials.  Moreover, they form the smallest
collection of spaces containing the Kleene--Kreisel spaces and
satisfying this condition. For example, the discrete space
$3=\{-1,0,1\}$ and the exponential $3^\N$ are $kk$-spaces but not
Kleene--Kreisel spaces.
\begin{lemma} \label{kk:subspace} \leavevmode
  \begin{enumerate}
  \item \label{kk:1} Any exhaustible $kk$-space is compact, and
    searchable if it is non-empty.
\item \label{kk:2} An exhaustible space is a $kk$-space iff
  it is a subspace of some Kleene--Kreisel space.
  \end{enumerate}
\end{lemma}
\begin{proof}
  (\ref{kk:1}):~Because $kk$-spaces are subspaces of Kleene--Kreisel
  spaces. (\ref{kk:2}):~Because any non-empty exhaustible subspace of
  a Kleene--Kreisel space is a computable retract.
\end{proof}

\subsection{Spaces with exhaustible $kk$-spaces of representatives}
\label{kk:exhaustible:section}

The remainder of this section is not needed until
Section~\ref{metric:spaces}.  We observed
in~\cite{escardo:exhaustible} that if a space $X$ is connected, as
will be the case in applications of
Theorem~\ref{unique:solution:metric} to analysis, computable maps $p
\colon X \to 2$ are constant, and hence $X$ is trivially exhaustible
if it has some computable point $x_0$, as its quantifier is computable
as $\forall_X(p)=p(x_0)$.  On the other hand, any $kk$-space $X$ is
totally separated (the clopen sets, or equivalently, the continuous
predicates $p \colon X \to 2$, separate the points), which implies
that it is totally disconnected (the connected components are
singletons). This partly explains why $kk$-spaces have the good
properties discussed above, for example that exhaustibility implies
compactness, which, as we have just seen, fails for connected spaces.
Moreover, $kk$-spaces are computationally totally separated, in the
sense that the computable predicates separate the points.  This
motivates the use, in Theorem~\ref{unique:solution:metric}, of
computational (metric) spaces~$X$ with exhaustible $kk$-spaces of
representatives. Such a space, being the computable (and hence
continuous) image of the representing map, is exhaustible and compact.

In general, $kk$-spaces are topological \emph{quotients} of subspaces
of hereditarily total elements of domains under the Scott topology, as
is well known.  By the continuous versions of the constructions
of~\cite{escardo:exhaustible}, every \emph{compact} $kk$-space arises
as a topological \emph{subspace} of total elements of such a domain.
Concrete examples are given below. Recall the notion of
representing space given in Definition~\ref{representation}.

\begin{definition} \label{kk:exhaustible} A computational space is
  called \emph{$kk$-exhaustible} if it is isomorphic to some space $X$
  such that its representing space~$\godel{X}$ is an exhaustible
  $kk$-space.  \qed
\end{definition}
\noindent
(The isomorphism is automatically required to be computable,
because the morphisms of our underlying category are computable maps.)
\begin{examples} \label{kk:examples} The following are $kk$-exhaustible:
  \begin{enumerate}
  \item \emph{Any interval $[a,b]$ with $a,b \in\R$ computable.}
    \begin{quote}
      Given our particular construction of $[-1,1]$ in
      Examples~\ref{real}, its representing space is not exhaustible.
      But, as is well known $[-1,1]$ is isomorphic to the set $[-1,1]$
      endowed with signed-digit binary representation, with
      representing space $3^\N$ where $3=\{-1,0,1\}$, which is an
      exhaustible $kk$-space. The general case $[a,b]$ is easily reduced to this.
    \end{quote}

  \item \emph{Finite products, countable powers and retracts of $kk$-exhaustible spaces.}

    \begin{quote}
      We omit the routine details, at least for the moment. 
    \end{quote}
  \end{enumerate}
Of course, the real line is not $kk$-exhaustible, because it is not compact. \qed
\end{examples}
\noindent
We'll see in Section~\ref{applications} that interesting subspaces of
analytic functions of the space $\R^{[-\epsilon,\epsilon]}$, with $0 <
\epsilon < 1$, are $kk$-exhaustible, which will allow us to compute
solutions of functional equations with analytic unknowns.


\section{Equations over Kleene--Kreisel spaces}
\label{equations} \label{proceed}

\begin{theorem} \label{unique:solution} \label{bijection} If $f\colon
  X \to Y$ is a computable map of $kk$-spaces with $X$ exhaustible,
  and $y \in Y$ is computable, then, uniformly in $\forall_X$, $f$,
  and~$y$:
\begin{enumerate}
\item \label{unique:solution:2} It is semi-decidable whether the
  equation $f(x) = y$ fails to have a solution $x \in X$.
\item \label{unique:solution:1} If $f(x) = y$ has a unique solution
  $x \in X$, then it is computable. 
\end{enumerate}
Hence if $f\colon X \to Y$ is a computable bijection then it has a
computable inverse, uniformly in~$\forall_X$ and~$f$.
\end{theorem}
\noindent
The conclusion is a computational version for $kk$-spaces of the
topological theorem that any continuous bijection of compact Hausdorff
spaces is a homeomorphism.
%
This gives an alternative route to the following fact
established in~\cite{escardo:exhaustible}:
\begin{theorem} \label{alternative}
  Any exhaustible $kk$-space
  is computably homeomorphic to an exhaustible subspace of the Baire
  space~$\N^\N$. 
\end{theorem}
\begin{proof}
  Let $K$ be an exhaustible $kk$-space, let $s \colon K \to \N^Z$ and
  $r \colon \N^Z \to K$ be computable maps with $r \comp s = \id_K$
  and $Z$ a Kleene--Kreisel space, and let $\delta_n \in Z$ be a
  computable dense sequence. The subspace $X = s(K) \subseteq \N^Z$,
  being a computable image of an exhaustible space, is itself
  exhaustible. As in~\cite{escardo:exhaustible}, we consider the map
  $X \to \N^\N$ that sends $u \in X$ to the
  sequence~$u(\delta_n)$, but we argue using
  Theorem~\ref{bijection} instead.  Let $f \colon X \to Y$ be the
  restriction of this map to its image $Y \subseteq \N^\N$.  By
  density, $f$ is one-to-one, and, by construction, it is onto, and
  hence it has a computable inverse. Therefore there is computable map
  $g \colon K \to Y$ defined by $g(k)=f(s(k))$ with computable inverse
  given by $g^{-1}(\alpha)=r(f^{-1}(\alpha))$.
\end{proof}

\medskip
We now prove Theorem~\ref{unique:solution}.  The following will be
applied to semi-decide that equations fail to have solutions:
\begin{lemma} \label{empty} Let $X$ be an exhaustible $kk$-space and
  $K_n \subseteq X$ be a sequence of sets that are decidable uniformly
  in $n$ and satisfy $K_n \supseteq K_{n+1}$.
\begin{quote}
  Emptiness of $\bigcap_n K_n$ is semi-decidable,
\end{quote}
uniformly in the quantifier of $X$ and the sequence of decision
procedures for $K_n$.
\end{lemma}
\begin{proof} 
  Because $X$ is compact by exhaustibility, $K_n$ is also compact as
  it is closed.  Because $X$ is Hausdorff, $\bigcap_n K_n = \emptyset$
  iff there is $n$ such that $K_n=\emptyset$.  But emptiness of this
  set is decidable uniformly in $n$ by the algorithm $\forall x \in X.
  x \not\in K_n$.  Hence a semi-decision procedure is given by
  $\exists n.\forall x \in X. x \not\in K_n$.
\end{proof}

As a preparation for a lemma that will be applied to
compute unique solutions, notice that if a singleton $\{u\} \subseteq
\N^Z$ is exhaustible, then the function $u$ is computable, because
$u(z) = \mu m.\forall v \in\{ u \}.u(z)=m$. Moreover, $u$ is
computable uniformly in~$\forall_{\{u\}}$, in the sense that there is
a computable functional
\[ \text{$U \colon S \to \N^Z$ \quad with \quad $S = \{ \phi \in
  2^{2^{\N^Z}} \mid \text{$\phi=\forall_{\{v\}}$ for some $v \in
    \N^Z$} \}$},
\]
such that \[ u=U\left(\forall_{\{u\}}\right),\] 
namely
\[
U(\phi)(z)=\mu m.\phi(\lambda u.u(z)=m).
\]
%In what follows, we won't explicitly exhibit uniformity functionals.
%
Lemma~\ref{unique} below generalizes this, using an
argument from~\cite{escardo:exhaustible} that was originally used to
prove that non-empty exhaustible subsets of $kk$-spaces are computable
images of the Cantor space and hence searchable.  Here we find
additional applications and further useful generalizations.

\pagebreak[3]
\begin{lemma} \label{unique} Let $X$ be a $kk$-space and $K_n
  \subseteq X$ be a sequence of sets that are exhaustible uniformly in
  $n$ and satisfy $K_n \supseteq K_{n+1}$.
  \begin{quote}
  If $\bigcap_n
    K_n$ is a singleton $\{x\}$, then $x$ is computable,
  \end{quote}
uniformly in the sequence $\forall_{K_n}$.
\end{lemma}
\begin{proof} 
  Let $X$ be a $kk$-space and $s \colon X \to \N^Z$ and $r \colon \N^Z
  \to X$ be computable functions with $r \comp s = \id_X$.  It
  suffices to show that the function $u=s(x) \in \N^Z$ is computable,
  because $x = r(u)$.  The sets $L_n = s(K_n) \subseteq \N^Z$, being
  computable images of exhaustible sets, are themselves exhaustible.
%
  For any $z \in Z$, the set $U_{z} = \{ v \in \N^Z \mid v(z) =
  u(z) \}$ is clopen and $\bigcap_n L_n = \{ u \} \subseteq U_{z}$.
  Because $\N^Z$ is Hausdorff, because $L_n \supseteq L_{n+1}$,
  because each $L_n$ is compact and because $U_{z}$ is open, there is
  $n$ such that $L_n \subseteq U_{z}$, that is, $v \in L_n$ implies
  $v(z) = u(z)$. Therefore, for every~$z \in Z$ there is~$n$ such that
  $v(z) = w(z)$ for all $v,w \in L_n$.
%
  Now, the function $n(z)=\mu n.\forall v,w \in L_n.  v(z)=w(z)$ is
  computable by the exhaustibility of $L_n$.  But $u \in L_{n(z)}$ for
  any~$z \in Z$ and therefore $u$ is computable by exhaustibility as
  $u(z)=\mu m.  \forall v \in L_{n(z)}.  v(z)=m$.
\end{proof}

To build sets $K_n$ suitable for applying this lemma, we use:
\begin{lemma} \label{equalupto} 
  For every $kk$-space $X$ there is a family $(=_n)$ of equivalence
  relations that are decidable uniformly in $n$ and satisfy
\begin{eqnarray*}
x=x' & \iff & \forall n.\,x =_n x', \\
x =_{n+1} x' & \implies & x =_n x'.
\end{eqnarray*}
\end{lemma}
\begin{proof}
  Let $X$ be a Kleene--Kreisel space and $s \colon X \to \N^Z$ and $r
  \colon \N^Z \to X$ be computable maps with $r \comp s = \id_X$. By
  the density theorem, there is a computable dense sequence $\delta_n
  \in Z$.  Then the definition
  \[
  x =_n x' \iff \forall i < n. s(x)(\delta_i) = s(x')(\delta_i)
  \]
  clearly produces an equivalence relation that is decidable uniformly
  in~$n$ and satisfies $x =_{n+1} x' \implies x =_n x'$. Moreover, $x
  = x'$ iff $s(x)=s(x')$, because $s$ is injective, iff
  $s(x)(\delta_n)=s(x')(\delta_n)$ for every $n$, by density, iff $x
  =_n x'$ for every $n$, by definition.
\end{proof}

%\medskip
\begin{proof}[Proof of Theorem~\ref{unique:solution}]
  The set $K_n = \{ x \in X \mid f(x) =_n y\}$, being a decidable
  subset of an exhaustible space, is exhaustible.  Therefore the
  result follows from Lemmas~\ref{empty} and~\ref{unique}, because $x
  \in \bigcap_n K_n$ iff $f(x) =_n y$ for every~$n$ iff $f(x) = y$ by
  Lemma~\ref{equalupto}.
\end{proof}

\begin{algorithms} \label{solve}
  In summary, the algorithm for semi-deciding non-existence of
  solutions is
\[ 
\exists n.\forall x \in X.f(x) \ne_n y,
\]
and that for computing the solution $x_0$ as a function of
$\forall_X$, $f$, and $y$ is:
\begin{eqnarray*}
\forall x \in K_n.p(x) & = & \forall x \in X. f(x) =_n y \implies p(x), \\
\forall v \in L_n.q(v) & = & \forall x \in K_n. q(s(x)), \\
n(z) & = & \mu n.\forall v,w \in L_n.  v(z)=w(z), \\
u(z)& = & \mu m.  \forall v \in L_{n(z)}.  v(z)=m, \\
  x_0 & = & r(u).
\end{eqnarray*}
Here the parameters $r$ and $s$ come from the assumption that $X$ is a
$kk$-space. \qed
\end{algorithms}

\begin{rem} \label{approximate} Even in the absence of uniqueness,
  \emph{approximate} solutions with precision~$n$ are trivially
  computable with the algorithm
\[ 
    \varepsilon_X(\lambda x.f(x)=_n y), 
\] 
using the fact that non-empty exhaustible subsets of $kk$-spaces are
searchable. But the above unique-solution algorithm uses the
quantification functional~$\forall_X$ rather than the selection
functional~$\varepsilon_X$. In the next section, we compute solutions
as limits of approximate solutions (cf.\
Remark~\ref{approximate:bis}). \qed
\end{rem}

% \begin{algorithm}
%   Regarding the isomorphism $g \colon K \to Y$
%   with $Y \subseteq \N^\N$ given in the proof of
%   Theorem~\ref{alternative}, we take $r$ and $s$ coming from the
%   assumption that $K$ is a $kk$-space, and the $r$ and $s$ of
%   Algorithms~\ref{solve} can be taken to be the identity, because the
%   $kk$-space under consideration is already of the form $\N^Z$, and
%   hence luckily don't appear, avoiding a potential name clash. The
%   algorithms given by unfolding the proofs are:
%  \begin{eqnarray*}
%    g(k) & = & \lambda i.s(k)(\delta_i), \\
%    g^{-1}(\alpha) & = & r(u_0),
%  \end{eqnarray*}
%  where 
%  \begin{eqnarray*}
%  \forall u \in K_n.p(u) & = & \forall k \in K. (\forall i<n.s(k)(\delta_i) = \alpha_i) \implies p(s(k)), \\
%  n(z) & = & \mu n.\forall v,w \in K_n.  v(z)=w(z), \\
%  u_0(z) & = & \mu m.  \forall v \in K_{n(z)}.  v(z)=m. \qquad\qquad\qquad\qquad\qquad\qed
%  \end{eqnarray*}
% \end{algorithm}

\section{Equations over metric spaces} \label{metric:spaces}

We first formulate the main result of this section and then define
the missing concepts. (Recall the notion of $kk$-exhaustibility
defined in Section~\ref{kk:exhaustible:section}.)
\begin{theorem} \label{unique:solution:metric} Let $X$ and $Y$ be
  computational metric spaces with $X$ computationally complete and
  $kk$-exhaustible.
\begin{quote}
  If $f\colon X \to Y$ and $y \in Y$ are computable, then, uniformly
  in $f$, $y$ and the exhaustibility condition:
\begin{enumerate}
\item \label{unique:solution:metric2} It is semi-decidable whether the
  equation $f(x) = y$ fails to have a solution $x \in X$.
\item \label{unique:solution:metric:1} If $f(x) = y$ has a unique solution
  $x \in X$, then it is computable.
\end{enumerate}
\end{quote}
Hence any computable bijection $f\colon X \to
  Y$ has a computable inverse, uniformly in $f$ and the exhaustibility
  condition.
\end{theorem}

\medskip There is a technical difficulty in the proof of the theorem:
at the intensional level, where computations take place, solutions are
unique only up to equivalence of representatives.  In order to
overcome this, we work with pseudo-metric spaces at the intensional
level and with a notion of decidable closeness for them.  Recall that
a \emph{pseudo-metric} on a set $X$ is a function $d \colon X \times X
\to [0,\infty)$ such that
\begin{enumerate}
\item $d(x,x)=0$,
\item $d(x,y) = d(y,x)$,
\item $d(x,z) \le d(x,y) + d(y,z)$.
\end{enumerate}
Then $d$ is a \emph{metric} if it additionally satisfies $d(x,y)=0 \implies x=y$.
If $d$ is only a pseudo-metric, then $(\sim)$ defined by
\[
x \sim y \iff d(x,y)=0
\]
is an equivalence relation, referred to as \emph{pseudo-metric
  equivalence}.  A pseudo-metric topology is Hausdorff iff it is $T_0$
iff the pseudo-metric is a metric. Moreover, two points are
equivalent iff they have the same neighbourhoods.  Hence any sequence
has at most one limit up to equivalence.
\begin{definition} We regard $[0,\infty)$ as a computational space as
  in Example~\ref{real}.
\begin{enumerate}
\item  A \emph{computational pseudo-metric space} is a computational space $X$
   endowed with a computable pseudo-metric, denoted by $d_X \colon X
   \times X \to [0,\infty)$ or simply~$d$. 

   We emphasize that we don't require the computational topology
   of $X$ to agree with the pseudo-metric topology generated by open
   balls (cf.\ Remark~\ref{firstly}).

 \item A \emph{computational metric space} is a computational
   pseudo-metric space in which the pseudo-metric is actually a
   metric, and hence we formulate the following definitions in
   the generality of pseudo-metric spaces.

 \item A \emph{fast Cauchy sequence} in a computational pseudo-metric
   space~$X$ is a sequence $x_n \in X$ with $d(x_n,x_{n+1}) < 2^{-n}$.
   The subspace of $X^\N$ consisting of fast Cauchy sequences is
   denoted by $\cauchy(X)$.

 \item A computational pseudo-metric space $X$ is called
   \emph{computationally complete} if every sequence $x_n \in
   \cauchy(X)$ has a limit uniformly in~$x_n$, i.e.\ there is a
   computable map $\lim \colon \cauchy(X) \to X$ such that $\lim_n
   (x_n)$ is a limit of $x_n$ for every sequence $x_n \in \cauchy(X)$.
\item A computational pseudo-metric space $X$ has \emph{decidable
    closeness} if there is a family of relations
  $\sim_n$ on~$X$ that are decidable uniformly in $n$ and satisfy:
\begin{enumerate}
\item $x \sim_n y \implies d(x,y) < 2^{-n}$, 
\item $x \sim y \implies \forall_n. x \sim_n y$.
\item $x \sim_{n+1} y \implies x \sim_n y$,
\item $x \sim_n y \iff y \sim_n x$,
\item $x \sim_{n+1} y \sim_{n+1} z \implies x \sim_n z$.
\end{enumerate}
The last condition is a counter-part of the triangle inequality.  It
follows from the first condition that if $x \sim_n y$ for every~$n$,
then $x \sim y$.  Write
\[
[x] = \{ y \in X \mid x \sim y\}, \qquad\qquad
[x]_n = \{ y \in X \mid x \sim_n y\}.
\]
Then the equivalence class $[x]$ is the closed ball
of radius zero centered at~$x$.
\end{enumerate}
We omit the qualification ``computational'' when it is clear
that the context requires it. \qed
\end{definition}
For instance, the computational spaces $\R$ and $[0,\infty)$ defined
in Example~\ref{real}, are complete metric spaces in this
computational sense under the Euclidean metric, as is well known, but
don't have decidable closeness.  We now proceed to prove the theorem.
\pagebreak[3]
\begin{lemma} \label{closeness:cover} For every computational metric
  space~$X$ there is a canonical computable pseudo-metric~$d =
  d_{\godel{X}}$ on the representing space~$\godel{X}$ such that:
  \begin{enumerate}
  \item The representation map $\rho = \rho_X \colon \godel{X} \to X$ is an isometry:
    \[ d(t,u) = d(\rho(t),\rho(u)). \]

  In particular:
    \begin{enumerate}
    \item $t \sim u \iff d(t,u)=0 \iff \rho(t)=\rho(u)$.
    \item If $f \colon X \to Y$ is a computable map of metric spaces,
      then any representative $\godel{f} \colon \godel{X} \to
      \godel{Y}$ preserves the relation~($\sim$).
  \end{enumerate}

  \item If $X$ is computationally complete then so is $\godel{X}$.

  \item The representing space $\godel{X}$ has decidable closeness.
  \end{enumerate}
\end{lemma}
\begin{proof}
  Construct $d_{\godel{X}} \colon \godel{X} \times \godel{X} \to
  {[0,\infty)}$ as the composition of a computable representative
  $\godel{d_X} \colon \godel{X} \times \godel{X} \to
  \godel{{[0,\infty)}}$ of $d_X \colon X \times X \to {[0,\infty)}$
  with the representation map $\rho_{{[0,\infty)}} \colon
  \godel{{[0,\infty)}} \to {[0,\infty)}$.  A limit operator for
  $\godel{X}$ from a limit operator for $X$ is constructed in a
  similar manner.  For given $t,u \in \godel{X}$, let $q_n$ be the
  $n$-th term of the sequence $\godel{d_X}(t,u) \in
  \godel{[0,\infty)}\subseteq \cauchy(\Q)$, and define $t \sim_n u$ to
  mean that the interval $[-2^{-n},2^{-n}]$ is contained in
  $[q_n-2^{-n+1},q_n+2^{-n+1}]$.
\end{proof}

\pagebreak[3] We now work at the intensional level and later transfer
the results to the extensional level with the aid of the above lemma.
\begin{lemma} \label{unique:bis:bis} Let $Z$ be a complete
  computational pseudo-metric $kk$-space with decidable closeness, and
  $K_n \subseteq Z$ be a sequence of sets that are exhaustible
  uniformly in $n$ and satisfy $K_n \supseteq K_{n+1}$.
  \begin{quote} If $\bigcap_n K_n$ is an equivalence class, then it
    has a computable member,
  \end{quote}
uniformly in the given computational data.
\end{lemma}
\noindent
\begin{proof}
  Let $z \in \bigcap_n K_n$. For any $m$, we
  have $\bigcap_n K_n = [z] \subseteq [z]_{m+1}$, and hence there is~$n$
  such that $K_n \subseteq [z]_{m+1}$, because the sets $K_n$ are compact,
  because $K_n \supseteq K_{n+1}$, because $Z$ is Hausdorff and
  because $[z]_{m+1}$ is open.  Hence for every $u \in K_n$ we have $u
  \sim_{m+1} z$, and so for all $u,v \in K_n$ we have $u \sim_m v$. By
  exhaustibility of $K_n$ and decidability of~$(\sim_n$), the function
  %\[
  $n(m) = \mu n. \forall u,v \in K_n.\, u \sim_m v$
  %\]
  is computable. By searchability of~$K_n$, there is a computable
  sequence $u_m \in K_{n(m)}$. Because $n(m) \le n(m+1)$, we have that
  $K_{n(m)} \supseteq K_{n(m+1)}$ and hence $u_m \sim_m u_{m+1}$ and
  so $d(u_m,u_{m+1}) < 2^{-m}$ and $u_m$ is a Cauchy sequence. By
  completeness, $u_m$ converges to a computable point $u_\infty$.
  Because $z \in K_{n(m)}$, we have $u_m \sim_m z$ for every $m$, and
  hence $d(u_m,z) < 2^{-m}$. And because $d(u_\infty,u_m) < 2^{-m+1}$,
  the triangle inequality gives $d(u_\infty,z) < 2^{-m} + 2^{-m+1}$
  for every~$m$ and hence $d(u_\infty,z)=0$ and therefore $u_\infty
  \in \bigcap_n K_n$.
\end{proof}
The proof of the following is essentially the same as that of
Theorem~\ref{unique:solution}, but uses Lemma~\ref{unique:bis:bis}
rather than Lemma~\ref{unique}, and doesn't rely on an analogue of
Lemma~\ref{equalupto}, which is built into the definition of decidable
closeness.
\pagebreak[3]
\begin{lemma} \label{unique:solution:bis} Let $Z$ and $W$ be
  pseudo-metric spaces with decidable closeness, and assume that $Z$
  is complete, exhaustible and a $kk$-space.  
\begin{quote}
If $g\colon Z \to W$ is
  a computable map that preserves pseudo-metric equivalence and $w \in
  W$ is computable, then, uniformly in $\forall_Z$, $g$, and~$w$:
\begin{enumerate}
\item \label{unique:solution:bis:2} It is semi-decidable whether the
  equivalence $g(z) \sim w$ fails to have a solution $z \in Z$.
\item \label{unique:solution:bis:1} If $g(z) \sim w$ has a unique solution
  $z \in Z$ up to equivalence, then some solution is computable. 
\end{enumerate}
\end{quote}
\end{lemma}
\begin{proof}
  The set $K_n = \{ z \in Z \mid g(z) \sim_n w\}$, being a decidable
  subset of an exhaustible space, is exhaustible.  Therefore the
  result follows from Lemmas~\ref{empty} and \ref{unique:bis:bis}, because $z
  \in \bigcap_n K_n$ iff $g(z) \sim_n w$ for every~$n$ iff $g(z) = w$.
\end{proof}
\begin{algorithm}
The algorithm for computing the solution $z = u_\infty$ from
$\forall_Z$, $g$ and $w$ is then the following, where we have expanded
$\forall_{K_n}$ as a quantification over~$Z$:
\begin{eqnarray*}
n(m) & = & \mu n. \forall u,v \in Z.\, g(u) \sim_n w \wedge g(v) \sim_n w \implies u \sim_m v,\\
u_\infty & = & \lim_m \varepsilon_K(\lambda z.g(z) \sim_{n(m)} w).
\end{eqnarray*}
Thus, although there are common ingredients with
Theorem~\ref{unique:solution}, the resulting algorithm is different,
because it relies on the limit operator (but see
Proposition~\ref{strictly:coarser} below). \qed
\end{algorithm}

\begin{rem} \label{approximate:bis}
Again, approximate solutions are computable as in
Remark~\ref{approximate}, and, in fact, we are computing the unique
solution as the limit of approximate solutions.  But, for
Theorem~\ref{unique:solution:metric}, approximate solutions are
computable uniformly in $\godel{f}$ and $\godel{y}$ only, as different
approximate solutions are obtained for different representatives of
$f$ and $y$. \qed
\end{rem}

\begin{proof}[Proof of Theorem~\ref{unique:solution:metric}.]
  Let $X$ and $Y$ be computational metric spaces with $X$
  $kk$-exhaustible, and let $f\colon X \to Y$ and $y \in Y$ be
  computable.  Now apply Lemma~\ref{unique:solution:bis} with
  $Z=\godel{X}$, $W=\godel{Y}$, $g=\godel{f}$, $w=\godel{y}$, using
  Lemma~\ref{closeness:cover} to fulfill the necessary hypotheses.  If
  $f(x)=y$ has a unique solution~$x$, then $g(z) \sim w$ has a unique
  solution~$z$ up to equivalence, and $x=\rho(z)$ for any
  solution~$z$, and hence $x$ is computable. Because $g$ preserves
  $(\sim)$ by Lemma~\ref{closeness:cover}, if $g(z) \sim w$ has a
  solution~$z$, then $x=\rho(z)$ is a solution of $f(x)=y$.  This
  shows that $f(x)=y$ has a solution iff $g(z)=w$ has a solution, and
  we can reduce the semi-decision of absence of solutions of $f(x)=y$
  to absence of solutions of $g(z)=w$.
\end{proof}

Before giving examples in computational real analysis, in
Section~\ref{applications}, we clarify some aspects of the above
development.
\begin{rem} \label{firstly} The metric topology of a computational
  space is always coarser than the computational topology, because, by
  continuity of the metric, open balls are open in the computational
  topology. Hence the computational topology of any computational
  metric space is Hausdorff. If $X$ is $kk$-exhaustible and the metric
  topology is compact, then both topologies agree, because no compact
  Hausdorff topology can be properly refined to another compact
  Hausdorff topology.  For e.g.\ the real line, the metric topology
  agrees with the computational topology, as is well-known, but in the
  example given in Proposition~\ref{strictly:coarser} below, it is
  strictly coarser.  It is also strictly coarser for any pseudo-metric
  $kk$-space, because $kk$-spaces are Hausdorff and pseudo-metric
  topologies are Hausdorff only for pseudo-metrics which are genuine
  metrics.  \qed
\end{rem}

Recall that an ultra-metric space is a metric space for which the
triangle inequality holds in the stronger form $d(x,z) \le
\max(d(x,y),d(y,z))$, and that an ultra-metric topology is
zero-dimensional because open balls are closed. Recall the equivalence
relations $(=_n)$ given in Lemma~\ref{equalupto}.
\begin{prop} \label{strictly:coarser}
  Any $kk$-space $X$ equipped with $d$ defined by
  \[
  d(x,y) = \inf \{2^{-n} \mid x =_n y\}
  \]
  is a computational ultra-metric space. Moreover:
\begin{enumerate}
\item The metric has decidable closeness given by $(\sim_n)=(=_n)$.
\item The metric topology is in general strictly coarser
than the computational topology, but both agree on compact subsets.
\item Exhaustible subspaces with the relative metric are computationally complete.
\end{enumerate}
\end{prop}
\begin{proof}
  Computability of the ultra-metric and decidability of closeness are
  easy.

  Let $s \colon X \to \N^Z$ and $r \colon \N^Z \to X$ be the same
  computable functions selected in the proof of Lemma~\ref{equalupto},
  let $\delta$ be the same dense sequence, and for $u,v \in \N^Z$
  define $u =_n v \iff \forall i<n.\,u(\delta_i)=v(\delta_i)$ so that
  $x =_n y$ iff $s(x) =_n s(y)$.

  \medskip
  \noindent
  \emph{Agreement of topologies:} A subbasic open set in the topology
  of pointwise convergence of $\N^Z$ is of the form $ N(z,V)=\{u \in
  \N^Z \mid u(z) \in V\}$ with $z \in Z$ and $V \subseteq \N$. Now $u
  =_n v$ iff $d(u,v) < 2^{-n}$, and hence the open ball
  $B_{2^{-n}}(u)$ is the intersection of the pointwise open sets
  $N(\delta_i,\{u(\delta_i)\})$, for $i<n$, and hence open balls are
  open in the topology of pointwise convergence.  For a compact
  subspace of $\N^Z$, density of $\delta$ gives that metric topology
  agrees with the pointwise topology. But the relative topology on
  compact subsets of~$\N^Z$ coincides with the topology of pointwise
  convergence, by classical, and easy, Arzela--Ascoli type arguments,
  and hence the three topologies agree. The reduction of this to $X$
  via the retraction is easy.

\pagebreak[3]
  \medskip
  \noindent
  \emph{Completeness:}
%
  Let $z_n \in Z$ be a fast Cauchy sequence. Then $z_n =_n z_{n+1}$ and
  hence $s(z_n) =_n s(z_{n+1})$. It suffices to show that the sequence
  $f_n=s(z_n)$ converges to a computable limit~$f_\infty$, because
  then the sequence $z_n = r(f_n)$ converges to the computable point
  $z_\infty = r(f_\infty)$ by continuity of~$r$. The set $L=s(K)$ is
  exhaustible because it is a computable image of an exhaustible set.
  For any $n$, the set $L_n = \{ g \in L \mid g =_n f_n \}$ is
  exhaustible because it is a decidable subset of an exhaustible set,
  and $f_n \in L_n$.  By compactness, $\bigcap_n L_n \ne \emptyset$
  because clearly $L_n \supseteq L_{n+1}$. If $g,h \in \bigcap_n L_n$,
  then $g =_n f_n =_n h$ for every $n$, and hence $g=h$, and so
  $\bigcap_n L_n = \{f_\infty\}$ for some computable $f_\infty$ by
  Lemma~\ref{unique}.  Because $f_\infty \in \bigcap_n L_n$, we have
  $f_\infty =_n f_n$ for every$~n$.  Hence if some ball $B_{2^{-k}}(h)$
  is a neighbourhood of $f_\infty$, then $h =_k
  f_\infty =_k f_n$ for all $n \ge k$, and hence $f_n \in B_k(h)$ for
  all $n \ge k$, which shows that $f_n \to f_\infty$.
\end{proof}
In view of this proposition, Lemma~\ref{unique:bis:bis} generalizes
Lemma~\ref{unique}. But Lemma~\ref{unique} cannot be eliminated,
because it is used to prove the proposition.  
\begin{algorithm}
  Expanding Lemma~\ref{unique}, the algorithm for computing
  $\lim_n f_n$ for a fast Cauchy sequence $f_n \in L \subseteq \N^Z$
  with $L$ exhaustible is:
\begin{eqnarray*}
  n(z) & = & \mu n.\forall g,h \in L.  g =_n f_n \wedge h =_n f_n \implies g(z) = h(z), \\
\lim_n f_n& = & \lambda z.\mu m.  \forall g \in L.  g =_{n(z)} f_{n(z)} \implies g(z)=m. 
\end{eqnarray*}
Notice that ``fast'' amounts to $f_n =_n f_{n+1}$. \qed
\end{algorithm}
\pagebreak[3]
\begin{rem}
  Independently of this, Matthias Schr\"oder (personal communication)
  showed that if a QCB space $X$ is the sequential coreflection of a
  zero-dimensional topology, then there is a metric $d$ on $X$ such
  that: (1) The topology induced by $d$ is coarser than that of $X$
  and than the zero-dimensional topology. (2) On compact subsets $X$,
  the three topologies agree.  (3) The image of $d$ is $\{ 0 \} \cup
  \{ 2^{-n} | n \in \N \}$.  This applies to all $kk$-spaces in
  particular, as their topologies satisfy the hypothesis. His
  construction uses countable pseudo-bases rather than dense
  sequences.  However, he hasn't considered the computational versions
  of these statements. \qed
\end{rem}


\section{Exhaustible spaces of analytic functions} \label{applications}

For any $\epsilon \in (0,1)$, any $x \in [-\epsilon,\epsilon]$, any $b>0$,
and any sequence $a \in [-b,b]^\N$, the Taylor series $\sum_n
a_n x^n$ converges to a number in the interval
$[-b/(1+\epsilon),b/(1-\epsilon)]$. 

\begin{lemma} \label{uniform:analytic}
  Any analytic function $f \in \R^{[-\epsilon,\epsilon]}$ of the form
  \[
  f(x) = \sum_n a_n x^n
  \]
  is computable uniformly in any given computable $\epsilon \in (0,1)$, $b >
  0$ and $a \in [-b,b]^\N$.
\end{lemma}
\begin{proof}
  Standard computational analysis argument. 
\end{proof}
\begin{definition}
  Denote by \[ A = A(\epsilon,b)\] the subspace of such analytic
  functions and by \[ T = T_{\epsilon,b} \colon [-b,b]^\N \to
  A(\epsilon,b)\] the functional that implements the uniformity
  condition, so that $f = T(a)$. \qed
\end{definition}
The following results also hold uniformly in $\epsilon$ and $b$,
but we omit explicit indications. Also, the results are uniform in the
exhaustibility assumptions. 

\begin{theorem}
  The space $A$ is $kk$-exhaustible.
\end{theorem}
\begin{proof}
  The space $[-b,b]^\N$ is $kk$-exhaustible by
  Examples~\ref{kk:examples}.  By Lemma~\ref{kk:exhaustible:image},
  any computable representative $\godel{T} \colon \godel{[-b,b]^\N}
  \to \godel{A}$ is a surjection, and hence
  $\godel{A}$ is exhaustible. It follows from
  Lemma~\ref{kk:subspace} that $\godel{A}$ is a
  $kk$-space, because the representing space $\godel{A}$
  is clearly a subspace of a Kleene--Kreisel space.
\end{proof}

Hence the solution of a functional equation with a unique analytic
unknown in~$A$ can be computed using
Theorem~\ref{unique:solution:metric}.

\begin{lemma}
  For any non-empty $kk$-exhaustible space $X$, the maximum- and
  minimum-value functionals
  \[
  {\max}_X, {\min}_X \colon \R^X \to \R
  \]
  are computable.
\end{lemma}
\noindent
NB.\ Of course, any $f \in \R^X$ attains its maximum value because it
is continuous and because $kk$-exhaustible spaces are compact.
\begin{proof}
  We discuss $\max$ only. By e.g.\ the algorithm given by
  Simpson~\cite{simpson:integration}, this is the case for $X=2^\N$.
  Because the representing space $\godel{X}$, being a non-empty
  exhaustible $kk$-space, is a computable image of the Cantor space,
  the space $X$ itself is a computable image of the cantor space, say
  with $q \colon 2^\N \to X$. Then the algorithm $\max_X(f)=\max_{2^N}(f
  \comp q)$ gives the required conclusion.
\end{proof}

\begin{cor}
  Any $kk$-exhaustible subspace $K$ of a metric space $X$ is
  computably located in the sense that the distance function $d_K
  \colon X \to \R$ defined by
  \[ d_K(x) = \min \{ d(x,y) \mid y \in K \} \]
  is computable.
\end{cor}
\begin{proof}
  $d_K(x)=\min_K(\lambda y.d(x,y))$.
\end{proof}

\begin{cor}
  For any non-empty $kk$-exhaustible metric space $X$, the max-metric
  on $\R^X$
  \[
  d(f,g) = \max \{ d(f(x),g(x)) \mid x \in X \}
  \]
  is computable.
\end{cor}
\noindent
\begin{proof}
  $d(f,g)=\max_X(\lambda x.d(f(x),g(x)))$.
\end{proof}
\begin{cor}
  For $f \in \R^X$, it is semi-decidable whether $f \not\in A$. 
\end{cor}
\begin{proof}
  Because $A$ is computationally located in
  $\R^{[-\epsilon,\epsilon]}$ as it is
  $kk$-exhaustible, and because $f \not\in A \iff d_A(f) \ne 0$.
\end{proof}

Another proof, which doesn't rely on the $kk$-exhaustibility of $A$,
uses Theorem~\ref{unique:solution:metric}: $f \not\in A$ iff the
equation $T(a)=f$ doesn't have a solution $a \in [-b,b]^\N$.  But this
alternative proof relies on a complete metric on $[-b,b]^\N$. For
simplicity, we consider a standard construction for $1$-bounded metric
spaces. This is no loss of generality for our purposes, because for
any metric $d$, the metric $d'(x,y)=\min(1,d(x,y))$ has the same
Cauchy sequences.  (Moreover, because we shall confine our attention
to $kk$-exhaustible metric spaces, this is no loss of generality with
the alternative reason that the diameter of such a space is computable
as $\max(\lambda x.\max(\lambda y.d(x,y))$.)

\begin{lemma} \label{product:metric} For any computational $1$-bounded
  metric space $X$, the metric on~$X^\N$ defined by
  \[
  d(x,y) = \sum_n 2^{-n-1} d(x_n,y_n) 
  \]
  is computable and $1$-bounded, and it is computationally complete if
  $X$ is.
\end{lemma}
\begin{proof}
  Use the fact that the map $[0,1]^\N \to [0,1]$ that sends a sequence
  $a \in [0,1]^\N$ to the number $\sum_n 2^{-n-1} a_n$ is computable.
  Regarding completeness, it is well known that a sequence in the
  space~$X^\N$ is Cauchy iff it is componentwise Cauchy in $X$, and in
  this case its limit is calculated componentwise. (To compute the
  limit componentwise, maybe we need suitable shifting to make sure
  that the components converge at the required speed --- I have to
  check this.)
\end{proof}

\begin{cor}
  The Taylor coefficients of any $f \in A$ can be
  computed from $f$.
\end{cor}
\begin{proof}
  The space $[-b,b]^\N$ is $kk$-exhaustible by
  Examples~\ref{kk:examples}, and hence the function $T$ is invertible
  by Theorem~\ref{unique:solution:metric} and
  Lemma~\ref{product:metric}.
\end{proof}

\medskip
Now, it remains to write down: analytic functions of the form $\sum_n a_n
x^n/n!$, with the same restrictions on $a$ and $\epsilon$, are also
$kk$-exhaustible, and are closed under differentiation, and they are
computably differentiable. Then it remains to think whether this can
be used to solve differential equations using e.g.\ the classical
Peano theorem.  In certain cases, it is known that analytic solutions
are unique, but this uses the complex plane. It should be easy to
generalize the above results to complex analytic functions. Can the
results discussed above be developed in the internal logic of the
category and then be extracted via realizability? Work on this paper will
have to be suspended for at least a month, unfortunately.

\pagebreak[3]
{%\footnotesize 
 \bibliographystyle{plain} 
\bibliography{references}
}

%\vfill

%{\footnotesize \tableofcontents} 

\end{document}
