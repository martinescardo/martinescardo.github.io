\documentclass[12pt]{article}

\usepackage{amssymb,amsmath,stmaryrd}

\usepackage[english]{babel}

\usepackage{a4wide}

\newcommand{\path}[2]{\operatorname{path}_{#1,#2}}
\newcommand{\test}[2]{\operatorname{test}_{#1,#2}}
\newcommand{\cons}[1]{\operatorname{cons}_{#1}}
\newcommand{\sizeof}[1]{|#1|}
\newcommand{\wpor}{\operatorname{wpor}}
\newtheorem{thm}{Theorem}[section]
\newtheorem{prop}{Proposition}[section]
\newtheorem{lem}{Lemma}[section]
\newtheorem{cor}{Corollary}[section]
\newtheorem{definition}{Definition}[section]
\newtheorem{fact}{Fact}[section]
\newtheorem{rem}{Remark}[section]
\newtheorem{conj}{Conjecture}[section]
\def\QED {{\unskip\nobreak\hfill\penalty50
  \hskip2em\hbox{}\nobreak\hfill$\square$
  \parfillskip=0pt \finalhyphendemerits=0 \par}}
% Proof environment
\newenvironment{proof}{\paragraph{Proof.}}{\QED\paragraph{}\noindent}

\newenvironment{defn}{\begin{definition}\em}{\end{definition}}

               
% self-defined commands
\newcommand{\meet}{\sqcap}
\newcommand{\arr}{\to}%{\rightarrow}
\newcommand{\obs}{\sqsubseteq}
\newcommand{\Pot}{\mathcal{P}}
\newcommand{\I}{\mathcal{I}}
\newcommand{\R}{\mathcal{R}}
\newcommand{\LL}{L}%{\mathcal{L}}
\newcommand{\RR}{R}%{\mathcal{R}}
\newcommand{\N}{{\Bbb N}}
\newcommand{\A}{{\Bbb A}}
\newcommand{\B}{{\Bbb B}}
\newcommand{\C}{{\Bbb C}}
\newcommand{\E}{{\Bbb E}}
\newcommand{\T}{{\Bbb T}}
\newcommand{\X}{{\Bbb X}}
\newcommand{\eps}{\varepsilon}
\newcommand{\aeps}{a_{\varepsilon}}
\def\Two{{2 \kern-0.4em 1}}
\def\TWO{{2 \kern-0.5em 1}}
\def\lbr{\mathopen{{[\kern-0.14em[}}}   % opening [[ value delimiter
\def\rbr{\mathclose{{]\kern-0.14em]}}}  % closing ]] value delimiter
\newcommand{\sem}[1]{\lbr#1\rbr}
\newcommand{\inc}{\hookrightarrow}

\newcommand{\True}{\operatorname{\mathrm{True}}}
\newcommand{\False}{\operatorname{\mathrm{False}}}
\newcommand{\por}{\operatorname{\mathrm{por}}}
\newcommand{\red}{\to}
\newcommand{\Meaning}[1]{\llbracket #1 \rrbracket}

\begin{document}

\title{On the non-sequential nature of \\ the 
  interval-domain model of real-number computation}

  \author{Mart\'\i n Escard\'o, Martin Hofmann, Thomas Streicher}

\date{April 1999, version of \today}

\maketitle

\begin{abstract}
  We show that real-number computations in the interval-domain
  environment are ``inherently parallel'' in a precise mathematical
  sense. We do this by reducing computations of the weak parallel-or
  operation on the Sierpinski domain to computations of the addition
  operation on the interval domain.
\end{abstract}

\section{Introduction}

\newcommand{\bool}{\textit{bool}}
\newcommand{\real}{\textit{real}}

Suppose we extend a functional programming language with an abstract
data type $\real$ meant to represent infinite-precision real
numbers~\cite{boehm:cartwright}.  What is a sensible semantics for
such a type or how does one explain to users what their programs
involving $\real$ actually mean?  It is probably not sensible to
interpret $\real$ as just the set of real numbers as this would not
assign meaning to nonterminating computations (``$\bot$'') nor to
partial computations, say one that outputs the information that a
number lies between $0$ and $0.5$ but does not terminate upon requests
for higher accuracy.

Scott proposed the \emph{interval domain} as a possible answer to this
question~\cite{scott:intervaldomain}. For simplicity and without
essential loss of generality, we consider here only its restriction to
the unit interval $[0,1]$. The elements of the interval domain
$\mathcal{I}$ are the closed intervals $x=[\underline
x,\overline x]$ where $0\leq \underline x\leq \overline x\leq 1$.
These are ordered by reverse inclusion, i.e., $x\sqsubseteq y$ if and
only if $\underline x\leq \underline y$ and $\overline y\leq\overline
x$.  Then $(\mathcal{I},\sqsubseteq)$ becomes a directed complete
partial order and thus supports general recursion via fixed points.
The idea of the relation $x \sqsubseteq y$ is that $y$ gives at least
as much information as $x$ about an unknown real number in the unit
interval, and possibly more.  If $y$ is a singleton, that is, a
maximal element in the ordering, then it gives complete or
\emph{total} information.  The ordered set~$\I$ is a typical example
of a \emph{continuous Scott domain}.  For an exposition of the general
theory, the reader is referred to Abramsky and
Jung~\cite{abramsky:jung}, Amadio and Curien~\cite{amadio:curien} or
to the forthcoming treatise by Gierz et al.~\cite{newcompendium}.  For
a topological view, see the survey paper by
Mislove~\cite{mislove:survey}, and, for recent applications, the
reader is referred to the survey paper by Edalat~\cite{Eda97}.

The first-named author has developed a lazy functional programming
language called Real PCF based on the interval domain. One of the
central results concerning Real PCF is that its interpretation in
terms of the interval domain is \emph{computationally adequate} with
respect to a lazy operational semantics~\cite{escardo:pcf}.  This
operational semantics associates with expressions of type $\real$ a
shrinking sequence of rational intervals.  Computational adequacy
asserts that the intersection of these intervals agrees with the
denotational semantics in $\mathcal{I}$ of such an expression.

In contrast to most ordinary functional languages, Real PCF contains a
construct whose operational semantics requires parallel evaluation (or
``dovetailing'' in the terminology of recursion theory~\cite{rogers}).
More precisely, one has a construct
$\textsf{pif}:\bool\times\real\times\real\rightarrow\real$ defined by
$\textsf{pif}(\textsf{true},x,y)=x$ and
$\textsf{pif}(\textsf{false},x,y)=y$ and
$\textsf{pif}(\bot,x,y)=x\sqcap y$ where $x\sqcap y$ is the least
interval containing both $x$ and~$y$. In particular,
$\textsf{pif}(\bot,x,x)=x$.  It is clear that when evaluating a term
$\textsf{pif}(b,x,y)$ we must evaluate $b,x,y$ in parallel and output
partial results while the computation of $b$ has not yet finished.
This construct is used in order to ensure that Real PCF is
Turing-universal, in the sense that all real functions and functionals
that are computable in the sense of recursion theory are programmable
in the language~\cite{MR2000b:03216}.

It is natural to ask whether the presence of this parallel construct
is an artifact of Real PCF or whether it is required for intrinsic
reasons. The result of this paper confirms that the latter is the
case.  More precisely, we show that, under very mild and reasonable
assumptions, any language that is computationally adequate with
respect to the interval domain must necessarily allow the definition
of the ``weak parallel-or'' construct $\textsf{wpor}$ defined on the
Sierpinski domain $\{ \bot, \top \}$ (interpreting the unit
type or arising as a retract of any other type), given by
$\textsf{wpor}(x,y)=\top$ if and only if $x=\top$ or $y=\top$.
Clearly, to evaluate $\textsf{wpor}(x,y)$ one must evaluate $x$ and
$y$ in parallel in order to be able to terminate as soon as either $x$
or $y$ has terminated.

The presence of such a parallel construct is a disadvantage from a
practical point of view because it renders the evaluation process
inefficient due to the exponential spawning of processes in recursive
definitions involving them.  However, the ultimate practical
consequences of our result, which, we emphasize, is specific to the
interval-domain model, are not clear cut. On the one hand, it may be
that there exists an alternative mathematical model, yet to be found,
which doesn't exhibit this parallel behaviour.  In this case, the
interpretation of our result would be that the interval-domain model
is at fault, and hence one should look for another, sequential model.
%(Notice that in the usual categories of stable or strongly stable
%domains the retracts are algebraic again and, therefore, don't allow
%for the interval domain.)  
On the other hand, it may be that this is
not a fault of the interval-domain model, but rather an intrinsic
property of the real numbers --- this is supported, for instance, by
the work of Luckhardt~\cite{luckhardt}.  In this case, one should look
for a proof that any reasonable model exhibits the same behaviour.  We
have at this point little theoretical or empirical evidence for any of
these two conclusions and therefore leave the interpretation to the
reader and to future work.

We now proceed to a more technical introduction to our result, in
which we give a simple proof of an important particular case.

Let's take intervals with rational end-points as the ``observable
tokens'' in infinite-precision real-number computations (of course,
one could instead work with dyadic numbers or any other convenient
recursively enumerable dense set).  A computation can be taken as an
r.e.\ ascending sequence $r_i$ of tokens. This computes the interval
$\bigsqcup_i r_i = \bigcap_i r_i = [\sup_i \underline{r_i}, \inf_i
\overline{r_i}]$. If the end-points of this are equal, then this is a
computation of a real number; otherwise, it is a \emph{partial
  computation} of a real number, or, as we prefer to see it, a
computation of a \emph{partial(ly defined) real number}.

Suppose that we wish to compute the midpoint operation $m:[0,1]^2 \to
[0,1]$ defined by
\[
m(x,y) = (x+y)/2.
\]
Then we need a process that, given computations~$r_i$ of~$x$ and~$s_i$
of~$y$, produces a computation~$t_i$ of~$m(x,y)$. In this case, it is
obvious what to do: just define $t_i = m(r_i,s_i)$, where the midpoint
of two intervals is taken pointwise. This process not only computes
the midpoint operation $m:[0,1]^2 \to [0,1]$, but also its
pointwise extension to partial numbers $\hat{m}:\I^2 \to \I$
defined by
\[
\hat{m}(x,y) = 
[m(\underline{x},\underline{y}),m(\overline{x},\overline{y})].
\]
But this is just one among many possible ways of performing the
computation. Another, perhaps artificial, way is this: At stage~$i$ of
the computation, read two tokens~$r_i$ and~$s_i$ from the input. If
they have an upper bound in the ordering (that is, overlap as
intervals) output their meet $t_i = r_i \sqcap s_i$ (that is, their
union as intervals). Otherwise, as in the previous example, output the
pointwise midpoint $t_i = m(r_i,s_i)$. For total(ly defined) numbers,
this behaves as the midpoint operation. More generally, this gives
rise to the extension $\tilde{m}:\I^2 \to \I$ defined by
\[
\tilde{m}(x,y) = 
\begin{cases}
x \meet y & \text{if $x$ and $y$ have an upper bound,} \\
\hat{m}(x,y) & \text{otherwise.} 
\end{cases}
\]
The above two extensions of the midpoint operation are easily seen to
be Scott continuous.  Our last example of a computation illustrates a
different phenomenon: At stage $i$, read two tokens $r_i$ and $s_i$
from the input. If the sum of their lengths is strictly bigger
than~$1$, output $t_i = [0,1]$, the bottom element of~$\I$.
Otherwise, as in the first example, output the pointwise midpoint $t_i
= m(r_i,s_i)$. For total numbers, this again computes the midpoint.
However, this computation cannot be modelled by a third extension of
the midpoint operation to the interval domain. In fact, consider the
case in which the inputs are $x=y=[0,1/2]$. One possible pair of
computations of $x$ and $y$ are just the constant sequences
$r_i=s_i=[0,1/2]$, for which we get the constant sequence
$t_i=[0,1/2]$ as output. But another possible pair of computations is
$r_i=s_i=[0,1/2+1/(i+1)]$, for which we get the constant sequence
$t_i=[0,1]$ as output. Thus, the output of this algorithm depends not
only on the input but also on the way the input is presented.

If one is interested only in total numbers, there is nothing wrong
with the last example. But if one works with the interval domain to
attach mathematical meaning to partial computations and uses functions
to account for the input-output behaviour of algorithms, such
computations are ruled out. Notice that the phenomenon illustrated by
the last example is not particular to the interval domain and can be
reproduced in more traditional Scott domains, where the tokens are
elements of a countable basis of the domain.

In practice, one doesn't work with computations as above. A
paradigmatic example is the functional programming language
PCF~\cite{plotkin:lcf}.  The terms of the language can be interpreted
either as mathematical expressions (via the denotational semantics) or
as algorithms (via the operational semantics).  The operational
semantics consists of a set of mathematically valid equations and
algorithmic rules for applying them. For example, out of the term
``3+7'' the calculation produces the term~``10''.  The case of
real-number computation is a bit subtler, as the ``final value''
cannot be computed in finitely many steps and exhibited at once, but
is based on the same principle~\cite{escardo:pcf}.

In both cases, it is of interest to know whether the operational
semantics is ``sequential''. To explain this, we refer to the process
of replacing a subterm of a term by an equal term, according to the
equations of the operational semantics, as \emph{reduction}.  Then the
language is said to be sequential if any unevaluated term~$M$ has a
subterm that has to be reduced first in order to evaluate~$M$.  In the
case of PCF, this is made precise by the Activity
Lemma~\cite{plotkin:lcf}.

The aim of this paper is to show that, under reasonable assumptions, a
language with a data type for real numbers which is mathematically
interpreted as the interval domain and whose constructs are
interpreted as Scott continuous functions cannot have a sequential
operational semantics.  Of course, we assume that computational
adequacy holds: the mathematical and operational meanings 
of the language coincide for terms of base types.

A simple instance of the failure of sequentiality is incorporated in
the first extension of the midpoint operation. In order to see this,
consider the Sierpinski domain $\Sigma = \{ \bot,\top \}$ with $\bot
\sqsubseteq \top$, where, computationally, $\bot$ denotes
non-termination and $\top$ denotes termination (without any further
output), and the two computable functions
$e:\Sigma \to \I$ and $p: \I \to \Sigma$ defined by
\begin{gather*}
e(b) = \begin{cases}
1 & \text{if $b=\top$,} \\
[0,1] & \text{otherwise,}
\end{cases} \qquad\qquad
p(x)  =  \begin{cases}
\top & \text{if $\underline{x} > 1/3$,} \\
\bot & \text{otherwise.}  
\end{cases} 
\end{gather*}
Here $1$ stands for the singleton interval $[1,1]$, and, following the
philosophy discussed above, we generally identify, conceptually and
notationally, singleton intervals with real numbers.  These functions
are intuitively sequential: For the first, we just wait for the input
to terminate; if it does, then we output any computation of~$1$;
otherwise we just wait for ever in vain, so that our output is a
non-terminating computation, which corresponds to complete absence of
information (that is, the bottom interval~$[0,1]$). For the second, we
just read tokens from the input until some satisfies the required
condition, in which case we terminate; if this condition is never
met, then our output is a non-terminating computation, as required.

Now consider the function $f:\Sigma^2 \to \Sigma$ defined by
\[
f(b,c) = p(m(e(b),e(c)).
\]
If $m$ were sequential, then so would be $f$. But $f$ is a
prototypical example of a non-sequentially computable function, the
so-called weak parallel or discussed above:
\begin{gather*}
  f(\bot,\bot) = p(m([0,1],[0,1])) = p([0,1]) = \bot, \\
  f(\bot,\top) = p(m([0,1],1)) = p([1/2,1]) = \top,  \\
  f(\top,\bot) = p(m(1,[0,1])) = p([1/2,1]) = \top, \\
  f(\top,\top) = p(m(1,1)) = p(1) = \top. \\
\end{gather*}
That is, the computation of $f$ terminates if and only if the
computation of at least one of its arguments terminates, so that it is
not possible to choose one argument to evaluate first in order to
correctly compute the output --- one has to alternate between watching
the first and second arguments until one becomes~$\top$.

One might suspect that the problem here is that one is working with
the \emph{greatest} continuous extension of the midpoint operation.
With a more complicated argument, we show in Section~\ref{proof} below
that, no matter which continuous extension is chosen, weak parallel or
is sequentially definable from it.

The midpoint operation and the unit interval domain~$\I$ have been
chosen for technical and expositional reasons. One can equally well
work with addition and the full interval domain~$\R$. But the
corresponding result for this follows by reduction to the previous
case.  We just observe that (1)~the pointwise extension of the
operation $x \mapsto x/2$ to the full interval domain is manifestly
sequential, (2)~the function $r:\mathbb{R} \to [0,1]$ given by
\begin{gather*}
  r(x) = \max(0,\min(x,1)),
\end{gather*}
extends to a computable function $r:\R \to \I$ which is manifestly
sequential, and (3)~any continuous extension of addition to the full
interval domain gives rise to a continuous extension of the midpoint
operation on the unit interval domain by composition with the
functions defined in (1) and (2).


\section{Inherently parallel functions}

For the purposes of the present paper, it is convenient to refer to
the sublanguage of Real PCF consisting of terms in which the parallel
conditional doesn't occur as Weak Real PCF. The operational semantics
of the sublanguage is clearly sequential, and the ``manifestly
sequential'' functions considered in the introduction and in the
development that follows are programmable in the sublanguage.  Notice
that all of them are \emph{unary} functions. But, we emphasize, for
the purposes of the arguments that follow, no previous knowledge of
Real PCF is needed

After briefly summarizing some of the notation and terminology
established in the introduction, we precisely formulate the theorem to
be proved.

Let $\I$ be the continuous Scott domain of non-empty closed
subintervals of $[0,1]$ ordered by reverse inclusion denoted
by~$\sqsubseteq$ and referred to as the information order. Hence
$\bot=[0,1]$ is the least element of $\I$.  Any $x \in \I$ is of the
form $[\underline{x},\overline{x}]$ with $0 \leq \underline{x} \leq
\overline{x} \leq 1$.  We write $\sizeof{x}$ for the size of an
interval $x$, given as $\overline{x}-\underline{x}$.  The elements $x$
with $\underline{x}<\overline{x}$ are called partial real numbers;
those with $\underline{x}=\overline{x}$ are called total real numbers;
we identify the total elements of~$\I$ with points of $[0,1]$. A
rational interval is one with both endpoints rational numbers.

Any two members of~$\I$ have a meet, or greatest lower bound, in the
information order, namely the smallest closed interval containing
their set-theoretical union.  Two elements of~$\I$ are
\emph{compatible} if they have an upper bound in the information order
(that is, overlap as intervals).  In this case, their set-theoretical
union is already an interval and hence gives their meet.

We denote by $\Sigma$ the Sierpinki domain $\{\bot,\top\}$ with the
ordering $\bot\sqsubseteq\top$.  The ``weak parallel-or'' function
$\wpor:\Sigma^2 \to\Sigma$ is given by $\wpor(x,y)=\top$ if and only
if $x=\top$ or $y=\top$.  Our goal is to define $\wpor$ from any
continuous extension of the midpoint operation and certain basic
functions which are regarded as undoubtedly sequential and we now
describe. As explained above, all of them are Weak Real PCF definable.

First, we need test functions from $\I$ to $\Sigma$. We can leave
these largely unspecified and simply assume that for any $a,b\in\I$
with
\[
\text{$a\sqsubseteq b$ and $a \ne b$} 
\]
there is a test function
\[
\test{a}{b}:\I\arr \Sigma
\]
such that
\begin{gather*}
\test{a}{b}(a)=\bot, \qquad
\test{a}{b}(b)=\top.
\end{gather*}
%
For any rational $p,q\in \I$ with 
\[
p \sqsubseteq q
\]
we
have the path function 
\[
\path{p}{q} : \Sigma\to \I 
\]
with
\begin{gather*}
\path{p}{q}(\bot)=p, \qquad
\path{p}{q}(\top)=q.
\end{gather*}
%
For a rational interval $q$ we let 
\[
\cons{q}:\I\arr\I
\]
be the unique increasing affine transformation sending $\bot=[0,1]$ to
$q$. In fact, we have 
\[
\cons{q}(x)=a x+b = [a \underline{x}+b, a \overline{x}+b]
\]
when 
\[
q=[b,a+b].
\]
\begin{defn}
  Let $f:\I^m \to \I$ be a function.  A function $h:\I^n \to \I$ is
  called $f$-definable if it can be obtained from $f$, the test
  functions $\test{a}{b}$, the path functions $\path{p}{q}$, the
  affine transformations $\cons{q}$, and the constants
  $\bot,\top\in\Sigma$, using projections and substitution.
\end{defn}
In other words, the $f$-definable functions are the clone generated by
$f$ and the basic functions.
%
We consider the function $\wpor$ as indisputably parallel and use
definability of $\wpor$ as definition of inherent parallelism.
\begin{defn}
A function $f$ is \emph{inherently parallel} if the function $\wpor$
is $f$-definable. 
\end{defn}
A function $f:\I^2 \arr \I$ is an extension of the midpoint
operation if for total elements $x,y\in \I$ the value $f(x,y)$ is also
total and moreover
$
f(x,y) = (x+y)/2.
$
We are interested in (Scott) \emph{continuous} extensions of the
midpoint operation.  Two examples of extensions have been considered in
the introduction.
%In particular, we have the greatest continuous
%extension given by
%\[
%m(x,y) = 
%[m(\underline{x},\underline{y}),m(\overline{x},\overline{y})].
%\] 
%Another continuous extension has been exhibited in the introduction.
%\[
%\tilde{m}(x,y) = 
%\begin{cases}
%x \meet y & \text{if $x$ and $y$ have an upper bound,} \\
%m(x,y) & \text{otherwise.} 
%\end{cases}
%\]

The next section is devoted to showing that any continuous extension of
the midpoint operation is inherently parallel.

\section{Mediation is inherently parallel} \label{proof}

We begin our study of inherent parallelism with a design for a
definition of $\wpor$ which works in particular for the greatest
continuous extension of the midpoint operation.
%For the sake of contradiction, we assume that
%$f:\I^2 \to \I$ is a continuous extension of the 
%midpoint operation which is \emph{not} inherently parallel.
\begin{lem}\label{designone}
  Any continuous function $f:\I^2 \to \I$ for which there are
  $a,a',b,b' \in \I$ with $a\sqsubseteq a'$, $b\sqsubseteq b'$ and
%\[
$f(a,b) \ne f(a',b) \meet f(a,b')$
%\] 
is inherently parallel. 
\end{lem}
\begin{proof}
  By monotonicity, we have $f(a,b) \sqsubseteq f(a',b)\meet
  f(a,b')$, and, by continuity, we can assume w.l.o.g.\ that $a,a',b,b'$
  are rational intervals. The result then follows from the fact that
\[
\wpor(x,y) = \test{f(a,b)}{f(a',b)\meet
  f(a,b')}(f(\path{a}{a'}(x),\path{b}{b'}(y))).
\] 
\end{proof}

\begin{cor}
  The greatest continuous extension of the midpoint operation is
  inherently parallel.
\end{cor}
\begin{proof}
  Apply Lemma~\ref{designone} with  $a=b=\bot$ and $a'=b'=1$.
\end{proof}

The next result shows that unless a function is inherently parallel
one can construct from it a unary non-stable function which moreover
exhibits its non-stability at an a priori chosen pair of intervals.
\begin{lem}\label{nonstab}
  Let $f:\I^2 \to \I$ be a continuous extension of the midpoint
  operation. Then $f$ is either inherently parallel or
for any two compatible intervals $a,b\in\I$ such that
  $a\meet b \ne a$ and $a\meet b \ne b$ there is an $f$-definable
  function $h:\I \to \I$ with
\[
h(a\meet b) \ne h(a)\meet h(b).
\]
\end{lem}
\begin{proof}
Assume that $f$ is not inherently parallel.  We use
Lemma~\ref{designone} to produce the desired function $h$.  By the
assumption on $a,b$, there is a number $\lambda<1$ such that
$\max(\sizeof{a},\sizeof{b}) < \lambda \sizeof{a\meet b}$.
%  
Let $u$ be the lower endpoint of $a\meet b$ which w.l.o.g.\ we assume
to be the lower endpoint of $a$ as well. Let $v$ be any rational point
in $a\meet b$ but not $a$.
%  
Lemma~\ref{designone} shows that
\begin{equation}
f(a\meet b,a\meet b) = f(a\meet b,v)\meet f(v,a\meet b).
\end{equation}
Since $f$ is an extension of the midpoint operation, $u\in f(a\meet
b,a\meet b)$, and hence either $u \in f(a\meet b,v)$ or $u\in
f(v,a\meet b)$.  W.l.o.g., we assume the former:
\begin{equation}\label{assone}
u\in f(a\meet b,v).
\end{equation} 
We then claim that there exists a
rational interval $d$ such that 
\begin{equation}\label{claim}
f(\cons{d}(a\meet b),v) \ne f(\cons{d}(a),v)\meet
f(\cons{d}(b),v)
\end{equation}
so we would be done with 
\begin{equation}\label{nonstablefunction}
h(x)= f(\cons{d}(x),v).
\end{equation} 
%If we want to exclude total elements from the $f$-definable ones we
%can of course replace $v$ by a sufficiently close rational interval.
%
For the sake of contradiction, assume that
\begin{equation}\label{asscontr}
f(\cons{d}(a\meet b),v)= f(\cons{d}(a),v)\meet
f(\cons{d}(b),v)
\end{equation}
for each rational interval $d$. 
%
We define inductively a 
 sequence $(d_i)_{i\in\mathbb{N}}$ of rational intervals such that 
\begin{equation}\label{asstwo}
u\in f(\cons{d_i}(a\meet b),v).
\end{equation}
and $\cons{d_i}(a\meet b)$ converges to a total real number. This
will give the desired contradiction as $u \ne (z + v)/2=f(z,v)$ for any
total $z\in a\meet b$. 
%
We put $d_0=\bot$; then $f(\cons{d_0}(a\meet b),v)=f(a\meet b,v)$ and
$u\in f(\cons{d_0}(a\meet b),v)$ by assumption (\ref{assone}). If
$d_i$ has already been constructed then by (\ref{asstwo}) we know that
$u\in f(\cons{d_i}(a\meet b),v)$ and assumption (\ref{asscontr}) shows
that $u$ belongs to either $f(\cons{d_i}(a),v)$ or
$f(\cons{d_i}(b),v)$. W.l.o.g.\ we may assume the former and let
$d_{i+1}$ be a rational interval satisfying
\begin{enumerate}
\item $\cons{d_{i+1}}(a\meet b)\sqsubseteq \cons{d_i}(a)$,
\item $\sizeof{\cons{d_{i+1}}(a\meet b)}\leq \lambda
  \sizeof{\cons{d_i}(a\meet b)}$.
\end{enumerate}
(Notice that if $a,b$ are rational intervals, we can simply choose
$d_i$ as the unique rational interval making the first part of the
specification an equality. Otherwise we slightly enlarge it so that
the second part still holds.)  By construction (\ref{assone}) holds
for $d_{i+1}$ and, since $\lambda<1$, the sequence of intervals
$\cons{d_i}(a\meet b)$ shrinks and converges to a total number $w\in
a\meet b$.
%
Now $f(w,v)=(w+v)/2$ is a total number different from $u$. Therefore,
by continuity of~$f$, we must have
\[
u\not\in \bigsqcup_i f(\cons{d_i}(a\meet b,v)
\]
and hence $u\not\in f(\cons{d_i}(a\meet b),v)$ for some~$i$,
contradicting~(\ref{asstwo}).
\end{proof}

\begin{rem}
  If $f$ is inherently parallel, it is still the case that a unary
  non-stable function can be defined from~$f$. In fact, consider
  intervals $c \sqsubseteq c'$ with $c \neq c'$.  Since $\wpor$ is
  $f$-definable by assumption, we can take
\[
h(x) = \path{c}{c'}(\wpor(\test{a \meet b}{a}(x),\test{a \meet
  b}{b}(x))).
\]
But this is not needed for our purposes.
\end{rem}

We now establish a relationship between inherent parallelism and a
naive generalization of sequentiality in the sense of Vuillemin.
Assume that in a computation of $z=f(x,y)$, approximations $u
\sqsubseteq x$ and $v \sqsubseteq y$ have been read from the input and
that $w=f(u,v)$ is the partial output so far. If $f$ were
``intuitively sequential'' then it would have to be that in this
situation either $f$ awaits progress in its left argument, i.e.,
$f(u,v')=w$ for all $v'\sqsupseteq v$ or the other way round, i.e.,
$f(u',v)=w$ for all $u'\sqsupseteq u$. The following asserts that any
$f$ that is not inherently parallel exhibits this property.
\begin{lem}\label{designtwo}
  If a continuous extension $f:\I^2 \to \I$ of the midpoint operation is
  \emph{not} inherently parallel, then for any two intervals $u,v$ one
  of the following cases holds:
\begin{enumerate}
\item $f(u,v)=f(u,v')$ whenever $v\sqsubseteq v'$, or
\item $f(u,v)=f(u',v)$ whenever $u\sqsubseteq u'$.
\end{enumerate}
\end{lem}
\begin{proof}
  Suppose, for the sake of contradiction, that we are given intervals
  $u,u',v,v'\in \I$ with $u\sqsubseteq u'$ and $v\sqsubseteq v'$ such
  that
\begin{enumerate}
\item $f(u,v)\ne f(u',v)$, and
\item  $f(u,v)\ne f(u,v')$.
\end{enumerate}
Let $a=f(u',v)$ and $b=f(u,v')$. By Lemma~\ref{designone} we have
$f(u,v)=a\meet b$.  So, in this case $a\meet b\ne a$ and $a\meet b\ne
b$.  Hence by Lemma~\ref{nonstab} we can find an $f$-definable
function $h:\I \to \I$ such that
\[
h(a\meet b) \ne h(a)\meet h(b).
\]
It follows that 
\[
\wpor(x,y)=\test{h(a\meet b)}{h(a)\meet
  h(b)}(h(f(\path{u}{u'}(x),\path{v}{v'}(y)))),
\]
as required.
\end{proof}

\pagebreak[3] We have now laid the groundwork to prove our main
result.
\begin{thm}
  Any continuous extension of the midpoint operation is inherently
  parallel.
\end{thm}
\begin{proof}
  For a given continuous extension $f$ of the midpoint operation,
  define subsets $\LL,\RR$ of $\I^2$ by
\[\begin{array}{l}
(a,b) \in  \LL \quad \Leftrightarrow \quad 
  \forall b' \sqsupseteq b. \; f(a,b) = f(a,b'),\\
(a,b) \in \RR \quad \Leftrightarrow \quad 
  \forall a' \sqsupseteq a.\; f(a,b) = f(a',b).
\end{array}\]
In other words, $\LL$ consists of all arguments $(a,b)$ where
improving the second argument does not improve output, i.e.\ ``$f$
looks at its left argument first'', and analogously for $\RR$.

We claim that the sets $\LL$ and $\RR$ are each closed under 
under directed suprema. 
Suppose that $(a_i,b_i)_{i \in I}$ is a directed family in $\LL$.
Suppose $b' \sqsupseteq \bigsqcup_{i\in I} b_i$. Then for all $a_i$ we have
$f(a_i,b') = f(a_i,b_i)$. Accordingly, we have
\begin{equation}
f(\bigsqcup_{i\in I} a_i , \bigsqcup_{i\in I} b_i) = 
  \bigsqcup_{i\in I} f(a_i, b_i) = \bigsqcup_{i\in I} f(a_i, b') =
  f(\bigsqcup_{i\in I} a_i , b')
\end{equation}
and, therefore, $(\bigsqcup_{i\in I} a_i , \bigsqcup_{i\in I} b_i) \in
\LL$. The proof of the claim for $\RR$ is symmetric.

Now assume for the sake of contradiction that $f$ is not inherently
parallel.  Then Lemma~\ref{designtwo} states that
$\LL\cup\RR=\I^2$.  Then for all total $x \in [0,1]$ we have $(x
, \bot) \notin \LL$ as otherwise $f(x,y) = f(x,\bot)$ for all total $y
\in [0,1]$ which is impossible as $f(x , y_1) \neq f(x , y_2)$ for
different $y_1 , y_2 \in [0,1]$ as $f$ agrees with the midpoint
operation at total elements. In particular, we have that $(0,\bot)
\notin \LL$. Pick any $\epsilon>0$.  Since $\LL$ is closed under
directed suprema this means that already $(a,\bot)\not\in\LL$ for some
non-total element $a\in\I$ with $\sizeof{a}\leq\epsilon$ and $0 \in
a$. Since $\LL\cup\RR=\I^2$ we must therefore have
$(a,\bot)\in\RR$.

By a similar argument as above we have for all total $y \in [0,1]$
that the pair $(a,y) \notin \RR$ and that, therefore, for each such
$y$ we can find a non-total element $b_y$ containing $y$ such that
$(a,b_y) \notin \RR$, and hence $(a,b_y)\in\LL$.  Since the intervals
$b_y$ cover $[0,1]$, compactness of $[0,1]$ gives finitely many
intervals $b_1 , \dots , b_n$ with
\begin{enumerate}
\item[(1)] $b_1 \cup \dots \cup b_n = [0,1]$ and 
\item[(2)] $(a,b_i) \in \LL$ for $i=1,\dots,n$.
\end{enumerate}
It follows from (2) that whenever $z \in b_i \cap b_j$ then $f(a, b_i)
= f(a,z)=f(a,b_j)$.  By iterating this argument using (1) we get that
$f(a,b_i)=f(a,b_j)$ for any two $i,j$. It follows from (2) that
$f(a,0)=f(a,1)$ because there are $i$ and $j$ with $0 \in b_i$ and $1
\in b_j$.

Summing up, for each $\epsilon>0$ we can find an interval $a$ with $0
\in a$ and with $\sizeof{a}\leq \epsilon$ such that $f(a,0)=f(a,1)$.
In particular $1/2\in f(a,0)$, and continuity of $f$ then gives
$1/2\in f(0,0)$, which contradicts the assumption that $f$ extends the
midpoint operation.
\end{proof}

\pagebreak[3]

\section{Remarks and questions}

This work could be improved in a number of directions.  

Firstly, we ask whether the main theorem holds for any domain (among a
class of domains of interest) with subspace of maximal elements (with
the relative Scott topology) homeomorphic to the Euclidean real line
or unit interval. More generally, we can consider domains with a
(densely) embedded copy of line or unit interval.

In fact, there are many well-known variations of the interval domain,
which have been used for different purposes. For example, one can
consider: (1) The set of all non-empty closed and bounded real
intervals --- this fails to be a continuous Scott domain only by
lacking a bottom element; to remedy this, just include the interval
$(-\infty,+\infty)$. (2) The previous together with the empty
interval~$\emptyset$, a top element, so that one gets a continuous
lattice --- this is the original interval domain considered by Scott.
(3) The set of non-empty compact intervals of the two-point
compactification $[-\infty,+\infty]$ of the real line. (4) The set of
non-empty compact connected subsets of the one-point compactification
of the line. This is isomorphic to the poset of non-empty closed
intervals of the circle.  It is not a continuous Scott domain, but it
is an FS-domain in the sense of Jung. (5) The set of all non-empty
compact subsets of (some compactification of) the real line or of some
subspace of the real line such as the unit interval. 

It is easy to see that our result applies to these domains as well,
with slight adaptations in the proofs. However, the domains having the
real line as the subspace of maximal points are endless. For example,
if $R$ is any such domain and $D$ is any domain with a top element,
then $R \times D$ and $[D \to R]$ are also domains with the real line
as the subspace of maximal points.  We suspect that our result should
apply to any domain environment for the line or unit interval.  But
notice that this doesn't rule out the possibility of sequential models
of (higher type) real-number computation other than Scott domains.

Secondly, we have claimed that Weak Real PCF is manifestly sequential.
While hardly anyone would cast doubt on this claim, one would like to
have a mathematical formulation and proof. Amin Farjudian has recently
performed some steps in this direction, by showing that all
first-order Weak Real PCF definable functions are Vuillemin
sequential~\cite{amin:wrpcf} (moreover, his argument shows that any
extension of the language with continuous first-order unary functions
has the same property). A further case for manifest sequentiality of
Weak Real PCF would be its conservativity over PCF.

Thirdly, it would be interesting to know how parallel the midpoint
operation is. For instance, is (strong) parallel or sequentially
definable from it?

\paragraph{Acknowledgment}
This work was supported by an EPSRC grant number \\ GR/M64840.

\bibliographystyle{plain}
{\footnotesize \bibliography{references}}

\end{document}
