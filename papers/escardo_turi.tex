\documentclass[11pt,a4paper]{article}
\usepackage{amssymb,latexsym}
\usepackage{a4wide}

\usepackage[all]{xy}
%\CompileMatrices
\UseComputerModernTips

\setlength{\parindent}{0pt}
\setlength{\parskip}{1ex plus 0.5ex minus 0.2ex}

\input{Dmacros} %Daniele's macros 
\input{Mmacros} 
\input{definitions}

\newcommand{\tuple}[1]{\langle #1 \rangle }
\newcommand{\implies}{\;\; \Rightarrow \;\;}
\newcommand{\cppobot}{\mathit{Cppo}_\bot}
\newcommand{\cpo}{\mathit{Cpo}}
\renewcommand{\sets}{\mathit{Set}}
\newcommand{\coalg}[1]{#1 \mbox{-\textit{Coalg}}}
\renewcommand{\alg}[1]{#1 \mbox{-\textit{Alg}}}
\newcommand{\recu}{\sharp}
\newcommand{\ideps}{\id^\varepsilon}
\newcommand{\Ideps}{\Id_\varepsilon}
\newcommand{\monad}{T}
\newcommand{\efunctor}{S}
\newcommand{\contr}{\sigma}
\newcommand{\contrcat}{\sigma \mbox{-\textit{EndoMaps}}}
\newcommand{\contcat}{\sigma \mbox{-\textit{Maps}}}
\newcommand{\factored}[1]{\overline{#1}}
\newcommand{\omegabar}{\Omega}
\newcommand{\delay}{\delta}
\newcommand{\fp}{\mathbf{Y}}
\newcommand{\fpp}{\overline{\fp}}
\newcommand{\fpc}{\mathbf{Y}}
\newcommand{\fppc}{\overline{\fp}}
\newcommand{\ms}{\mathit{MS}}
\newcommand{\cms}{\mathit{CMS}}
\newcommand{\pointedfunctor}{\tuple{\efunctor , \contr}}
\newcommand{\isom}[1]{\varphi_{#1}}
\newcommand{\comparing}{\varphi}
\newcommand{\inial}{\alpha '}
\newcommand{\ti}{T}
\newcommand{\esse}{S}
\newcommand{\esseti}{ST}
\newcommand{\nameof}[1]{\ulcorner #1 \urcorner}
\renewcommand{\strength}{\mathfrak{t}}
\newcommand{\stre}{\mathfrak{s}}
\newcommand{\monoi}{\mathfrak{m}}
\newcommand{\lolly}[2]{[#1 ,#2 ]} 
                      % {#1 \triangleright #2}  %{( #1 \! \multimap \! #2 )}
\newcommand{\f}{\mathfrak{f}}
\newcommand{\g}{\mathfrak{g}}
\newcommand{\computation}[2]{#2^{(#1)}}
\newcommand{\h}{\varepsilon}
\newcommand{\dd}{\overline{d}}
\newcommand{\order}{\sqsubseteq}


\title{{\Large\bf Fixpoint Monads}
\\
{\large\bf Reconciling Domain and Metric Theoretic Fixpoints}
}

\usepackage[english]{babel}

\date{24th March 2000}

\author{
\begin{tabular}{c}
Mart\'{\i}n H.\ Escard\'{o}\\
%Division of Computer Science\\
%School of Mathematical and Computational Sciences\\
%Univ.\ of St Andrews\\
{\tt <mhe@dcs.st-and.ac.uk>}
\end{tabular}
\hspace{7mm}
\begin{tabular}{c}
Daniele Turi\thanks{Research supported by EPSRC grant R34723.}
\\
%LFCS\\
%Division of Informatics\\
%Univ.\ of Edinburgh\\
{\tt <dt@dcs.ed.ac.uk>}
\end{tabular}
}

%\date{\today}

\begin{document}

\maketitle


\begin{abstract}
\noindent
We show that least fixpoints of endomaps on pointed domains
and unique fixpoints of contractive endomaps on complete metric spaces
arise from one and the same categorical structure,
namely that of a \emph{fixpoint monad} corresponding to 
a pointed endofunctor.
This specializes to the existing notion of a fixpoint object.
As an application we give a criterion for algebraic compactness.
\end{abstract}


\section{Introduction}

Traditionally,
denotational models of computation have been
based on some form of partial order (\emph{domain}) \cite{domains},
with recursion modelled using Tarski's
least fixpoint theorem.
An alternative approach,
developed especially for modelling concurrent computations,
is the metric-theoretic one \cite{ACG},
where recursion is modelled using Banach's unique
fixpoint theorem.
Various attempts have already been made to reconcile these
two different approaches (see, eg, \cite{Smyth,Wagner}).
Here we address one particular question,
namely whether it is possible to provide
a uniform framework for understanding
both Tarski's and Banach's fixpoint theorems,
despite their apparent heterogeneous nature.

A motivation for our investigation
is \cite{escardo:metric}, where
a metric model of PCF is given
which is very close to the traditional
domain-theoretic model \cite{Scott}.
Here we give a conceptual explanation
of some constructions performed in that work.
A crucial observation is 
that $\varepsilon$-contractive maps
$X\rightarrow Y$
between metric spaces 
are in bijection with 
maps of type $\Id_\varepsilon X \rightarrow Y$,
where $\Id_\varepsilon$ is the endofunctor
scaling down the distance
between the points of a space by a factor $\varepsilon \in [0,1)$.
Contractive \emph{endomaps} thus appear as algebras 
$\Id_\varepsilon X \rightarrow X$
of the endofunctor $\Id_\varepsilon$.

% rewrite this introduction completely


We use this correspondence as a basis for
an axiomatization of the notion of contractive endomap
and Banach's fixpoint theorem,
following closely the
domain-theoretic axiomatization
which led to the notion of \emph{fixpoint object} \cite{fpobject}.
%The latter is based on the notion of lifting monad.
However, we do not use 
the notion of lifting monad as our starting point.
There are at least two reasons for that. 
Firstly,
the the obvious candidate for 
a metric counterpart of lifting
is some form
of down-scaler endofunctor,
but down-scalers cannot be given the structure of a monad.
The problem is that there is no map from $\Id_\varepsilon X$ to $X$ which
is non-distance increasing, hence, in particular, no multiplication.
Secondly,
while
the initial algebra of the \emph{endofunctor part} of
the lifting monad is a fixpoint object,
providing the basic structure to compute least fixpoints
of endomaps on pointed domains,
the initial algebra of any
metric down-scaler endofunctor is the trivial empty metric space.

We overcome the second obstacle by noticing
that although the initial algebra,
i.e.\ the free algebra over the initial object,
is trivial,
all other free algebras of $\Id_\varepsilon$
can be used to obtain the 
fixpoints of $\varepsilon$-contractive maps given by Banach's theorem
in essentially the same way as for the fixpoints
of endomaps on cppos from the fixpoint object.
We proceed then by axiomatizing
the relevant structure involved
in the above remarks,
obtaining the new notions of $\contr$-contractive map
and of fixpoint monad relative to arbitrary pointed endofunctors
$\pointedfunctor$.

The theorems we prove are routine generalizations
of those for fixpoint objects,
once one realizes various unexploited generalities in the latter.
For instance, one can start from a pointed endofunctor
rather than from a monad in order to define the fixpoint object.
This is crucial, because the down-scaler $\Id_\varepsilon$ comes with
a natural point, namely the evident `almost identity' map 
$\id^\varepsilon : \Id \rightarrow \Id_\varepsilon$,
hence the structure of the lifting monad relevant 
for fixpoints is shared by the down-scalers.
In fact, we generalize the motivating theorem for fixpoint objects
\cite[Thm 4.3]{alex}
by considering fixpoints of (possibly weakly) $\contr$-contractive maps.
This not only accounts for contractive maps in the metric sense,
but also for all endomaps on cppos, the
least element of a cppo giving algebraic structure which
makes any endomap into an algebra of the lifting
endofunctor.

A further sign of the soundness of our generalization is
that the theorems relating fixpoint objects
and final coalgebras \cite{Fre90}
also generalize to fixpoint monads.
Indeed, we easily prove that the existence of
a certain family of final coalgebras
gives rise to a fixpoint monad.
In the converse direction,
we use the notion of 
\emph{locally contractive} endofunctors
on monoidal closed categories
and the fact that they are algebraically compact.







%This work is based on the combination of two elementary observations.
%The first observation,
%made in \cite{escardo:metric},
%was that $\varepsilon$-contractive maps
%$X\rightarrow Y$
%between metric spaces 
%are in 1-1 correspondence with 
%maps of type $\Id_\varepsilon X \rightarrow Y$,
%where $\Id_\varepsilon$ is the `down-scaler' endofunctor
%on metric spaces which behaves as the identity on
%the underlying sets while scaling down the distance
%between the points by a factor $\varepsilon \in [0,1)$.
%In particular,
%$\varepsilon$-contractive \emph{endomaps} are then algebras 
%$\Id_\varepsilon X \rightarrow X$
%of the endofunctor $\Id_\varepsilon$.

%When one tries to relate metric spaces to domains
%one naturally looks for a metric notion of lifting.
%A first candidate is some form of down-scaler endofunctor,
%but that is bound to fail for down-scalers cannot be monads,
%because there is no non-distance increasing map
%from $\Id_\varepsilon X$ to $X$, hence no multiplication.
%Morever, the initial algebra of the \emph{endofunctor part} of
%the lifting monad is a fixpoint object,
%providing the basic structure to compute least fixpoints
%of endomaps on pointed domains,
%while the initial algebra of any
%metric down-scaler endofunctor is the trivial empty metric space.
%It is here that our second observation took place.
%We noted that although the initial algebra,
%ie the free algebra over the initial object,
%is trivial,
%all other free algebras of $\Id_\varepsilon$
%can be used to obtain the 
%fixpoints of $\varepsilon$-contractive maps given by Banach's theorem
%in essentially the same way as for the fixpoints
%of endomaps on cppos from the fixpoint object.


%\paragraph{Contents.}
%In Section \ref{terminology} we 
%recall a few notions regarding metric spaces and
%algebras in order to fix some terminology and notation.
%In Section \ref{contract}
%we introduce the notion of contractive maps with
%respect to a pointed endofunctor.
%In Section \ref{fix} we introduce the notion of fixpoint monad
%and prove the main theorem for it, generalizing
%\cite[Theorem ??]{fpobject}.
%Finally, in Section \ref{coalg}
%we introduce the notion of locally contractive endofunctor and
%generalize the results 




\section{Terminology and Notation}

\paragraph{Metric Spaces}
In this paper by a \emph{metric space} we
intend a \emph{non-empty} set $X$ together with
a \emph{bounded} metric $d: X\times X \rightarrow [0,\infty )$.
Without loss of generality, we assume the
bound to be 1, hence $d: X\times X \rightarrow [0,1]$.
Such metric spaces form a category
with morphisms 
$\tuple{X,d}\rightarrow \tuple{X',d'}$
given by non-distance increasing functions
$f:X\rightarrow X'$, ie
$d' (f(x_1 ) ,f(x_2 ))\leq d(x_1 ,x_2 )$
for all $x_1$ and $x_2$ in $X$.
A special case of non-distance increasing functions
are the $\varepsilon$-contractive ones,
where $\varepsilon \in [0,1)$ and
$d' (f(x_1 ) ,f(x_2 ))\leq \varepsilon \cdot d(x_1 ,x_2 )$
for all $x_1$ and $x_2$ in $X$.
(See, eg, \cite{Dugu} for more on metric spaces.)

\paragraph{Algebras}
A pointed endofunctor $\pointedfunctor$ 
consists of an endofunctor on a category $\CC$
and a natural transformation $\sigma : \Id \Rightarrow S$.
(The latter is called the point of $S$, but will shall call
points also the maps $1\rightarrow X$ from a terminal object.)
In this paper we consider algebras of endofunctors $S$,
of pointed endofunctors $\tuple{S,\sigma}$,
and of monads $\tuple{S,\sigma ,\mu}$ on a category $\CC$.
All three consist of maps $f:SX\rightarrow X$,
but this map has to satisfy the equations
$\id_X = f \comp \sigma_X$ and $f\comp Sf = f \comp \mu_X$
if $S$ is a monad, only the first equation if $S$ is
a pointed endofunctor, and no equation at all if $S$
is just an endofunctor.
When we say that $f$ is an algebra of an endofunctor $S$
we mean that $f$ has to satisfy no equation at all,
even if $S$ is a pointed endofunctor or a monad.

The definition of algebra homomorphism 
$h :\tuple{X,f} \rightarrow \tuple{Y,g}$
is independent from
the equations for the algebra structure;
that is, in all three cases it consists of a morphism
$h: X \rightarrow Y$ in the underlying category
such that
$$\xymatrix{
\esse X \ar[d]_{f} \ar[r]^{\esse h} 
     & \esse Y\ar[d]^{g}
\\
X \ar[r]_h &  Y
}$$
commutes.
In the sequel, $\alg{S}$ 
will denote the category of
algebras of the endofunctor $S$.
The symbol $T$ will be reserved for a monad,
and $\alg{T}$ for the corresponding category.


\section{$\contr$-Contractive Maps}

\begin{definition}
  Let $\tuple{\esse , \contr}$ be a pointed endofunctor.  We say that
  a map $f:X\rightarrow Y$ is \emph{weakly $\contr$-contracting} if it
  factors through $\contr_X : X \rightarrow \esse X$; if there is a
  \emph{unique} $\factored{f} : \esse X \rightarrow Y$ such that
$$\xymatrix{
X \ar[r]^{\contr_X}  \ar[rd]_{f}  & {\esse X} \ar@{-->}[d]^{\factored{f}}  
\\
& Y
}$$
commutes then we say that $f$ is \emph{$\contr$-contracting}
and we call such $\factored{f}$ the $\contr$-factor of $f$.
\close
\end{definition}

\begin{example}
  Lifting with eta in cpos. A map is weakly contractive iff its image
  has a lower bound. In this case a factor maps bottom to a lower
  bound. Hence, strong contractivity hardly ever occurs.
\end{example}

\begin{example}
  The functor is down-scaler for a given fixed $\varepsilon$, the
  point is the ``identity''. Contractive maps are contractive maps in
  the metric sense, with contractivity factor~$\varepsilon$. Since the
  ``identity'' is epi, weak and strong contractivity coincide, as the
  following proposition shows.
\end{example}


\begin{proposition}
  \begin{enumerate}
  \item 
  If the pushout of $\contr_X : \esse X \rightarrow Y$ along
$f$ exists and it is the identity on $Y$
then $f$ is $\contr$-contracting.
$$\xymatrix{
X \ar[r]^{\contr_X}  \ar[d]_{f}  & {\esse X} \ar[d]^{\factored{f}}  
\\
Y \ar[r]_{\id_Y} & Y
}$$ 
\item 
If $\contr$ is pointwise epi and $f$ is weakly $\contr$-contracting,
then the pushout of 
$\contr_X : \esse X \rightarrow Y$ along
$f$ exists and it is the identity on $Y$,
hence $f$ is $\contr$-contracting.
  \end{enumerate}

\emph{Proof.}
1) If $g:SX \rightarrow Y$ is such that $g\comp \contr_X = f$,
then, by the universal property of pushouts,
there is a unique $g': Y\rightarrow Y$ such that
$g' \comp \id_Y = \id_Y$ and $g'\comp g = \factored{f}$,
which implies that $g=\factored{f}$. 

2) Let $\factored{f}$ be a $\contr$-factor for $f$.
Take $h:SX \rightarrow Z$ and $k:Y\rightarrow Z$
such that $k\comp f = h \comp \contr_X$. 
Then $k\comp \factored{f} \comp \contr_X = h \comp \contr_X$,
hence, since $\contr_X$ is epi, $k\comp \factored{f} = h $,
which means that $k$ is the desired 
unique mediating morphism from $Y$ to $Z$.
\close
\end{proposition}


%\emph{Alternative definition.}
%We say that $f:X\rightarrow Y$ is \emph{$\contr$-contracting}
%if the pushout of $\contr_X : \esse X \rightarrow Y$ along
%$f$ exists and it is the identity on $Y$.
%$$\xymatrix{
%X \ar[r]^{\contr_X}  \ar[d]_{f}  & {\esse X} \ar[d]^{\factored{f}}  
%\\
%Y \ar[r]_{\id_Y} & Y
%}$$
%We call the pushout $\factored{f}$ of $f$ along
%$\contr_X : \esse X \rightarrow Y$ 
%the $\contr$-factor of $f$.

Note that if $f:X\rightarrow Y$ is (weakly) $\contr$-contractive
so are the composites $f\comp g$ and $h\comp f$,
for all maps $g:X' \rightarrow X$ and $h: Y\rightarrow Y'$.
In particular, since any map $X\rightarrow 1$
into a terminal object is trivially contracting,
constant maps $X\rightarrow 1 \rightarrow Y$ are 
contractive.

If $\tuple{\esse , \contr , \mu}$ is a monad
and $X$ is an algebra for the \emph{monad} $\esse$,
then every map $f: X\rightarrow Y$ is trivially 
weakly $\contr$-contractive,
because the algebra structure cancels 
the unit $\contr_X$ of the monad $\esse$.
This is the case when $X$ is a cppo and $\esse$
is the lifting monad (although $\contr$ is not pointwise epi).

We now consider the $\contr$-contractive 
maps for a pointed endofunctor $\pointedfunctor$
on a category $\CC$ as the \emph{objects} of a category.
The arrows $f\rightarrow g$
are pairs of maps $h$ and $k$ in $\CC$ such that the diagram
$$\xymatrix{
X \ar[d]_{f} \ar[r]^{h} 
     & X'\ar[d]^{g}
\\
Y \ar[r]_k &  Y'
}$$
commutes.
Then, clearly, such a category is isomorphic to the 
comma category
$(\esse \downarrow \Id )$:
$$\xymatrix{
{\esse X} \ar[d]_{\factored{f}} \ar[r]^{\esse h} 
     & {\esse X'} \ar[d]^{\factored{g}}
\\
Y \ar[r]_k &  Y'
}$$
This restricts to an isomorphism
between the subcategory of $\contr$-contractive \emph{endomaps}
and the category $\alg{\esse}$ of algebras $\esse X \rightarrow X$
of the endofunctor $\esse$.


%This gives a 1-1 correspondence between
%$\contr$-contractive endomaps and $\esse$-algebras.
%The following proposition shows that this extends to 
%an isomorphism between the category $\alg{\esse}$ of 
%$\esse$-algebras and the natural category $\contrcat$
%of $\contr$-contractive endomaps.

%\begin{proposition}
%  If $\contr$ is pointwise epi then
%the first diagram commutes if and only if the second does,
%for all $\contr$-contractive maps $f$ and $f'$ and
%all maps $h$ and $k$:
%$$\xymatrix{
%X \ar[d]_{f} \ar[r]^{h} 
%     & X'\ar[d]^{g}
%\\
%Y \ar[r]_k &  Y'
%}
%\HSP
%\xymatrix{
%{\esse X} \ar[d]_{\factored{f}} \ar[r]^{\esse h} 
%     & {\esse X'} \ar[d]^{\factored{g}}
%\\
%Y \ar[r]_k &  Y'
%}$$
%\close
%\end{proposition}



\section{Fixpoint Monads}

If the (evident) forgetful functor $U : \alg{\esse} \rightarrow \CC$
has a left adjoint, then
we denote the monad generated by this adjunction by 
$\ti = \tuple{\ti ,\eta ,\mu}$,
and the free $\esse$-algebra over an object $X$ by
$$\delta_X : \esse \ti X \rightarrow \ti X$$
Note that $\delta_X$ is natural in $X$.
We call $\ti$ the \emph{monad generated by $\esse$}.

\begin{definition}
Let $T$ be a monad on a category $\CC$ with a terminal object 1.
Let 1 also denote the constantly 1 endofunctor on $\CC$.
Then $\ti$ 
is a \emph{fixpoint monad} for 
a pointed endofunctor $(\esse , \contr )$
if 
it is generated by the endofunctor $\esse$
and the parallel natural transformations
$\delta \comp \contr_\ti$ and $\id_T$
have an equalizer $\infty : 1 \rightarrow \ti $.
\close
\end{definition}
Note that $\infty$ is then a fixpoint for $\delta \comp \contr_\ti$
\begin{equation}
  \label{eq:fpmonad}
  \xymatrix{
1 \ar[d]_-{\infty}  \ar[rrd]^{\infty} & &
\\
\ti  \ar[r]_-{\contr_{\ti}} & 
{\esse \ti} \ar[r]_{\delta} & \ti
}
\end{equation}
and that the naturality of $\infty_X$ is a consequence
of the universal property of equalizers.

\begin{example}
  The monad $T$ generated by the endofunctor $\Id_\varepsilon$ is the
  monad studied in \cite{escardo:metric}.  Its action on a (non-empty,
  1-bounded) complete metric space $X=\tuple{X,d}$ is the set
  $$\ti X = (X\times \Nat ) \cup \s{\infty}$$
  endowed with the metric
  $\dd :\ti X \times \ti X \rightarrow [0,1]$ defined by
$$
\dd (\infty,\infty) = 0, \qquad 
\dd (\computation{n}{x} ,\infty) = \dd (\infty,\computation{n}{x}) = \h^{n}
$$
$$
\dd (\computation{n}{x},\computation{n}{y})= \h^{n}\cdot d(x,y), \qquad
\dd (\computation{n}{x},\computation{m}{y})=\h^{\min(n,m)} \qquad
\mbox{for $n \ne m$}. 
$$
where $\computation{n}{x}$ stands for $(x,n)$.  We regard the
points of $\ti X$ as ``abstract computations'' of elements of~$X$.
The free algebra structure $\delta_X : \esse \ti X \rightarrow \ti X$
is the \emph{delay operator}
\[
\delay_{X}(\computation{n}{x}) = \computation{n+1}{x}, \qquad
\delay_{X}(\infty) = \infty.
\]
Finally, the unit $\eta_X : X \rightarrow \ti X$ and the
multiplication $\mu_X = \isom{\delta_X} : \ti^2 X \rightarrow \ti X$
are given as follows:
\[ 
\eta_X(x)  = \computation{0}{x}, \qquad
\mu_X\left(\computation{m}{\left(\computation{n}{x}\right)}\right) =
\computation{m+n}{x}.
\] 
\end{example}

\begin{example}
  The monad $T$ generated by the lifting endofunctor $L$ is
  defined as follows.  Its action on a cpo $X$ is the set
  $$\ti X = (X\times \Nat ) \cup (\Nat \cup \s{\infty})$$
  ordered by:
$$
\computation{n}{x} \order \computation{n}{y}\iff x\order y \qquad
n \order \computation{m}{x} \iff n \leq m
$$
$$
n\order m \iff n \leq m \qquad n \order \infty
$$
where, again, $\computation{n}{x}$ stands for $(x,n)$.
The free algebra structure is defined in a similar way as in the
metric case, and for the extra points of $\Nat$ it is the
successor. So is the unit [??what about multiplication??]
\end{example}

\begin{proposition}
If an endofunctor $\esse$ generates a monad $\ti$
then the canonical comparison functor
$$K:\alg{\esse} \rightarrow \alg{\ti}$$
between the algebras of the \emph{endofunctor} $\esse$ and
the algebras of the \emph{monad} $\ti$ is an isomorphism.
%
%\emph{Proof.}
%$$
%\isom{\f} \comp \delta_X \comp \esse \eta_X
%= \f \comp \esse \isom{\f} \comp \esse \eta_X
%= \f
%$$
%$$
%\g\comp \delta_X = \g\comp \delta_X \comp \esse \mu_X \comp \esse \eta_{TX}
%= \g\comp \mu_X \comp \delta_{TX} \comp \esse \eta_{TX}
%= \g\comp T\g \comp \delta_{TX} \comp \esse \eta_{TX}
%$$
%$$
%= \g\comp \delta_X \comp \esse \ti \g \comp \esse \eta_{TX}
%= \g\comp \delta_X \comp \esse \eta_{X} \comp \esse \g
%$$
\close
\end{proposition}
Before giving concrete examples, we give the general construction of
this folklore isomorphism.

The functor $K$ maps an $\esse$-algebra $\f: \esse X \rightarrow X$ to
the $\ti$-algebra $\isom{\f} : TX \rightarrow X$, obtained by
transposing the identity $\id_X :X\rightarrow X =U\tuple{X,\f}$ across
the adjunction from $\CC$ to $\alg{\esse}$:
\begin{equation}
  \label{eq:kappa}
  \xymatrix{
{\esse TX} \ar[d]_{\delta_X} \ar[r]^{\esse \isom{\f}} 
     & {\esse X} \ar[d]^{\f}
\\
TX \ar[r]_{\isom{\f}} &  X
}
\end{equation}

\begin{example}
Given a map $f:LX\rightarrow X$, the structure map sends
$n$ to $f^n (f(\bot ))$, 
and a point $\computation{n}{x}$ to $f^n (x)$.
\end{example}

\begin{example}
  Given an $\varepsilon$-contractive map $f$
the structure map sends $\infty$ to the fixpoint of $f$
and a point $\computation{n}{x}$ to $f^n (x)$.
\end{example}


In other words, $\isom{\f}$ is the counit of the 
adjunction applied to the algebra $\tuple{X,\f}$.
On morphisms, the comparison functor behaves as the identity.
Therefore:

\begin{remark}
  $\comparing_{\f}$ is natural in $\f$, in the sense
that for every commuting square
$$\xymatrix{
\esse X \ar[d]_{\f} \ar[r]^{\esse h} 
     & \esse Y\ar[d]^{\g}
\\
X \ar[r]_h &  Y
}$$
we have $\isom{\g}\comp Th = h\comp \isom{\f}$:
$$\xymatrix{
\ti X \ar[d]_{\isom{\f}} \ar[r]^{\esse h} 
     & \esse \ti Y\ar[d]^{\isom{\g}}
\\
X \ar[r]_h &  Y
}$$
\close
\end{remark}

The inverse of $K$ acts on a $\ti$-algebra $\ti X \rightarrow X$
by precomposing it with 
$\esse X \stackrel{\esse \eta_X}{\rightarrow}\esse \ti X 
  \stackrel{\delta_X}{\rightarrow} \ti X$.


\begin{theorem} 
If $\ti$ is
a fixpoint monad for a pointed endofunctor $(\esse , \contr )$,
then, for every $\esse$-algebra structure $\f:\esse X\rightarrow X$
the point
$$\fpc_{\f} \defin \isom{\f} \comp \infty_X : 1\rightarrow X$$
has the following two properties:
\begin{itemize}
\item[1.] (\emph{$\contr$-fixpoint})
$\fpc_\f$ is a $\contr$-fixpoint for $\f$,
ie $\fpc_\f = \f \comp \contr_X \comp \fpc_\f$:
$$\xymatrix{
1 \ar[d]_-{\fpc_\f}  \ar[rrd]^-{\fpc_\f} & & 
\\
X \ar[r]_{\contr_X} & \esse X \ar[r]_-\f & X
}$$
\item[2.] (\emph{uniformity}) $\fpc_\f$ is natural in $\f$ in the sense that 
for all $\esse$-algebra homomorphisms 
$h:\tuple{X,\f} \rightarrow \tuple{Y,\g}$
$$\xymatrix{
\esse X \ar[d]_{\f} \ar[r]^{\esse h} 
     & \esse Y\ar[d]^{\g}
\\
X \ar[r]_h &  Y
}$$
we have $\fpc_\g = h\comp \fpc_\f$:
$$\xymatrix{
1 \ar[d]_-{\fpc_\f}  \ar[rd]^-{\fpc_\g} & 
\\
X \ar[r]_-h & Y
}$$
\end{itemize}
Moreover:
\begin{itemize}
\item[3.] (\emph{uniqueness})
$\fpc$ is the unique operator verifying the two properties above.
\end{itemize}

\emph{Proof.}
1)
Just note that the following diagram commutes
because of diagrams (\ref{eq:fpmonad}) and (\ref{eq:kappa})
and of the naturality of $\contr$
$$\xymatrix{
1\ar[r]^-{\infty_X} \ar[rdd]_-{\infty_X}
   & TX \ar[r]^{\isom{\f}} \ar[d]^{\contr_{TX}} 
   & X \ar[d]^{\contr_X} %\ar@/^/[dd]
\\
   & \esse TX \ar[r]^{\esse \isom{\f}} \ar[d]^{\delta_X}
   & \esse X \ar[d]^{\f}
\\
   & TX \ar[r]_{\isom{\f}} 
   & X
}$$
2)
By naturality of $\infty$ and $\comparing$:
$$\fpc_\g = \isom{\g} \comp \infty_Y = 
\isom{\g}\comp Th \comp \infty_X 
  = h \comp \isom{\f} \comp \infty_X = h \comp \fpc_\f$$
3)
By the equalizing property of $\infty$,
we have that any other $\contr$-fixpoint operator $\fppc$
has the property that $\fppc_{\delta_X} = \infty_X$.
Then, if $\fppc$ satisfies the second property,
$$\fppc_\f = \isom{\f} \comp \fppc_{\delta_X}
= \isom{\f} \comp \infty_X = \fpc_\f$$
because, by definition,
$\isom{\f}$ is an $\esse$-algebra homomorphism
between $\delta_X$ and $\f$.
\close
\end{theorem}
(Cf \cite[Thm 4.3]{alex} and \cite[Thm 3.12]{Mulry}.)

\begin{example}
  Cpos.

The familiar fixpoint operator of domain theory 
is recovered by considering maps 
$$\esse X \stackrel{x}{\rightarrow} X \stackrel{f}{\rightarrow} X$$
where $x$ is an algebra for the \emph{pointed} endofunctor $\pointedfunctor$,
hence $x\comp \contr_x = \id_X$.
Then we have that for every endomap $f:X\rightarrow X$
on a $(\esse , \contr )$-algebra $\tuple{X,x }$
the point
$$\fp^x_f \defin \isom{f\comp x} \comp \infty_X : 1\rightarrow X$$
is a fixpoint for $f$:
$$\fp^x_f = f \comp x \comp \contr_X \comp \fp^x_f = f \comp \fp^x_f$$
Moreover,
$\fp$ is natural in $\tuple{f,x}$ in the sense that 
for all endomaps 
$f: X\rightarrow X$ and $g:Y\rightarrow Y$
on $(\esse , \contr )$-algebras $\tuple{X,x }$ and $\tuple{Y,\upsilon }$
and all homomorphisms $h:X \rightarrow Y$
$$\xymatrix{
\esse X \ar[d]_{x} \ar[r]^{\esse h} 
     & \esse Y\ar[d]^{y}
\\
X \ar[d]_{f} \ar[r]^{h} 
     & Y\ar[d]^{g}
\\
X \ar[r]_h &  Y
}
$$
we have $\fp^y_g = h\comp \fp^x_f$.
$\fp$ is the unique operator verifying the two properties above.
Note that we do not need to assume that $\esse$ is a monad,
although when we take $\esse$ to be the
lifting monad on cpos then its (monad) algebras
are property rather than
structure, hence the superscript $x$ can be dropped.
The fixpoint operator is then the one given
by the least fixpoint theorem for 
endomaps on cppos.
\end{example}

\begin{example}
We can define a fixpoint operator
$\fp$ for $\contr$-contractive endomaps
$f:X\rightarrow X$:
$$\fp_f \defin \isom{\factored{f}} \comp \infty_X : 1\rightarrow X$$
Clearly:
$$\fp_f = f \comp \fp_f$$
and
$\fp_f$ is natural in $f$ in the sense that 
for all $\contr$-contractive endomaps 
$f: X\rightarrow X$ and $g:Y\rightarrow Y$
and all commuting squares
$$\xymatrix{
X \ar[d]_{f} \ar[r]^{h} 
     & Y\ar[d]^{g}
\\
X \ar[r]_h &  Y
}$$
we have $\fp_g = h\comp \fp_f$.
Moreover,
$\fp$ is the unique operator verifying the two properties above.
It corresponds to the fixpoint operator given
by Banach's fixpoint theorem.
\end{example}

\begin{remark}
  If $\ti$
is a fixpoint monad for
a pointed endofunctor on a category
$\CC$ with an initial object $0$,
then all equalizers $\infty_X$
are determined by $\infty_0$,
because $\infty_X = \fpc_{\delta_X}$
hence, by uniformity,
$\infty_X$ is the composition of
$\infty_0 = \fpc_{\delta_0}$
with the unique homomorphism from the initial algebra $\delta_0$
to $\delta_X$.
This explains why
in categories with initial objects
one can focus on the \emph{fixpoint object} $\ti 0$
rather than on the whole monad $\ti$.
\end{remark}


In particular, for cppos, $T0$ is given by the ordered natural numbers
together with a top infinity point.  In the metric case, when
$\pointedfunctor$ is $(\Id_\varepsilon ,\id^\varepsilon )$, there is
no algebra $x: \Id_\varepsilon X \rightarrow X$ such that $x\comp
\id_X^\varepsilon = \id_X$.  Thus, instead of arbitrary endomaps
between algebras of the \emph{pointed} endofunctor $(\Id_\varepsilon
,\id^\varepsilon )$, one considers $\varepsilon$-contractive endomaps.
Since $\id^\varepsilon$ is pointwise epi, there is an isomorphism
$f\mapsto \factored{f}$ between the category of
$\varepsilon$-contractive endomaps and the category of
$\Id_\varepsilon$-algebras.

%In general, even when $\contr$ is not pointwise epi
%one can define such a fixpoint operator for
%$\contr$-contractive maps
%\emph{relative} to a functor
%$$\ ^- : \contrcat \rightarrow \alg{\esse}$$
%which enables to select one $\contr$-factor $\factored{f}$
%for each $\contr$-contractive endomap.




\section{Fixpoints and Final Coalgebras}

Let us consider the case when the category $\CC$ has binary coproducts
and the endofunctor $\esse$ \emph{freely generates a monad}
$\ti$, in the sense that
$\ti X \cong X+\esse \ti X$
is an initial algebra for the endofunctor $(X+\esse )$
for every $X$ in $\CC$.


\begin{proposition}
\label{propoabove}
  If, for every $X$ in $\CC$,
$\ti X\cong X+\esse \ti X$ is a 
final $(X+\esse )$-coalgebra
then
$\ti$ is a fixpoint monad
for $\pointedfunctor$
for every point $\sigma :\Id \Rightarrow S$ of $S$.
\end{proposition}

We need the following technical lemma.

\begin{lemma}
Let $G$ and $H$ be two endofunctors on a category $\CC$
connected by a natural transformation $\tau : H\Rightarrow GH$.
Let $a : A\rightarrow GHA$ be a final $GH$-coalgebra with inverse
$a ' :GHA \rightarrow A$.
Then, the algebra $a ' \comp \tau_A : HA \rightarrow A$
is final with respect to homomorphisms from $H$-coalgebras,
ie
for every $H$-coalgebra $y: Y \rightarrow HY$,
there exists a unique $h: Y \rightarrow A$ 
such that the diagram
$$\xymatrix{
Y \ar[r]^-{h}  \ar[d]_-{y} & A
\\
HY \ar[r]_-{Hh} & HA \ar[u]_-{a '\comp \tau_A} 
}$$
commutes.

\emph{Proof.} Simply turn $y : Y \rightarrow HY$
into a $GH$-coalgebra by composing it
with $\tau_Y : HY \rightarrow GHY$
and then use finality.
\close
%Let $A\cong X+HA$ be a final $(X+H)$-coalgebra
%and let $[f,\alpha]: X+HA \rightarrow A$ be its inverse.
%Then, the algebra $\alpha : HA \rightarrow A$
%is final with respect to homomorphisms from $H$-coalgebras,
%ie
%for every $H$-coalgebra $\upsilon : Y \rightarrow HY$,
%there exists a unique $h: Y \rightarrow A$ 
%such that the diagram
%$$\xymatrix{
%Y \ar[r]^-{h}  \ar[d]_-{\upsilon} & A
%\\
%HY \ar[r]_-{Hh} & HA \ar[u]_-{\alpha} 
%}$$
%commutes.

%\emph{Proof.} Simply turn $\upsilon : Y \rightarrow HY$
%into an $(X+H)$-coalgebra by composing it
%with the right injection $HY \rightarrow X+HY$
%and then use finality.
\end{lemma}

\emph{Proof of Proposition \ref{propoabove}.}
Define $\infty_X$ using the finality property of $\delta_X$
given by the above lemma for $\tau =\inr : \esse \Rightarrow X+ \esse$:
$$\xymatrix{
1 \ar@{-->}[r]^-{\infty_X}  \ar[d]_{\contr_1} & TX
\\
\esse 1 \ar[r]_-{\esse (\infty_X )} & {\esse TX} \ar[u]_{\delta_X} 
}$$
Then, by the naturality of $\contr$:
$$\delta_X \comp \contr_{TX} \comp \infty_X 
  = \delta_X \comp \esse (\infty_X) \comp \contr_1
  = \infty_X
$$
Further, that $\infty$ is an equalizer can 
be proved as follows.
Consider a map $f: A \rightarrow TX$ such that
\begin{equation}
  \label{eq:fequa}
  f = \delta_X \comp \contr_{TX} \comp f
\end{equation}
The claim is that $f$ factors through $\infty_X : 1\rightarrow TX$.
By the finality property of $\delta_X$ and the definition of $\infty_X$,
it suffices to show that the following two diagrams commute:
$$\xymatrix{
A \ar[r]^-{f}  \ar[d]_{\contr_A} & TX
\\
\esse A \ar[r]_-{\esse f} & {\esse TX} \ar[u]_{\delta_X} 
}
\HSP
\xymatrix{
A \ar[r]^-{!}  \ar[d]_{\contr_A} & 1 \ar[d]^{\contr_1}
\\
\esse A \ar[r]_-{\esse !} & {\esse 1} 
}
$$
But the commutativity of the second is trivial
and the commutativity of the first follows from
the naturality of $\contr$ and the assumption (\ref{eq:fequa}).
\close


\subsection*{Algebraic Compactness}

In the rest of this section we assume
familiarity with the results in \cite{Kock}.
Thus recall that a monoidal pointed endofunctor $\pointedfunctor$
on a monoidal category $\CC = \tuple{\CC , \otimes , I}$
is equipped with
a natural transformation
$$\monoi_{X,Y} : X\otimes Y \rightarrow \esse (X\otimes Y)$$
which gives rise to a tensorial strength
$\strength_{X,Y} : X\otimes \esse Y \rightarrow \esse (X\otimes Y)$
$$\xymatrix{
X\otimes \esse Y \ar[r]^{\contr \otimes \id}  \ar[rd]_-{\strength}  
  & \esse X\otimes \esse Y \ar[d]^{\monoi}  
\\
& \esse (X\otimes Y)
}$$
Also recall that in a (\emph{monoidal}) \emph{closed category}, 
ie a category where $\fstarg \otimes X$ has
a right adjoint $\lolly{X}{\fstarg}$,
there is a 1-1 correspondence
between tensorial strengths and ordinary strengths;
the latter internalize the action of endofunctors on morphisms:
$$\stre_{X,Y} : \lolly{X}{Y}\rightarrow \lolly{\esse X}{\esse Y}$$
(see \cite[Thm 1.3]{Kock}).


\begin{lemma}
\label{exp}
If $\pointedfunctor$ 
is a monoidal pointed endofunctor
on a closed category $\CC$,
then 
$\stre_{X,A} : \lolly{X}{A}\rightarrow \lolly{\esse X}{\esse A}$
is weakly $\contr$-contractive for all $X$ and $A$ in $\CC$.

\emph{Proof.}
By definition, 
$\stre_{X,A} : \lolly{X}{A}\rightarrow \lolly{\esse X}{\esse A}$
is obtained by transposing the map
$S\varepsilon \comp \monoi \comp (\contr \otimes \id )$,
therefore it is the transposition of $S\varepsilon \comp \monoi$
precomposed with $\contr$, which gives the desired factorization.
\close
\end{lemma}

\begin{definition}
Let $H$ be a monoidal endofunctor on a closed 
category $\CC$.
We say that $H$ is \emph{locally $\contr$-contractive}
if its strength is $\contr$-contractive.
\end{definition}

\begin{theorem}
Let $\pointedfunctor$ be a monoidal pointed endofunctor
on a closed category $\CC = (\CC ,\otimes, I)$
generating a fixpoint monad.
Then every locally $\contr$-contractive
endofunctor $H$ on $\CC$ is algebraically compact,
ie if $\alpha :HA\cong A$ is an initial $H$-algebra then
its inverse $\inial :A \cong HA$ is a
final $H$-coalgebra.

\emph{Proof.}
We have to prove that for every $g: X\rightarrow HX$
there exists a unique $h: X\rightarrow A$ making the diagram
$$\xymatrix{
X \ar[r]^-h  \ar[d]_g & A  \ar@/_/[d]_-{\inial}
\\
HX \ar[r]_-{Hh} 
   & HA \ar@/_/[u]_-{\alpha} 
}
$$
commute.
By monoidal closure,
the mapping $h \mapsto \alpha \comp Hh \comp g$
corresponds to a map
$$\Phi_g : \lolly{X}{A} \rightarrow \lolly{X}{A}$$
which is $\contr$-contracting because it factors
through the (by hypothesis) $\contr$-contractive map 
$\stre_{X,A} : \lolly{X}{A} \rightarrow \lolly{HX}{HA}$
as follows:
$$\xymatrix{
\lolly{X}{A} \ar[r]^-{\stre_{X,A}}
 & \lolly{HX}{HA} \ar[r]^-{\lolly{g}{\alpha}}
 & \lolly{X}{A}
}
$$
By, again, monoidal closure,
$\fp_{\Phi_g}$ is the name of a coalgebra map
from $A$ to $\ti X$.
This proves the existence part of the theorem.

Next, uniqueness.
Note that, by the initiality of $A$,
the map $\Phi_{\inial}$ can have only one fixpoint,
namely (the name of) the identity on $\lolly{A}{A}$.
Therefore, by uniformity, 
$\Phi_g$ is completely determined by
a (necessarily unique) homomorphism
from
$\Phi_{\inial}$ to $\Phi_g$.
But, as the following diagram shows, 
such a homomorphism is given by
$\lolly{h}{A}$,
for any coalgebra
homomorphism $h: X \rightarrow A$.
$$\xymatrix{
\lolly{A}{A} \ar[r]^-{\stre_{A,A}} \ar[d]_-{\lolly{h}{A}}
 & \lolly{HA}{HA} \ar[r]^-{\lolly{\alpha '}{\alpha}} \ar[d]_-{\lolly{Hh}{HA}}
 & \lolly{A}{A} \ar[d]^-{\lolly{h}{A}}
\\
\lolly{X}{A} \ar[r]_-{\stre_{X,A}}
 & \lolly{HX}{HA} \ar[r]_-{\lolly{g}{\alpha}}
 & \lolly{X}{A}
}
$$
The first square commutes by naturality and the second
by the fact that $h$ is a coalgebra homomorphism.
\close
\end{theorem}

\begin{remark}
[Note that this theorem does not apply to cppo...Strictness..]
But maybe the idea could be elaborated to cover this case.
\end{remark}


By taking $H=X+S$ we have:
\begin{corollary}
If $\ti$ is a fixpoint monad freely generated by
a monoidal pointed endofunctor
$\pointedfunctor$ on a closed category $\CC$,
then, for every $X$ in $\CC$,
$\ti X\cong X+\esse \ti X$ is a 
final $(X+\esse )$-coalgebra.
\end{corollary}

\begin{example}
  ???
\end{example}

\paragraph{Future work.}
The main question we would like to address next is whether
it is possible to glue the various down-scalers together
and obtain an endofunctor whose algebras are
contractive maps, irrespective of their contraction
factor. 



\paragraph{Acknowledgements.} 
We gratefully acknowledge discussions with Alex Simpson.

\bibliographystyle{alpha}
\bibliography{fixpoints}




\end{document}



\begin{theorem}
Let $\pointedfunctor$ be an affine monoidal pointed endofunctor
on an affine monoidal closed category $\CC = (\CC ,\otimes, 1)$
generating a fixpoint monad
and let $H$ be an endofunctor on $\CC$
such that either $H\cong KS$ or $H\cong SK$
for some strong endofunctor $K$.

If $\alpha :HA\cong A$ is an initial $H$-algebra then
its inverse $\inial :A \cong HA$ is a
final $H$-coalgebra.

\emph{Proof.}
We have to prove that for every $g: X\rightarrow HX$
there exists a unique $h: X\rightarrow A$ making the diagram
$$\xymatrix{
X \ar[r]^-h  \ar[d]_g & A  \ar@/_/[d]_-{\inial}
\\
HX \ar[r]_-{Hh} 
   & HA \ar@/_/[u]_-{\alpha} 
}
$$
%$$\xymatrix{
%A \ar[r]^-h  \ar[d]_g & \ti X
%\\
%X+ \esse A \ar[r]_-{X+ \esse h} 
%   & X+ \esse \ti X \ar[u]_{[\eta_X ,\delta_X ]} 
%}
%$$
commute.
By monoidal closure,
the mapping $h \mapsto \alpha \comp Hh \comp g$
induces a map
$$\Phi_g : \lolly{X}{A} \rightarrow \lolly{X}{A}$$
If it is $\contr$-contracting then 
clearly we can prove the existence part of the theorem.
In order to prove $\Phi_g$ $\contr$-contracting
let us make its definition explicit as follows,
where we have assumed that $H\cong KS$
and where $\stre^\esse$ and $\stre^K$ are the strengths
of $\esse$ and $K$ respectively.
$$\xymatrix{
\lolly{X}{A} \ar[r]^-{\stre^\esse}
 & \lolly{\esse X}{\esse A}\ar[r]^-{\stre^K}
 & \lolly {K\esse X}{K\esse A} \ar[r]^-{\cong}
 & \lolly{HX}{HA} \ar[r]^-{\lolly{HX}{\alpha}}
 & \lolly{HX}{A} \ar[r]^-{\lolly{g}{A}}
 & \lolly{X}{A}
}
$$
%$$\xymatrix{
%\ti X^A \ar[r] 
% & \esseti X^{\esse A} \ar[r]^-{\delta_X^{\esse A}}
% & \ti X^{\esse A} \; \cong \; 1\times \ti X^{\esse A}
%   \ar[r]^-{\nameof{\eta_X} \times \id } % \ti X^{\esse A} }
% & \ti X^X \times \ti X^{\esse A} \; \cong \;
%   \ti X^{X+ \esse A} \ar[r]^-{\ti X^g}
% & \ti X^A
%}
%$$
For every coalgebra $g: X \rightarrow HX$,
weak $\contr$-contractivity is then immediate,
because $\Phi_g$ factors
through the map 
$\stre : \lolly{X}{A} \rightarrow \lolly{\esse X}{\esse A}$
which,
by Lemma \ref{exp}, 
is weakly $\contr$-contractive.
Because $\CC$ is affine, ie the unit of the tensor is the terminal object,
$\fp (\Phi_g )$ is the name of a coalgebra map
from $A$ to $\ti X$.
The same holds if $H\cong SK$.

Next, uniqueness.
We consider the case $H\cong KS$, but 
the same holds for $H\cong SK$.
Note that, by the initiality of $A$,
the map $\Phi_{\inial}$ can have only one fixpoint,
namely (the name of) the identity on $\lolly{A}{A}$.
Therefore, by uniformity, 
$\Phi_g$ is completely determined by
a (necessarily unique) homomorphism
from
$\Phi_{\inial}$ to $\Phi_g$.
But, as the following diagram shows, 
such a homomorphism is given by
$\lolly{h}{A}$,
for any coalgebra
homomorphism $h: X \rightarrow A$.
$$\xymatrix{
\lolly{A}{A} \ar[r]^-{\stre^\esse} \ar[d]_-{\lolly{h}{A}}
 & \lolly{\esse A}{\esse A}\ar[r]^-{\stre^K} \ar[d]_-{\lolly{\esse h}{\esse A}}
 & \lolly {K\esse A}{K\esse A} \ar[r]^-{\cong}
   \ar[d]_-{\lolly{K\esse h}{K\esse A}}
 & \lolly{HA}{HA} \ar[r]^-{\lolly{HA}{\alpha}} \ar[d]_-{\lolly{Hh}{HA}}
 & \lolly{HA}{A} \ar[r]^-{\lolly{\alpha '}{A}} \ar[d]_-{\lolly{Hh}{A}}
 & \lolly{A}{A} \ar[d]^-{\lolly{h}{A}}
\\
\lolly{X}{A} \ar[r]_-{\stre^\esse}
 & \lolly{\esse X}{\esse A}\ar[r]_-{\stre^K}
 & \lolly {K\esse X}{K\esse A} \ar[r]_-{\cong}
 & \lolly{HX}{HA} \ar[r]_-{\lolly{HX}{\alpha}}
 & \lolly{HX}{A} \ar[r]_-{\lolly{g}{A}}
 & \lolly{X}{A}
}
$$
All squares commute by naturality, but for the last,
which commutes by the fact that $h$ is a coalgebra homomorphism.
%(For the second square, note that $\delta_X^C$ is natural in $C$
%because
%$f^{( \midarg)}: A^{( \midarg)} \rightarrow B^{( \midarg)}$ 
%is natural for any $f:A\rightarrow B$.)
\close
\end{theorem}
(Cf Theorem 7.3 in \cite{TR98}.)

By taking $H=X+S$ we have:
\begin{corollary}
If $\ti$ is a fixpoint monad freely generated by
an affine monoidal pointed endofunctor
$\pointedfunctor$ on an affine monoidal closed category $\CC$,
then, for every $X$ in $\CC$,
$\ti X\cong X+\esse \ti X$ is a 
final $(X+\esse )$-coalgebra.
\end{corollary}
Note that we do not need to assume that
$\infty$ is an equalizer.
%Also note that
%in the cpo case we need the extra assumption that $\esse$
%is a commutative monad in order to ensure that $\ti X^h$
%is strict, which fact we do not need here.

\section{Rest}

\begin{proposition}
  If $F:X\rightarrow Y$
is (weakly) $\contr$-contractive
for a strong pointed endofunctor $\pointedfunctor$
then $\lolly{A}{f} : \lolly{A}{X} \rightarrow \lolly{A}{Y}$
is (weakly) $\contr$-contractive as well.
\end{proposition}

The proof is a consequence of the following lemma.

\begin{lemma}
  If $\pointedfunctor$ is a strong pointed endofunctor
on a monoidal closed category $\CC$,
then $\lolly{A}{\contr_X } : \lolly{A}{X} \rightarrow \lolly{A}{SX}$
is weakly $\contr$-contractive for all $X$ and $A$ in $\CC$.

\emph{Proof.}
First consider the following commutative diagram.
$$Bla$$
Then transpose the two composites in the outer diagram
$$bla$$
and note that this gives the desired factorization.
\close
\end{lemma}

\nocite{fpobject,Fre90,escardo:metric,alex}


\paragraph{Acknowledgements.} 
Alex Simpson.

\bibliographystyle{alpha}
\bibliography{metric}




\end{document}




\begin{theorem} \textbf{[Cpo case]}
If $\ti$ is
a fixpoint monad for a pointed endofunctor $(\esse , \contr )$,
then, for every endomap $f:X\rightarrow X$
on a $(\esse , \contr )$-algebra $\tuple{X,\xi }$
the point
$$\fp_f \defin \isom{f\comp \xi} \comp \infty_X : 1\rightarrow X$$
has the following two properties:
\begin{enumerate}
\item (\emph{fixpoint})
$\fp_f$ is a fixpoint for $f$,
ie $\fp_f = f\comp \fp_f$:
$$\xymatrix{
1 \ar[d]_-{\fp_f}  \ar[rd]^-{\fp_f} & 
\\
X \ar[r]_-f & X
}$$
\item (\emph{uniformity}) $\fp_f$ is natural in $f$ in the sense that 
for all endomaps 
$f: X\rightarrow X$ and $g:Y\rightarrow Y$
on $(\esse , \contr )$-algebras $\tuple{X,\xi }$ and $\tuple{Y,\upsilon }$
and all commuting diagrams
$$\xymatrix{
\esse X \ar[d]_{\xi} \ar[r]^{\esse h} 
     & \esse Y\ar[d]^{\upsilon}
\\
X \ar[d]_{f} 
     & Y\ar[d]^{g}
\\
X \ar[r]_h &  Y
}$$
we have $\fp_g = h\comp \fp_f$:
$$\xymatrix{
1 \ar[d]_-{\fp_f}  \ar[rd]^-{\fp_g} & 
\\
X \ar[r]_-h & Y
}$$
Moreover:
\item (\emph{uniqueness})
$\fp$ is the unique operator verifying the two properties above.
\end{enumerate}

\emph{Proof.}
1)
$$\xymatrix{
1\ar[r]^-{\infty_X} \ar[rdd]_-{\infty_X}
   & TX \ar[r]^{\isom{f}} \ar[d]^{\contr_{TX}} 
   & X \ar[d]_{\contr_X} %\ar@/^/[dd]
\\
   & \esse TX \ar[r]^{\esse \isom{f}} \ar[d]^{\delta_X}
   & \esse X \ar[d]_{\factored{f}}
\\
   & TX \ar[r]_{\isom{f}} 
   & X
}$$
2)
By naturality of $\infty$ and $\comparing$:
$$\fp_g = \isom{g\comp \upsilon} \comp \infty_Y = 
\isom{g\comp \upsilon}\comp Th \comp \infty_X 
  = h \comp \isom{f\comp \xi} \comp \infty_X = h \comp \fp_f$$
3)
By the equalizing property of $\infty$,
we have that any other fixpoint operator $\fpp$
has the property that $\fpp (\delta_X \comp \contr_{TX} ) = \infty_X$.
Then, if $\fpp$ satisfies the second property,
$$\fpp f = \isom{f\comp \xi} \comp \fpp (\delta_X \comp \contr_{TX} ) 
= \isom{f\comp \xi} \comp \infty_X = \fp f$$
because $\isom{f\comp \xi}$ is an $\esse$-algebra homomorphism
between $\delta_X$ and $f\comp \xi$.
\close
\end{theorem}



\begin{theorem}\textbf{[Metric case]}
If $\ti$ is
a fixpoint monad for a pointed endofunctor $(\esse , \contr )$,
then, for every $\contr$-contracting endomap $f:X\rightarrow X$,
the point
$$\fp_f \defin \isom{f} \comp \infty_X : 1\rightarrow X$$
has the following three properties:
\begin{enumerate}
\item (\emph{fixpoint})
$\fp_f$ is a fixpoint for $f$,
ie $\fp_f = f\comp \fp_f$:
$$\xymatrix{
1 \ar[d]_-{\fp_f}  \ar[rd]^-{\fp_f} & 
\\
X \ar[r]_-f & X
}$$
\item (\emph{uniformity}) $\fp_f$ is natural in $f$ in the sense that 
for all $\contr$-contractive endomaps 
$f: X\rightarrow X$ and $g:Y\rightarrow Y$
and all commuting squares
$$\xymatrix{
X \ar[d]_{f} \ar[r]^{h} 
     & Y\ar[d]^{g}
\\
X \ar[r]_h &  Y
}$$
we have $\fp_g = h\comp \fp_f$:
$$\xymatrix{
1 \ar[d]_-{\fp_f}  \ar[rd]^-{\fp_g} & 
\\
X \ar[r]_-h & Y
}$$
\item (\emph{uniqueness})
$\fp$ is the unique operator verifying the two properties above.
\end{enumerate}

\emph{Proof.}
1)
$$\xymatrix{
1\ar[r]^-{\infty_X} \ar[rdd]_-{\infty_X}
   & TX \ar[r]^{\isom{f}} \ar[d]^{\contr_{TX}} 
   & X \ar[d]_{\contr_X} %\ar@/^/[dd]
\\
   & \esse TX \ar[r]^{\esse \isom{f}} \ar[d]^{\delta_X}
   & \esse X \ar[d]_{\factored{f}}
\\
   & TX \ar[r]_{\isom{f}} 
   & X
}$$
2)
By naturality of $\infty$ and $\comparing$:
$$\fp_g = \isom{g}\comp \infty_Y = \isom{g}\comp Th \comp \infty_X 
  = h \comp \isom{f} \comp \infty_X = h \comp \fp_f$$
3)
By the equalizing property of $\infty$,
we have that any other fixpoint operator $\fpp$
has the property that $\fpp (\delta_X \comp \contr_{TX} ) = \infty_X$.
Then, if $\fpp$ satisfies the second property,
$$\fpp f = \isom{f} \comp \fpp (\delta_X \comp \contr_{TX} ) 
= \isom{f} \comp \infty_X = \fp f$$
because $\isom{f}$ is an $\esse$-algebra homomorphism
between $\delta_X$ and $\factored{f}$.
\close
\end{theorem}

\begin{remark}
  If $\ti$
is a fixpoint monad for
a pointed endofunctor on a category
$\CC$ with an initial object,
then all equalizers $\infty_X$
are determined by $\infty_0$,
because, by uniformity,
$\infty_X = \delta_X^\recu \comp \infty_0$.
One then can focus on the \emph{fixpoint object} $\ti 0$
for the pointed endofunctor.
\close
\end{remark}


\begin{proposition}
  If, for every $X$, there exists a unique fixpoint 
$\infty_X : 1 \rightarrow TX$ for $\delta_X \comp \contr_{TX}$,
then $\s{\infty_X}_{X\in \CC}$ is the unique
family of points such that 
$h\comp \infty_X = \infty_Y$
for every $\esse$-algebra homomorphism $h: TX \rightarrow TY$.
The converse also holds under the assumption that
$(\esse , \contr )$ is well-pointed,
ie $\esse \contr = \contr_\esse$.

\emph{Proof.}
$$\xymatrix{
1 \ar[d]_-{\infty_X}  \ar[rrd]^{\infty_X} & &
\\
TX \ar[d]_h \ar[r]_{\contr_{TX}} & 
{\esse TX} \ar[d]^{\esse h} \ar[r]_{\delta_X} & TX \ar[d]^h
\\
TY \ar[r]_{\contr_{TY}} & 
{\esse TY} \ar[r]_{\delta_Y} & TY
}$$
hence $h\comp \infty_X = \infty_Y$ by the uniqueness of
$\infty_Y$ for $\delta_Y \comp \contr_{TY}$.

For the converse use well-pointedness as follows.
$$\xymatrix{
1 \ar[d]_-{\infty_X}  \ar[rrd]^{\infty_X} & &
\\
TX  \ar[r]_{\contr_{TX}} & 
{\esse TX} \ar[r]_{\delta_X} & TX 
\\
{\esse TX} \ar[u]^{\delta_X} \ar@/^/[r]^{\contr_{\esse TX}} 
                                 \ar@/_/[r]_{\esse \contr_{TX}} & 
{\esse^2 TX} \ar[u]_{\esse \delta_X} \ar[r]_{\esse \delta_X} &
{\esse TX} \ar[u]_{\delta_X} 
}$$
\close
\end{proposition}