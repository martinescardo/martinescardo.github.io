\documentclass[%dvips,% for generating postscript printouts
%25pt,%
a4paper,%
landscape]{foils}

\usepackage{graphicx}

\newcommand{\hash}{\operatorname{\sharp}}

\usepackage{diagrams}
%\usepackage{diagrams}
\newarrow{Into} C--->
\newarrow{Onto} ----{>>}
\newarrow{Mapsto} |--->
\newarrow{Dashto} {}{dash}{}{dash}>

\newcommand{\bnf}{\mathrel{::=}}
\newcommand{\lift}[1]{#1_{\bot}}
\newcommand{\N}{\mathbb{N}}
\newcommand{\R}{\mathbb{R}}
\newcommand{\Bool}{\mathbb{B}}
\newcommand{\Sierp}{\mathcal{S}}
\newcommand{\pN}{\mathcal{N}}
\newcommand{\pBool}{\mathcal{B}}
\newcommand{\If}{\,\mathrel{\operatorname{if}}}
\newcommand{\Then}{\mathrel{\operatorname{then}}}
\newcommand{\Else}{\mathrel{\operatorname{else}}}
\newcommand{\True}{1}%{\operatorname{tt}}%{\operatorname{true}}
\newcommand{\False}{0}%{\operatorname{ff}}%{\operatorname{false}}
\newcommand{\domain}[1]{{\D_{#1}}}
\newcommand{\total}[1]{{\T_{#1}}}
\newcommand{\totaleq}[1]{\sim_{#1}}
\newcommand{\quo}[1]{\qq_{#1}}
\newcommand{\qq}{\rho}
\newcommand{\D}{D}
\newcommand{\E}{E}
\newcommand{\C}{C}
\newcommand{\T}{T}
\newcommand{\kk}[1]{{\C_{#1}}}
\newcommand{\Scott}{\operatorname{Scott}}
\newcommand{\tproduct}{\sigma \times \tau}
\newcommand{\tfunction}{\sigma \to \tau}
\newcommand{\rh}[1]{\rho_{#1}}
\newcommand{\comp}{\circ}
\newcommand{\siff}{\iff}%{\Leftrightarrow}
\newcommand{\Search}{\operatorname{\mathcal{S}}}
\newcommand{\g}{\operatorname{\text{\boldmath{$g$}}}}
\newcommand{\f}{\operatorname{\text{\boldmath{$f$}}}}
\newcommand{\x}{\operatorname{\text{\boldmath{$x$}}}}
\newcommand{\e}{\operatorname{\text{\boldmath{$e$}}}}
\newcommand{\sK}{\operatorname{\text{\boldmath{$K$}}}}
\newcommand{\overhead}[1]{\foilhead{\darkblue{#1}} \vspace{-4ex}}
\newcommand{\cons}{\hash}

\newcommand{\Bool}{\operatorname{Bool}}
\newcommand{\prodf}{\operatorname{\darkblue{prod}}}

\newcommand{\hash}{\operatorname{\sharp}}

\newcommand{\bnf}{\mathrel{::=}}
\newcommand{\lift}[1]{#1_{\bot}}
\newcommand{\N}{\mathbb{N}}
\newcommand{\R}{\mathbb{R}}
%\newcommand{\Bool}{\mathbb{B}}
\newcommand{\Sierp}{\mathcal{S}}
\newcommand{\pN}{\mathcal{N}}
\newcommand{\pBool}{\mathcal{B}}
\newcommand{\If}{\,\mathrel{\operatorname{if}}}
\newcommand{\Then}{\mathrel{\operatorname{then}}}
\newcommand{\Else}{\mathrel{\operatorname{else}}}
\newcommand{\True}{\operatorname{True}}%{\operatorname{true}}
\newcommand{\False}{\operatorname{False}}%{\operatorname{false}}
\newcommand{\domain}[1]{{\D_{#1}}}
\newcommand{\total}[1]{{\T_{#1}}}
\newcommand{\totaleq}[1]{\sim_{#1}}
\newcommand{\quo}[1]{\qq_{#1}}
\newcommand{\qq}{\rho}
\newcommand{\D}{D}
\newcommand{\E}{E}
\newcommand{\C}{C}
\newcommand{\T}{T}
\newcommand{\kk}[1]{{\C_{#1}}}
\newcommand{\Scott}{\operatorname{Scott}}
\newcommand{\tproduct}{\sigma \times \tau}
\newcommand{\tfunction}{\sigma \to \tau}
\newcommand{\rh}[1]{\rho_{#1}}
\newcommand{\comp}{\circ}
\newcommand{\siff}{\iff}%{\Leftrightarrow}
\newcommand{\Search}{\operatorname{\mathcal{S}}}
\newcommand{\g}{\operatorname{\text{\boldmath{$g$}}}}
\newcommand{\f}{\operatorname{\text{\boldmath{$f$}}}}
\newcommand{\x}{\operatorname{\text{\boldmath{$x$}}}}
\newcommand{\e}{\operatorname{\text{\boldmath{$e$}}}}
\newcommand{\sK}{\operatorname{\text{\boldmath{$K$}}}}
\newcommand{\overhead}[1]{\foilhead[-4ex]{\darkblue{#1}}}
\newcommand{\cons}{\hash}

%\usepackage[notref,notcite]{showkeys}
%\usepackage[light,none,bottom]{draftcopy}

% \now command (By Manfred Kerber and then Riccardo Poli)

\usepackage{calc}
\newcounter{hours}\newcounter{minutes}
\newcommand{\now}{ %
\setcounter{hours}{\time/60} %
\setcounter{minutes}{\time-\value{hours}*60} %
\ifnum\theminutes <10 \thehours :0\theminutes\else \thehours:\theminutes\fi}

\newcommand{\myalpha}{s}
\newcommand{\mybeta}{t}

% sample usage: \date{\today \now\ hours}

%\usepackage{a4wide}

%\usepackage{a4}

%\usepackage{times}

\usepackage{url}

\newcommand{\qqquad}{\qquad\qquad}

\usepackage[english]{babel}

\usepackage{latexsym,amssymb,amsmath,stmaryrd,color}

\newcommand{\myparagraph}[1]{\vspace{2ex}\noindent{\bf #1}}

% mathematical macros

\newcommand{\Top}{\operatorname{Top}}
\newcommand{\tinduced}[3]{{#1}({#2},{#3})} % subbasic in the induced topology
%\newcommand{\E}{\varepsilon}
\newcommand{\eval}{\operatorname{\varepsilon}}%{\operatorname{eval}}
\newcommand{\curry}[1]{\overline{\scriptstyle #1}}
\newcommand{\Curry}[1]{\overline{#1}}
%\newcommand{\N}{\mathbb{N}}
\newcommand{\K}{\mathbb{K}}
\newcommand{\NN}{\mathcal{N}}
\newcommand{\Z}{\mathbb{Z}}
%\newcommand{\R}{\mathbb{R}}
\newcommand{\RR}{\mathcal{R}}
\newcommand{\Q}{\mathbb{Q}}
\newcommand{\B}{2}%{\mathbb{B}}
\newcommand{\Nat}{\operatorname{Nat}}
%\newcommand{\Bool}{\operatorname{Bool}}
\newcommand{\dual}[1]{{#1}^\wedge} % furst Lawson dual
\newcommand{\dualdual}[1]{{#1}^{\wedge\wedge}} % second Lawson dual
%\newcommand{\e}{e} % canonical map into second Lawson dual
\renewcommand{\d}{d} % its inverse when it exists.
\newcommand{\onlineremark}[1]{{\footnotesize(#1)}}
\newcommand{\old}[1]{}

\newcommand{\U}{{\cal U}}
\newcommand{\V}{{\cal V}}
\newcommand{\assembly}{\operatorname{N}}
\newcommand{\preassembly}{\operatorname{P}}
\newcommand{\Vstar}{\V^*}
\newcommand{\op}{\rm op}
\newcommand{\powerset}{\operatorname{\mathcal{P}}}
\newcommand{\pneg}{\operatorname{\underline{\neg}}}
\newcommand{\phull}[1]{\underline{#1}}
\newcommand{\wellinside}{\eqslantless}
\newcommand{\waybelow}{\ll}
\newcommand{\eqdef}{\stackrel{{\rm\scriptscriptstyle  def}}{=}}
\newcommand{\himplies}{\Rightarrow}
\newcommand{\functorfont}[1]{\operatorname{#1}}
\newcommand{\catfont}[1]{\mathbf{#1}}
\newcommand{\Loc}{\catfont{Loc}}
\newcommand{\Frm}{\catfont{Frm}}
\newcommand{\Set}{\catfont{Set}}
%\newcommand{\K}{K} % was{\operatorname{K}} % ``kelleyification''
%\newcommand{\Sierp}{\mathbb{S}}
\newcommand{\Stone}{\catfont{Stone}}
\newcommand{\Coh}{\catfont{Spec}}
\newcommand{\CReg}{\catfont{KReg}}
\newcommand{\RLC}{\catfont{RLK}}
\newcommand{\SC}{\catfont{SK}}
\newcommand{\SLC}{\catfont{SLK}}
\newcommand{\BReg}{\catfont{BReg}}

%\newcommand{\clsb}[2]{#2_{#1}}
\newcommand{\clsb}[2]{#1\setminus #2}
\newcommand{\open}{\circ}
\newcommand{\closed}{\boxempty}
\newcommand{\cocompact}[1]{\kappa_{#1}}
\newcommand{\Patch}{\operatorname{Patch}}
\newcommand{\Opens}{\operatorname{\mathcal{O}}}%[1]{\operatorname{\mathcal{O}}{#1}} 
\renewcommand{\O}{\Opens}
\newcommand{\Cocompacts}{\operatorname{\mathcal{C}}}
%\newcommand{\C}[2]{\operatorname{C}(#1,#2)}
\newcommand{\Regions}{\operatorname{\mathcal{R}}}%[1]{\operatorname{\mathcal{R}}{#1}} 
\newcommand{\frameleft}[1]{#1^{*}} \newcommand{\frameright}[1]{#1_{*}}
\newcommand{\arrow}{\to}%{\rightarrow}
%\newcommand{\comp}{\mathrel{\circ}}
\newcommand{\id}{\operatorname{id}}
\newcommand{\join}{\sqcup}
\newcommand{\meet}{\sqcap}
\newcommand{\himp}{\mathrel{\Rightarrow}}
\newcommand{\bigjoin}{\bigsqcup}
\newcommand{\dirjoin}{\sideset{}{^{\uparrow}}\bigjoin}
\newcommand{\bigmeet}{\bigsqcap}
\newcommand{\below}{\sqsubseteq}
\renewcommand{\above}{\sqsupseteq}
\newcommand{\farrow}[1]{\stackrel{#1}{\longrightarrow}}
\newenvironment{eqn}{\[\begin{array}{rcll}}{\end{array}\]}
\newcommand{\qtext}[1]{\quad\text{#1}}
\newcommand{\up}{\operatorname{\uparrow}}
\newcommand{\down}{\operatorname{\downarrow}}
\newcommand{\upup}{\mathord{\hbox{\makebox[0pt][l]{\raise .6mm\hbox{$\uparrow$}}$\uparrow$}}}
\newcommand{\fix}{\operatorname{fix}}
\newcommand{\Infl}[1]{\hat{#1}}
\newcommand{\infl}{\Infl{\top}} %{\Infl{\operatorname{\i d}}}


% defined terms and index entries

\newcommand{\fontfordefinition}[1]{\emph{#1}}%{{\bfseries\itshape #1}}
% \newcommand{\df}[1]{\fontfordefinition{#1}} % no index required
\newcommand{\dfkey}[2]{\fontfordefinition{#1}\index{#2}}
\newcommand{\dfemph}[1]{\emph{#1}\index{#1}}
\newcommand{\cindex}[1]{#1} % was {#1\index{#1}}
\newcommand{\tindex}[1]{#1\index{#1}}
\newcommand{\df}[1]{\dfkey{#1}{#1}}
\newcommand{\dfn}[1]{\fontfordefinition{#1}}
\newcommand{\ind}[1]{#1\index{#1}}
\newcommand{\indemph}[1]{\emph{#1}\index{#1}}

\makeindex

% Omission macro

\newcommand{\Omit}[1]{}

% Chronology macro

\newcommand{\chron}[1]{}
                     %{\begin{footnotesize}(Chronology: #1\end{footnotesize})}

\newcommand{\smalltext}[1]{\begin{footnotesize} #1 \end{footnotesize}}

%\newcommand{\myparagraph}[1]{\medskip\emph{#1}}

\newcommand{\lbracket}{(}
\newcommand{\rbracket}{)}
\newcommand{\QTop}{\operatorname{{QTop}}}

\newcommand{\numbered}[1]{\grey{#1}}

\definecolor{darkgreen}{rgb}{0,0.5,0}
\definecolor{darkblue}{rgb}{0,0.0,0.5}
\definecolor{darkred}{rgb}{1,0,0}
\definecolor{grey}{rgb}{0.4,0.4,0.4}
\newcommand{\black}{\textcolor{black}}
\newcommand{\grey}{\textcolor{grey}}
\newcommand{\blue}{\textcolor{blue}}
\newcommand{\red}{\textcolor{red}}
\newcommand{\darkblue}{\textcolor{darkblue}}
\newcommand{\darkred}{\textcolor{darkred}}
\newcommand{\darkgreen}{\textcolor{darkgreen}}
\newcommand{\green}{\textcolor{green}}
\newcommand{\yellow}{\textcolor{yellow}}
\newcommand{\cyan}{\textcolor{cyan}}
\newcommand{\magenta}{\textcolor{magenta}}

\newcommand{\qed}{}%{$\boxempty$}

\newcommand{\sexample}[1]{\noindent\textbf{Example} \blue{#1} }
\newcommand{\slemma}[1]{\noindent\darkgreen{\textbf{Lemma}} \\[1ex] {\emtheorem #1}}
\newcommand{\ulemma}[1]{\noindent\darkgreen{\textbf{Lemma}} {\emtheorem #1}}
\newcommand{\proposition}[1]{\noindent\darkgreen{\textbf{Proposition}} {\emtheorem #1}}
\newcommand{\proof}[1]{\noindent{\textbf{Proof}} {#1}}
\newcommand{\sdefinition}[1]{\noindent\darkgreen{\textbf{Definition}} \\[1ex] {\emtheorem #1}}
\newcommand{\scorollary}[1]{\noindent\darkgreen{\textbf{Corollary}} {\emtheorem #1}}
\newcommand{\stheorem}[1]{\noindent\darkgreen{\textbf{Theorem}}  {\emtheorem #1}}
\newcommand{\texample}[2]{\noindent\textbf{Example} \blue{#1}}
\newcommand{\tlemma}[2]{\noindent\darkgreen{\textbf{Lemma} (#1)}  \\[1ex] {\emtheorem #2}}
\newcommand{\ttlemma}[2]{\noindent\darkgreen{\textbf{Lemma} (#1)} {\emtheorem #2}}
\newcommand{\utlemma}[2]{\noindent\darkgreen{\textbf{Lemma} (#1)} {\emtheorem #2}}
\newcommand{\tdefinition}[2]{\noindent\darkgreen{\textbf{Definition} (#1)} \\[1ex] {\emtheorem #2}}
\newcommand{\tcorollary}[2]{\noindent\darkgreen{\textbf{Corollary} (#1)}  \\[1ex] {\emtheorem #2}}
\newcommand{\ttheorem}[2]{\noindent\darkgreen{\textbf{Theorem} (#1)}  \\[1ex] {\emtheorem #2}}

\newcommand{\utheorem}[2]{\noindent\darkgreen{\textbf{Theorem} (#1)} {\emtheorem #2}}

\newcommand{\uapplication}[1]{\noindent\darkgreen{\textbf{Application}} {#1}}

\newcommand{\emtheorem}{\em}

\LogoOff

\newcommand{\twostar}{2^{*}}
\newcommand{\twoomega}{2^\infty}
\newcommand{\prefix}{\sqsubseteq}
\newcommand{\all}[3]{\forall\, #1 \,{\in}\, #2\,.\left(#3\right)}
\newcommand{\some}[3]{\exists\, #1 \,{\in}\, #2\,.\left(#3\right)}
\newcommand{\exactlyone}[3]{\exists!\, #1 \,{\in}\, #2\,.\left(#3\right)}
\newcommand{\lam}[3]{\lambda #1 \,{\in}\, #2\,.\left(#3\right)}
\newcommand{\uall}[2]{\forall\, #1\,.\left(#2\right)}
\newcommand{\usome}[2]{\exists\, #1\,.\left(#2\right)}
\newcommand{\uexactlyone}[3]{\exists!\, #1\,.\left(#2\right)}
\newcommand{\ulam}[2]{\lambda #1 .\left(#2\right)}
\newcommand{\xall}[3]{\forall\, #1 \,{\in}\, #2\,.\,#3}
\newcommand{\xsome}[3]{\exists\, #1 \,{\in}\, #2\,.\,#3}
\newcommand{\xexactlyone}[3]{\exists!\, #1 \,{\in}\, #2\,.\,#3}
\newcommand{\xuall}[2]{\forall\, #1\,.\,#2}
\newcommand{\xusome}[2]{\exists\, #1\,.\,#2}
\newcommand{\xuexactlyone}[2]{\exists!\, #1,.\,#2}
\newcommand{\xlam}[3]{\lambda #1 \,{\in}\, #2\,.\,#3}
\newcommand{\xulam}[2]{\lambda #1 .\,#2}
\newcommand{\tlam}[3]{\lambda #1 \,{:}\, #2\,.\,\left(#3\right)}
\newcommand{\xtlam}[3]{\lambda #1 \,{:}\, #2\,.\,#3}


%%% Local Variables: 
%%% mode: latex
%%% TeX-master: "escardo-mfps2010"
%%% End: 


\title{\darkblue{Computability of continuous solutions of} \\ \darkblue{higher-type equations}}

\date{{\normalsize CiE 2009} \\[2ex] 
{\normalsize Heidelberg, 19-24 July 2009}}


\author{\darkblue{Mart\'{\i}n Escard\'o} \\[1ex]
  {\small School of Computer Science,
    Birmingham University, UK}\\[2ex]}


\begin{document}

%\usefont{T1}{cmfr}{m}{it}
%\usefont{T1}{cmr}{m}{n}

\raggedright

\maketitle

\overhead{Summary}

\vfill

{1.} \darkblue{Brief background.} 

\qquad\qquad \darkblue{(a)} Kleene--Kreisel spaces. \\
\qquad\qquad \darkblue{(b)} Their exhaustible subspaces ($\darkblue{\approx}$ compact subspaces).

\vfill

{2.} \darkblue{Computability of solutions of equations over Kleene--Kreisel spaces.} % (Speed of sound.)

\vfill

\qquad\qquad Enough/necessary to assume that: \\
\qquad\qquad\darkblue{(a)} The solution is unique. \\
\qquad\qquad\darkblue{(b)} The unknown belongs to an exhaustible space.


{3.} \darkblue{Beyond Kleene--Kreisel spaces.}

{4.} \darkblue{Interesting examples of exhaustible ranges for the unknowns}.

\qquad\qquad Certain sets \darkblue{$K \subseteq \R^{[-\epsilon,\epsilon]}$} of analytic
  functions.

\overhead{Kleene--Kreisel spaces}

\darkblue{Characterization.} Least collection of objects

\qquad \darkblue{1.} containing the \darkblue{(discrete)} natural-number object $\darkblue{\N}$,

\qquad \darkblue{2.} closed under finite products \darkblue{$X \times Y$} and exponentials \darkblue{$Y^X$},

\quad in a suitable cartesian closed category of \darkblue{space-like objects}.

\vfill

{The following categories work, among others:} 

\quad \grey{Super-categories of $\operatorname{Top}$:}
filter spaces, limit spaces, equilogical spaces.

\quad \grey{Sub-categories of $\operatorname{Top}$:}
sequential spaces, $k$-spaces, QCB spaces. 

\darkblue{Definition.} A $kk$-space is a computable retract of a Kleene--Kreisel space.


\overhead{Exhaustible spaces}

\darkblue{Idea.} Can algorithmically check all points in finite time.

\vfill

\darkblue{Definition.}
A space \darkblue{$X$} is \darkblue{exhaustible} iff the functional
   \[
   \darkblue{\forall \colon 2^X \to 2}
   \]
defined by
   \[
   \darkblue{\forall(p)=1 \black{\iff} \text{$p(x)=1$ for all $x \in X$}}
   \]
is \darkblue{computable}.

\vfill

\darkblue{NB.}
Any \darkblue{$p \in 2^X$} is the characteristic function of a \darkblue{clopen} set.




\overhead{Main topological tool}

\darkblue{Lemma.} For any $kk$-space \darkblue{$X$},

the functional \darkblue{$\forall \colon 2^X \to 2$} is
continuous $\iff$ \darkblue{$X$} is compact.

\vfill

\darkblue{Corollary} \\
Exhaustible $kk$-spaces are compact.

\darkblue{Proof.} Computable functions are continuous.

\overhead{Theorems [Escard\'o 2007 (LICS) \& 2008 (LMCS)]}

\darkblue{1.} Finite $kk$-spaces are exhaustible \grey{(of course)}.

\darkblue{2.} Exhaustible $kk$-spaces are closed under finite and countable products.

\darkblue{3.} Hence e.g.\ the Cantor space $2^\N$ is exhastible \grey{(previously Berger)}.

\darkblue{4.} Computable images of exhaustible spaces are exhaustible.

\darkblue{5.} Any non-empty exhaustible $kk$-space is a computable image of $2^\N$.

\darkblue{6.} Any exhaustible $kk$-space is computably homeomorphic to
\\ ~~\, an exhaustible subspace of the Baire space $\N^\N$ \grey{(and hence is a Stone space)}.

\darkblue{7.} Any exhaustible non-empty subspace of a Kleene-Kreisel space is a
\\ ~~\, computable retract \grey{(and hence a $kk$-space)}.

\darkblue{8.} Arzela--Ascoli type characterization.

\overhead{Computability of solutions of higher-type equations}

\darkblue{Theorem.} 
Assume: \\[1ex]
\qquad (a) \darkblue{$X$} and \darkblue{$Y$} are $kk$-spaces, \\
\qquad (b) \darkblue{$X$} is exhaustible, \\
\qquad (c) \darkblue{$f\colon X \to Y$} is computable, \\
\qquad (d) \darkblue{$y \in Y$} is computable. \\

\vfill
Then, uniformly in the above data,

\vfill

\qquad 1. If \darkblue{$f(x) = y$} has a unique solution
  \darkblue{$x \in X$}, then it is computable. 

\vfill

\qquad 2. The non-solvability of the equation \darkblue{$f(x) = y$} 
is semi-decidable.

\vfill



\overhead{}

\vfill

\darkblue{Corollary.}

If \darkblue{$f\colon X \to Y$} is a computable bijection of
exhaustible $kk$-spaces, then it has a computable inverse.

\vfill

\darkgreen{Cf:} a continuous bijection of compact Hausdorff spaces is
a homeomorphism.

\overhead{Subsummed by the above theorem}

\vfill

\darkblue{1.} Equations of the form \darkblue{$g(x)=h(x)$}, even with parameters.

\qquad \grey{(Easy group-theoretical trick.)}

\vfill

\darkblue{2.} Finite systems of equations with finitely many variables. 

\qquad \grey{(Because $kk$-spaces are closed under finite products.)}

\vfill

\darkblue{3.} Certain countable systems with countably many variables.

\qquad \grey{(Because $kk$-spaces are closed under countable cartesian powers.)}


\overhead{Applied to compute unique solutions:}

\vfill

\darkblue{Lemma.} Assume:
\\[1ex]
\qquad (a) \darkblue{$X$} is a $kk$-space. \\
\qquad (b) \darkblue{$K_n \subseteq X$} is a sequence of sets that are
exhaustible uniformly in \darkblue{$n$}. \\
\qquad (c) \darkblue{$K_n \supseteq K_{n+1}$}.

    If \darkblue{$\bigcap_n K_n$} is a singleton \darkblue{$\{x\}$},
    then \darkblue{$x$} is computable, uniformly in the data.

\overhead{Applied to semi-decide non-existence of solutions:}

\vfill

\darkblue{Lemma.} Assume:
\\[1ex]
\qquad (a) \darkblue{$X$} is an exhaustible $kk$-space. \\
\qquad (b) \darkblue{$K_n \subseteq X$} is a sequence of sets that are
decidable uniformly in \darkblue{$n$}. \\
\qquad (c) \darkblue{$K_n  \supseteq K_{n+1}$}.

Emptiness of \darkblue{$\bigcap_n K_n$} is semi-decidable, uniformly in the data.

\overhead{Used to build sets \darkblue{$K_n$} suitable for the application \\ of
the above two lemmas:}

\vfill

\darkblue{Lemma.}  For every $kk$-space \darkblue{$X$} there is a
family \darkblue{$(=_n)$} of equivalence relations that are decidable
uniformly in \darkblue{$n$} and satisfy \darkblue{
\begin{eqnarray*}
x=y & \iff & \forall n.\,x =_n y, \\
x =_{n+1} y & \implies & x =_n y.
\end{eqnarray*}
}

The proof uses the Kleene--Kreisel density theorem.

\overhead{Proof of the theorem}

\vfill

The set \darkblue{$K_n = \{ x \in X \mid f(x) =_n y\}$} is exhaustible, 
because it is a decidable subset of an exhaustible space.

\vfill

Therefore the result follows from the above lemmas, because

\qquad \darkblue{$x \in \bigcap_n K_n$} $\iff$ 
\darkblue{$\forall n.f(x) =_n y$} $\iff$ 
\darkblue{$f(x) = y$}.

\vfill


\overhead{To go beyond $kk$-spaces, can use representations}

E.g.\ The compact interval $[-1,1]$ has an exhaustible set $3^\omega$ of representatives.

\vfill

Binary representation with digit set $3 = \{-1,0,1\}$.

\vfill

\darkblue{Definition.} A represented space is \emph{exhaustible} if it
has an exhaustible set of representatives.

\overhead{Computability of solutions of higher-type equations II}

\darkblue{Theorem.} 
Assume: \\[1ex]
\qquad (a) \darkblue{$X$} and \darkblue{$Y$} are computational metric spaces, \\
\qquad (b) \darkblue{$X$} is computationally complete and exhaustible, \\
\qquad (c) \darkblue{$f\colon X \to Y$} is computable, \\
\qquad (d) \darkblue{$y \in Y$} is computable. \\

\vfill
Then, uniformly in the above data,

\vfill

\qquad 1. If \darkblue{$f(x) = y$} has a unique solution
  \darkblue{$x \in X$}, then it is computable. 

\vfill

\qquad 2. The non-solvability of the equation \darkblue{$f(x) = y$} 
is semi-decidable.

\overhead{Example: Exhaustible spaces of analytic functions}

\vfill

\darkblue{Theorem.}
  Let \darkblue{$\epsilon \in (0,1)$} and \darkblue{$b >
  0$} be computable.

\begin{quote}
The space 
$
\darkblue{A} = \darkblue{A(\epsilon,b)}
$
of 
analytic functions  \darkblue{$f \in \R^{[-\epsilon,\epsilon]}$} of the form
  \[
  \darkblue{f(x) = \sum_n a_n x^n}
  \]
  with \darkblue{$a_n \in [-b,b]$} is exhaustible.
\end{quote}

\overhead{Corollary.}

\quad 1. The Taylor coefficients of any \darkblue{$f \in A$} can be computed from \darkblue{$f$}.

\quad 2. For \darkblue{$f \in \R^{[-\epsilon,\epsilon]}$}, it is
semi-decidable whether \darkblue{$f \not\in A$}.

\overhead{References and advertisement}

\vfill

\small

\grey{1.} Normann. \darkblue{Computing with functionals -- computability theory or computer science?} Bulletin of Symbolic logic, 12 (2006), no.~1, 43--59. 

\vfill

\grey{2.} Normann. \darkblue{50 years of continuous functionals.} Slides of CiE 08 invited talk.

\vfill

\grey{3.} Escard\'o. \darkblue{Infinite sets that admit fast exhaustive search}. LICS, IEEE, pp.443-452, 2007.

\vfill

\grey{4.} Escard\'o. \darkblue{Exhaustible sets in higher-type computation.}
LMCS-4 (3:3) 2008.

\vfill

\grey{5.} Escard\'o and Oliva. \darkblue{Selection functions, bar recursion, and backward induction}. Preprint, 2009. 


\vfill

\darkblue{Thank you.}

\end{document}
