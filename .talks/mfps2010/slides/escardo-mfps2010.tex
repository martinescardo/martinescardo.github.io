\documentclass%
[%dvips,% for generating postscript printouts
%25pt,%
%a4paper,%
%landscape
Screen4to3,
%Screen16to9
]{foils}

\newcommand{\Bool}{\operatorname{Bool}}
\newcommand{\prodf}{\operatorname{\darkblue{prod}}}

\newcommand{\hash}{\operatorname{\sharp}}

\newcommand{\bnf}{\mathrel{::=}}
\newcommand{\lift}[1]{#1_{\bot}}
\newcommand{\N}{\mathbb{N}}
\newcommand{\R}{\mathbb{R}}
%\newcommand{\Bool}{\mathbb{B}}
\newcommand{\Sierp}{\mathcal{S}}
\newcommand{\pN}{\mathcal{N}}
\newcommand{\pBool}{\mathcal{B}}
\newcommand{\If}{\,\mathrel{\operatorname{if}}}
\newcommand{\Then}{\mathrel{\operatorname{then}}}
\newcommand{\Else}{\mathrel{\operatorname{else}}}
\newcommand{\True}{\operatorname{True}}%{\operatorname{true}}
\newcommand{\False}{\operatorname{False}}%{\operatorname{false}}
\newcommand{\domain}[1]{{\D_{#1}}}
\newcommand{\total}[1]{{\T_{#1}}}
\newcommand{\totaleq}[1]{\sim_{#1}}
\newcommand{\quo}[1]{\qq_{#1}}
\newcommand{\qq}{\rho}
\newcommand{\D}{D}
\newcommand{\E}{E}
\newcommand{\C}{C}
\newcommand{\T}{T}
\newcommand{\kk}[1]{{\C_{#1}}}
\newcommand{\Scott}{\operatorname{Scott}}
\newcommand{\tproduct}{\sigma \times \tau}
\newcommand{\tfunction}{\sigma \to \tau}
\newcommand{\rh}[1]{\rho_{#1}}
\newcommand{\comp}{\circ}
\newcommand{\siff}{\iff}%{\Leftrightarrow}
\newcommand{\Search}{\operatorname{\mathcal{S}}}
\newcommand{\g}{\operatorname{\text{\boldmath{$g$}}}}
\newcommand{\f}{\operatorname{\text{\boldmath{$f$}}}}
\newcommand{\x}{\operatorname{\text{\boldmath{$x$}}}}
\newcommand{\e}{\operatorname{\text{\boldmath{$e$}}}}
\newcommand{\sK}{\operatorname{\text{\boldmath{$K$}}}}
\newcommand{\overhead}[1]{\foilhead[-4ex]{\darkblue{#1}}}
\newcommand{\cons}{\hash}

%\usepackage[notref,notcite]{showkeys}
%\usepackage[light,none,bottom]{draftcopy}

% \now command (By Manfred Kerber and then Riccardo Poli)

\usepackage{calc}
\newcounter{hours}\newcounter{minutes}
\newcommand{\now}{ %
\setcounter{hours}{\time/60} %
\setcounter{minutes}{\time-\value{hours}*60} %
\ifnum\theminutes <10 \thehours :0\theminutes\else \thehours:\theminutes\fi}

\newcommand{\myalpha}{s}
\newcommand{\mybeta}{t}

% sample usage: \date{\today \now\ hours}

%\usepackage{a4wide}

%\usepackage{a4}

%\usepackage{times}

\usepackage{url}

\newcommand{\qqquad}{\qquad\qquad}

\usepackage[english]{babel}

\usepackage{latexsym,amssymb,amsmath,stmaryrd,color}

\newcommand{\myparagraph}[1]{\vspace{2ex}\noindent{\bf #1}}

% mathematical macros

\newcommand{\Top}{\operatorname{Top}}
\newcommand{\tinduced}[3]{{#1}({#2},{#3})} % subbasic in the induced topology
%\newcommand{\E}{\varepsilon}
\newcommand{\eval}{\operatorname{\varepsilon}}%{\operatorname{eval}}
\newcommand{\curry}[1]{\overline{\scriptstyle #1}}
\newcommand{\Curry}[1]{\overline{#1}}
%\newcommand{\N}{\mathbb{N}}
\newcommand{\K}{\mathbb{K}}
\newcommand{\NN}{\mathcal{N}}
\newcommand{\Z}{\mathbb{Z}}
%\newcommand{\R}{\mathbb{R}}
\newcommand{\RR}{\mathcal{R}}
\newcommand{\Q}{\mathbb{Q}}
\newcommand{\B}{2}%{\mathbb{B}}
\newcommand{\Nat}{\operatorname{Nat}}
%\newcommand{\Bool}{\operatorname{Bool}}
\newcommand{\dual}[1]{{#1}^\wedge} % furst Lawson dual
\newcommand{\dualdual}[1]{{#1}^{\wedge\wedge}} % second Lawson dual
%\newcommand{\e}{e} % canonical map into second Lawson dual
\renewcommand{\d}{d} % its inverse when it exists.
\newcommand{\onlineremark}[1]{{\footnotesize(#1)}}
\newcommand{\old}[1]{}

\newcommand{\U}{{\cal U}}
\newcommand{\V}{{\cal V}}
\newcommand{\assembly}{\operatorname{N}}
\newcommand{\preassembly}{\operatorname{P}}
\newcommand{\Vstar}{\V^*}
\newcommand{\op}{\rm op}
\newcommand{\powerset}{\operatorname{\mathcal{P}}}
\newcommand{\pneg}{\operatorname{\underline{\neg}}}
\newcommand{\phull}[1]{\underline{#1}}
\newcommand{\wellinside}{\eqslantless}
\newcommand{\waybelow}{\ll}
\newcommand{\eqdef}{\stackrel{{\rm\scriptscriptstyle  def}}{=}}
\newcommand{\himplies}{\Rightarrow}
\newcommand{\functorfont}[1]{\operatorname{#1}}
\newcommand{\catfont}[1]{\mathbf{#1}}
\newcommand{\Loc}{\catfont{Loc}}
\newcommand{\Frm}{\catfont{Frm}}
\newcommand{\Set}{\catfont{Set}}
%\newcommand{\K}{K} % was{\operatorname{K}} % ``kelleyification''
%\newcommand{\Sierp}{\mathbb{S}}
\newcommand{\Stone}{\catfont{Stone}}
\newcommand{\Coh}{\catfont{Spec}}
\newcommand{\CReg}{\catfont{KReg}}
\newcommand{\RLC}{\catfont{RLK}}
\newcommand{\SC}{\catfont{SK}}
\newcommand{\SLC}{\catfont{SLK}}
\newcommand{\BReg}{\catfont{BReg}}

%\newcommand{\clsb}[2]{#2_{#1}}
\newcommand{\clsb}[2]{#1\setminus #2}
\newcommand{\open}{\circ}
\newcommand{\closed}{\boxempty}
\newcommand{\cocompact}[1]{\kappa_{#1}}
\newcommand{\Patch}{\operatorname{Patch}}
\newcommand{\Opens}{\operatorname{\mathcal{O}}}%[1]{\operatorname{\mathcal{O}}{#1}} 
\renewcommand{\O}{\Opens}
\newcommand{\Cocompacts}{\operatorname{\mathcal{C}}}
%\newcommand{\C}[2]{\operatorname{C}(#1,#2)}
\newcommand{\Regions}{\operatorname{\mathcal{R}}}%[1]{\operatorname{\mathcal{R}}{#1}} 
\newcommand{\frameleft}[1]{#1^{*}} \newcommand{\frameright}[1]{#1_{*}}
\newcommand{\arrow}{\to}%{\rightarrow}
%\newcommand{\comp}{\mathrel{\circ}}
\newcommand{\id}{\operatorname{id}}
\newcommand{\join}{\sqcup}
\newcommand{\meet}{\sqcap}
\newcommand{\himp}{\mathrel{\Rightarrow}}
\newcommand{\bigjoin}{\bigsqcup}
\newcommand{\dirjoin}{\sideset{}{^{\uparrow}}\bigjoin}
\newcommand{\bigmeet}{\bigsqcap}
\newcommand{\below}{\sqsubseteq}
\renewcommand{\above}{\sqsupseteq}
\newcommand{\farrow}[1]{\stackrel{#1}{\longrightarrow}}
\newenvironment{eqn}{\[\begin{array}{rcll}}{\end{array}\]}
\newcommand{\qtext}[1]{\quad\text{#1}}
\newcommand{\up}{\operatorname{\uparrow}}
\newcommand{\down}{\operatorname{\downarrow}}
\newcommand{\upup}{\mathord{\hbox{\makebox[0pt][l]{\raise .6mm\hbox{$\uparrow$}}$\uparrow$}}}
\newcommand{\fix}{\operatorname{fix}}
\newcommand{\Infl}[1]{\hat{#1}}
\newcommand{\infl}{\Infl{\top}} %{\Infl{\operatorname{\i d}}}


% defined terms and index entries

\newcommand{\fontfordefinition}[1]{\emph{#1}}%{{\bfseries\itshape #1}}
% \newcommand{\df}[1]{\fontfordefinition{#1}} % no index required
\newcommand{\dfkey}[2]{\fontfordefinition{#1}\index{#2}}
\newcommand{\dfemph}[1]{\emph{#1}\index{#1}}
\newcommand{\cindex}[1]{#1} % was {#1\index{#1}}
\newcommand{\tindex}[1]{#1\index{#1}}
\newcommand{\df}[1]{\dfkey{#1}{#1}}
\newcommand{\dfn}[1]{\fontfordefinition{#1}}
\newcommand{\ind}[1]{#1\index{#1}}
\newcommand{\indemph}[1]{\emph{#1}\index{#1}}

\makeindex

% Omission macro

\newcommand{\Omit}[1]{}

% Chronology macro

\newcommand{\chron}[1]{}
                     %{\begin{footnotesize}(Chronology: #1\end{footnotesize})}

\newcommand{\smalltext}[1]{\begin{footnotesize} #1 \end{footnotesize}}

%\newcommand{\myparagraph}[1]{\medskip\emph{#1}}

\newcommand{\lbracket}{(}
\newcommand{\rbracket}{)}
\newcommand{\QTop}{\operatorname{{QTop}}}

\newcommand{\numbered}[1]{\grey{#1}}

\definecolor{darkgreen}{rgb}{0,0.5,0}
\definecolor{darkblue}{rgb}{0,0.0,0.5}
\definecolor{darkred}{rgb}{1,0,0}
\definecolor{grey}{rgb}{0.4,0.4,0.4}
\newcommand{\black}{\textcolor{black}}
\newcommand{\grey}{\textcolor{grey}}
\newcommand{\blue}{\textcolor{blue}}
\newcommand{\red}{\textcolor{red}}
\newcommand{\darkblue}{\textcolor{darkblue}}
\newcommand{\darkred}{\textcolor{darkred}}
\newcommand{\darkgreen}{\textcolor{darkgreen}}
\newcommand{\green}{\textcolor{green}}
\newcommand{\yellow}{\textcolor{yellow}}
\newcommand{\cyan}{\textcolor{cyan}}
\newcommand{\magenta}{\textcolor{magenta}}

\newcommand{\qed}{}%{$\boxempty$}

\newcommand{\sexample}[1]{\noindent\textbf{Example} \blue{#1} }
\newcommand{\slemma}[1]{\noindent\darkgreen{\textbf{Lemma}} \\[1ex] {\emtheorem #1}}
\newcommand{\ulemma}[1]{\noindent\darkgreen{\textbf{Lemma}} {\emtheorem #1}}
\newcommand{\proposition}[1]{\noindent\darkgreen{\textbf{Proposition}} {\emtheorem #1}}
\newcommand{\proof}[1]{\noindent{\textbf{Proof}} {#1}}
\newcommand{\sdefinition}[1]{\noindent\darkgreen{\textbf{Definition}} \\[1ex] {\emtheorem #1}}
\newcommand{\scorollary}[1]{\noindent\darkgreen{\textbf{Corollary}} {\emtheorem #1}}
\newcommand{\stheorem}[1]{\noindent\darkgreen{\textbf{Theorem}}  {\emtheorem #1}}
\newcommand{\texample}[2]{\noindent\textbf{Example} \blue{#1}}
\newcommand{\tlemma}[2]{\noindent\darkgreen{\textbf{Lemma} (#1)}  \\[1ex] {\emtheorem #2}}
\newcommand{\ttlemma}[2]{\noindent\darkgreen{\textbf{Lemma} (#1)} {\emtheorem #2}}
\newcommand{\utlemma}[2]{\noindent\darkgreen{\textbf{Lemma} (#1)} {\emtheorem #2}}
\newcommand{\tdefinition}[2]{\noindent\darkgreen{\textbf{Definition} (#1)} \\[1ex] {\emtheorem #2}}
\newcommand{\tcorollary}[2]{\noindent\darkgreen{\textbf{Corollary} (#1)}  \\[1ex] {\emtheorem #2}}
\newcommand{\ttheorem}[2]{\noindent\darkgreen{\textbf{Theorem} (#1)}  \\[1ex] {\emtheorem #2}}

\newcommand{\utheorem}[2]{\noindent\darkgreen{\textbf{Theorem} (#1)} {\emtheorem #2}}

\newcommand{\uapplication}[1]{\noindent\darkgreen{\textbf{Application}} {#1}}

\newcommand{\emtheorem}{\em}

\LogoOff

\newcommand{\twostar}{2^{*}}
\newcommand{\twoomega}{2^\infty}
\newcommand{\prefix}{\sqsubseteq}
\newcommand{\all}[3]{\forall\, #1 \,{\in}\, #2\,.\left(#3\right)}
\newcommand{\some}[3]{\exists\, #1 \,{\in}\, #2\,.\left(#3\right)}
\newcommand{\exactlyone}[3]{\exists!\, #1 \,{\in}\, #2\,.\left(#3\right)}
\newcommand{\lam}[3]{\lambda #1 \,{\in}\, #2\,.\left(#3\right)}
\newcommand{\uall}[2]{\forall\, #1\,.\left(#2\right)}
\newcommand{\usome}[2]{\exists\, #1\,.\left(#2\right)}
\newcommand{\uexactlyone}[3]{\exists!\, #1\,.\left(#2\right)}
\newcommand{\ulam}[2]{\lambda #1 .\left(#2\right)}
\newcommand{\xall}[3]{\forall\, #1 \,{\in}\, #2\,.\,#3}
\newcommand{\xsome}[3]{\exists\, #1 \,{\in}\, #2\,.\,#3}
\newcommand{\xexactlyone}[3]{\exists!\, #1 \,{\in}\, #2\,.\,#3}
\newcommand{\xuall}[2]{\forall\, #1\,.\,#2}
\newcommand{\xusome}[2]{\exists\, #1\,.\,#2}
\newcommand{\xuexactlyone}[2]{\exists!\, #1,.\,#2}
\newcommand{\xlam}[3]{\lambda #1 \,{\in}\, #2\,.\,#3}
\newcommand{\xulam}[2]{\lambda #1 .\,#2}
\newcommand{\tlam}[3]{\lambda #1 \,{:}\, #2\,.\,\left(#3\right)}
\newcommand{\xtlam}[3]{\lambda #1 \,{:}\, #2\,.\,#3}


%%% Local Variables: 
%%% mode: latex
%%% TeX-master: "escardo-mfps2010"
%%% End: 


\usepackage{hyperref}

\newcommand{\next}{\operatorname{next}}
\newcommand{\myitem}[1]{\item[\darkblue{#1}]}

\newcommand{\Maybe}{\operatorname{Maybe}}

\title{\darkblue{Topology, computation, monads, games and proofs}}

% \title{\darkblue{Topological ideas and techniques} \\
% \darkblue{in computation and its frontiers}}

\author{\darkblue{Mart\'{\i}n Escard\'o} \\[10ex]
$\frac{3}{4}$ of this talk is joint work with \darkblue{Paulo Oliva} from Queen Mary, London. \\[10ex] ~}

\date{\sc MFPS 2010, Ottawa, May 6-10, 2010}

\begin{document}

\maketitle

% \overhead{About}

% \begin{enumerate}
% \item[\grey{1.}] 
% \darkblue{Survey} of both my own long-going work on \darkblue{topology in computation},

% and work with Paulo Oliva on ramifications to \darkblue{game theory} and
% \darkblue{proof theory}.

% \item[\grey{2.}]  This is originally motivated and shaped by topological ideas.

% \item[\grey{3.}]  
% But now the notion of \darkblue{selection function} for a generalized quantifier

% has emerged with a prominent and unifying role, 

% and with strong categorical underpinnings.

% \item[\grey{4.}] 
% Paulo and I have found that this is only the \darkblue{tip of the iceberg}, and are currently working on many papers in this direction. 
% \end{enumerate}

\overhead{Contents plan}

\vfill

\darkblue{I.} Topology in computation.

\grey{Exhaustive search.}

\vfill

\darkblue{II.} The selection monad.

\grey{Selection functions for generalized quantifiers.}

\vfill

\darkblue{III.} Game theory.

\grey{Optimal plays and strategies.}

\vfill

\darkblue{IV.} Proof theory.

\grey{Computational extraction of witnesses from classical proofs.}


\overhead{Unifying concept: selection functions for quantifiers}

A certain \emph{countable product of selection functions} implements:

~

\darkblue{I.} Topology in computation: \darkblue{Tychonoff theorem.}

\vfill

\darkblue{II.} The selection monad: \darkblue{Strength.}

\vfill

\darkblue{III.} Game theory: \darkblue{Optimal plays and strategies.}

\vfill

\darkblue{IV.} Proof theory: \darkblue{Double-negation shift. \grey{Bar recursion.}}

\overhead{Time plan}

Each part has half of the time allocated to the previous part.

Even though it probably deserves twice as much.


\overhead{I. Topology and computation.}

\darkblue{Old theorem.}
A set of natural numbers is exhaustively searchable iff it is finite.

\vfill

\emph{Intuition:}
How could one possibly check infinitely many cases in finite time?

\emph{Proof:} Removed from this intuition (Halting problem, diagonalization).


\overhead{Common wisdom}

A set of whatever-you-can-think-of is exhaustively searchable iff it is finite.




\vfill

E.g.\ types in 
\begin{enumerate}
\item[\grey{1.}] System $T$, PCF
\item[\grey{2.}] FPC, ML, Haskell etc.
\end{enumerate}

\overhead{Can't always trust common wisdom}


\vfill

\darkblue{E.g.} The total elements of \darkblue{$\Nat \to \Bool$} are
exhaustively searchable.

\vfill

Many other examples.

\overhead{Corollary}

The type \darkblue{$(\Nat \to \Bool) \to \Nat$} has decidable equality.

\vfill

\darkblue{Proof.}
Given $\darkblue{f,g \colon (\Nat \to \Bool) \to \Nat}$.

Check whether \darkblue{$f(\alpha) = g(\alpha)$} for every \darkblue{$\alpha \colon \Nat \to \Bool$}.

\overhead{Exhaustible set}

A set \darkblue{$K \subseteq X$} is 
\emph{exhaustible} iff there is an an algorithm s.t.

\vfill

\begin{enumerate}
\item[\grey{1.}] \emph{Input:} \darkblue{$p \colon X \to \Bool$} decidable.
\item[\grey{2.}] \emph{Output:} \darkblue{$\True$} or \darkblue{$\False$}.
\item[\grey{3.}] \emph{Specification:} output \darkblue{$\True$} iff \darkblue{$\exists k
    \in K.p(k) = \True$}.
\end{enumerate}

\vfill

The algorithm has higher type \darkblue{$(X \to \Bool) \to \Bool$}.

\overhead{Searchable set}

A set \darkblue{$K \subseteq X$} is 
\emph{searchable} iff there is an an algorithm s.t.

\vfill

\begin{enumerate}
\item[\grey{1.}] \emph{Input:} \darkblue{$p \colon X \to \Bool$} decidable.
\item[\grey{2.}] \emph{Output:} Either \darkblue{fail} or
  some \darkblue{$k \in K$}.
\item[\grey{3.}] \emph{Specification:} output \darkblue{fail}
    if \darkblue{$\forall k \in K.p(k) = \False$}, 
    \\ \phantom{\emph{Specification:}} or else \darkblue{$k \in K$} with
  \darkblue{$p(k) = \True$}.
\end{enumerate}

\vfill

The algorithm has higher type \darkblue{$(X \to \Bool) \to 1+X$}.


\overhead{Of course}

Searchable $\implies$ exhaustible.

\vfill

This is so by definition. 

\overhead{Searchable set, slightly different notion and formulation}

A set \darkblue{$K \subseteq X$} is 
\emph{searchable} iff there is an an algorithm s.t.

\vfill

\begin{enumerate}
\item[\grey{1.}] \emph{Input:} \darkblue{$p \colon X \to \Bool$} decidable.
\item[\grey{2.}] \emph{Output:} \darkblue{$k \in K$}.
\item[\grey{3.}] \emph{Specification:} If \darkblue{$\exists x \in
    K.p(x)=\True$} then \darkblue{$p(k)=\True$}. \\ 
  \phantom{\emph{Specification:}} Otherwise
  \darkblue{$p(k)=\False$},  of course.
\end{enumerate}

\vfill

\darkblue{Only difference:} previous accounts for the empty set, this doesn't.

\vfill

The algorithm has higher type \darkblue{$(X \to \Bool) \to X$}.

\overhead{Still}

Searchable $\implies$ exhaustible.

\vfill

Given the potential example \darkblue{$k \in K$}, 

check whether \darkblue{$p(k)=\True$} or \darkblue{$p(k)=\False$}.

\vfill

\darkblue{Better:} the answer to the exhaustion procedure is just \darkblue{$p(k)$}. 


\overhead{Summary of the two notions}

\darkblue{$K \subseteq X$}
\begin{enumerate}
\item[] \darkblue{Exhaustible:} algorithm \darkblue{$\exists_K \colon (X \to \Bool) \to \Bool$}.

The boolean \darkblue{existential quantifier} is computable.

\item[] \darkblue{Searchable:} algorithm \darkblue{$\varepsilon_K \colon (X \to \Bool) \to X$}.

The set $K$ has a computable a \darkblue{selection function}.

\end{enumerate}

\vfill

\qquad Derived functions:

\qquad \qquad \darkblue{$\exists_K(p) = p(\varepsilon_K(p))$}.

\qquad \qquad \darkblue{$\forall_K(p) = \neg \exists_K(\neg \comp p)$}.

\overhead{Theorem (LMCS'2008, ENTCS'2004)}

Exhaustible sets (hence searchable sets) are \darkblue{topologically compact}.

\vfill

Types with decidable equality are \darkblue{topologically discrete}.

\overhead{First examples and counter-examples}

\begin{enumerate}
\item[\grey{1.}] 
A set of natural numbers is compact iff it is finite.

\item[\grey{2.}]
The maximal elements of the \darkblue{lazy natural numbers} are searchable. 

\grey{(Which amount to the \darkblue{one-point compactification of discrete natural numbers}.)}

\item[\grey{3.}]
The set of all sequences \darkblue{$\alpha \colon \Nat \to \Nat$} is \emph{not}
searchable.

\item[\grey{4.}]
The set of sequences \darkblue{$\alpha \colon \Nat \to \Nat$} such that 
\darkblue{$\alpha_k < 17$} is searchable. 

\item[\grey{5.}]
The set of sequences \darkblue{$\alpha \colon \Nat \to \Nat$} such that 
\darkblue{$\alpha_k < k$} is searchable. 

\item[\grey{6.}]  The set of sequences \darkblue{$\alpha \colon \Nat
    \to \Nat$} such that \darkblue{$\alpha_k < \beta_k$} is searchable, for
  any given sequence \darkblue{$\beta \colon \Nat \to \Nat$}.

  \end{enumerate}

\overhead{More examples (LMCS'2008)}

\newcommand{\compact}{\operatorname{compact}}
\newcommand{\discrete}{\operatorname{discrete}}

Consider the types defined by the following grammar:

\begin{gather*}
  \compact  ::=  1 | \compact + \compact | \compact \times \compact | \discrete \to \compact, \\
  \discrete  ::=  1 | \Nat | \discrete + \discrete | \discrete \times \discrete | \compact \to \discrete.
\end{gather*}

\noindent\darkblue{Theorem.}

\begin{enumerate}
\item[\grey{1.}] 
\darkblue{Compact types} are \darkblue{searchable}.
\item[\grey{2.}]
\darkblue{Discrete types} have \darkblue{decidable equality} of total elements.
\end{enumerate}

\vfill

\noindent 
\darkblue{E.g.} $(((\Nat \to 1+1) \to \Nat) \to 1+1+1) \to ((\Nat \to 1+1+1+1) \to \Nat)$
has decidable equality.

\overhead{Dictionary between topology and computation}

\darkblue{Open set.} Semi-decidable set.

\darkblue{Closed and open set.} Decidable set.

\darkblue{Continuous map.} Computable function.

\darkblue{Compact set.} Exhaustively searchable set.

\darkblue{Discrete space.} Type with decidable equality.

\darkblue{Hausdorff space.} Space with semi-decidable apartness.

\vfill
\noindent
Take a theorem in topology, \\
apply the dictionary, \\
get a theorem in computability theory. \\
\grey{Unfortunately you have to come up with a new proof.}

\overhead{Theorems (LMCS'2008)}

\begin{enumerate}
\item[\grey{1.}] The non-empty exhaustible sets are the computable images of the
  Cantor space $(\Nat \to \Bool)$.
\item[\grey{2.}] Searchable sets are closed under computable images.
\item[\grey{3.}] Hence hence the non-empty exhaustible sets are searchable. 

\grey{(Given yes/no algorithm, get an algorithm for witnesses.)}

\item[\grey{4.}] Searchable sets are closed under countable products.

\grey{(Tychonoff theorem.)}

\item[\grey{5.}] And under intersections with decidable sets. 
\item[\grey{6.}] They are retracts of the types where they live. 
%\item[\grey{7.}] Arzela--Ascoli type characterization of searchable sets.
% (For details see publications.)
\end{enumerate}

\overhead{Is this feasible?}

I have some counter-intuitively fast examples to show you.


\overhead{II. Selection functions for generalized quantifiers.}

Pause to look at some motivating examples.

\overhead{Mean-value theorem}

\darkblue{$\int_0^1 f = f(a)$}.

\vfill

The mean value is attained.

\vfill

\noindent
If you travelled from London to Ottawa and your journey took 12 hours,
then at some point you were travelling at $5379 / 12 \approx 440$km/h.


\overhead{Maximum-value theorem}

\darkblue{$\sup_0^1 f = f(a)$}.

\vfill

The maximum value is attained.

\overhead{Universal-value theorem}

\darkblue{$\forall p = p(a)$}.

\vfill

The universal value is attained. 

\vfill

Known as \darkblue{Drinker paradox}: 

In every pub there is a 
person \darkblue{$a$} such that everybody drinks iff \darkblue{$a$} drinks.

\overhead{Existential-value theorem}

\darkblue{$\exists p = p(a)$.}

\vfill

The existential value is attained. 

\vfill

Another version of the \darkblue{Drinker paradox}: 

In every pub there is a 
person \darkblue{$a$} such somebody drinks iff \darkblue{$a$} drinks.

\overhead{General pattern}

\darkblue{$\phi(p) = p(a)$}.

$R$ type of results.

$p \colon X \to R$.

$\phi \colon (X \to R) \to R$. \qquad \grey{Lives in the continuation monad.}

$a \in X$.

\vfill

\darkblue{We want to find $a$ from given $p$, as $a = \varepsilon(p)$.}

$\varepsilon \colon (X \to R) \to X$. \qquad \grey{Lives in the selection monad.}

\darkblue{$\phi(p) = p(\varepsilon(p))$.}

\overhead{Continuation monad}

\darkblue{$KX = ((X \to R) \to R)$}.

\vfill

Well known, with many theoretical and practical uses.

\overhead{Selection monad}

\darkblue{$JX = ((X \to R) \to X)$.}

\darkblue{Images of searchable sets are searchable:}

$f \colon X \to Y$ 

$J f \colon J X \to J Y$, \qquad $Jf\varepsilon = \lambda
q.f(\varepsilon(\lambda x.f(q(x))$.


\vfill

\darkblue{Singletons are searchable:}

$\eta \colon X \to J X$, \qquad $\eta(x) = \lambda p.x$.



\vfill

\darkblue{The union of a searchable set of searchable sets is searchable:}

$\mu \colon J J X \to J X$, \qquad $\mu(E) = \lambda p. E(\lambda\varepsilon.p(\varepsilon(p)))(p)$.


\vfill

\overhead{Monad morphism $J \to K$}

\darkblue{$\varepsilon \mapsto \phi$} where \darkblue{$\phi(p) = p(\varepsilon(p))$. }

\vfill

We write \darkblue{$\phi = \overline{\varepsilon}$}.

\vfill

Then \darkblue{$\overline{\varepsilon}(p) = p(\varepsilon(p))$}.

\vfill

\darkblue{Definition.} 

A quantifier \darkblue{$\phi \in KX$} is \darkblue{attainable} if it has a selection function \darkblue{$\varepsilon \in JX$}:
\[
 \darkblue{\phi = \overline{\varepsilon}}.
\]

\overhead{$J$ and $K$ are strong}

The strengths

\darkblue{$X \times T Y \to T(X \times Y)$.}

extend to

\darkblue{$\otimes \colon TX \times T Y \to T(X \times Y)$.}

\vfill



\darkblue{NB.} The monads are not commutative.

The extension of the co-strengths \darkblue{$T X \times Y \to T(X \times Y)$}

give different maps \darkblue{$\otimes' \colon TX \times T Y \to T(X \times Y)$.}

\overhead{Terminology and examples}

\grey{1.} \darkblue{$\otimes \colon KX \times KY \to K(X \times Y)$.}

 \qquad \qquad \emph{Product of quantifiers.}


\qquad \qquad \qquad \darkblue{$(\forall_X \otimes \exists_Y)(p) = \forall x \in X.\exists y \in Y.p(x,y)$}.

\qquad \qquad \qquad \darkblue{$\exists_X \otimes \exists_Y = \exists_{X\times Y}$}.

\vfill

\grey{2.} \darkblue{$\otimes \colon JX \times JY \to J(X \times Y)$.}

\qquad\qquad \emph{Product of selection functions.}

\qquad \qquad\qquad \darkblue{$(\varepsilon \otimes \delta)(p) = (a,b(a))$}

\qquad \qquad \qquad where \darkblue{$b(x) = \delta(\lambda y.p(x,y)$},

\qquad \qquad \qquad \phantom{where} \darkblue{$a = \varepsilon(\lambda x.p(x,b(x))$}. 

\overhead{Theorem}

Attainable quantifiers are closed under finite products.

\vfill

\darkblue{Proof.}

\quad The monad morphism gives
\[
\darkblue{\overline{\varepsilon} \otimes \overline{\delta} = 
\overline{\varepsilon \otimes \delta}.}
\]

\quad Hence if \darkblue{$\phi = \overline{\varepsilon}$} 
and \darkblue{$\gamma = \overline{\delta}$},

\quad then \darkblue{$\phi \otimes \gamma = \overline{\varepsilon \otimes \delta}$},

\quad and so \darkblue{$\varepsilon \otimes \delta$} is a selection function for
\darkblue{$\phi \otimes \gamma$}.

\overhead{Example 1}

Binary Tychonoff theorem.

\vfill

The product of two searchable sets is searchable.

\vfill

\darkblue{Proof.}

\darkblue{$\varepsilon \in J X$} selection function for \darkblue{$\exists_K \in KX$} with \darkblue{$K \subseteq X$}.

\darkblue{$\delta \in J Y$} selection function for \darkblue{$\exists_L \in KY$} with \darkblue{$L \subseteq Y$}.

\darkblue{$\varepsilon \otimes \delta$} selection function for \darkblue{$\exists_K \otimes \exists_L = \exists_{K \times L} \in K(X \times Y)$}.


\overhead{Example 2}

In every pub there are a man \darkblue{$a$} and a woman \darkblue{$b$} such that

every man buys a drink to some woman iff \darkblue{$a$} buys a drink to \darkblue{$b$}.

\vfill

\darkblue{Proof.}

This amounts to \darkblue{$\left(\forall x \in X.\exists y \in Y.p(x,y)\right) = p(a,b)$.}

By the Drinker paradoxes, the quantifiers \darkblue{$\forall_X$} and \darkblue{$\exists_Y$}

have
selection functions \darkblue{$A$} and \darkblue{$E$}. 

By the above theorem, the quantifier \darkblue{$\forall_X \otimes \exists_Y$} has
a selection function \darkblue{$A \otimes E$}. 

Hence we can take \darkblue{$(a,b) = (A \otimes E)(p)$}.


\overhead{Countable Tychonoff theorem for searchable sets}

Arbitrary products of compact sets are compact.

Countable products of searchable sets are searchable.

\vfill

Countable ``strength'':

\qquad \qquad \darkblue{$\bigotimes \colon \prod_i J X_i \to J \prod_i X_i$},


\vfill

characterized by


\qquad \qquad \darkblue{$\bigotimes_i \varepsilon_i = \varepsilon_0 \otimes \bigotimes_i \varepsilon_{i+1}$.}

\vfill

\darkblue{NB.} This exists only in particular categories.


\overhead{III. Game theory.}

Products of selection functions calculate
\begin{enumerate}
\item[\grey{1.}] optimal plays, and
\item[\grey{2.}] optimal strategies.
\end{enumerate}

\overhead{Example 1}

\begin{enumerate}
\item 
Two-person game that finishes after exactly \darkblue{$n$} moves.

\item 
Eloise starts and alternates playing with Abelard. \darkblue{One of them wins.}

\item \darkblue{$i$}-th move is an element of the set \darkblue{$X_i$}.

\item A predicate \darkblue{$p \colon \prod_{i=0}^{n-1}
  X_i \to \Bool$} tells whether Eloise wins.

\item  Eloise can
  win iff
\[
\darkblue{\exists x_0 \in X_0 \quad \forall x_1 \in X_1 \quad \exists x_{2} \in X_{2} \quad \forall x_{3} \in X_{3} \cdots p(x_0, \ldots, x_{n-1}).}
\]

\item 
If \darkblue{$\phi_{2i}
= \exists_{X_{2i}}$} and \darkblue{$\phi_{2i+1} = \forall_{X_{2i+1}}$},
this amounts to
\darkblue{$
 \left(\bigotimes_{i=1}^n \phi_i\right)(p). 
$}
\end{enumerate}

\overhead{Example 2}


\begin{enumerate}
\item 
Two-person game that finishes after exactly \darkblue{$n$} moves.

\item 
Eloise starts and alternates playing with Abelard. \darkblue{Lose, draw, win.}

\item \darkblue{$i$}-th move is an element of the set \darkblue{$X_i$}.

\item A predicate \darkblue{$p \colon \prod_{i=0}^{n-1}
  X_i \to \{-1,0,1\}$}.

\item  The optimal outcome of the game is
\[
\darkblue{\sup_{x_0 \in X_0} \quad \inf_{x_1 \in X_1} \quad \sup_{ x_{2} \in X_{2}} \quad \inf_{x_{3} \in X_{3}} \cdots \quad p(x_0, \ldots, x_{n-1}).}
\]

\item 
If \darkblue{$\phi_{2i}
= \sup_{X_{2i}}$} and \darkblue{$\phi_{2i+1} = \inf_{X_{2i+1}}$},
this again amounts to
\darkblue{$
 \left(\bigotimes_{i=1}^n \phi_i\right)(p). 
$}
\end{enumerate}

\overhead{Sequential game of length $n$}


\darkblue{$X_0, \dots, X_{n-1}$} sets of possible moves at rounds
\darkblue{$0,\dots,n-1$}.

\darkblue{$p \colon \prod_{i=0}^{n-1} X_i \to R$} outcome (or pay-off) function.

\darkblue{$\phi_0 \in K X_0$}, \dots, \darkblue{$\phi_{n-1} \in K X_{n-1}$} quantifiers for each round.

\vfill

We don't stipulate who plays at each round.

This is implicit in the choice of quantifiers.


\overhead{Subgame}

Determined by a partial play \darkblue{$a = (a_0,\dots,a_{n-1}) \in \prod_{i=0}^{k-1} X_i$} for \darkblue{$k \le n$}:
 \[ \darkblue{(X_i , p_{{a}}, \phi_i)}.\]  

\vfill

Here \darkblue{$p_{{a}} \colon \prod_{i=k}^{n-1} X_i \to R$} is defined by
\[
\darkblue{p_{{a}}(x_k,\dots,x_{n-1}) =
  p(a_0,\dots,a_{k-1},x_k,\dots,x_{n-1})},
\]

\vfill

Like the original game but starts at the position determined by the
moves~\darkblue{$a$}.

\overhead{Optimal outcomes and plays}

%\vspace{-2ex}

  \begin{enumerate}
\item[\grey{1.}] The \emph{optimal outcome} of the game is
\darkblue{$w = \left(\bigotimes_{i=0}^{n-1} \phi_i\right)(p)$}.


\item[\grey{2.}] A play \darkblue{$(a_0,\dots,a_{n-1})$} is \emph{optimal} if
\[
\darkblue{w_{()} = w_{(a_0)} = w_{(a_0,a_1)} = w_{(a_0,a_1,a2)} = \cdots = w_{(a_0,a_1,\dots,a_{n-1})}}.
\]

All players have played as best as they could.
\end{enumerate}

\overhead{Example}

\newcommand{\play}[9]{\text{%
\begin{tabular}{c|c|c}
$#1$ & $#2$ & $#3$ \\ \hline
$#4$ & $#5$ & $#6$ \\ \hline
$#7$ & $#8$ & $#9$
\end{tabular}%
}%
}

\newcommand{\X}{\text{X}}
\renewcommand{\O}{\text{O}}
\newcommand{\n}{\phantom{\O}}


The optimal outcome for tic-tac-toe is a draw.

An optimal play is

% [0,4,1,2,6,3,5,7,8]
%  x o x o x o x o x

$
\begin{array}{ccc}
%     0    1   2   3   4   5   6   7   8
\play{\X}{\n}{\n}{\n}{\n}{\n}{\n}{\n}{\n} & \quad
\play{\X}{\n}{\n}{\n}{\O}{\n}{\n}{\n}{\n} & \quad
\play{\X}{\X}{\n}{\n}{\O}{\n}{\n}{\n}{\n} \\[7ex]

\play{\X}{\X}{\O}{\n}{\O}{\n}{\n}{\n}{\n} & \quad
\play{\X}{\X}{\O}{\n}{\O}{\n}{\X}{\n}{\n} & \quad
\play{\X}{\X}{\O}{\O}{\O}{\n}{\X}{\n}{\n} \\[7ex]

\play{\X}{\X}{\O}{\O}{\O}{\X}{\X}{\n}{\n} & \quad
\play{\X}{\X}{\O}{\O}{\O}{\X}{\X}{\O}{\n} & \quad
\play{\X}{\X}{\O}{\O}{\O}{\X}{\X}{\O}{\X}
\end{array}
$

\overhead{Optimal moves and strategies}

\begin{enumerate}
\item A move \darkblue{$a_k \in X_k$} is \emph{optimal} for a subgame
  \darkblue{$(a_0,\dots,a_{k-1}) \in \prod_{i=0}^{k-1} X_i$} if it
  doesn't change the optimal outcome.

\[  \darkblue{w_{ (a_0,\dots,a_{k-1}) } = w_{ (a_0,\dots,a_{k-1},a_k)}}. \]

\item A \emph{strategy} is a family of functions,
% with \darkblue{$k < n$},
%
\[\darkblue{\next_k : \prod_{i=0}^{k-1} X_i \to X_k.} \]

\item A strategy is \emph{optimal} if the move \darkblue{$\next_k(a)$}
  is optimal for every partial play \darkblue{$a$}.

\end{enumerate}

\overhead{Policy functions for the game}

\noindent 
A \emph{policy} is a sequence of
selection functions \darkblue{$\varepsilon_i \colon (X_i \to R) \to
  X_i$} for the game quantifiers.

\vfill

\noindent
E.g., if the policy of the player is to maximize the payoff, then
\darkblue{$\varepsilon(p)$} is a point where \darkblue{$p$} attains
its maximum value.



\overhead{Calculating optimal plays and strategies} 

\noindent\darkblue{Theorem.} 
Let \darkblue{$(X_i,p,\phi_i)$} be a game with policy functions
\darkblue{$\varepsilon_i$}.
\begin{enumerate}
\item[\grey{1.}] 
An optimal play is given by
\[
\darkblue{{a} = \left(\bigotimes_{i=k}^{n-1}  \varepsilon_i\right)(p)}.
\]


\item[\grey{2.}] 
An optimal strategy is given by
\[
\darkblue{\next_k({a}) = \left(\left(\bigotimes_{i=k}^{n-1} \varepsilon_i\right)(p_{a})\right)_0}.
\]
\end{enumerate}

\overhead{Nash equilibria for sequential games}

Calculated as in the theorem, with \darkblue{$R=\R^n$} and \darkblue{$\phi_i = \sup$}.


\vfill

(Simultaneous games are a completely different story.)

\overhead{Dependent product of selection functions} 

  Sometimes  the allowed moves depend on the played moves at previous rounds.

\vfill

  Consider ``dependent product''
\darkblue{$T X \times (X \to T Y) \to T(X \times Y)$}. 

%specialized to \darkblue{$T=J$} and \darkblue{$T=K$}.

\vfill

E.g., \darkblue{$\left(\exists x \in X. \forall y \in Y_x.p(x,y)\right) =
  (\phi \otimes \gamma)(p)$} for \darkblue{$\phi = \exists_X$} and
  \darkblue{$\gamma(x) = \forall_{Y_x}$}.

\vfill

Iterating this (in)finitely often, we get
\[
\darkblue{\prod_i \left(\prod_{k < i} X_k \to T X_i \right) \to T\left(\prod_i X_i\right)}.
\]

Optimal plays and strategies calculated using this.

\overhead{Let's run an example in the computer}

If there is enough time left.


\overhead{IV. Proof theory.}

Algebras of the monad \darkblue{$J$}:

\qquad \darkblue{$J A \to A$}.

\qquad \darkblue{$((A \to R) \to A) \to A$}.

\qquad Propositions that satisfy \darkblue{Peirce's Law}.

\vfill

\darkblue{Get proof translation that eliminates Peirce's Law directly.}

Connection with the double-negation translation via the morphism \darkblue{$J \to K$}.

\overhead{Double negation shift}

$\forall i \in \N.\darkblue{\neg \neg} A(i) \implies \darkblue{\neg \neg} \forall i \in \N.A(i)$.

\vfill

Used by Spector (1962) to interpret the classical axiom of countable choice.

\vfill

Can be written as a \darkblue{$K$-shift}:

\qquad 
$\forall i \in \N. \darkblue{K} A(i) \implies \darkblue{K} \forall i \in \N.A(i)$.

\qquad

Non-intuitionistic principle, realized by \emph{Spector bar recursion}.

\overhead{$J$-shift}

The countable product functional \darkblue{$\bigotimes \colon \prod_i J X_i \to J \prod_i X_i$} realizes the \darkblue{$J$-shift}

\vfill

\qquad 
$\forall i \in \N. \darkblue{J} A(i) \implies \darkblue{J} \forall i \in \N.A(i)$.

\vfill

More general than the \darkblue{$K$}-shift.

\vfill

\darkblue{$\bigotimes$} is yet another form of bar recursion.



\overhead{Unifying concept: selection functions for quantifiers}

\vfill

\darkblue{I.} Topology in computation.

\grey{Exhaustive search.}

\vfill

\darkblue{II.} The selection monad.

\grey{Selection functions for generalized quantifiers.}

\vfill

\darkblue{III.} Game theory.

\grey{Optimal plays and strategies.}

\vfill

\darkblue{IV.} Proof theory.

\grey{Computational extraction of witnesses from classical proofs.}


\overhead{Unifying concept: selection functions for quantifiers}

The product of selection functions \darkblue{$\bigotimes \colon \prod_i J X_i \to J \prod_i X_i$} gives:

~

\darkblue{I.} Topology in computation: \darkblue{Tychonoff theorem.}

\vfill

\darkblue{II.} The selection monad: \darkblue{Strength.}

\vfill

\darkblue{III.} Game theory: \darkblue{Optimal plays and strategies.}

\vfill

\darkblue{IV.} Proof theory: \darkblue{Generalized double-negation shift, bar recursion.}

~

\darkblue{\phantom{Thanks!}}

~

\vfill

\overhead{Unifying concept: selection functions for quantifiers}

The product of selection functions \darkblue{$\bigotimes \colon \prod_i J X_i \to J \prod_i X_i$} gives:

~

\darkblue{I.} Topology in computation: \darkblue{Tychonoff theorem.}

\vfill

\darkblue{II.} The selection monad: \darkblue{Strength.}

\vfill

\darkblue{III.} Game theory: \darkblue{Optimal plays and strategies.}

\vfill

\darkblue{IV.} Proof theory: \darkblue{Generalized double-negation shift, bar recursion.}

~

\darkblue{Thanks!}

~

\vfill

\overhead{References}

%\vspace{-2ex}

\begin{small}
  
\noindent\grey{1.}
MHE. \darkblue{Synthetic topology of data types and classical spaces.}
ENTCS'2004.

\noindent\grey{2.}
MHE. \darkblue{Infinite sets that admit fast exhaustive search.} LICS'2007.

\noindent\grey{3.}
MHE. \darkblue{Exhaustible sets in higher-type computation.} LMCS'2008.

\noindent\grey{4.}
MHE. \darkblue{Computability of continuous solutions of higher-type equations},
LNCS'2009.

\noindent\grey{5.}
MHE~\&~PO. \darkblue{Selection functions, bar recursion, and backward
  induction}, MSCS'2010.

\noindent\grey{6.} MHE~\&~PO. \darkblue{Searchable Sets, Dubuc-Penon Compactness, Omniscience Principles,} \\
\noindent\phantom{\grey{6.} MHE~\&~PO.}
\darkblue{and the Drinker Paradox}. CiE'2010.

\noindent\grey{7.}
MHE~\&~PO. \darkblue{The Peirce translation and the double negation shift}. 
LNCS'2010.

\noindent\grey{8.}
MHE~\&~PO. \darkblue{Computational interpretations of analysis via
  products of selection functions}, \\
\noindent\phantom{\grey{8.} MHE~\&~PO.}
LNCS'2010.

% \noindent\grey{9.} MHE~\&~PO. \darkblue{Selection functions in computability and proof theory}. EPSRC grant application, 2009. 

\end{small}


\overhead{Links}

\begin{footnotesize}
  
\url{http://math.andrej.com/2007/09/28/seemingly-impossible-functional-programs/}

\url{http://math.andrej.com/2008/11/21/a-haskell-monad-for-infinite-search-in-finite-time/}

\url{http://www.cs.bham.ac.uk/~mhe/papers/index.html}

Maybe add links to the Haskell programs here.

\end{footnotesize}



\end{document}
