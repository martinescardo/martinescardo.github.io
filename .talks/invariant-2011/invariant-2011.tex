\documentclass%
[%dvips,% for generating postscript printouts
%25pt,%
%a4paper,%
%landscape
Screen4to3,
%Screen16to9
]{foils}

\newcommand{\Bool}{\operatorname{Bool}}
\newcommand{\prodf}{\operatorname{\darkblue{prod}}}

\newcommand{\hash}{\operatorname{\sharp}}

\newcommand{\bnf}{\mathrel{::=}}
\newcommand{\lift}[1]{#1_{\bot}}
\newcommand{\N}{\mathbb{N}}
\newcommand{\R}{\mathbb{R}}
%\newcommand{\Bool}{\mathbb{B}}
\newcommand{\Sierp}{\mathcal{S}}
\newcommand{\pN}{\mathcal{N}}
\newcommand{\pBool}{\mathcal{B}}
\newcommand{\If}{\,\mathrel{\operatorname{if}}}
\newcommand{\Then}{\mathrel{\operatorname{then}}}
\newcommand{\Else}{\mathrel{\operatorname{else}}}
\newcommand{\True}{\operatorname{True}}%{\operatorname{true}}
\newcommand{\False}{\operatorname{False}}%{\operatorname{false}}
\newcommand{\domain}[1]{{\D_{#1}}}
\newcommand{\total}[1]{{\T_{#1}}}
\newcommand{\totaleq}[1]{\sim_{#1}}
\newcommand{\quo}[1]{\qq_{#1}}
\newcommand{\qq}{\rho}
\newcommand{\D}{D}
\newcommand{\E}{E}
\newcommand{\C}{C}
\newcommand{\T}{T}
\newcommand{\kk}[1]{{\C_{#1}}}
\newcommand{\Scott}{\operatorname{Scott}}
\newcommand{\tproduct}{\sigma \times \tau}
\newcommand{\tfunction}{\sigma \to \tau}
\newcommand{\rh}[1]{\rho_{#1}}
\newcommand{\comp}{\circ}
\newcommand{\siff}{\iff}%{\Leftrightarrow}
\newcommand{\Search}{\operatorname{\mathcal{S}}}
\newcommand{\g}{\operatorname{\text{\boldmath{$g$}}}}
\newcommand{\f}{\operatorname{\text{\boldmath{$f$}}}}
\newcommand{\x}{\operatorname{\text{\boldmath{$x$}}}}
\newcommand{\e}{\operatorname{\text{\boldmath{$e$}}}}
\newcommand{\sK}{\operatorname{\text{\boldmath{$K$}}}}
\newcommand{\overhead}[1]{\foilhead[-4ex]{\darkblue{#1}}}
\newcommand{\cons}{\hash}

%\usepackage[notref,notcite]{showkeys}
%\usepackage[light,none,bottom]{draftcopy}

% \now command (By Manfred Kerber and then Riccardo Poli)

\usepackage{calc}
\newcounter{hours}\newcounter{minutes}
\newcommand{\now}{ %
\setcounter{hours}{\time/60} %
\setcounter{minutes}{\time-\value{hours}*60} %
\ifnum\theminutes <10 \thehours :0\theminutes\else \thehours:\theminutes\fi}

\newcommand{\myalpha}{s}
\newcommand{\mybeta}{t}

% sample usage: \date{\today \now\ hours}

%\usepackage{a4wide}

%\usepackage{a4}

%\usepackage{times}

\usepackage{url}

\newcommand{\qqquad}{\qquad\qquad}

\usepackage[english]{babel}

\usepackage{latexsym,amssymb,amsmath,stmaryrd,color}

\newcommand{\myparagraph}[1]{\vspace{2ex}\noindent{\bf #1}}

% mathematical macros

\newcommand{\Top}{\operatorname{Top}}
\newcommand{\tinduced}[3]{{#1}({#2},{#3})} % subbasic in the induced topology
%\newcommand{\E}{\varepsilon}
\newcommand{\eval}{\operatorname{\varepsilon}}%{\operatorname{eval}}
\newcommand{\curry}[1]{\overline{\scriptstyle #1}}
\newcommand{\Curry}[1]{\overline{#1}}
%\newcommand{\N}{\mathbb{N}}
\newcommand{\K}{\mathbb{K}}
\newcommand{\NN}{\mathcal{N}}
\newcommand{\Z}{\mathbb{Z}}
%\newcommand{\R}{\mathbb{R}}
\newcommand{\RR}{\mathcal{R}}
\newcommand{\Q}{\mathbb{Q}}
\newcommand{\B}{2}%{\mathbb{B}}
\newcommand{\Nat}{\operatorname{Nat}}
%\newcommand{\Bool}{\operatorname{Bool}}
\newcommand{\dual}[1]{{#1}^\wedge} % furst Lawson dual
\newcommand{\dualdual}[1]{{#1}^{\wedge\wedge}} % second Lawson dual
%\newcommand{\e}{e} % canonical map into second Lawson dual
\renewcommand{\d}{d} % its inverse when it exists.
\newcommand{\onlineremark}[1]{{\footnotesize(#1)}}
\newcommand{\old}[1]{}

\newcommand{\U}{{\cal U}}
\newcommand{\V}{{\cal V}}
\newcommand{\assembly}{\operatorname{N}}
\newcommand{\preassembly}{\operatorname{P}}
\newcommand{\Vstar}{\V^*}
\newcommand{\op}{\rm op}
\newcommand{\powerset}{\operatorname{\mathcal{P}}}
\newcommand{\pneg}{\operatorname{\underline{\neg}}}
\newcommand{\phull}[1]{\underline{#1}}
\newcommand{\wellinside}{\eqslantless}
\newcommand{\waybelow}{\ll}
\newcommand{\eqdef}{\stackrel{{\rm\scriptscriptstyle  def}}{=}}
\newcommand{\himplies}{\Rightarrow}
\newcommand{\functorfont}[1]{\operatorname{#1}}
\newcommand{\catfont}[1]{\mathbf{#1}}
\newcommand{\Loc}{\catfont{Loc}}
\newcommand{\Frm}{\catfont{Frm}}
\newcommand{\Set}{\catfont{Set}}
%\newcommand{\K}{K} % was{\operatorname{K}} % ``kelleyification''
%\newcommand{\Sierp}{\mathbb{S}}
\newcommand{\Stone}{\catfont{Stone}}
\newcommand{\Coh}{\catfont{Spec}}
\newcommand{\CReg}{\catfont{KReg}}
\newcommand{\RLC}{\catfont{RLK}}
\newcommand{\SC}{\catfont{SK}}
\newcommand{\SLC}{\catfont{SLK}}
\newcommand{\BReg}{\catfont{BReg}}

%\newcommand{\clsb}[2]{#2_{#1}}
\newcommand{\clsb}[2]{#1\setminus #2}
\newcommand{\open}{\circ}
\newcommand{\closed}{\boxempty}
\newcommand{\cocompact}[1]{\kappa_{#1}}
\newcommand{\Patch}{\operatorname{Patch}}
\newcommand{\Opens}{\operatorname{\mathcal{O}}}%[1]{\operatorname{\mathcal{O}}{#1}} 
\renewcommand{\O}{\Opens}
\newcommand{\Cocompacts}{\operatorname{\mathcal{C}}}
%\newcommand{\C}[2]{\operatorname{C}(#1,#2)}
\newcommand{\Regions}{\operatorname{\mathcal{R}}}%[1]{\operatorname{\mathcal{R}}{#1}} 
\newcommand{\frameleft}[1]{#1^{*}} \newcommand{\frameright}[1]{#1_{*}}
\newcommand{\arrow}{\to}%{\rightarrow}
%\newcommand{\comp}{\mathrel{\circ}}
\newcommand{\id}{\operatorname{id}}
\newcommand{\join}{\sqcup}
\newcommand{\meet}{\sqcap}
\newcommand{\himp}{\mathrel{\Rightarrow}}
\newcommand{\bigjoin}{\bigsqcup}
\newcommand{\dirjoin}{\sideset{}{^{\uparrow}}\bigjoin}
\newcommand{\bigmeet}{\bigsqcap}
\newcommand{\below}{\sqsubseteq}
\renewcommand{\above}{\sqsupseteq}
\newcommand{\farrow}[1]{\stackrel{#1}{\longrightarrow}}
\newenvironment{eqn}{\[\begin{array}{rcll}}{\end{array}\]}
\newcommand{\qtext}[1]{\quad\text{#1}}
\newcommand{\up}{\operatorname{\uparrow}}
\newcommand{\down}{\operatorname{\downarrow}}
\newcommand{\upup}{\mathord{\hbox{\makebox[0pt][l]{\raise .6mm\hbox{$\uparrow$}}$\uparrow$}}}
\newcommand{\fix}{\operatorname{fix}}
\newcommand{\Infl}[1]{\hat{#1}}
\newcommand{\infl}{\Infl{\top}} %{\Infl{\operatorname{\i d}}}


% defined terms and index entries

\newcommand{\fontfordefinition}[1]{\emph{#1}}%{{\bfseries\itshape #1}}
% \newcommand{\df}[1]{\fontfordefinition{#1}} % no index required
\newcommand{\dfkey}[2]{\fontfordefinition{#1}\index{#2}}
\newcommand{\dfemph}[1]{\emph{#1}\index{#1}}
\newcommand{\cindex}[1]{#1} % was {#1\index{#1}}
\newcommand{\tindex}[1]{#1\index{#1}}
\newcommand{\df}[1]{\dfkey{#1}{#1}}
\newcommand{\dfn}[1]{\fontfordefinition{#1}}
\newcommand{\ind}[1]{#1\index{#1}}
\newcommand{\indemph}[1]{\emph{#1}\index{#1}}

\makeindex

% Omission macro

\newcommand{\Omit}[1]{}

% Chronology macro

\newcommand{\chron}[1]{}
                     %{\begin{footnotesize}(Chronology: #1\end{footnotesize})}

\newcommand{\smalltext}[1]{\begin{footnotesize} #1 \end{footnotesize}}

%\newcommand{\myparagraph}[1]{\medskip\emph{#1}}

\newcommand{\lbracket}{(}
\newcommand{\rbracket}{)}
\newcommand{\QTop}{\operatorname{{QTop}}}

\newcommand{\numbered}[1]{\grey{#1}}

\definecolor{darkgreen}{rgb}{0,0.5,0}
\definecolor{darkblue}{rgb}{0,0.0,0.5}
\definecolor{darkred}{rgb}{1,0,0}
\definecolor{grey}{rgb}{0.4,0.4,0.4}
\newcommand{\black}{\textcolor{black}}
\newcommand{\grey}{\textcolor{grey}}
\newcommand{\blue}{\textcolor{blue}}
\newcommand{\red}{\textcolor{red}}
\newcommand{\darkblue}{\textcolor{darkblue}}
\newcommand{\darkred}{\textcolor{darkred}}
\newcommand{\darkgreen}{\textcolor{darkgreen}}
\newcommand{\green}{\textcolor{green}}
\newcommand{\yellow}{\textcolor{yellow}}
\newcommand{\cyan}{\textcolor{cyan}}
\newcommand{\magenta}{\textcolor{magenta}}

\newcommand{\qed}{}%{$\boxempty$}

\newcommand{\sexample}[1]{\noindent\textbf{Example} \blue{#1} }
\newcommand{\slemma}[1]{\noindent\darkgreen{\textbf{Lemma}} \\[1ex] {\emtheorem #1}}
\newcommand{\ulemma}[1]{\noindent\darkgreen{\textbf{Lemma}} {\emtheorem #1}}
\newcommand{\proposition}[1]{\noindent\darkgreen{\textbf{Proposition}} {\emtheorem #1}}
\newcommand{\proof}[1]{\noindent{\textbf{Proof}} {#1}}
\newcommand{\sdefinition}[1]{\noindent\darkgreen{\textbf{Definition}} \\[1ex] {\emtheorem #1}}
\newcommand{\scorollary}[1]{\noindent\darkgreen{\textbf{Corollary}} {\emtheorem #1}}
\newcommand{\stheorem}[1]{\noindent\darkgreen{\textbf{Theorem}}  {\emtheorem #1}}
\newcommand{\texample}[2]{\noindent\textbf{Example} \blue{#1}}
\newcommand{\tlemma}[2]{\noindent\darkgreen{\textbf{Lemma} (#1)}  \\[1ex] {\emtheorem #2}}
\newcommand{\ttlemma}[2]{\noindent\darkgreen{\textbf{Lemma} (#1)} {\emtheorem #2}}
\newcommand{\utlemma}[2]{\noindent\darkgreen{\textbf{Lemma} (#1)} {\emtheorem #2}}
\newcommand{\tdefinition}[2]{\noindent\darkgreen{\textbf{Definition} (#1)} \\[1ex] {\emtheorem #2}}
\newcommand{\tcorollary}[2]{\noindent\darkgreen{\textbf{Corollary} (#1)}  \\[1ex] {\emtheorem #2}}
\newcommand{\ttheorem}[2]{\noindent\darkgreen{\textbf{Theorem} (#1)}  \\[1ex] {\emtheorem #2}}

\newcommand{\utheorem}[2]{\noindent\darkgreen{\textbf{Theorem} (#1)} {\emtheorem #2}}

\newcommand{\uapplication}[1]{\noindent\darkgreen{\textbf{Application}} {#1}}

\newcommand{\emtheorem}{\em}

\LogoOff

\newcommand{\twostar}{2^{*}}
\newcommand{\twoomega}{2^\infty}
\newcommand{\prefix}{\sqsubseteq}
\newcommand{\all}[3]{\forall\, #1 \,{\in}\, #2\,.\left(#3\right)}
\newcommand{\some}[3]{\exists\, #1 \,{\in}\, #2\,.\left(#3\right)}
\newcommand{\exactlyone}[3]{\exists!\, #1 \,{\in}\, #2\,.\left(#3\right)}
\newcommand{\lam}[3]{\lambda #1 \,{\in}\, #2\,.\left(#3\right)}
\newcommand{\uall}[2]{\forall\, #1\,.\left(#2\right)}
\newcommand{\usome}[2]{\exists\, #1\,.\left(#2\right)}
\newcommand{\uexactlyone}[3]{\exists!\, #1\,.\left(#2\right)}
\newcommand{\ulam}[2]{\lambda #1 .\left(#2\right)}
\newcommand{\xall}[3]{\forall\, #1 \,{\in}\, #2\,.\,#3}
\newcommand{\xsome}[3]{\exists\, #1 \,{\in}\, #2\,.\,#3}
\newcommand{\xexactlyone}[3]{\exists!\, #1 \,{\in}\, #2\,.\,#3}
\newcommand{\xuall}[2]{\forall\, #1\,.\,#2}
\newcommand{\xusome}[2]{\exists\, #1\,.\,#2}
\newcommand{\xuexactlyone}[2]{\exists!\, #1,.\,#2}
\newcommand{\xlam}[3]{\lambda #1 \,{\in}\, #2\,.\,#3}
\newcommand{\xulam}[2]{\lambda #1 .\,#2}
\newcommand{\tlam}[3]{\lambda #1 \,{:}\, #2\,.\,\left(#3\right)}
\newcommand{\xtlam}[3]{\lambda #1 \,{:}\, #2\,.\,#3}


%%% Local Variables: 
%%% mode: latex
%%% TeX-master: "escardo-mfps2010"
%%% End: 


%\usepackage{hyperref}

\newcommand{\db}{\darkblue}
\newcommand{\myquad}{\phantom{xxxx}}
\newcommand{\mybox}[2]{\fbox{\parbox{#1\linewidth}{#2}}}
\newcommand{\next}{\operatorname{next}}
\newcommand{\myitem}[1]{\grey{#1}}

\newcommand{\Maybe}{\operatorname{Maybe}}

\usepackage{parskip}

\title{\darkblue{Mathematics of computation with infinite objects}}

\author{\darkblue{Mart\'{\i}n Escard\'o} \\ School of Computer Science, University of Birmingham}


%  \\[10ex]
% Part of this talk is joint work with \darkblue{Paulo Oliva} from Queen Mary, London. \\[8ex] ~}

\date{\sc Invariants, Oxford, 25th Oct 2011}

\begin{document}

\setlength\textheight{\paperheight}
\setlength\headheight{-7ex}
\setlength\footskip{20ex}
%\addtolength\foilheadskip{-20ex}
%\addtolength\voffset{20ex}

\maketitle

\overhead{Let me start with six questions for you}

Here is a hint:

\vfill


\begin{quote}
2 questions have \darkblue{positive} answer.

2 questions have \darkblue{negative} answer. 

1 question has \darkblue{unknown} answer. 

1 question \darkblue{doesn't fall in the above three categories}.
\end{quote}

\vfill

\overhead{Question 1}

Suppose you have written computer programs for calculating functions
\[
\darkblue{f, g \colon \Z \to \Z}.
\]
Can the computer decide, in finite time, whether or not \darkblue{$f = g$}?

\vfill

Here \darkblue{$\Z$} is the set of integers. 

\vfill
\darkblue{Input:} Two functions \darkblue{$f, g \colon \Z \to \Z$}.

\darkblue{Output:} An element of the set \darkblue{$2 = \{0,1\}$}.

\vfill

\overhead{Question 2}

Consider the set \darkblue{$S$} of infinite binary sequences \db{$x_0,x_1,x_2,\dots, x_n, \dots$} with \darkblue{$x_i \in 2$}.

(For example we can use them for representing real numbers in binary.)

\vfill

Does it make sense to write programs for computing e.g. functions
\begin{eqnarray*}
  \darkblue{f  \colon} & \darkblue{S \to \Z,} & \qquad \text{(given a sequence, calculate an integer);}\\
  \darkblue{g  \colon} & \darkblue{S \to S,} & \qquad \text{(given a real number, calculate another real number)}.
\end{eqnarray*}

\vfill

\overhead{Question 3}

Suppose you have written computer programs for calculating functions
\[
\darkblue{f, g \colon S \to \Z}.
\]
Can the computer decide, in finite time, whether or not \darkblue{$f = g$}?


\vfill
\darkblue{Input:} Two functions \darkblue{$f, g \colon S \to \Z$}.

\darkblue{Output:} An element of the set \darkblue{$2 = \{0,1\}$}.

\vfill

\overhead{Question 4}

Suppose you have written computer programs for calculating functions
\[
\darkblue{f, g \colon S \to S}
\]
Can the computer decide, in finite time, whether or not \darkblue{$f = g$}?


\vfill
\darkblue{Input:} Two functions \darkblue{$f, g \colon S \to S$}.

\darkblue{Output:} An element of the set \darkblue{$2 = \{0,1\}$}.

\vfill

\overhead{Question 5}

Can we plot the Mandelbrot set?

Given a point that is not in the boundary, can the computer tell
whether it is inside or outside the set?

\overhead{Question 6}

It is easy to write a program that systematically lists all provable
statements.

 (Let's say formally written in the language of set theory.)

\vfill

We can also write it so that it stops if it ever lists a statement \darkblue{$A$} and also its negation \darkblue{$\neg A$}. 

\vfill

Will this program stop? 

\vfill

\overhead{Answers}

\begin{quote}
2 questions have \darkblue{positive} answer.

2 questions have \darkblue{negative} answer. 

1 question has \darkblue{unknown} answer. 

1 question \darkblue{doesn't fall in the above three categories}.

\quad (But what else can we say apart from yes, no or don't know???)
\end{quote}

Let's leave the questions unanswered for the moment.

\overhead{A simple computation}

Fix a constant \darkblue{$a \in [0,4]$}.

Given \darkblue{$x \in [0,1]$}, it is easy to see that \darkblue{$ax(1-x) \in [0,1]$}.

\vfill

Rudimentary model of population growth \grey{(Pierre Verhulst (1840s))}.
\begin{quote}
\darkblue{$x = 0$:} zero fish in a lake. \\
\darkblue{$x = 1$:} the maximum capacity of fish in the lake. \\
\darkblue{$x_n$} fish in the \darkblue{$n$th} generation
$\implies$ \darkblue{$ax_n(1-x_n)$} fish in the next generation.
\end{quote}

\vfill

\overhead{A simple computation}

Given \darkblue{$x_0 \in [0,1]$}, define a sequence by the recurrence
\[
\darkblue{x_{n+1} = ax_n(1-x_n)}.
\]
We want to compute \darkblue{$x_{60}$}.

\darkblue{How many fish will the lake have in the 60th generation?}

\overhead{Let's calculate with a computer}

Choose \darkblue{$a = 4$} and \darkblue{$x_0 = 43 / 64 = 0.671875$}.

\grey{Many other choices will have a similar  effect.}

\vfill

Using \darkblue{7} decimal digits \grey{(simple precision)} and
\darkblue{16} \grey{(double precision)} for the calculations, the
computer claims that \darkblue{$x_{60}$} is

\begin{verbatim}
     0.934518     0.928604
\end{verbatim}
Rounding to two decimal digits, we get \darkblue{$x_{60} \approx 0.93$}.

\vfill

\darkblue{Is that correct?}

\overhead{Let's show intermediate generation counts}

This is what the computer claims:
\begin{verbatim}
 n      simple       double           
-----------------------------
 0    0.671875      0.671875
 5    0.384327      0.384327
10    0.313034      0.313037
15    0.022702      0.022736
20    0.983813      0.982892
25    0.652837      0.757549 <------ start to differ
30    0.934927      0.481445 <------ completely different
35    0.848153      0.313159
40    0.057696      0.024009
45    0.991613      0.930892
50    0.042174      0.625693
55    0.108415      0.637033
60    0.934518      0.928604 <------ similar by coincidence
\end{verbatim}
Both are completely wrong. The correct answer is \darkblue{$x_{60} = 0.315445$}.

\overhead{How to get this right using a computer}

Could use rational arithmetic, with arbitrary-size integers for the
numerators and denominators, keeping then in lowest terms.

\vfill

\red{But:} The computation of \darkblue{$x_n$} runs out of memory for
\darkblue{$n \ge 30$} or so.

\vfill

With 1Gb of memory. \grey{This is a lot of memory!}

\vfill

\grey{Pause to run something in the computer.}

\vfill

\overhead{How to get this right using a computer}

Code numbers using infinitely many digits.

\vfill

\grey{Pause to run something in the computer.}

\vfill


\overhead{Computing with infinite sequences}

\darkblue{$\Z^\N$} space of infinite sequences of natural numbers.

Schematic model of computation for functions \darkblue{$f \colon \Z^\N \to \Z^\N$}:

Given \darkblue{$\alpha \in \Z^\N$}, calculate \darkblue{$\beta = f(\alpha) \in \Z^\N$} as follows:

\begin{diagram}
  \text{\phantom{x}} & \rTo^{\darkblue{\dots \alpha_3, \alpha_2, \alpha_1, \alpha_0}} & \mybox{0.6}{In order to calculate \darkblue{$f$}, repeatedly do: \\
\myquad\myitem{1.} read finitely many integers \\
\myquad\myitem{2.} calculate for a finite amount of time \\
\myquad\myitem{3.} write finitely many integers
} & 
 \rTo^{\darkblue{\dots \beta_3, \beta_2, \beta_1, \beta_0}}
\end{diagram}

\overhead{Example}

Suppose \db{$f \colon \Z^\N \to \Z^\N$} codes the logistic map \db{$x \mapsto
4x(1-x)$} in decimal notation. 

Want to calculate \db{$f(f(f(f(x_0))))$}.

We get the following pipeline:
\[
\db{\longrightarrow \text{\framebox{$f$}}\longrightarrow \text{\framebox{$f$}}\longrightarrow \text{\framebox{$f$}}\longrightarrow \text{\framebox{$f$}}\longrightarrow}
\]
Given a pipeline of digits for \db{$x_0$}, this will produce a pipeline for digits for \db{$x_4$}.

\red{NB.} In the programming language I am using, you still write
\[
\darkblue{x_4 = f(f(f(f(x_0)))).}
\]
The pipeline is created by the internal mechanisms of the language.

\overhead{First continuity theorem}

If \db{$f \colon \Z^\N \to \Z^\N$} is computable in the above fashion,
then finite segments of the output can only depend on finite segments
of the input.

\vfill

To make this precise, for \db{$\alpha, \beta \in \Z^\N$} and \db{$p \in \N$}, define
\begin{quote}
  \db{$\alpha =_p \beta \iff \alpha_n = \beta_n$} for all \db{$i < p$}.
\end{quote}
We say that \db{$\alpha$} and \db{$\beta$} agree with precision \db{$p$}.

\vfill

\overhead{First continuity theorem}

\darkgreen{Theorem.}  If \db{$f \colon \Z^\N \to \Z^\N$} is computable
in the above fashion, then for every input \db{$\alpha \in \Z^\N$} and
every desired output precision \db{$p \in \N$} 
there is a sufficient input precision \db{$q \in \N$}
such that
\[
\db{\alpha =_q \beta \implies f(\alpha) =_p f(\beta)}.
\]

\vfill

To calculate \db{$f(\alpha)$} with precision \db{$p$}, it is enough to
know \db{$\alpha$} with precision \db{$q$}.

\vfill

\overhead{First continuity theorem}

\darkgreen{Theorem.}  If \db{$f \colon \Z^\N \to \Z^\N$} is computable
in the above fashion, then for every input \db{$\alpha \in \Z^\N$} and
every desired output precision \db{$p \in \N$} 
there is a sufficient input precision \db{$q \in \N$}
such that
\[
\db{\alpha =_q \beta \implies f(\alpha) =_p f(\beta)}.
\]

To calculate \db{$f(\alpha)$} with precision \db{$p$}, it is enough to
know \db{$\alpha$} with precision \db{$q$}.

\vfill

\red{Compare with:}

\darkgreen{Definition.}  A function \db{$f \colon \R \to \R$} is
called continuous iff for every input
\db{$x \in \R$} and every desired output precision \db{$\epsilon > 0$}
there is a sufficient input precision
\db{$\delta > 0$} such that
\[
\db{|x - y| < \delta \implies |f(x) - f(y)| < \epsilon}.
\]

\vfill





\overhead{Computing with decimal notation}

\newarrow{Mapsto}|--->
\newarrow{Dotsto}....>


\begin{diagram}
 & \Z^\N & & \Z^\N \\
 & \uInto^{\text{subspace}} & & \uInto_{\text{subspace}} \\
\alpha & \text{decimal expansions} & \rDotsto^{\text{calculate here}} & \text{decimal expansions} & \beta \\
\dMapsto^{\text{\small decoding map}} & \dTo &   & \dTo & \dMapsto_{\text{\small decoding map}}  \\
\sum_{n \ge 0} \alpha_i 10^{-n} & \R & \rTo_{\text{given $f$}} & \R & \sum_{n \ge 0} \beta_i 10^{-n} \\
\end{diagram}


~

Assume $\alpha_0 \in \Z$ and $\alpha_{i+1} \in \{ 0,\dots,9\}$.

\overhead{Computing with decimal notation}

\newarrow{Mapsto}|--->
\newarrow{Dotsto}....>


\begin{diagram}
\text{decimal expansions} & \rDotsto^{\text{computer implementation of $f$}} & \text{decimal expansions}\\
 \dTo^{\text{\small decoding map}} &   & \dTo_{\text{\small decoding map}}  \\
 \R & \rTo_{\text{$f$}} & \R & 
\end{diagram}

\vfill

\darkgreen{Second Continuity Theorem.}
If \db{$f$} has a computer implementation, then it is continuous. 

\overhead{But decimal notation isn't good for infinite computation}

This was discovered by Brouwer in the 1920's.

\vfill

Not even multiplication by three can be computed.

\vfill

\overhead{But decimal notation isn't good for infinite computation}

This was discovered by Brouwer in the 1920's.

\begin{diagram}
\text{decimal expansions} & \rDotsto^{\text{nothing continuous here}} & \text{decimal expansions}\\
 \dTo^{\text{\small decoding map}} &   & \dTo_{\text{\small decoding map}}  \\
 \R & \rTo_{\text{$f(x)=3x$}} & \R & 
\end{diagram}

Suppose the input starts \darkblue{$0.333333$}. 

$\bullet$ The first digit of the output must be \darkblue{$0$} if we eventually
read a digit \darkblue{$< 3$}.

$\bullet$ The first digit of the output must be \darkblue{$1$} if we eventually
read a digit \darkblue{$> 3$}.

$\bullet$ While we read digits \darkblue{$=3$}, we cannot produce any output for
lack of information.

\darkblue{$\implies$} If the input is \darkblue{$1/3$}, the next
output digit depends on \emph{all} input digits.

\overhead{Other bases have the same problem}

So changing to e.g.\ binary notation doesn't help.

\overhead{Perhaps shocking, but no reason to abandon hope}

Already discussed by Brouwer in the 1920's.

\vfill

Change the notation for real numbers.

\vfill

The problem is boundary cases. 

\vfill

There are many theoretical solutions, used in practice.

\vfill

I will discuss one of them (the one I used above in practice).

\vfill

\overhead{Use binary or decimal notation with negative digits}

Actually Cauchy (1840) proposed this to simplify calculation by hand.

\vfill

\db{$1\overline{8}40 = 1000 - 800 + 40 = 240$}.

\vfill

\red{Intuition.} If we are stuck as above, make a reasonable guess, and adjust it later using a negative digit if necessary.

\vfill

\overhead{Computing with decimal notation with negative digits}

\newarrow{Mapsto}|--->
\newarrow{Dotsto}....>


\begin{diagram}
\text{decimal expansions} & \rDotsto^{\text{implementation of $f$}} & \text{decimal expansions}\\
 \dTo^{\text{\small decoding map}} &   & \dTo_{\text{\small decoding map}}  \\
 \R & \rTo_{\text{$f$}} & \R & 
\end{diagram}

\vfill

\darkgreen{Second Continuity Theorem Still holds.}
If \db{$f$} has a computer implementation, then it is continuous. 

\darkgreen{Third Continuity Theorem.}
If \db{$f$} is continuous, then it has a continuous implementation.

\vfill

\overhead{But is this just a hack?}

A coding \db{$A \subseteq \Z^\N$} with decoding map
 \[ \db{\text{decode}_A \colon A \to \R} \] 
is called \red{admissible} if given any other
coding \db{$B \subseteq \Z^\N$} with decoding map
 \[ \db{\text{decode}_B \colon B \to \R}, \] 
there is a computer program 
\[
\db{\text{translate}  \colon B \to A} 
\]
that translates from \db{$B$} codes to \db{$A$} codes.


\db{Theorem.}
\begin{enumerate}
\item Decimal notation with negative digits is an admissible coding.
\item Any two admissible codings can be translated to each other.
\end{enumerate}


\overhead{More general picture}

\newarrow{Into}C--->

\begin{diagram}
  \Z^\N & & \Z^\N \\
  \uInto^{\text{subspace}} & & \uInto_{\text{subspace}} \\
  A & \rTo^{\text{concrete $f$}} & B & \text{\darkblue{concrete spaces}}\\ 
  \dTo^{\text{decoding map}} &   & \dTo_{\text{decoding map}} \\
  X & \rTo_{\text{mathematical $f$}} & Y & \text{\darkblue{abstract spaces}}\\
\end{diagram}

Similar continuity theorems.

Notion of admissibility also defined for this more general situation.

\overhead{Dictionary between topology and computation}

\darkblue{Continuous map.} Computable function.

\darkblue{Open set.} Semi-decidable set.

\darkblue{Closed and open set.} Decidable set.

\darkblue{Compact set.} Exhaustively searchable set.

\darkblue{Discrete space.} Type with decidable equality.

\darkblue{Hausdorff space.} Space with semi-decidable apartness.

\vfill
\noindent
Take a theorem in topology, \\
apply the dictionary, \\
get a theorem in computability theory. 

\vfill

\grey{Unfortunately you have to come up with a new proof.}


\overhead{Exhaustive search}

Algorithmically check all possibilities in finite time.

\qquad Either find something.

\qquad Or else report that there is nothing to be found.

\vfill

\darkblue{Old theorem.}
A set of natural numbers is exhaustively searchable iff it is finite.

\vfill

\red{Intuition:}
How could one possibly check infinitely many cases in finite time?

\darkgreen{Proof:} Removed from this intuition (Halting problem, diagonalization).


\overhead{Computational common wisdom}

A set of whatever-you-can-think-of is exhaustively searchable iff it is finite.


\overhead{Can't always trust common wisdom}


\darkblue{E.g.} The Cantor space \darkblue{$2^\N$} is
exhaustively searchable.

\vfill

\db{Theorem}
\begin{enumerate}
\item Exhaustively searchable sets are topologically compact.
\item They are precisely the computable images of the Cantor space.
\end{enumerate}

\vfill

% \overhead{We could finish now}

% And revisit the questions I posed to you.

% \vfill

% Or carry on for a little longer to show you a surprising program.

% \vfill

\overhead{Formal systems for computation}

\darkblue{Example 1.} G\"odel's system T.

\darkblue{Example 2.} Platek--Scott--Plotkin's PCF and FPC.

\darkblue{Example 3.} Escard\'o's Real PCF and Real FPC.

\darkblue{Example 4.} Practical programming languages based on (2).

\quad ML, OCaml, Haskell, \dots

\qquad Robin Milner is mainly responsible for making these

\qquad theoretical tools available for practical use with many insights.

\overhead{Some popularization references}

\begin{enumerate}
\item In Andrej Bauer's \emph{Mathematics and Computation} blog \url{http://math.andrej.com/}
\begin{enumerate}
\item Seemingly impossible functional programs
\item A Haskell monad for infinite search in finite time
\item Running a classical proof with choice in Agda
\end{enumerate}
\item Computing with real numbers represented as infinite sequences of digits in Haskell. \url{http://www.cs.bham.ac.uk/~mhe/papers/fun2011.lhs}
\item With Paulo Oliva. What Sequential Games, the Tychonoff Theorem and the Double-Negation Shift have in Common. 
\url{http://www.cs.bham.ac.uk/~mhe/papers/msfp2010/}
\end{enumerate}


\overhead{Let's quickly revisit the opening questions and then finish}

\overhead{Question 1}

Suppose you have written computer programs for calculating functions
\[
\darkblue{f, g \colon \Z \to \Z}
\]
Can the computer decide, in finite time, whether or not \darkblue{$f = g$}?

\vfill

Here \darkblue{$\Z$} is the set of integers. 

\vfill
\darkblue{Input:} Two functions \darkblue{$f, g \colon \Z \to \Z$}.

\darkblue{Output:} An element of the set \darkblue{$2 = \{0,1\}$}.

\vfill

\overhead{Question 2}

Consider the set \darkblue{$2^\N$} of infinite binary sequences \db{$x_0,x_1,x_2,\dots, x_n, \dots$} with \darkblue{$x_i \in 2$}.

(For example we can use them for representing real numbers in binary.)

\vfill

Does it make sense to write programs for computing e.g. functions
\begin{eqnarray*}
  \darkblue{f  \colon} & \darkblue{2^\N \to \Z,} & \qquad \text{(given a sequence, calculate an integer);}\\
  \darkblue{g  \colon} & \darkblue{2^\N \to 2^\N,} & \qquad \text{(given a real number, calculate another real number)}.
\end{eqnarray*}

\vfill

\overhead{Question 3}

Suppose you have written computer programs for calculating functions
\[
\darkblue{f, g \colon 2^\N \to \Z}
\]
Can the computer decide, in finite time, whether or not \darkblue{$f = g$}?


\vfill
\darkblue{Input:} Two functions \darkblue{$f, g \colon 2^\N \to \Z$}.

\darkblue{Output:} An element of the set \darkblue{$2 = \{0,1\}$}.

\vfill

\overhead{Question 4}

Suppose you have written computer programs for calculating functions
\[
\darkblue{f, g \colon 2^\N \to 2^\N}
\]
Can the computer decide, in finite time, whether or not \darkblue{$f = g$}?


\vfill
\darkblue{Input:} Two functions \darkblue{$f, g \colon 2^\N \to 2^\N$}.

\darkblue{Output:} An element of the set \darkblue{$2 = \{0,1\}$}.

\vfill

\overhead{Question 5}

Can we plot the Mandelbrot set?

Given a point that is not in the boundary, can the computer tell
whether it is inside or outside the set?

\overhead{Question 6}

It is easy to write a program that systematically lists all provable
statements.

 (Let's say formally written in the language of set theory.)

\vfill

We can also write it so that it stops if it ever lists a statement \darkblue{$A$} and also its negation \darkblue{$\neg A$}. 

\vfill

Will this program stop? 

\vfill


\overhead{Concluding discussion}

\darkblue{I.} General topology is naturally linked with infinite computation because
it is a theory of approximation.

\darkblue{II.} Category theory is also very useful to understand and apply
computation with infinite objects (e.g.\ final co-algebras, monads).

\darkblue{III.} I had to abandon the ambitious plan to also discuss constructive
mathematics. 
\begin{quote}
In a nutshell: 
\begin{enumerate}
\item 
Give up some logical principles (e.g.\ excluded middle)
and, \item
 axioms (e.g.\ powerset), and \item
develop proofs in the normal
mathematical way. \item
Automatically get programs, even if you were not
thinking of them (BHK interpretation). 
\item Linked to topology: any
function you prove to exist will be continuous. \\ (But this is a contentious subject.)
\end{enumerate}
\end{quote}




\end{document}