\documentclass%
[%dvips,% for generating postscript printouts
%25pt,%
%a4paper,%
%landscape
Screen4to3,
%Screen16to9
]{foils}

\newcommand{\Bool}{\operatorname{Bool}}
\newcommand{\prodf}{\operatorname{\darkblue{prod}}}

\newcommand{\hash}{\operatorname{\sharp}}

\newcommand{\bnf}{\mathrel{::=}}
\newcommand{\lift}[1]{#1_{\bot}}
\newcommand{\N}{\mathbb{N}}
\newcommand{\R}{\mathbb{R}}
%\newcommand{\Bool}{\mathbb{B}}
\newcommand{\Sierp}{\mathcal{S}}
\newcommand{\pN}{\mathcal{N}}
\newcommand{\pBool}{\mathcal{B}}
\newcommand{\If}{\,\mathrel{\operatorname{if}}}
\newcommand{\Then}{\mathrel{\operatorname{then}}}
\newcommand{\Else}{\mathrel{\operatorname{else}}}
\newcommand{\True}{\operatorname{True}}%{\operatorname{true}}
\newcommand{\False}{\operatorname{False}}%{\operatorname{false}}
\newcommand{\domain}[1]{{\D_{#1}}}
\newcommand{\total}[1]{{\T_{#1}}}
\newcommand{\totaleq}[1]{\sim_{#1}}
\newcommand{\quo}[1]{\qq_{#1}}
\newcommand{\qq}{\rho}
\newcommand{\D}{D}
\newcommand{\E}{E}
\newcommand{\C}{C}
\newcommand{\T}{T}
\newcommand{\kk}[1]{{\C_{#1}}}
\newcommand{\Scott}{\operatorname{Scott}}
\newcommand{\tproduct}{\sigma \times \tau}
\newcommand{\tfunction}{\sigma \to \tau}
\newcommand{\rh}[1]{\rho_{#1}}
\newcommand{\comp}{\circ}
\newcommand{\siff}{\iff}%{\Leftrightarrow}
\newcommand{\Search}{\operatorname{\mathcal{S}}}
\newcommand{\g}{\operatorname{\text{\boldmath{$g$}}}}
\newcommand{\f}{\operatorname{\text{\boldmath{$f$}}}}
\newcommand{\x}{\operatorname{\text{\boldmath{$x$}}}}
\newcommand{\e}{\operatorname{\text{\boldmath{$e$}}}}
\newcommand{\sK}{\operatorname{\text{\boldmath{$K$}}}}
\newcommand{\overhead}[1]{\foilhead[-4ex]{\darkblue{#1}}}
\newcommand{\cons}{\hash}

%\usepackage[notref,notcite]{showkeys}
%\usepackage[light,none,bottom]{draftcopy}

% \now command (By Manfred Kerber and then Riccardo Poli)

\usepackage{calc}
\newcounter{hours}\newcounter{minutes}
\newcommand{\now}{ %
\setcounter{hours}{\time/60} %
\setcounter{minutes}{\time-\value{hours}*60} %
\ifnum\theminutes <10 \thehours :0\theminutes\else \thehours:\theminutes\fi}

\newcommand{\myalpha}{s}
\newcommand{\mybeta}{t}

% sample usage: \date{\today \now\ hours}

%\usepackage{a4wide}

%\usepackage{a4}

%\usepackage{times}

\usepackage{url}

\newcommand{\qqquad}{\qquad\qquad}

\usepackage[english]{babel}

\usepackage{latexsym,amssymb,amsmath,stmaryrd,color}

\newcommand{\myparagraph}[1]{\vspace{2ex}\noindent{\bf #1}}

% mathematical macros

\newcommand{\Top}{\operatorname{Top}}
\newcommand{\tinduced}[3]{{#1}({#2},{#3})} % subbasic in the induced topology
%\newcommand{\E}{\varepsilon}
\newcommand{\eval}{\operatorname{\varepsilon}}%{\operatorname{eval}}
\newcommand{\curry}[1]{\overline{\scriptstyle #1}}
\newcommand{\Curry}[1]{\overline{#1}}
%\newcommand{\N}{\mathbb{N}}
\newcommand{\K}{\mathbb{K}}
\newcommand{\NN}{\mathcal{N}}
\newcommand{\Z}{\mathbb{Z}}
%\newcommand{\R}{\mathbb{R}}
\newcommand{\RR}{\mathcal{R}}
\newcommand{\Q}{\mathbb{Q}}
\newcommand{\B}{2}%{\mathbb{B}}
\newcommand{\Nat}{\operatorname{Nat}}
%\newcommand{\Bool}{\operatorname{Bool}}
\newcommand{\dual}[1]{{#1}^\wedge} % furst Lawson dual
\newcommand{\dualdual}[1]{{#1}^{\wedge\wedge}} % second Lawson dual
%\newcommand{\e}{e} % canonical map into second Lawson dual
\renewcommand{\d}{d} % its inverse when it exists.
\newcommand{\onlineremark}[1]{{\footnotesize(#1)}}
\newcommand{\old}[1]{}

\newcommand{\U}{{\cal U}}
\newcommand{\V}{{\cal V}}
\newcommand{\assembly}{\operatorname{N}}
\newcommand{\preassembly}{\operatorname{P}}
\newcommand{\Vstar}{\V^*}
\newcommand{\op}{\rm op}
\newcommand{\powerset}{\operatorname{\mathcal{P}}}
\newcommand{\pneg}{\operatorname{\underline{\neg}}}
\newcommand{\phull}[1]{\underline{#1}}
\newcommand{\wellinside}{\eqslantless}
\newcommand{\waybelow}{\ll}
\newcommand{\eqdef}{\stackrel{{\rm\scriptscriptstyle  def}}{=}}
\newcommand{\himplies}{\Rightarrow}
\newcommand{\functorfont}[1]{\operatorname{#1}}
\newcommand{\catfont}[1]{\mathbf{#1}}
\newcommand{\Loc}{\catfont{Loc}}
\newcommand{\Frm}{\catfont{Frm}}
\newcommand{\Set}{\catfont{Set}}
%\newcommand{\K}{K} % was{\operatorname{K}} % ``kelleyification''
%\newcommand{\Sierp}{\mathbb{S}}
\newcommand{\Stone}{\catfont{Stone}}
\newcommand{\Coh}{\catfont{Spec}}
\newcommand{\CReg}{\catfont{KReg}}
\newcommand{\RLC}{\catfont{RLK}}
\newcommand{\SC}{\catfont{SK}}
\newcommand{\SLC}{\catfont{SLK}}
\newcommand{\BReg}{\catfont{BReg}}

%\newcommand{\clsb}[2]{#2_{#1}}
\newcommand{\clsb}[2]{#1\setminus #2}
\newcommand{\open}{\circ}
\newcommand{\closed}{\boxempty}
\newcommand{\cocompact}[1]{\kappa_{#1}}
\newcommand{\Patch}{\operatorname{Patch}}
\newcommand{\Opens}{\operatorname{\mathcal{O}}}%[1]{\operatorname{\mathcal{O}}{#1}} 
\renewcommand{\O}{\Opens}
\newcommand{\Cocompacts}{\operatorname{\mathcal{C}}}
%\newcommand{\C}[2]{\operatorname{C}(#1,#2)}
\newcommand{\Regions}{\operatorname{\mathcal{R}}}%[1]{\operatorname{\mathcal{R}}{#1}} 
\newcommand{\frameleft}[1]{#1^{*}} \newcommand{\frameright}[1]{#1_{*}}
\newcommand{\arrow}{\to}%{\rightarrow}
%\newcommand{\comp}{\mathrel{\circ}}
\newcommand{\id}{\operatorname{id}}
\newcommand{\join}{\sqcup}
\newcommand{\meet}{\sqcap}
\newcommand{\himp}{\mathrel{\Rightarrow}}
\newcommand{\bigjoin}{\bigsqcup}
\newcommand{\dirjoin}{\sideset{}{^{\uparrow}}\bigjoin}
\newcommand{\bigmeet}{\bigsqcap}
\newcommand{\below}{\sqsubseteq}
\renewcommand{\above}{\sqsupseteq}
\newcommand{\farrow}[1]{\stackrel{#1}{\longrightarrow}}
\newenvironment{eqn}{\[\begin{array}{rcll}}{\end{array}\]}
\newcommand{\qtext}[1]{\quad\text{#1}}
\newcommand{\up}{\operatorname{\uparrow}}
\newcommand{\down}{\operatorname{\downarrow}}
\newcommand{\upup}{\mathord{\hbox{\makebox[0pt][l]{\raise .6mm\hbox{$\uparrow$}}$\uparrow$}}}
\newcommand{\fix}{\operatorname{fix}}
\newcommand{\Infl}[1]{\hat{#1}}
\newcommand{\infl}{\Infl{\top}} %{\Infl{\operatorname{\i d}}}


% defined terms and index entries

\newcommand{\fontfordefinition}[1]{\emph{#1}}%{{\bfseries\itshape #1}}
% \newcommand{\df}[1]{\fontfordefinition{#1}} % no index required
\newcommand{\dfkey}[2]{\fontfordefinition{#1}\index{#2}}
\newcommand{\dfemph}[1]{\emph{#1}\index{#1}}
\newcommand{\cindex}[1]{#1} % was {#1\index{#1}}
\newcommand{\tindex}[1]{#1\index{#1}}
\newcommand{\df}[1]{\dfkey{#1}{#1}}
\newcommand{\dfn}[1]{\fontfordefinition{#1}}
\newcommand{\ind}[1]{#1\index{#1}}
\newcommand{\indemph}[1]{\emph{#1}\index{#1}}

\makeindex

% Omission macro

\newcommand{\Omit}[1]{}

% Chronology macro

\newcommand{\chron}[1]{}
                     %{\begin{footnotesize}(Chronology: #1\end{footnotesize})}

\newcommand{\smalltext}[1]{\begin{footnotesize} #1 \end{footnotesize}}

%\newcommand{\myparagraph}[1]{\medskip\emph{#1}}

\newcommand{\lbracket}{(}
\newcommand{\rbracket}{)}
\newcommand{\QTop}{\operatorname{{QTop}}}

\newcommand{\numbered}[1]{\grey{#1}}

\definecolor{darkgreen}{rgb}{0,0.5,0}
\definecolor{darkblue}{rgb}{0,0.0,0.5}
\definecolor{darkred}{rgb}{1,0,0}
\definecolor{grey}{rgb}{0.4,0.4,0.4}
\newcommand{\black}{\textcolor{black}}
\newcommand{\grey}{\textcolor{grey}}
\newcommand{\blue}{\textcolor{blue}}
\newcommand{\red}{\textcolor{red}}
\newcommand{\darkblue}{\textcolor{darkblue}}
\newcommand{\darkred}{\textcolor{darkred}}
\newcommand{\darkgreen}{\textcolor{darkgreen}}
\newcommand{\green}{\textcolor{green}}
\newcommand{\yellow}{\textcolor{yellow}}
\newcommand{\cyan}{\textcolor{cyan}}
\newcommand{\magenta}{\textcolor{magenta}}

\newcommand{\qed}{}%{$\boxempty$}

\newcommand{\sexample}[1]{\noindent\textbf{Example} \blue{#1} }
\newcommand{\slemma}[1]{\noindent\darkgreen{\textbf{Lemma}} \\[1ex] {\emtheorem #1}}
\newcommand{\ulemma}[1]{\noindent\darkgreen{\textbf{Lemma}} {\emtheorem #1}}
\newcommand{\proposition}[1]{\noindent\darkgreen{\textbf{Proposition}} {\emtheorem #1}}
\newcommand{\proof}[1]{\noindent{\textbf{Proof}} {#1}}
\newcommand{\sdefinition}[1]{\noindent\darkgreen{\textbf{Definition}} \\[1ex] {\emtheorem #1}}
\newcommand{\scorollary}[1]{\noindent\darkgreen{\textbf{Corollary}} {\emtheorem #1}}
\newcommand{\stheorem}[1]{\noindent\darkgreen{\textbf{Theorem}}  {\emtheorem #1}}
\newcommand{\texample}[2]{\noindent\textbf{Example} \blue{#1}}
\newcommand{\tlemma}[2]{\noindent\darkgreen{\textbf{Lemma} (#1)}  \\[1ex] {\emtheorem #2}}
\newcommand{\ttlemma}[2]{\noindent\darkgreen{\textbf{Lemma} (#1)} {\emtheorem #2}}
\newcommand{\utlemma}[2]{\noindent\darkgreen{\textbf{Lemma} (#1)} {\emtheorem #2}}
\newcommand{\tdefinition}[2]{\noindent\darkgreen{\textbf{Definition} (#1)} \\[1ex] {\emtheorem #2}}
\newcommand{\tcorollary}[2]{\noindent\darkgreen{\textbf{Corollary} (#1)}  \\[1ex] {\emtheorem #2}}
\newcommand{\ttheorem}[2]{\noindent\darkgreen{\textbf{Theorem} (#1)}  \\[1ex] {\emtheorem #2}}

\newcommand{\utheorem}[2]{\noindent\darkgreen{\textbf{Theorem} (#1)} {\emtheorem #2}}

\newcommand{\uapplication}[1]{\noindent\darkgreen{\textbf{Application}} {#1}}

\newcommand{\emtheorem}{\em}

\LogoOff

\newcommand{\twostar}{2^{*}}
\newcommand{\twoomega}{2^\infty}
\newcommand{\prefix}{\sqsubseteq}
\newcommand{\all}[3]{\forall\, #1 \,{\in}\, #2\,.\left(#3\right)}
\newcommand{\some}[3]{\exists\, #1 \,{\in}\, #2\,.\left(#3\right)}
\newcommand{\exactlyone}[3]{\exists!\, #1 \,{\in}\, #2\,.\left(#3\right)}
\newcommand{\lam}[3]{\lambda #1 \,{\in}\, #2\,.\left(#3\right)}
\newcommand{\uall}[2]{\forall\, #1\,.\left(#2\right)}
\newcommand{\usome}[2]{\exists\, #1\,.\left(#2\right)}
\newcommand{\uexactlyone}[3]{\exists!\, #1\,.\left(#2\right)}
\newcommand{\ulam}[2]{\lambda #1 .\left(#2\right)}
\newcommand{\xall}[3]{\forall\, #1 \,{\in}\, #2\,.\,#3}
\newcommand{\xsome}[3]{\exists\, #1 \,{\in}\, #2\,.\,#3}
\newcommand{\xexactlyone}[3]{\exists!\, #1 \,{\in}\, #2\,.\,#3}
\newcommand{\xuall}[2]{\forall\, #1\,.\,#2}
\newcommand{\xusome}[2]{\exists\, #1\,.\,#2}
\newcommand{\xuexactlyone}[2]{\exists!\, #1,.\,#2}
\newcommand{\xlam}[3]{\lambda #1 \,{\in}\, #2\,.\,#3}
\newcommand{\xulam}[2]{\lambda #1 .\,#2}
\newcommand{\tlam}[3]{\lambda #1 \,{:}\, #2\,.\,\left(#3\right)}
\newcommand{\xtlam}[3]{\lambda #1 \,{:}\, #2\,.\,#3}


%%% Local Variables: 
%%% mode: latex
%%% TeX-master: "escardo-mfps2010"
%%% End: 


%\usepackage{hyperref}

\newcommand{\concat}{\operatorname{::}} %\hash}
\newcommand{\arginf}{\operatorname{arginf}}
\newcommand{\argsup}{\operatorname{argsup}}
\newcommand{\argexists}{\operatorname{arg-\exists}}
\newcommand{\false}{\operatorname{false}}
\newcommand{\true}{\operatorname{true}}

\newcommand{\next}{\operatorname{next}}
\newcommand{\myitem}[1]{\item[\darkblue{#1}]}

\newcommand{\Maybe}{\operatorname{Maybe}}

\title{\darkblue{The ubiquitous selection monad}}

\author{\darkblue{Mart\'{\i}n Escard\'o} \\
University of Birmingham, UK\\[10ex]
Joint work with \darkblue{Paulo Oliva}, Queen Mary, London, UK. \\[9ex] ~}

\date{\sc Wessex Seminar, Swansea, 7th April 2011}

\begin{document}

\setlength{\parindent}{0in}

\maketitle

\overhead{Seemingly disparate constructions}

\begin{enumerate}
\item[\grey{1.}] \darkblue{Game theory.}

Optimal plays of sequential games of unbounded length.

\item[\grey{2.}]  \darkblue{Proof theory.} \grey{Double Negation Shift:}

$\forall n \in \N(\blue{\neg \neg} A(n)) \implies \blue{\neg \neg} \forall n \in \N(A(n)).$


\item[\grey{3.}]  \darkblue{Topology.} \grey{Tychonoff Theorem:}

$X_i$ compact $\implies$ $\prod_i X_i$ also compact.  

\item[\grey{4.}]  \darkblue{Higher-type computation.} \grey{Computational Tychonoff Theorem:}

$X_n$ exhaustibly searchable $\implies$ $\prod_n X_n$ also exhaustibly searchable.  

\grey{The point is that we get infinite exhaustively searchable sets.}
\end{enumerate}

\overhead{What do they have in common?}

\vfill

Implemented/realized by a certain \darkblue{infinite product of selection functions.}

\vfill

Explained by a certain \darkblue{selection monad}.

\vfill

\overhead{Countable-product functional}

\vfill

In a simply typed formalism:
\[
\darkblue{\left(\black{\N \to ((X \to R) \to X)}\right)} \to \darkblue{\left(\black{(\N \to X) \to R}\right)} \to \darkblue{\left(\black{\N \to X}\right)}.
\]

\vfill

In a dependently typed formalism: %, where $n$ ranges over $\N$:
\[
\darkblue{\left(\black{\prod_{n  \in \N}(X_n \to R) \to X_n}\right)} \to
\green{\left(\darkblue{\left(\black{\prod_{n \in \N} X_n}\right)} \black{\to R}\right)} \to
\darkblue{\left(\black{\prod_{n \in \N} X_n}\right)}.
\]

\vfill

\overhead{There is structure in the above types}

Write \darkblue{$J X = ((X \to R) \to X)$} where \darkblue{$R$} is
fixed in advance.

\vfill

In a simply typed formalism:
\[
(\N \to \darkblue{J} X) \to \darkblue{J}(\N\to X).
\]

In a dependently typed formalism:
\[
\prod_{n \in \N} \darkblue{J} X_n  \to \darkblue{J} \prod_{n \in \N} X_n.
\]

% \framebox{This is a \darkblue{$J$-shift}.}

\overhead{Selection functions \black{$(X \to R) \to X$}}

\blue{$X$} set of things.

\quad \grey{\small Goods in a store; possible moves of a game;
proofs of a proposition;
points of a space.}

\blue{$R$} set of values.

\quad \grey{\small Prices; outcomes win, lose, draw; how much money you win;
true or false;
proofs again.}

\blue{$X \stackrel{p}{\longrightarrow} R$} value judgement.

\quad \grey{\small How you value it;
how much it costs you;
pay-off of a move;
propositional function.}


\blue{$(X \to R) \stackrel{\varepsilon}{\longrightarrow} X$} selects
something according to some criterion.

\quad \grey{\small The best, the cheapest, any, something odd.}

\overhead{Example 1}

\vspace{-2ex}

\blue{$X$} set of goods.

\blue{$R$} set of prices.

\blue{$X \stackrel{p}{\longrightarrow} R$} table of prices.

\blue{$(X \to R) \stackrel{\varepsilon}{\longrightarrow} X$} selects
a \darkblue{cheapest good} in a given table.

\blue{$(X \to R) \stackrel{\phi}{\longrightarrow} R$} 
determines the \darkblue{lowest price} in a given table.

\vfill

\noindent\darkgreen{Fundamental equation:}
\[
\blue{p(\varepsilon(p)) = \phi(p).}
\]
This says that the price of a cheapest good is the lowest in the table.
\begin{gather*}
  \phi  =  \inf \qquad \varepsilon =  \arginf, \\
  \blue{p(\arginf(p)) = \inf(p).}
\end{gather*}

\overhead{Example 2}

\vspace{-2ex}

\blue{$X$} set of individuals.

\blue{$R$} set of booleans \blue{$\false = 0 < 1 = \true$}.

\blue{$X \stackrel{p}{\longrightarrow} R$} property.

\blue{$(X \to R) \stackrel{\varepsilon}{\longrightarrow} X$} selects
an individual with the highest truth value. 

\blue{$(X \to R) \stackrel{\phi}{\longrightarrow} R$} 
determines the highest value of a given property. 

\vfill

\darkgreen{Fundamental equation:}
\[
\blue{p(\varepsilon(p)) = \phi(p)}
\]
\begin{eqnarray*}
  \phi & = & \sup = \exists \\
  \varepsilon & = & \argsup = \argexists = \text{Hilbert's choice operator}
\end{eqnarray*}
\[
\blue{p(\varepsilon(p)) = \exists(p)} \qquad \grey{\text{Hilbert's definition of $\exists$ in his $\varepsilon$-calculus}}
\]

\overhead{Maximum-Value Theorem}

Let \blue{$X$} be a compact non-empty topological space.

\vfill

\darkblue{\framebox{\black{Any continuous function \blue{$p \colon X \to \R$} attains its
maximum value.}}}

\vfill

This means that there is \blue{$a \in X$} such that
\[
\blue{\sup p = p(a)}. 
\]

\vfill

\grey{However, the proof is non-constructive when e.g.\ \blue{$X=[0,1]$}.}

\grey{A maximizing argument \blue{$a$} cannot be algorithmically
calculated from
\blue{$p$}.} % with an algorithm that works for all~\blue{$p$}.

\vfill

Of course, there is a \darkblue{Minimum-Value Theorem} too.


\overhead{Mean-Value Theorem}

\darkblue{\framebox{\black{Any continuous function \blue{$p \colon
        [0,1] \to \R$} attains its mean value.}}}

\vfill

There is \blue{$a \in [0,1]$} such that
\[
\blue{\int p = p(a)}.
\]

\vfill

\grey{Again this \blue{$a$} cannot be found from \blue{$p$} using an algorithm.} 
\overhead{Universal-Value Theorem}

Let \blue{$X$} be a non-empty set and \blue{$2=\{0,1\}$} be the set of booleans.

\darkblue{\framebox{\black{Any \blue{$p \colon
        X \to 2$} attains its universal value.}}}

\vfill

There is \blue{$a \in X$} such that
\[
\blue{\forall p = p(a)}.
\]

\grey{This is again a classical statement if the set \darkblue{$X$} is infinite.}

\vfill

This is usually formulated as the \darkblue{Drinker Paradox}:

\darkblue{\framebox{\black{In any inhabited pub there is a person \blue{$a$} s.t.\ if 
\blue{$a$} drinks then everybody drinks.}}}

We've also met the \darkblue{Existential-Value Theorem}.


\overhead{General situation}

%The general equation we are considering here is thus

With \blue{$\phi$} among \blue{$\exists, \forall, \sup, \inf, \int, \dots$}, we have that
\[
\text{\framebox{$\blue{\phi(p) = p(a)}$}}
\]
for some \blue{$a$} depending on \blue{$p$}.

\vfill

In favourable circumstances, \blue{$a$} can be calculated as 
\[ \blue{a =\varepsilon(p)},\] 
so that
\[ \text{\framebox{$\blue{\phi(p) = p(\varepsilon(p))}$}} \]

\overhead{Selection function}

\vfill

\darkgreen{Definition.}

A \darkblue{selection function} for a \grey{(logical, arithmetical, \dots)} {quantifier}
\[
\blue{\phi \colon (X \to R) \to R}
\]

is a functional 
\[
\blue{\varepsilon \colon (X \to R) \to X}
\]

such that 
\[ 
\blue{\phi(p) = p(\varepsilon(p))}.
\]

% \grey{NB.}

% \grey{A quantifier may or may not have a selection function.}

% \grey{A quantifier may have more than one selection function.}

\overhead{Monad morphism}

\darkblue{\black{\framebox{Every \blue{$\varepsilon \colon (X \to R) \to X$} is the selection function of some \blue{$\phi \colon (X \to R) \to R$.}}}}

 Namely \blue{$\phi = \overline{\varepsilon}$} defined by
\[
\blue{\phi(p) = p(\varepsilon(p))}.
\]

%Also written $\blue{\phi = \overline{\varepsilon}}$.

\vfill

This construction defines a monad morphism \blue{$\theta \colon J \to K$}:
\begin{eqnarray*}
\blue{\overbrace{\black{(X \to R) \to X}}^{{JX}}} & \stackrel{\blue{\Theta}}{\longrightarrow} & \blue{\overbrace{\black{(X \to R) \to R}}^{{KX}}} \\
\blue{\varepsilon} & \longmapsto & \blue{\overline{\varepsilon}} % \blue{\underbrace{\black{\lambda p.p(\varepsilon(p))}}_{\overline{\varepsilon}}}
\end{eqnarray*}

This is a morphism from the \darkblue{selection monad} to the \darkblue{quantifier monad}.

\grey{Oh, I mean to the continuation monad.}

\overhead{Units of the monads}

\begin{eqnarray*}
 X & \stackrel{\blue{\eta}}{\longrightarrow} & \blue{K} X \\
 x & \longmapsto & \lambda p. p(x).
\end{eqnarray*}
\framebox{Quantifies over the singleton $\blue{\{x\}} \subseteq X$.}
\[
\blue{\eta(x) = \exists_{\{x\}} = \forall_{\{x\}}}.
\]
\begin{eqnarray*}
 X & \stackrel{\blue{\eta}}{\longrightarrow} & \blue{J} X \\
 x & \longmapsto & \lambda p. x.
\end{eqnarray*}
\framebox{Produces a selection function for the singleton quantifier.}

\overhead{Functors of the monads}

Let \blue{$f \colon X \to Y$}.
\begin{eqnarray*}
  \blue{K} X & \stackrel{\blue{K} f}{\longrightarrow} & \blue{K} Y \\
  \phi & \longmapsto & \lambda p.\phi(\lambda x. p(f(x))).
\end{eqnarray*}
\framebox{If $\phi$ quantifies over a set $\darkblue{S} \subseteq X$, then $\blue{K}f(\phi)$ quantifies over the set $\blue{f(S)} \subseteq Y$.}

\begin{eqnarray*}
  \blue{J} X & \stackrel{\blue{J} f}{\longrightarrow} & \blue{J} Y \\
  \varepsilon & \longmapsto & \lambda p.f(\varepsilon(\lambda x. p(f(x)))).
\end{eqnarray*}

\framebox{If $\varepsilon$ is a selection function for $\phi$, then $\blue{J} f(\varepsilon)$ is a selection function for $\blue{K} f(\phi)$.}

\overhead{Multiplications}

They can be explained in intuitive terms, but this takes some time.

\begin{eqnarray*}
  \blue{K K} X & \stackrel{\mu}{\longrightarrow} & \blue{K} X \\
  \Phi & \longmapsto &  \lambda p.\Phi(\lambda \phi.\phi(p)).
\end{eqnarray*}

\begin{eqnarray*}
  \blue{J J} X & \stackrel{\mu}{\longrightarrow} & \blue{J} X \\
  E & \longmapsto &  \lambda p.E(\lambda \varepsilon.p(\varepsilon(p)))(p).
\end{eqnarray*}
\vfill

Use the selection function \darkblue{$E$} to find a selection function
\darkblue{$\varepsilon$} such that \darkblue{$p(\varepsilon(p))$}, and
apply this resulting selection function to \darkblue{$p$} to find an
element of \darkblue{$X$}.



\overhead{Monad algebras}

$\blue{K} A \to A$. 

\qquad $((A \to R) \to R) \to A$.

\qquad  \darkblue{Double-negation elimination.}

\framebox{Explains the G\"odel--Gentzen translation of classical into intuitionistic logic.}

\vfill

$\blue{J} A \to A$. 

\qquad $((A \to R) \to A) \to A$.

\qquad \darkblue{Peirce's Law.}

\vfill

\framebox{Get different proof translation of classical into intuitionistic logic.}

\overhead{\darkred{Aside:} we get a more conceptual explanation of call/cc}

The \darkblue{type} of call/cc can be written as $\darkblue{J K X \to K X}$. \\
\grey{(An instance of Peirce's Law, as discovered by Tim Griffin.)}

\vfill

Its \darkblue{$\lambda$-term} can be reconstructed as follows:
\begin{enumerate}
\item[\grey{1.}] \darkblue{$K X$} is a \darkblue{$K$}-algebra, with structure map
  $\darkblue{\mu \colon K K X \to K X}$.
\item[\grey{2.}] Because we have a morphism \darkblue{$J \stackrel{\theta}{\longrightarrow}  K$}, every
  \darkblue{$K$}-algebra is a \darkblue{$J$}-algebra:
\[
\darkblue{J A \stackrel{\theta_A}{\longrightarrow} K A \stackrel{\alpha}{\longrightarrow} A}.
\]
\item[\grey{3.}] Call/cc is what results for \darkblue{$A = K X$} and \darkblue{$\alpha = \mu$}:
\[
\darkblue{J K X \stackrel{\theta_{K X}}{\longrightarrow} K K X \stackrel{\mu}{\longrightarrow} K X}.
\]
\end{enumerate}

\overhead{Strengths}

\begin{eqnarray*}
  X \times  \blue{K} Y & \stackrel{t}{\longrightarrow} & \blue{K}(X \times Y) \\
  (x, \phi) & \longmapsto & \lambda p.\phi(\lambda y.p(x,y)).
\end{eqnarray*}

\framebox{If \darkblue{$\phi$} quantifies over \darkblue{$S \subseteq Y$}, then
\darkblue{$t(x,\phi)$} quantifies over \darkblue{$\{x\} \times S
  \subseteq X \times Y$}.}

\begin{eqnarray*}
  X \times  \blue{J} Y & \stackrel{t}{\longrightarrow} & \blue{J}(X \times Y) \\
  (x, \varepsilon) & \longmapsto & \lambda p.(x,\varepsilon(\lambda y.p(x,y))).
\end{eqnarray*}

\framebox{This produces a selection function for the above quantifier.}

\overhead{We have monoidal-monad structures}

Because we have strong monads \darkblue{$T=J$} and \darkblue{$T=K$} on
a ccc.

\begin{eqnarray*}
\blue{T} X \times \blue{T} Y & \stackrel{\otimes}{\longrightarrow} & \blue{T} (X \times Y ) \\
(u, v) & \longmapsto & (T (\lambda x.t_{X,Y}(x, v)))(u) \text{~~~\darkgreen{$\longleftarrow$ we want this one},}\\
(u, v) & \longmapsto & (T (\lambda y.t_{Y,X}(u, x)))(v) \text{~~~\,\darkred{$\longleftarrow$ not this one}.} \\
\end{eqnarray*}

\framebox{The monads are not commutative.} 

And this is good!

% \overhead{Monoidal monad structure}

% \begin{eqnarray*}
% \blue{T} X \times \blue{T} Y & \stackrel{\otimes}{\longrightarrow} & \blue{T} (X \times Y ) \\
% (u, v) & \longmapsto & (T (\lambda x.t(x, v)))(u).
% \end{eqnarray*}


% \begin{enumerate}
% \item Sequential games of length two.
% \item Binary Tychonoff Theorem.
% \item Bekic's Lemma. \grey{When $X=R$ and hence $JR = KR$, an element of $KR=((R \to R) \to R)$ is a fixed point operator iff it is its own selection function. Then Bekic's Lemma follows. }
% \end{enumerate}

\overhead{Examples}

\begin{eqnarray*}
\darkblue{K} X \times \darkblue{K} Y & \stackrel{\otimes}{\longrightarrow} & \blue{K} (X \times Y ) \\
(\exists_A, \exists_B) & \longmapsto & \exists_{A \times B}.
\end{eqnarray*}

\vfill

\begin{eqnarray*}
\darkblue{K} X \times \darkblue{K} Y & \stackrel{\otimes}{\longrightarrow} & \blue{K} (X \times Y ) \\
(\forall_A, \exists_B) & \longmapsto & \lambda p.\forall x \in A.\exists y \in B.p(x,y).
\end{eqnarray*}

\vfill

The other choice of $\otimes$ concatenates the quantifiers in
reverse order.

\overhead{Because we have a strong monad morphism:}

\[
\darkblue{\overline{\varepsilon \otimes \delta} = \overline{\varepsilon} \otimes \overline{\delta}.}
\]
In other words:

\vfill

\darkgreen{Theorem.}

\quad \darkgreen{If}

\qquad \darkblue{$\varepsilon \in J X$} is a selection function for
the quantifier \darkblue{$\phi \in K X$},

\qquad \darkblue{$\delta \in J Y$} is a selection function for the
quantifier \darkblue{$\gamma \in K Y$},

\quad \darkgreen{then}

\qquad \darkblue{$\varepsilon \otimes \delta$} is a selection function
for the quantifier \darkblue{$\phi \otimes \gamma$}.

% \darkblue{This is good for optimally playing games, as we'll see.}

\overhead{Binary product of quantifiers and selection functions}

In every pub there are a man~\darkblue{$b$} and a
woman~\darkblue{$c$} such that if~\darkblue{$b$} buys a drink
to~\darkblue{$c$} then every man buys a drink to some woman.

\vfill

\overhead{Binary product of quantifiers and selection functions}

In every pub there are a man~\darkblue{$b$} and a
woman~\darkblue{$c$} such that if~\darkblue{$b$} buys a drink
to~\darkblue{$c$} then every man buys a drink to some woman.

\vfill


If \darkblue{$X = \text{set of men}$} and \darkblue{$Y = \text{set
    of women}$}, and if we define % the combined quantifier
\darkblue{$\phi = \forall \otimes \exists$} by
\[
\darkblue{\phi(p) = \left(\forall x \in X \,\, \exists y \in Y \,\, p(x,y)\right)},
\]
then our claim amounts to
\[
\darkblue{\phi(p) = p(a)}
\]
for a suitable pair \darkblue{$a = (b,c) \in X \times Y$},

\vfill

This is calculated as \blue{$a = (\varepsilon \otimes \delta)(p)$} where
\blue{$\overline{\varepsilon} = \forall_X$} and \blue{$\overline{\delta} =
\exists_Y$}.

\overhead{The infinite strength of the selection monad}

\darkblue{In certain categories of interest}

There is a countable monoidal-monad structure
\[
\darkblue{\bigotimes} \colon \prod_n \blue{J} X_n \to \blue{J} \prod_n X_n 
\]

\darkblue{uniquely} determined by the equation
\[
\darkblue{\bigotimes_n \varepsilon_n = \varepsilon_o \otimes \bigotimes_n \varepsilon_{n+1}}.
\]

\framebox{Turns out to be an instance of the \darkblue{bar recursion} scheme.}


\overhead{The continuation monad lacks infinite strength}

However, if a sequence of quantifiers \darkblue{$\phi_n$} have
selection functions \darkblue{$\varepsilon_n$},

then their product can be defined as
\[
\darkblue{\bigotimes_n \phi_n = \overline{\bigotimes_n \varepsilon_n}}
\]

and satisfies

\[
\darkblue{\bigotimes_n \phi_n = \phi_o \otimes \bigotimes_n \phi_{n+1}}.
\]

This is useful for various applications.



\overhead{Playing games}

Products of selection functions compute \darkblue{optimal
plays} and \darkblue{strategies}. 

\overhead{First example}

\darkblue{Alternating, two-person game.}

\vfill

\grey{1.}~\darkblue{Eloise} plays first, against
\darkblue{Abelard}. One of them wins (no draw).

\vfill

\grey{2.}~The \darkblue{$i$-th} move is an
element of the set \darkblue{$X_i$}.

\vfill

\grey{3.}~The game is defined by
a predicate \darkblue{$p \colon \prod_{i} X_i \to \mathtt{Bool}$}
\\ \phantom{xx} that tells whether Eloise wins
a given play \darkblue{$x = (x_0, \dots, x_{n-1})$}. 
% \in \prod_{i<n} X_i$}. 

\vfill

\grey{4.}~Eloise has a winning strategy for the
game \darkblue{$p$} if and only if
\[
\darkblue{\exists x_0 \!\in\! X \forall x_1 \!\in\! Y\exists x_{2} \!\in\! X_{2} \forall x_{3} \!\in\! X_{3} \cdots p(x_0, x_1, x_2, x_3, \ldots)}.
\]

\overhead{First example}

\grey{4.}~Eloise has a winning strategy for the game \darkblue{$p$} if
and only if
\[
\darkblue{\exists x_0 \!\in\! X \forall x_1 \!\in\! Y\exists x_{2} \!\in\! X_{2} \forall x_{3} \!\in\! X_{3} \cdots p(x_0, x_1, x_2, x_3, \ldots)}.
\]
If we define
\[
  \darkblue{\phi_i =  
  \begin{cases}
    \exists_{X_i} & \text{if $i$ is even,} \\
    \forall_{X_i} & \text{if $i$ is odd,} 
  \end{cases}}
\]

then this condition for Eloise
having a winning strategy amounts to

\[ \darkblue{\left(\bigotimes_{i} \phi_i\right)(p)}. \] 


\overhead{Second example}
    
Choose \darkblue{$R = \{-1,0,1\}$}
instead, with the convention that 
\[
\darkblue{\begin{cases}
 -1 = \text{Abelard wins,} & \\
 \phantom{-}0 = \text{draw}, & \\
 \phantom{-}1 = \text{Eloise wins.} & 
\end{cases}}
\]
The existential and universal quantifiers get
replaced by \darkblue{$\sup$} and \darkblue{$\inf$}:
\[
  \darkblue{\phi_i =  
  \begin{cases}
    \sup_{X_i} & \text{if $i$ is even,} \\
    \inf_{X_i} & \text{if $i$ is odd.} 
  \end{cases}}
\]
The optimal outcome is calculated as \darkblue{$\bigotimes_{i}
  \phi_i$}, which %in this case 
amounts to
\[\darkblue{\darkblue{\sup_{x_0 \!\in\! X_0} \inf_{x_1 \!\in\! Y}\sup_{x_{2} \!\in\! X_{2}} \inf_{x_{3} \!\in\! X_{3}} \cdots p(x_0, x_1, x_2, x_3, \ldots)}}.
\]

% \darkblue{$-1$} if Abelard has a winning strategy, \darkblue{0} if
% the game is a draw, \darkblue{1} if Eloise has a winning strategy.

\overhead{General non-history dependent case}

A \darkblue{sequential game} is given by
\begin{enumerate}
\item[\grey{1.}] Sets of moves \darkblue{$X_0, X_1, X_2, \dots$}.
\item[\grey{2.}] A set \darkblue{$R$} of possible outcomes.
\item[\grey{3.}] An outcome function \darkblue{$p \colon \prod_i X_i \to R$},
\item[\grey{4.}] A quantifier \darkblue{$\phi_i \in K X_i$} for each stage of the game.
\item[\grey{5.}] Optionally a selection function
  \darkblue{$\varepsilon_i$} for the quantifier \darkblue{$\phi_i$}.
\end{enumerate}

\vfill

\framebox{These are games in \darkblue{normal form}.}

For games in \darkblue{extensive form}, the outcome function is
presented as a tree.


\overhead{Calculating the optimal outcome of a game}

The value
\[ \darkblue{\left(\bigotimes_{i} \phi_i\right)(p)} \]

gives the \darkblue{optimal outcome} of the game.

\vfill

\framebox{This takes place when all players play as best as they can.}

\vfill

In the first example, the optimal outcome is
\darkblue{$\mathtt{True}$} if Eloise has a winning strategy, and
\darkblue{$\mathtt{False}$} if Abelard has a winning strategy.

\overhead{Calculating an optimal play}

Suppose each quantifier \darkblue{$\phi_i$} has a selection function
\darkblue{$\varepsilon_i$}.

% (thought of as a policy function for the \darkblue{$i$-th} move).

\vfill

\darkgreen{Theorem.}
The sequence \[ \darkblue{a = (a_0,a_1, \dots,a_{i},\dots) = \left(\bigotimes_{i}
    \varepsilon_i\right)(p)}\] is an \darkblue{optimal play}. 

\vfill

\framebox{\parbox{60ex}{This means that for every stage
    \darkblue{$i$} of the game, the move \darkblue{$a_i$} is optimal
    given that the moves \darkblue{$a_0,\dots,a_{i-1}$} have been
    played.}}

\overhead{Calculating an optimal strategy}

For a \darkgreen{partial play} \darkblue{$a \in \prod_{\darkgreen{i < k}} X_i$},
we have a \darkgreen{subgame} \darkblue{$p_a \colon \prod_{\darkgreen{i \ge k}} X_i
  \to R$},
\[
\darkblue{p_a(x) = p(a \cdot x).}
\]

\darkgreen{Corollary.}
The function \darkblue{$f_k : \prod_{i < k} X_i
\to X_k$} defined by %, for $k < n-1$,
        \[
        \darkblue{f_k(a) = \left(\left(\bigotimes_{i=k}^{n-1}
    \varepsilon_i\right)(p_a )\right)_0}
        \]

    is an \darkblue{optimal strategy} for playing the game. 

\vfill

\framebox{\parbox{60ex}{This means that given that the sequence of moves
\darkblue{$a_0,\dots,a_{k-1}$} have been played, the move
\darkblue{$a_k=f_k(a_0,\dots,a_{k-1})$} is optimal.}}

\overhead{Program extraction from classical proofs with choice}

Start with intuitionistic choice
\[
\forall i \in I \left(\exists x \in X_i \left(A(i,x)\right)\right)
\implies 
\exists \vec{x} \in \prod_i X_i\left(\forall i \in I\left(A(i,x_i)\right) \right).
\]

\vfill

Apply the \darkblue{$T$}-translation, say for \darkblue{$T=K$} or
\darkblue{$T=J$}:
\[
\forall i \in I \left(\darkblue{T} \exists x \in X_i \left(A(i,x)\right)\right)
\implies 
\darkblue{T} \exists \vec{x} \in \prod_i X_i\left(\forall i \in I\left(A(i,x_i)\right) \right).
\]
\framebox{Is that realizable?}

\overhead{The $J$-shift}

Think of \darkblue{$JA = ((A \to R) \to A)$} as a logical \darkgreen{modality}.

\vfill

\darkgreen{Theorem}

The product functional \darkblue{$\darkgreen{\bigotimes} \colon \prod_n J X_n \to
J\left(\prod_n X_n\right)$} realizes the \darkgreen{$J$-shift} \\[2ex]
\[
\darkblue{\forall n (J(A(n)) \to
J\left(\forall n(A(n))\right)}.
\]

\vfill

\grey{To guess the theorem, apply Curry--Howard.}

\grey{To prove it, use bar induction.}


\overhead{Countable choice}

\grey{1.} Start again with intuitionistic choice, but countable this time:
\[
\forall n \in \N \left(\exists x \in X_n \left(A(n,x)\right)\right)
\implies 
\exists \vec{x} \in \prod_n X_n\left(\forall n \in \N\left(A(n,x_n)\right) \right).
\]
\grey{2.} Apply the functor \darkgreen{$J$}:
\[
\darkgreen{J} \darkgreen{\left(\black{\forall n \in \N \left(\exists x \in X_n \left(A(n,x)\right)\right)}\right)}
\implies 
\darkgreen{J}
\exists \vec{x} \in \prod_n X_n\left(\forall n \in \N\left(A(n,x_n)\right) \right).
\]
\grey{3.} Finally pre-compose with the instance of the \darkgreen{$J$-shift} 
\[
\forall n \in \N \left(\darkgreen{J} \exists x \in X_n \left(A(n,x)\right)\right)
\implies
\darkgreen{J} \left(\black{\forall n \in \N \left(\exists x \in X_n \left(A(n,x)\right)\right)}\right).
\]
% to get the \darkgreen{$J$}-translation of choice.

\framebox{\darkgreen{Theorem.} The \darkgreen{$J$}-translation of countable choice is realizable.}

\overhead{Countable choice}

\grey{1.} Start again with intuitionistic choice, but countable this time:
\[
\forall n \in \N \left(\exists x \in X_n \left(A(n,x)\right)\right)
\implies 
\exists \vec{x} \in \prod_n X_n\left(\forall n \in \N\left(A(n,x_n)\right) \right).
\]
\grey{2.} Apply the functor \darkgreen{$J$}:
\[
\underline{\darkgreen{J} \darkgreen{\left(\black{\forall n \in \N \left(\exists x \in X_n \left(A(n,x)\right)\right)}\right)}}
\implies 
\darkgreen{J}
\exists \vec{x} \in \prod_n X_n\left(\forall n \in \N\left(A(n,x_n)\right) \right).
\]
\grey{3.} Finally pre-compose with the instance of the \darkgreen{$J$-shift} 
\[
\forall n \in \N \left(\darkgreen{J} \exists x \in X_n \left(A(n,x)\right)\right)
\implies
\underline{\darkgreen{J} \left(\black{\forall n \in \N \left(\exists x \in X_n \left(A(n,x)\right)\right)}\right)}.
\]
% to get the \darkgreen{$J$}-translation of choice.

\framebox{\darkgreen{Theorem.} The \darkgreen{$J$}-translation of countable choice is realizable.}


\overhead{Topology and computation}

\vfill

\framebox{I need a large class of topological spaces to formulate a
  computational theorem.}

\vfill

\darkgreen{Kleene--Kreisel spaces} are good for total higher-type
computation.

\vfill

But perhaps a bit limited.

\vfill

\darkgreen{Enlarge} by closing under retracts. 

\vfill

Denote by \darkblue{$2=\{0,1\}$} the space of booleans.

\overhead{Effective compactness}



\vfill

\grey{1.} \darkgreen{Theorem} (topological).

~~~ A space \darkblue{$X$} is compact \darkgreen{$\iff$}
has a continuous quantifier \darkblue{$(X \to 2)
\to 2$}.

\vfill

\grey{2.} \darkgreen{Definition} (computational).

~~~~ A space \darkblue{$X$} is \darkgreen{effectively compact} if it has a computable
quantifier \darkblue{$(X \to 2) \to 2$}.

\vfill

\grey{3.} \darkgreen{Theorem} (computational).

~~~ A space \darkblue{$X$} is effectively compact \darkgreen{$\iff$} \\
\phantom{xxx}it has a computable selection function \darkblue{$(X \to 2) \to X$}.
% for its existential quantifier.

\vfill

\framebox{This says that two different, common forms of exhaustive search are equivalent.}

\overhead{Computational Tychonoff Theorem}

\darkgreen{Theorem} 

Effectively compact spaces are closed under the formation of countable products.

\vfill

\framebox{This is implemented again by the infinite product of selection functions.}

\vfill

\grey{We have a Haskell implementation that runs fast in counter-intuitive examples.}


\overhead{Conclusion}

\vfill

\begin{center}
\Huge Selection functions everywhere.
\end{center}

\vfill

\overhead{\darkred{Appendix}}

\overhead{The monads defined as Kleisli triples}

Define the (internalized) extension operators:
\begin{eqnarray*}
  (X \to K Y) & \to & (KX \to KY) \\
    f & \mapsto & f^\sharp \\
    f & \mapsto & \lambda \phi.\lambda p.(\phi(\lambda x.p(f x))). 
\end{eqnarray*}

\vfill

\darkblue{Example:} Take $\darkblue{X=Y=\N}$ and $\darkblue{f(k)(p) = \exists n < k(p(k))}$. 
Then 
\[
\darkblue{f^\sharp(\forall_N)(p) = \forall k \in \N(\exists n < k(p(k)))}.
\]

\overhead{Kleisli extension for $J$}


\begin{eqnarray*}
  (X \to J Y) & \to & (JX \to JY) \\
   g & \mapsto & g^\sharp. \\
%   g & \mapsto & \lambda \varepsilon.\lambda p.(g(\varepsilon(\lambda x.p(g x p))) p.
\end{eqnarray*}
For a suitable \darkblue{$x \in X$} constructed from \darkblue{$\varepsilon$}, we will define:
$
\darkblue{g^\sharp(\varepsilon)(p) = g x p}.
$

Before such an \darkblue{$x \in X$} is available, we abstract it: 
\darkblue{$\lambda x.g x p$}.

But now we can apply \darkblue{$\varepsilon$} to this, to find
\darkblue{$x_0 \stackrel{\text{def}}{=} \varepsilon(\lambda x.g x p) \in X$}, and define
\[
\darkblue{g^\sharp(\varepsilon)(p) = g x_0 p}.
\]
Expanding the definition, we get $\darkblue{g^\sharp(\varepsilon)(p) = g (\varepsilon(\lambda x.g x p)) p}$.


\overhead{Extension operators related by the monad morphism}

\[
\darkblue{\overline{g^\sharp(\varepsilon)} = \left(\lambda x.\overline{g(x)}\right)^\sharp(\overline{\varepsilon})}.
\]
In other words:

\vfill

\darkgreen{Theorem.}
Let \darkblue{$f \colon X \to KY$} and \darkblue{$g \colon X \to JY$}. 

\darkgreen{If} \darkblue{$g(x) \in J Y$} is a selection function of the
quantifier \darkblue{$f(x) \in KY$} for all \darkblue{$x \in X$},

\darkgreen{and} \darkblue{$\varepsilon \in KY$} is a selection
function for the quantifier \darkblue{$\phi \in KX$},

\darkgreen{then} \darkblue{$g^\sharp(\varepsilon) \in J X$} is a
selection function for the quantifier \darkblue{$f^\sharp(\phi) \in K
  X$}.


\end{document}
