\batchmode
\documentclass[12pt]{report}
\makeatletter

\usepackage{amsmath,amscd,latexsym,amssymb,stmaryrd}
\usepackage{../packages/QED}

\usepackage{a4}

\usepackage[english]{babel}

\title{{\small Introduction to} \\{\sc exact numerical computation} \\[1ex]
       {\small Notes for a tutorial at ISSAC 2000}}

\author{Mart\'\i n Escard\'o \\
        School of Computer Science \\
        St Andrews University \\[1ex]
        {\tt http://www.dcs.st-and.ac.uk/\~{}mhe/}}

\providecommand{\on}{\operatorname}
\providecommand{\Cases}{\operatorname{cases}}
\providecommand{\LCases}{\operatorname{lexcases}}
\providecommand{\qarrow}{\twoheadrightarrow}
\providecommand{\sgn}{\operatorname{sgn}}
\providecommand{\parrow}{\rightharpoonup}
\providecommand{\arrow}{\rightarrow}
\providecommand{\two}{\mathbf{2}}
\providecommand{\ten}{D}
\providecommand{\three}{\mathbf{3}}
\providecommand{\twomega}{\mathbf{2}^{\mathbb{N}}}
\providecommand{\tenomega}{D^{\mathbb{N}}}
\providecommand{\twoinfty}{\mathbf{2}^{\infty}}
\providecommand{\thomega}{\mathbf{3}^{\mathbb{N}}}
\providecommand{\thinfty}{\mathbf{3}^{\infty}}
\providecommand{\Real}{\mathbb{R}}
\providecommand{\R}{\mathbb{R}}
\providecommand{\myomega}{\mathbb{N}}
\providecommand{\sscript}[1]{_{\empty_{#1}}}
\providecommand{\id}{\operatorname{id}}
\providecommand{\True}{\operatorname{true}}
\providecommand{\False}{\operatorname{false}}
\providecommand{\Meaning}[1]{\llbracket #1 \rrbracket}
\providecommand{\norm}{\operatorname{norm}}

\providecommand{\myemph}[1]{\emph{#1}}
\providecommand{\smallemph}[1]{{\small\emph{#1}}}
\providecommand{\df}[1]{\emph{#1}}%
\providecommand{\sltitle}[1]{#1}%
\providecommand{\warning}[1]{{\bf #1}}
\providecommand{\mybf}[1]{{\bf #1}}
\providecommand{\mytt}[1]{{\tt #1}}
\providecommand{\cyanbf}[1]{{\bf {#1}}}
\providecommand{\black}[1]{{#1}}
\providecommand{\mathcolor}[1]{#1}%
\providecommand{\mc}[1]{$#1$ }
\providecommand{\md}[1]{\begin{displaymath}#1\end{displaymath}
}
\providecommand{\bitem}{\qquad $\bullet$ ~}
\par\usepackage[dvips]{color}
\pagecolor[gray]{.7}

\AtBeginDocument{\input /home/staff/mhe/papers/issac/issac.aux }

\makeatletter
\count@=\the\catcode`\_ \catcode`\_=8 
\newenvironment{tex2html_wrap}{}{} \catcode`\_=\count@
\makeatother
\let\mathon=$
\let\mathoff=$
\ifx\AtBeginDocument\undefined \newcommand{\AtBeginDocument}[1]{}\fi
\newbox\sizebox
\setlength{\hoffset}{0pt}\setlength{\voffset}{0pt}
\addtolength{\textheight}{\footskip}\setlength{\footskip}{0pt}
\addtolength{\textheight}{\topmargin}\setlength{\topmargin}{0pt}
\addtolength{\textheight}{\headheight}\setlength{\headheight}{0pt}
\addtolength{\textheight}{\headsep}\setlength{\headsep}{0pt}
\setlength{\textwidth}{349pt}
\newwrite\lthtmlwrite
\makeatletter
\let\realnormalsize=\normalsize
\global\topskip=2sp
\def\preveqno{}\let\real@float=\@float \let\realend@float=\end@float
\def\@float{\let\@savefreelist\@freelist\real@float}
\def\end@float{\realend@float\global\let\@freelist\@savefreelist}
\let\real@dbflt=\@dbflt \let\end@dblfloat=\end@float
\let\@largefloatcheck=\relax
\def\@dbflt{\let\@savefreelist\@freelist\real@dbflt}
\def\adjustnormalsize{\def\normalsize{\mathsurround=0pt \realnormalsize
 \parindent=0pt\abovedisplayskip=0pt\belowdisplayskip=0pt}\normalsize}%
\def\lthtmltypeout#1{{\let\protect\string\immediate\write\lthtmlwrite{#1}}}%
\newcommand\lthtmlhboxmathA{\adjustnormalsize\setbox\sizebox=\hbox\bgroup}%
\newcommand\lthtmlvboxmathA{\adjustnormalsize\setbox\sizebox=\vbox\bgroup%
 \let\ifinner=\iffalse }%
\newcommand\lthtmlboxmathZ{\@next\next\@currlist{}{\def\next{\voidb@x}}%
 \expandafter\box\next\egroup}%
\newcommand\lthtmlmathtype[1]{\def\lthtmlmathenv{#1}}%
\newcommand\lthtmllogmath{\lthtmltypeout{l2hSize %
:\lthtmlmathenv:\the\ht\sizebox::\the\dp\sizebox::\the\wd\sizebox.\preveqno}}%
\newcommand\lthtmlfigureA[1]{\let\@savefreelist\@freelist
       \lthtmlmathtype{#1}\lthtmlvboxmathA}%
\newcommand\lthtmlfigureZ{\lthtmlboxmathZ\lthtmllogmath\copy\sizebox
       \global\let\@freelist\@savefreelist}%
\newcommand\lthtmldisplayA[1]{\lthtmlmathtype{#1}\lthtmlvboxmathA}%
\newcommand\lthtmldisplayB[1]{\edef\preveqno{(\theequation)}%
  \lthtmldisplayA{#1}\let\@eqnnum\relax}%
\newcommand\lthtmldisplayZ{\lthtmlboxmathZ\lthtmllogmath\lthtmlsetmath}%
\newcommand\lthtmlinlinemathA[1]{\lthtmlmathtype{#1}\lthtmlhboxmathA  \vrule height1.5ex width0pt }%
\newcommand\lthtmlinlineA[1]{\lthtmlmathtype{#1}\lthtmlhboxmathA}%
\newcommand\lthtmlinlineZ{\egroup\expandafter\ifdim\dp\sizebox>0pt %
  \expandafter\centerinlinemath\fi\lthtmllogmath\lthtmlsetinline}
\newcommand\lthtmlinlinemathZ{\egroup\expandafter\ifdim\dp\sizebox>0pt %
  \expandafter\centerinlinemath\fi\lthtmllogmath\lthtmlsetmath}
\def\lthtmlsetinline{\hbox{\vrule width.1em\vtop{\vbox{%
  \kern.1em\copy\sizebox}\ifdim\dp\sizebox>0pt\kern.1em\else\kern.3pt\fi
  \ifdim\hsize>\wd\sizebox \hrule depth1pt\fi}}}
\def\lthtmlsetmath{\hbox{\vrule width.1em\vtop{\vbox{%
  \kern.1em\kern0.8 pt\hbox{\hglue.17em\copy\sizebox\hglue0.8 pt}}\kern.3pt%
  \ifdim\dp\sizebox>0pt\kern.1em\fi \kern0.8 pt%
  \ifdim\hsize>\wd\sizebox \hrule depth1pt\fi}}}
\def\centerinlinemath{%\dimen1=\ht\sizebox
  \dimen1=\ifdim\ht\sizebox<\dp\sizebox \dp\sizebox\else\ht\sizebox\fi
  \advance\dimen1by.5pt \vrule width0pt height\dimen1 depth\dimen1 
 \dp\sizebox=\dimen1\ht\sizebox=\dimen1\relax}

\def\lthtmlcheckvsize{\ifdim\ht\sizebox<\vsize\expandafter\vfill
  \else\expandafter\vss\fi}%
\makeatletter \tracingstats = 1 


\begin{document}
\pagestyle{empty}\thispagestyle{empty}%
\lthtmltypeout{latex2htmlLength hsize=\the\hsize}%
\lthtmltypeout{latex2htmlLength vsize=\the\vsize}%
\lthtmltypeout{latex2htmlLength hoffset=\the\hoffset}%
\lthtmltypeout{latex2htmlLength voffset=\the\voffset}%
\lthtmltypeout{latex2htmlLength topmargin=\the\topmargin}%
\lthtmltypeout{latex2htmlLength topskip=\the\topskip}%
\lthtmltypeout{latex2htmlLength headheight=\the\headheight}%
\lthtmltypeout{latex2htmlLength headsep=\the\headsep}%
\lthtmltypeout{latex2htmlLength parskip=\the\parskip}%
\lthtmltypeout{latex2htmlLength oddsidemargin=\the\oddsidemargin}%
\makeatletter
\if@twoside\lthtmltypeout{latex2htmlLength evensidemargin=\the\evensidemargin}%
\else\lthtmltypeout{latex2htmlLength evensidemargin=\the\oddsidemargin}\fi%
\makeatother
\stepcounter{chapter}
\stepcounter{paragraph}
\stepcounter{chapter}
{\newpage\clearpage
\lthtmlinlinemathA{tex2html_wrap_inline802}%
$\mathbb{M} \subseteq \mathbb{R}$%
\lthtmlinlinemathZ
\hfill\lthtmlcheckvsize\clearpage}

\stepcounter{paragraph}
{\newpage\clearpage
\lthtmlinlinemathA{tex2html_wrap_inline812}%
$f:[0,1] \to [0,1]$%
\lthtmlinlinemathZ
\hfill\lthtmlcheckvsize\clearpage}

{\newpage\clearpage
\lthtmlinlinemathA{tex2html_wrap_inline820}%
$x_0, \, f(x_0),\,f(f(x_0)),\, \dots,\, f^n(x_0),\,\dots$%
\lthtmlinlinemathZ
\hfill\lthtmlcheckvsize\clearpage}

\stepcounter{chapter}
\stepcounter{paragraph}
\stepcounter{paragraph}
\stepcounter{paragraph}
\stepcounter{paragraph}
\stepcounter{paragraph}
{\newpage\clearpage
\lthtmlinlinemathA{tex2html_wrap_inline898}%
$0 \cdot 3$%
\lthtmlinlinemathZ
\hfill\lthtmlcheckvsize\clearpage}

{\newpage\clearpage
\lthtmldisplayA{displaymath762}%
\begin{displaymath}3 \times
0.333333333333333333333333333333333333333333 \end{displaymath}%
\lthtmldisplayZ
\hfill\lthtmlcheckvsize\clearpage}

{\newpage\clearpage
\lthtmldisplayA{eqnarraystar197}%
\begin{eqnarray*}9\bar{1}{5}\bar{2} & = & 9 \times 10^3 - 1 \times 10^2 + 5 \times 2^1
    - 2 \times 2^0 \\& = & 9000-100+50-2  \\& = & 8948,
\end{eqnarray*}%
\lthtmldisplayZ
\hfill\lthtmlcheckvsize\clearpage}

{\newpage\clearpage
\lthtmlinlinemathA{tex2html_wrap_inline922}%
$\bar{1}$%
\lthtmlinlinemathZ
\hfill\lthtmlcheckvsize\clearpage}

{\newpage\clearpage
\lthtmlinlinemathA{tex2html_wrap_inline924}%
$1.\bar{1}$%
\lthtmlinlinemathZ
\hfill\lthtmlcheckvsize\clearpage}

\stepcounter{paragraph}
{\newpage\clearpage
\lthtmlinlinemathA{tex2html_wrap_inline944}%
$\lceil (6+1) \log_2 10 \rceil = 24$%
\lthtmlinlinemathZ
\hfill\lthtmlcheckvsize\clearpage}

\stepcounter{paragraph}
\stepcounter{paragraph}
\stepcounter{chapter}
\stepcounter{paragraph}
\stepcounter{paragraph}
\stepcounter{paragraph}
\stepcounter{paragraph}
\stepcounter{paragraph}
\stepcounter{chapter}
{\newpage\clearpage
\lthtmlinlinemathA{tex2html_wrap_inline960}%
$\pi$%
\lthtmlinlinemathZ
\hfill\lthtmlcheckvsize\clearpage}

{\newpage\clearpage
\lthtmlinlinemathA{tex2html_wrap_inline962}%
$\alpha$%
\lthtmlinlinemathZ
\hfill\lthtmlcheckvsize\clearpage}

{\newpage\clearpage
\lthtmlinlinemathA{tex2html_wrap_inline964}%
$\beta$%
\lthtmlinlinemathZ
\hfill\lthtmlcheckvsize\clearpage}

{\newpage\clearpage
\lthtmlinlinemathA{tex2html_wrap_inline974}%
$\epsilon-\delta$%
\lthtmlinlinemathZ
\hfill\lthtmlcheckvsize\clearpage}

{\newpage\clearpage
\lthtmlinlinemathA{tex2html_wrap_inline976}%
$\alpha
\equiv_n \beta$%
\lthtmlinlinemathZ
\hfill\lthtmlcheckvsize\clearpage}

{\newpage\clearpage
\lthtmlinlinemathA{tex2html_wrap_inline984}%
$\alpha_i = \beta_i$%
\lthtmlinlinemathZ
\hfill\lthtmlcheckvsize\clearpage}

{\newpage\clearpage
\lthtmlinlinemathA{tex2html_wrap_inline990}%
$\alpha \in U$%
\lthtmlinlinemathZ
\hfill\lthtmlcheckvsize\clearpage}

{\newpage\clearpage
\lthtmlinlinemathA{tex2html_wrap_inline994}%
$\beta \in U$%
\lthtmlinlinemathZ
\hfill\lthtmlcheckvsize\clearpage}

{\newpage\clearpage
\lthtmlinlinemathA{tex2html_wrap_inline996}%
$\beta \equiv_n \alpha$%
\lthtmlinlinemathZ
\hfill\lthtmlcheckvsize\clearpage}

{\newpage\clearpage
\lthtmlinlinemathA{tex2html_wrap_inline998}%
$\alpha_i$%
\lthtmlinlinemathZ
\hfill\lthtmlcheckvsize\clearpage}

{\newpage\clearpage
\lthtmlinlinemathA{tex2html_wrap_inline1008}%
$\alpha_i \in U$%
\lthtmlinlinemathZ
\hfill\lthtmlcheckvsize\clearpage}

{\newpage\clearpage
\lthtmlinlinemathA{tex2html_wrap_inline1010}%
$i \ge n$%
\lthtmlinlinemathZ
\hfill\lthtmlcheckvsize\clearpage}

{\newpage\clearpage
\lthtmlinlinemathA{tex2html_wrap_inline1012}%
$\phi$%
\lthtmlinlinemathZ
\hfill\lthtmlcheckvsize\clearpage}

{\newpage\clearpage
\lthtmlinlinemathA{tex2html_wrap_inline1020}%
$\alpha \equiv_k \beta$%
\lthtmlinlinemathZ
\hfill\lthtmlcheckvsize\clearpage}

{\newpage\clearpage
\lthtmlinlinemathA{tex2html_wrap_inline1022}%
$\phi(\alpha) \equiv_n
\phi(\beta)$%
\lthtmlinlinemathZ
\hfill\lthtmlcheckvsize\clearpage}

{\newpage\clearpage
\lthtmlinlinemathA{tex2html_wrap_inline1026}%
$D= \{ 0,1,\dots,9 \}$%
\lthtmlinlinemathZ
\hfill\lthtmlcheckvsize\clearpage}

{\newpage\clearpage
\lthtmlinlinemathA{tex2html_wrap_inline1028}%
$D^{\mathbb{N}}$%
\lthtmlinlinemathZ
\hfill\lthtmlcheckvsize\clearpage}

\stepcounter{chapter}
\stepcounter{paragraph}
{\newpage\clearpage
\lthtmlinlinemathA{tex2html_wrap_inline1032}%
$\alpha \in D^{\mathbb{N}}$%
\lthtmlinlinemathZ
\hfill\lthtmlcheckvsize\clearpage}

{\newpage\clearpage
\lthtmldisplayA{displaymath763}%
\begin{displaymath}\llbracket \alpha \rrbracket = \sum_{i \geqslant 0} \alpha_i \cdot 10^{-(i+1)} \in
[0,1].
\end{displaymath}%
\lthtmldisplayZ
\hfill\lthtmlcheckvsize\clearpage}

{\newpage\clearpage
\lthtmlinlinemathA{tex2html_wrap_inline1034}%
$\alpha \mapsto \llbracket \alpha \rrbracket$%
\lthtmlinlinemathZ
\hfill\lthtmlcheckvsize\clearpage}

{\newpage\clearpage
\lthtmldisplayA{displaymath764}%
\begin{displaymath}q:D^{\mathbb{N}}\twoheadrightarrow[0,1]. 
\end{displaymath}%
\lthtmldisplayZ
\hfill\lthtmlcheckvsize\clearpage}

{\newpage\clearpage
\lthtmlinlinemathA{tex2html_wrap_inline1036}%
$m/10^n \in
(0,1)$%
\lthtmlinlinemathZ
\hfill\lthtmlcheckvsize\clearpage}

\stepcounter{paragraph}
{\newpage\clearpage
\lthtmlinlinemathA{tex2html_wrap_inline1042}%
$\phi:D^{\mathbb{N}}\rightarrow
D^{\mathbb{N}}$%
\lthtmlinlinemathZ
\hfill\lthtmlcheckvsize\clearpage}

{\newpage\clearpage
\lthtmlinlinemathA{tex2html_wrap_inline1044}%
$f:[0,1] \rightarrow[0,1]$%
\lthtmlinlinemathZ
\hfill\lthtmlcheckvsize\clearpage}

{\newpage\clearpage
\lthtmldisplayA{displaymath765}%
\begin{displaymath}f(\llbracket \alpha \rrbracket)=\llbracket \phi(\alpha) \rrbracket, \end{displaymath}%
\lthtmldisplayZ
\hfill\lthtmlcheckvsize\clearpage}

{\newpage\clearpage
\lthtmldisplayA{displaymath766}%
\begin{displaymath}\begin{CD}
D^{\mathbb{N}}@>{\phi}>> D^{\mathbb{N}}\\
@V{q}VV @VV{q}V \\
[0,1] @>>{f}> [0,1].
\end{CD} 
\end{displaymath}%
\lthtmldisplayZ
\hfill\lthtmlcheckvsize\clearpage}

{\newpage\clearpage
\lthtmlinlinemathA{tex2html_wrap_inline1048}%
$\phi(\alpha)$%
\lthtmlinlinemathZ
\hfill\lthtmlcheckvsize\clearpage}

{\newpage\clearpage
\lthtmlinlinemathA{tex2html_wrap_inline1058}%
$q:D^{\mathbb{N}}\twoheadrightarrow[0,1]$%
\lthtmlinlinemathZ
\hfill\lthtmlcheckvsize\clearpage}

{\newpage\clearpage
\lthtmlfigureA{proof285}%
\begin{proof}Simply because it is continuous, and a continuous surjection of
compact Hausdorff spaces is always a quotient map. 
\end{proof}%
\lthtmlfigureZ
\hfill\lthtmlcheckvsize\clearpage}

{\newpage\clearpage
\lthtmlfigureA{proof293}%
\begin{proof}Let $\phi:D^{\mathbb{N}}\to D^{\mathbb{N}}$\space be a realizer of $f$ . Then, for
  any $\alpha$\space and any $n \ge 0$ , the value of $\phi(3^n2\alpha)$\space is
  of the form $0\beta$ . Thus, if $\phi$\space were continuous,
  $\phi(3^\omega)$\space would be of the form $0\beta'$ . But, similarly, for
  any $\alpha$\space and any $n \ge 0$ , the value of $\phi(3^n4\alpha)$\space is
  of the form $1\beta$ , and continuity of $\phi$\space would imply that
  $\phi(3^\omega)$\space would be of the form $1\beta''$ .  Thus,
  $\phi(3^\omega)$\space would have to be of the forms $0\beta'$\space and
  $1\beta''$\space at the same time.  Since this is impossible, we conclude
  that $\phi$\space cannot be continuous.
\end{proof}%
\lthtmlfigureZ
\hfill\lthtmlcheckvsize\clearpage}

\stepcounter{chapter}
\stepcounter{paragraph}
{\newpage\clearpage
\lthtmlinlinemathA{tex2html_wrap_inline1118}%
$\mathbf{3}=\{\bar{1},0,1\}$%
\lthtmlinlinemathZ
\hfill\lthtmlcheckvsize\clearpage}

{\newpage\clearpage
\lthtmlinlinemathA{tex2html_wrap_inline1124}%
$\alpha \in \mathbf{3}^{\mathbb{N}}$%
\lthtmlinlinemathZ
\hfill\lthtmlcheckvsize\clearpage}

{\newpage\clearpage
\lthtmldisplayA{displaymath767}%
\begin{displaymath}\llbracket \alpha \rrbracket = \sum_{i \geqslant 0} \alpha_i \cdot 2^{-(i+1)} \in [-1,1],
\end{displaymath}%
\lthtmldisplayZ
\hfill\lthtmlcheckvsize\clearpage}

{\newpage\clearpage
\lthtmlinlinemathA{tex2html_wrap_inline1128}%
$q:\mathbf{3}^{\mathbb{N}}\twoheadrightarrow[-1,1]$%
\lthtmlinlinemathZ
\hfill\lthtmlcheckvsize\clearpage}

{\newpage\clearpage
\lthtmlinlinemathA{tex2html_wrap_inline1130}%
$f:[-1,1]\rightarrow[-1,1]$%
\lthtmlinlinemathZ
\hfill\lthtmlcheckvsize\clearpage}

{\newpage\clearpage
\lthtmlinlinemathA{tex2html_wrap_inline1132}%
$\phi:\mathbf{3}^{\mathbb{N}}\rightarrow\mathbf{3}^{\mathbb{N}}$%
\lthtmlinlinemathZ
\hfill\lthtmlcheckvsize\clearpage}

{\newpage\clearpage
\lthtmlfigureA{proof333}%
\begin{proof}M{\"u}ller~\cite{Mue86a}, and Weihrauch and
  Kreitz~\cite{weihrauch:kreitz,kreitz:weihrauch}, showed that the quotient map
  $q:\mathbf{3}^{\mathbb{N}}\twoheadrightarrow[-1,1]$\space is \emph{admissible} (or maximal) in the
  following sense.  For every quotient map $q':\mathbf{3}^{\mathbb{N}}\twoheadrightarrow
  [-1,1]$ , there is a (far from unique) continuous map $t:\mathbf{3}^{\mathbb{N}}
  \rightarrow\mathbf{3}^{\mathbb{N}}$\space which translates from $q'$ -notation to
  $q$ -notation, meaning that $q' = q \circ t$ .  The same argument
  shows that, more generally, for every continuous map $g:\mathbf{3}^{\mathbb{N}}
  \rightarrow[-1,1]$\space there is a continuous map $\phi:\mathbf{3}^{\mathbb{N}}\rightarrow
  \mathbf{3}^{\mathbb{N}}$\space such that $g = q \circ \phi$ .  (In other words, the space
  $\mathbf{3}^{\mathbb{N}}$\space is projective over the quotient map $q$ .) Then the result
  follows by taking $g=f \circ q$ .  \qed\end{proof}%
\lthtmlfigureZ
\hfill\lthtmlcheckvsize\clearpage}

\stepcounter{paragraph}
{\newpage\clearpage
\lthtmlinlinemathA{tex2html_wrap_inline1162}%
$\mathbf{3}^{\mathbb{N}}$%
\lthtmlinlinemathZ
\hfill\lthtmlcheckvsize\clearpage}

{\newpage\clearpage
\lthtmlinlinemathA{tex2html_wrap_inline1170}%
$\mathbb{R}\setminus \{0\}$%
\lthtmlinlinemathZ
\hfill\lthtmlcheckvsize\clearpage}

\stepcounter{chapter}
{\newpage\clearpage
\lthtmlinlinemathA{tex2html_wrap_inline1178}%
$m/2^n \in (0,1)$%
\lthtmlinlinemathZ
\hfill\lthtmlcheckvsize\clearpage}

{\newpage\clearpage
\lthtmlinlinemathA{tex2html_wrap_inline1180}%
$\equiv$%
\lthtmlinlinemathZ
\hfill\lthtmlcheckvsize\clearpage}

{\newpage\clearpage
\lthtmlinlinemathA{tex2html_wrap_inline1182}%
$\alpha \equiv \beta$%
\lthtmlinlinemathZ
\hfill\lthtmlcheckvsize\clearpage}

{\newpage\clearpage
\lthtmlinlinemathA{tex2html_wrap_inline1184}%
$\llbracket \alpha \rrbracket=\llbracket \beta \rrbracket$%
\lthtmlinlinemathZ
\hfill\lthtmlcheckvsize\clearpage}

{\newpage\clearpage
\lthtmlinlinemathA{tex2html_wrap_inline1186}%
$\alpha \in \mathbf{3}^{\star}$%
\lthtmlinlinemathZ
\hfill\lthtmlcheckvsize\clearpage}

{\newpage\clearpage
\lthtmlinlinemathA{tex2html_wrap_inline1188}%
$\beta \in \mathbf{3}^{\mathbb{N}}$%
\lthtmlinlinemathZ
\hfill\lthtmlcheckvsize\clearpage}

{\newpage\clearpage
\lthtmlinlinemathA{tex2html_wrap_inline1190}%
$\alpha 0 \bar{1} \beta \equiv \alpha \bar{1} 1 \beta$%
\lthtmlinlinemathZ
\hfill\lthtmlcheckvsize\clearpage}

{\newpage\clearpage
\lthtmlinlinemathA{tex2html_wrap_inline1192}%
$\alpha 0 1 \beta \equiv \alpha 1 \bar{1} \beta$%
\lthtmlinlinemathZ
\hfill\lthtmlcheckvsize\clearpage}

{\newpage\clearpage
\lthtmlinlinemathA{tex2html_wrap_inline1194}%
$\alpha < \beta$%
\lthtmlinlinemathZ
\hfill\lthtmlcheckvsize\clearpage}

{\newpage\clearpage
\lthtmlinlinemathA{tex2html_wrap_inline1198}%
$\alpha_k < \beta_k$%
\lthtmlinlinemathZ
\hfill\lthtmlcheckvsize\clearpage}

{\newpage\clearpage
\lthtmlfigureA{proof362}%
\begin{proof}The preimage of a point by the quotient map $q:\mathbf{3}^{\mathbb{N}}\rightarrow
  [-1,1]$\space is a closed set because $q$\space is continuous. But topologically
  closed subsets of~$\mathbf{3}^{\mathbb{N}}$\space are closed under non-empty infima and
  suprema in the lexicographical order (in fact, this characterizes
  topologically closed subsets).  \qed\end{proof}%
\lthtmlfigureZ
\hfill\lthtmlcheckvsize\clearpage}

{\newpage\clearpage
\lthtmlinlinemathA{tex2html_wrap_inline1210}%
$c:\mathbf{3}^{\mathbb{N}}\rightarrow\mathbf{3}^{\mathbb{N}}$%
\lthtmlinlinemathZ
\hfill\lthtmlcheckvsize\clearpage}

{\newpage\clearpage
\lthtmlinlinemathA{tex2html_wrap_inline1212}%
$c(q^{-1}(\{x\}))$%
\lthtmlinlinemathZ
\hfill\lthtmlcheckvsize\clearpage}

{\newpage\clearpage
\lthtmlinlinemathA{tex2html_wrap_inline1214}%
$x \in [-1,1]$%
\lthtmlinlinemathZ
\hfill\lthtmlcheckvsize\clearpage}

{\newpage\clearpage
\lthtmlfigureA{proof368}%
\begin{proof}If there were, $[-1,1]$\space would be a retract of $\mathbf{3}^{\mathbb{N}}$ . But
  $\mathbf{3}^{\mathbb{N}}$\space is a totally disconnected space, and such spaces are
  closed under retracts. On the other hand, $[-1,1]$\space is connected.
\end{proof}%
\lthtmlfigureZ
\hfill\lthtmlcheckvsize\clearpage}

\stepcounter{chapter}
{\newpage\clearpage
\lthtmlinlinemathA{tex2html_wrap_inline1224}%
$\alpha,\beta$%
\lthtmlinlinemathZ
\hfill\lthtmlcheckvsize\clearpage}

{\newpage\clearpage
\lthtmlinlinemathA{tex2html_wrap_inline1226}%
$\alpha \leqslant \beta$%
\lthtmlinlinemathZ
\hfill\lthtmlcheckvsize\clearpage}

{\newpage\clearpage
\lthtmlinlinemathA{tex2html_wrap_inline1228}%
$\llbracket \alpha \rrbracket \leqslant \llbracket \beta \rrbracket$%
\lthtmlinlinemathZ
\hfill\lthtmlcheckvsize\clearpage}

{\newpage\clearpage
\lthtmlinlinemathA{tex2html_wrap_inline1230}%
$\alpha=\bar{1}1^\omega$%
\lthtmlinlinemathZ
\hfill\lthtmlcheckvsize\clearpage}

{\newpage\clearpage
\lthtmlinlinemathA{tex2html_wrap_inline1232}%
$\beta=0\bar{1}^\omega$%
\lthtmlinlinemathZ
\hfill\lthtmlcheckvsize\clearpage}

{\newpage\clearpage
\lthtmlinlinemathA{tex2html_wrap_inline1236}%
$\llbracket \alpha \rrbracket=0 \not\leqslant -1/2 =
\llbracket \beta \rrbracket$%
\lthtmlinlinemathZ
\hfill\lthtmlcheckvsize\clearpage}

{\newpage\clearpage
\lthtmlinlinemathA{tex2html_wrap_inline1238}%
$\alpha=10^{\omega}$%
\lthtmlinlinemathZ
\hfill\lthtmlcheckvsize\clearpage}

{\newpage\clearpage
\lthtmlinlinemathA{tex2html_wrap_inline1240}%
$\beta=01^{\omega}$%
\lthtmlinlinemathZ
\hfill\lthtmlcheckvsize\clearpage}

{\newpage\clearpage
\lthtmlinlinemathA{tex2html_wrap_inline1242}%
$\llbracket \alpha \rrbracket=\llbracket \beta \rrbracket=1/2$%
\lthtmlinlinemathZ
\hfill\lthtmlcheckvsize\clearpage}

{\newpage\clearpage
\lthtmlinlinemathA{tex2html_wrap_inline1244}%
$\alpha \not\leqslant
\beta$%
\lthtmlinlinemathZ
\hfill\lthtmlcheckvsize\clearpage}

\stepcounter{paragraph}
{\newpage\clearpage
\lthtmlinlinemathA{tex2html_wrap_inline1246}%
$(\alpha,\beta)$%
\lthtmlinlinemathZ
\hfill\lthtmlcheckvsize\clearpage}

{\newpage\clearpage
\lthtmlinlinemathA{tex2html_wrap_inline1250}%
$\llbracket \alpha \rrbracket < \llbracket \beta \rrbracket$%
\lthtmlinlinemathZ
\hfill\lthtmlcheckvsize\clearpage}

{\newpage\clearpage
\lthtmlinlinemathA{tex2html_wrap_inline1252}%
$\alpha = \beta$%
\lthtmlinlinemathZ
\hfill\lthtmlcheckvsize\clearpage}

{\newpage\clearpage
\lthtmlinlinemathA{tex2html_wrap_inline1254}%
$\alpha > \beta$%
\lthtmlinlinemathZ
\hfill\lthtmlcheckvsize\clearpage}

{\newpage\clearpage
\lthtmlinlinemathA{tex2html_wrap_inline1256}%
$\llbracket \alpha \rrbracket > \llbracket \beta \rrbracket$%
\lthtmlinlinemathZ
\hfill\lthtmlcheckvsize\clearpage}

{\newpage\clearpage
\lthtmlinlinemathA{tex2html_wrap_inline1264}%
$(\alpha,\alpha)$%
\lthtmlinlinemathZ
\hfill\lthtmlcheckvsize\clearpage}

{\newpage\clearpage
\lthtmldisplayA{displaymath768}%
\begin{displaymath}\operatorname{norm}:\mathbf{3}^{\mathbb{N}}\times \mathbf{3}^{\mathbb{N}}\rightarrow\mathbf{3}^{\mathbb{N}}\times \mathbf{3}^{\mathbb{N}}
  \end{displaymath}%
\lthtmldisplayZ
\hfill\lthtmlcheckvsize\clearpage}

{\newpage\clearpage
\lthtmlinlinemathA{tex2html_wrap_inline1266}%
$\operatorname{norm}(\alpha,\beta)=(\alpha',\beta')$%
\lthtmlinlinemathZ
\hfill\lthtmlcheckvsize\clearpage}

{\newpage\clearpage
\lthtmlinlinemathA{tex2html_wrap_inline1268}%
$(\llbracket \alpha \rrbracket,\llbracket \beta \rrbracket)=(\llbracket \alpha' \rrbracket,\llbracket \beta' \rrbracket)$%
\lthtmlinlinemathZ
\hfill\lthtmlcheckvsize\clearpage}

{\newpage\clearpage
\lthtmlinlinemathA{tex2html_wrap_inline1270}%
$(\alpha',\beta')$%
\lthtmlinlinemathZ
\hfill\lthtmlcheckvsize\clearpage}

{\newpage\clearpage
\lthtmlinlinemathA{tex2html_wrap_inline1272}%
$\operatorname{norm}(\alpha',\beta')=(\alpha',\beta')$%
\lthtmlinlinemathZ
\hfill\lthtmlcheckvsize\clearpage}

{\newpage\clearpage
\lthtmlinlinemathA{tex2html_wrap_inline1274}%
$\sim$%
\lthtmlinlinemathZ
\hfill\lthtmlcheckvsize\clearpage}

{\newpage\clearpage
\lthtmlinlinemathA{tex2html_wrap_inline1276}%
$\alpha 0 \bar{1} \beta \sim \alpha \bar{1} 1 \beta$%
\lthtmlinlinemathZ
\hfill\lthtmlcheckvsize\clearpage}

{\newpage\clearpage
\lthtmlinlinemathA{tex2html_wrap_inline1278}%
$\alpha 0 1 \beta \sim \alpha 1 \bar{1} \beta$%
\lthtmlinlinemathZ
\hfill\lthtmlcheckvsize\clearpage}

{\newpage\clearpage
\lthtmlinlinemathA{tex2html_wrap_inline1284}%
$1\bar{1}^\omega \equiv \bar{1}1^\omega$%
\lthtmlinlinemathZ
\hfill\lthtmlcheckvsize\clearpage}

{\newpage\clearpage
\lthtmlinlinemathA{tex2html_wrap_inline1298}%
$\mathbf{3}^{\mathbb{N}}\times \mathbf{3}^{\mathbb{N}}$%
\lthtmlinlinemathZ
\hfill\lthtmlcheckvsize\clearpage}

\stepcounter{chapter}
{\newpage\clearpage
\lthtmldisplayA{displaymath769}%
\begin{displaymath}\min(x,y)  =  \begin{cases}
                    x & \text{if $x \leqslant y$ ,} \\
                    y & \text{otherwise,}
                    \end{cases}
\end{displaymath}%
\lthtmldisplayZ
\hfill\lthtmlcheckvsize\clearpage}

{\newpage\clearpage
\lthtmldisplayA{displaymath770}%
\begin{displaymath}\operatorname{cases}:[-1,1]^4 \rightharpoonup[-1,1]
\end{displaymath}%
\lthtmldisplayZ
\hfill\lthtmlcheckvsize\clearpage}

{\newpage\clearpage
\lthtmldisplayA{displaymath771}%
\begin{displaymath}\operatorname{cases}(x,t,y,z) = \begin{cases}
                    y & \text{if $x < t$\space or $y=z$ ,} \\
                    z & \text{if $x > t$\space or $y=z$ ,}
                    \end{cases}
\end{displaymath}%
\lthtmldisplayZ
\hfill\lthtmlcheckvsize\clearpage}

{\newpage\clearpage
\lthtmlinlinemathA{tex2html_wrap_inline1320}%
$ \{ (x,t,y,z) \mid \text{$x = t$\space implies $y=z$ }
\}.  $%
\lthtmlinlinemathZ
\hfill\lthtmlcheckvsize\clearpage}

{\newpage\clearpage
\lthtmldisplayA{displaymath772}%
\begin{displaymath}\operatorname{cases}(x,t,y,z) = \begin{cases}
                    y & \text{if $x \leqslant t$ ,} \\
                    z & \text{if $x \geqslant t$ .}
                    \end{cases}
\end{displaymath}%
\lthtmldisplayZ
\hfill\lthtmlcheckvsize\clearpage}

{\newpage\clearpage
\lthtmlfigureA{proof437}%
\begin{proof}We first consider an analogous lexicographical case-analysis operator
$
\operatorname{lexcases}:\left(\mathbf{3}^{\mathbb{N}}\right)^4 \rightharpoonup\mathbf{3}^{\mathbb{N}}
$ such that
\begin{displaymath}
\operatorname{lexcases}(\alpha,\beta,\gamma,\delta) = \begin{cases}
                    \gamma & \text{if $\alpha < \beta$\space or $\gamma=\delta$ ,} \\
                    \delta & \text{if $\alpha > \beta $\space or $\gamma=\delta$ ,}
                    \end{cases}
\end{displaymath}

with domain of definition $ \{ (\alpha,\beta,\gamma,\delta) \mid
\text{$\alpha = \beta$\space implies $\gamma=\delta$ } \}.$   In order to compute $\operatorname{lexcases}(\alpha,\beta,\gamma,\delta)$ , we can
  first output the greatest common prefix of $\gamma$\space and $\delta$ . If
  it is finite then $\gamma \ne \delta$\space and hence we must have $\alpha
  < \beta$\space or $\alpha > \beta$ . In the first case we output the
  remainder of $\gamma$\space and in the second the remainder of $\delta$ .
Now, a realizer for the numerical case-analysis operator is given by
$  %
\phi(\alpha,\beta,\gamma,\delta)  =
  \operatorname{lexcases}(\alpha',\beta',\gamma',\delta')
$\space %
where 
$(\alpha',\beta') =\operatorname{norm}(\alpha,\beta)$   and 
$(\gamma',\delta')=\operatorname{norm}(\gamma,\delta)$ .
\qed\end{proof}%
\lthtmlfigureZ
\hfill\lthtmlcheckvsize\clearpage}

{\newpage\clearpage
\lthtmldisplayA{displaymath774}%
\begin{displaymath}\min(x,y)= \operatorname{cases}(x,y,x,y),
\end{displaymath}%
\lthtmldisplayZ
\hfill\lthtmlcheckvsize\clearpage}

{\newpage\clearpage
\lthtmlinlinemathA{tex2html_wrap_inline1364}%
$\min$%
\lthtmlinlinemathZ
\hfill\lthtmlcheckvsize\clearpage}

{\newpage\clearpage
\lthtmlinlinemathA{tex2html_wrap_inline1366}%
$f,g:\mathbb{R}\rightarrow
\mathbb{R}$%
\lthtmlinlinemathZ
\hfill\lthtmlcheckvsize\clearpage}

{\newpage\clearpage
\lthtmlinlinemathA{tex2html_wrap_inline1370}%
$h:\mathbb{R}\rightarrow\mathbb{R}$%
\lthtmlinlinemathZ
\hfill\lthtmlcheckvsize\clearpage}

{\newpage\clearpage
\lthtmldisplayA{displaymath775}%
\begin{displaymath}h(x)  =  \begin{cases} 
  f(x) & \text{if $x \leqslant x_0$ ,} \\
  g(x) & \text{if $x \geqslant x_0$ }
  \end{cases}
\end{displaymath}%
\lthtmldisplayZ
\hfill\lthtmlcheckvsize\clearpage}

{\newpage\clearpage
\lthtmldisplayA{displaymath776}%
\begin{displaymath}h(x)=\operatorname{cases}(x,x_0,f(x),g(x)).
\end{displaymath}%
\lthtmldisplayZ
\hfill\lthtmlcheckvsize\clearpage}

{\newpage\clearpage
\lthtmldisplayA{displaymath777}%
\begin{displaymath}\operatorname{sgn}(x)  =  \begin{cases}
                    -1 & \text{if $x < 0$ ,} \\
                    1 & \text{if $x > 0$ }
                    \end{cases}
\end{displaymath}%
\lthtmldisplayZ
\hfill\lthtmlcheckvsize\clearpage}

{\newpage\clearpage
\lthtmldisplayA{displaymath778}%
\begin{displaymath}\operatorname{sgn}(x) = \operatorname{cases}(x,0,-1,1)
\end{displaymath}%
\lthtmldisplayZ
\hfill\lthtmlcheckvsize\clearpage}

{\newpage\clearpage
\lthtmlinlinemathA{tex2html_wrap_inline1386}%
$y \ne z$%
\lthtmlinlinemathZ
\hfill\lthtmlcheckvsize\clearpage}

\stepcounter{chapter}
\stepcounter{paragraph}
\stepcounter{paragraph}
\stepcounter{paragraph}
{\newpage\clearpage
\lthtmlinlinemathA{tex2html_wrap_inline1394}%
$R \to R$%
\lthtmlinlinemathZ
\hfill\lthtmlcheckvsize\clearpage}

{\newpage\clearpage
\lthtmlinlinemathA{tex2html_wrap_inline1396}%
$R \to B$%
\lthtmlinlinemathZ
\hfill\lthtmlcheckvsize\clearpage}

{\newpage\clearpage
\lthtmlinlinemathA{tex2html_wrap_inline1398}%
$N \to R$%
\lthtmlinlinemathZ
\hfill\lthtmlcheckvsize\clearpage}

{\newpage\clearpage
\lthtmlinlinemathA{tex2html_wrap_inline1400}%
$N \to (N \to R)$%
\lthtmlinlinemathZ
\hfill\lthtmlcheckvsize\clearpage}

{\newpage\clearpage
\lthtmlinlinemathA{tex2html_wrap_inline1402}%
$N \to (R
  \to R)$%
\lthtmlinlinemathZ
\hfill\lthtmlcheckvsize\clearpage}

{\newpage\clearpage
\lthtmlinlinemathA{tex2html_wrap_inline1404}%
$(N \to R) \to R$%
\lthtmlinlinemathZ
\hfill\lthtmlcheckvsize\clearpage}

{\newpage\clearpage
\lthtmlinlinemathA{tex2html_wrap_inline1406}%
$(R \to R)
  \to R$%
\lthtmlinlinemathZ
\hfill\lthtmlcheckvsize\clearpage}

{\newpage\clearpage
\lthtmlinlinemathA{tex2html_wrap_inline1408}%
$(R \to B) \to B$%
\lthtmlinlinemathZ
\hfill\lthtmlcheckvsize\clearpage}

{\newpage\clearpage
\lthtmlinlinemathA{tex2html_wrap_inline1410}%
$\sqrt{2}$%
\lthtmlinlinemathZ
\hfill\lthtmlcheckvsize\clearpage}

{\newpage\clearpage
\lthtmlinlinemathA{tex2html_wrap_inline1416}%
$\{\sqrt{2}\}$%
\lthtmlinlinemathZ
\hfill\lthtmlcheckvsize\clearpage}

\stepcounter{paragraph}
\stepcounter{paragraph}
{\newpage\clearpage
\lthtmlfigureA{tex2html_wrap2738}%
\OE%
\lthtmlfigureZ
\hfill\lthtmlcheckvsize\clearpage}

{\newpage\clearpage
\lthtmlinlinemathA{tex2html_wrap_inline2694}%
$\exists$%
\lthtmlinlinemathZ
\hfill\lthtmlcheckvsize\clearpage}

{\newpage\clearpage
\lthtmlinlinemathA{tex2html_wrap_inline2696}%
$4^{\rm
  th}$%
\lthtmlinlinemathZ
\hfill\lthtmlcheckvsize\clearpage}


\end{document}
