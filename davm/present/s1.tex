\documentclass{csslides}\raggedright

\usepackage{amsmath,amscd,amssymb,stmaryrd}

\newcommand{\Cases}{\operatorname{cases}}
\newcommand{\sscript}[1]{_{\empty_{#1}}}
\newcommand{\id}{\operatorname{id}}
\newcommand{\True}{\operatorname{true}}
\newcommand{\False}{\operatorname{false}}
\newcommand{\parrow}{\rightharpoonup}
\newcommand{\Meaning}[1]{\llbracket #1 \rrbracket}
\newcommand{\norm}{\operatorname{norm}}
\newcommand{\three}{\mathbf{3}}
\newcommand{\thomega}{{\three}^{\omega}}
\newcommand{\twomega}{\{0,1\}^{\omega}}
\newcommand{\myemph}[1]{{\it #1}\/}
\begin{document}

\title{Exact Arithmetic \\~ Using the Golden Ratio \\ [5ex]}
\author{David McGaw}
\date[CS]{}
\maketitle

\begin{slide}{}
\vfill

{\bf Solving}

\begin{center}
\begin{math}
\begin{array}{ccccc}
-1.41x_{1} & 2x_{2} & & = & 1 \\
x_{1} & -1.41x_{2} & x_{3} & = & 1 \\
 & 2x_{2} & -1.41x_{3}  & = & 1
\end{array}
\end{math}
\end{center}

\vfill

Rounding to 3 significant figures (without using row operations) gives:
\[ x_{1}=0.709, x_{2}=1, x_{3}=0.7.\]

The exact answer (to 3 significant figures) is: \[ x_{1}=x_{2}=x_{3}=1.69. \]

\vfill
\end{slide}


\begin{slide}{}

\vfill
{\bf Other Methods of Solving}
\begin{itemize}
\item Floating point with error analysis.
\item Interval Arithmetic.
\item Stochastic Rounding.
\item Symbolic Manipulation.
\end{itemize}
\vfill
\end{slide}


\begin{slide}{}
{\bf A Solution} \\
Represent numbers as infinite streams of digits.

\vfill

{\bf Problem with infinite streams in decimal}

\begin{center}
\begin{math}
\begin{array}{rll}
& 0.444\ldots & = 4/9 \\
+ & 0.555 \ldots & = 5/9 \\ \hline
& ?
\end{array}
\end{math}
\end{center}

\vfill

{\bf Definition} \\
Finite Character --- A finite number of output digits depends only on a finite number of input digits.
\vfill
\end{slide}

\begin{slide}{}
{\bf Why?} \\
\begin{picture}(400,200)
\put(125,80){\oval(150,90)[t]}
\put(275,80){\oval(150,90)[t]}
\put(200,100){\oval(60,18)[b]}
\put(200,100){\oval(20,10)[b]}
\put(50,50){\makebox[0in]{\(0=.00\ldots\)}}
\put(200,50){\makebox[0in]{\(0.99\ldots=1\)}}
\put(350,50){\makebox[0in]{\(1.99\ldots=2\)}}
\put(275,140){\(1\)}
\put(125,140){\(0\)}
\put(45,100){\vector(1,0){315}}
\end{picture}
\begin{center}
\begin{math}
\begin{array}{rll}
& 0.4 \\
+ & 0.5 \\ \hline
& ? & \in [0.9,1.1] \\
& 0.44 \\
+ & 0.55 \\ \hline
& ? & \in [0.99,1.01]
\end{array}
\end{math}
\end{center}
{\bf We Want} \\
\begin{picture}(400,200)
\put(145,100){\oval(190,70)[t]}
\put(255,100){\oval(190,70)[b]}
\put(50,100){\line(1,0){300}}
\end{picture}
\end{slide}

\begin{slide}{}
\vfill
{\bf Other Approaches to Exact Computation}
\begin{itemize}
\item Signed Digits \& Variations.
\item Limits of Effective Cauchy Sequence.
\item B-approximable.
\item Dedekind Cuts.
\item Continued Fractions.
\item Mobius Transformations.
\end{itemize}
\vfill
\end{slide}

\begin{slide}{}
{\bf Golden Ratio} \\
The Golden Ratio $\phi$ is defined s.t. $\phi^{2}=\phi+1$

In decimal
\[ \llbracket d \rrbracket = \sum_{i=0}^{\infty} d_{i}.10^{-i} = d_{0}.10^{0} + d_{1}.10^{-1} + \ldots \] \\
where \(d=d_{0}\/\cdot\/ d_{1} d_{2} d_{3} \ldots\) and \(d_{i} \in \{0, \ldots,9\}.\)

Similarly in Golden Ratio Notation
\[ \llbracket a \rrbracket =\sum_{i=0}^{\infty} a_{i}.\phi^{-i} = a_{0}.\phi^{0} + a_{1}.\phi^{-1} + \ldots \] \\
 where \(a=a_{0}\/\cdot\/a_{1} a_{2} a_{3} \ldots\) and \(a_{i} \in \{0, 1\}.\)

So
\begin{center}
\( \phi^{2} = \phi + 1                          \) \\
\( \phi^{2} = \phi^{1} + \phi^{0}                       \) \\
\( \phi^{0} = \phi^{-1} + \phi^{-2}                     \) \\
\( \llbracket 1 \cdot 000 \ldots \rrbracket = \llbracket 0 \cdot 110 \ldots \rrbracket \) \\
\end{center}
Giving us the identity \\
\begin{center}
\framebox{$\llbracket 100 \rrbracket = \llbracket 011 \rrbracket$ }
\end{center}
\end{slide}

\begin{slide}{}
\vfill
{\bf Aims} \\
The project should provide a number of useful operations for example:
\begin{itemize}
\item Conversion functions for input and output in decimal.
\item Basic arithmetic functions e.g.  addition, subtraction, multiplication and division.
\item Elementary functions e.g.  trigonometric and logarithmic functions.
\item Basic operations on functions e.g.  definite integration, global maxima and roots.
\end{itemize}
\vfill
\end{slide}
                                                                                                                                                                    
\end{document}