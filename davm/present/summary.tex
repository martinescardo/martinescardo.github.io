\documentclass[12pt]{article}

\usepackage{a4,stmaryrd,amsfonts}


\title{Exact Arithmetic \\~ Using the Golden Ratio}

\author{David McGaw  \\  
        University of Edinburgh}

\date{June 1999} 


\begin{document}

\maketitle

\section{Introduction} \label{ch:I}

The usual approach to real arithmetic on computers consists of using
floating point approximations. Unfortunately, floating point
arithmetic can sometimes produce wildly erroneous results. One
alternative approach is to use exact real arithmetic. Exact real
arithmetic allows exact real number computation to be performed
without the roundoff errors characteristic of other methods.
Conventional representations such as decimal and binary notation are
inadequate for this purpose. 

In this project we have considered an alternative representation
of real numbers, using the golden ratio. Firstly we have looked at the golden
ratio and its relation to the Fibonacci series, finding some
interesting and useful identities. We have then implemented algorithms for
basic arithmetic operations, trigonometric and logarithmic functions,
conversion and integration. 

We begin this summary by motivating the use of exact arithmetic and,
in particular, golden ratio notation. We then discuss the work
performed during the project.

\section{Problems with Floating-Point Representation}

In floating point representation, finitely many digits are kept.  This
introduces rounding errors, which when combined with other values can
make the error of the answer grow.  In some cases the rounding errors
in a calculation can be so large that the result is meaningless.

For example take the following system of equations~\cite{kn:Hall}:
\begin{center}
\[
\begin{array}{ccccc}
-1 \cdot 41x_{1} & 2x_{2} & & = & 1, \\
x_{1} & -1 \cdot 41x_{2} & x_{3} & = & 1, \\
 & 2x_{2} & -1 \cdot 41x_{3} & = & 1.
\end{array}
\]
\end{center}
Solving using Gaussian elimination (without row operations) and
rounding at each stage to 3 significant figures gives: \( x_{1}=0 \cdot 709,
x_{2}=1, x_{3}=0 \cdot 7.\) But the correct answer (to 3 significant
figures) is: \( x_{1}=x_{2}=x_{3}=1 \cdot 69. \) We can see that these
rounding errors have resulted in the answer that we get is completely
meaningless.

Increasing the number of digits of accuracy will not solve this
problem, it will only delay it, because by doing more operations we
could get the same magnitude of error.  Similarly allowing row
operations (swapping rows to make coefficients as large as possible)
would be useful but would not remove errors, since doing any
operations on a number containing an error can make the error grow
uncontrollably.

\section{Exact real number computation} \label{sec:stra}

We can argue that if errors are introduced because we truncate the
remaining digits of a number, we could get rid of this error by not
truncating the number, representing it as an infinite stream of
digits. Computation with infinite streams is supported by functional
programming languages such as Haskell and ML. We have chosen to
implement the algorithms developed in this project in ML.

We would ideally like to represent a number as an infinite stream of
decimal digits, but, perhaps surprisingly, this representation is not
suitable for exact computation. This fact, attributed to
Brouwer~\cite{kn:Brouwer}, is often discussed in the literature.  We
illustrate the algorithmic problem of infinite decimal expansions by
considering a hypothetical algorithm that performs multiplication
by~3. The key point is that the first digit of output has to be
produced in finite time and hence cannot depend on infinitely many
input digits.  More generally, any algorithmic function of streams of
digits has to have the property that finitely many output digits
depend only on finitely many input digits.

Suppose that, at some stage, our hypothetical algorithm for
multiplication by 3 has read ``$0 \cdot 3$'' from the input. If all remaining
digits are 3, then the input represents the number $\frac{1}{3}$. In
this case the result of this multiplication has to be 1, which can be
written as $1 \cdot 000\cdots$ or $0 \cdot 999\cdots$.  On the other hand, if we
eventually read a digit smaller than 3, then the input represents a
number smaller than $\frac{1}{3}$, and hence the output has to be
smaller than 1 and has to start with ``$0 \cdot$''. Similarly, if we
eventually read a digit greater than 3, then the output has to start
with ``$1 \cdot$''. Thus, while the next scanned digit is~3 we cannot decide
the first digit of output, because it is still possible that we will
eventually read a digit smaller or greater than~3. Hence, if the input
happens to be ``$0 \cdot 333\cdots$'' then we will not be able to decide the
first output digit in finite time. Therefore such an algorithm cannot
exist.

A graphical representation of this is in Figure \ref{fig:dec}.
\begin{figure}
\unitlength=.2mm
\begin{center}
\begin{picture}(400,100)
\put(95,30){\oval(150,90)[t]}
\put(245,30){\oval(150,90)[t]}
\put(170,50){\oval(60,18)[b]}
\put(170,50){\oval(20,10)[b]}
\put(20,0){\makebox[0in]{\(0=.00\cdots\)}}
\put(170,0){\makebox[0in]{\(0 \cdot 99\cdots=1\)}}
\put(320,0){\makebox[0in]{\(1 \cdot 99\cdots=2\)}}
\put(245,90){\(1 \cdot \)}
\put(95,90){\(0 \cdot \)}
\put(15,50){\vector(1,0){315}}
\end{picture}
\end{center}
\caption{Ranges in decimal notation} \label{fig:dec}
\end{figure}
The two large arcs represent the range of the possible values of
numbers beginning with the digit 0 and 1. We can see from the diagram
that the ranges of the two arcs intersect at one point $0 \cdot 99
\cdots = 1 \cdot 00 \cdots$. The small arcs represent the ranges of possible
values of the multiplication by 3 algorithm after we have looked at
each successive digit~3. We can see the range of the answer is getting
much smaller after looking at each successive digit. But no matter how
many digits we look at, in this case the value of the result will
always in both the interval beginning with ``$0 \cdot$'' and ``$1 \cdot$''.  Therefore we will never be able to determine the first digit of
the output by looking at a finite number of digits of the input.


There are many solutions to this problem, a number of which are
surveyed by Plume~\cite{kn:Plume}. Perhaps the simplest is to allow
negative digits.  The idea is that one can guess an output digit and
then, at a later stage, use a negative digit to adjust the answer if
necessary. In this project we have considered a solution proposed by
Di~Gianantonio~\cite{kn:DiGianantonio}, based on the golden ratio.

\section{The Golden Ratio Notation}

The golden ratio $\phi$ is the limit of the quotients of successive
terms of the Fibonacci series. This is approximately $1 \cdot 618$,
but we seldom use this fact. What is important for the purposes of
this work is that $\phi$ is the positive solution of the equation
$\phi^{2}=\phi+1$.  

Golden ratio notation is the same as decimal
notation, where $\phi$ replaces the base~$10$ and the binary digits
$0$ and $1$ replace the decimal digits $0,\dots, 9$.  Thus, a stream
\[
0 \cdot \alpha_1 \alpha_2 \cdots \alpha_n \cdots
\]
of binary digits $0$ and $1$ denotes the number
\[ \sum_{i=1}^{\infty} \alpha_{i}  \phi^{-i} = \alpha_{1}  \phi^{-1} + \alpha_{2} \phi^{-2} + \cdots + \alpha_{n} \phi^{-n} + \cdots. \] 
A graphical representation of golden ratio notation is in Figure
\ref{fig:rangegrn}.
\begin{figure} 
\begin{center}\unitlength=.2mm
\begin{picture}(400,100) 
\put(145,50){\oval(190,70)[t]}
\put(255,50){\oval(190,70)[b]}
\put(50,50){\line(1,0){300}}
\put(145,89){\makebox[0mm]{$0 \cdot$}}
\put(255,-2){\makebox[0mm]{$1 \cdot$}}
\put(40,47){\makebox[0mm]{0}}
\put(160,60){\makebox[0mm]{$\frac{1}{\phi}$}}
\put(240,30){\makebox[0mm]{1}}
\put(360,47){\makebox[0mm]{$\phi$}}
\end{picture}
\end{center}
\caption{Ranges in golden ratio notation} \label{fig:rangegrn}
\end{figure}
In golden ratio notation, the sequence of digits $011$ is equivalent
to the sequence $100$. This follows from the equation $\phi^2=\phi+1$.
All algorithms developed in this work are based on suitably rewriting
streams using this fundamental identity.


\section{Algorithms for Golden Ratio Notation}

This project can be split into two parts, one on the theory behind the
golden ratio notation and why we use it, and the other on algorithms
that we use to construct the exact calculator implemented in this
project. These can be further subdivided into 4 sections: manipulation
of streams, conversion between representations, limits of sequences,
and definite integration, as discussed below.

Manipulation of streams covers both algorithms that rewrite a stream
in a different, but equivalent, form and algorithms that compare two
streams. For decimal numerals, we can read off the numerical order
from the lexicographical order. This is not so for golden ratio
numerals, as it can be seen from Figure~\ref{fig:rangegrn}. However,
we have developed an algorithm that rewrites a pair of numerals, by
applying the identities discussed above, so that the numerical and
lexicographical orders do coincide.

For input and output purposes, we have developed algorithms to
convert between \emph{finite} decimal and golden ratio numerals. We
have also developed algorithms for conversion between \emph{infinite}
golden ratio and signed-digit binary streams, so that the algorithms
implemented in our calculator can be used by Plume's
calculator~\cite{kn:Plume}, and vice-versa.

Limits of sequences are applied to compute functions given by Taylor
series expansions. In particular, we compute the values of $e^{x},
\sin(x), \cos(x), \ln(x)$ and~$x^{y}$.

In order to deal with definite integration, we have modified an
algorithm originally developed by John Longley for signed-digit binary
notation~\cite{kn:Longley}. The modification is based on the basic
identities for golden ratio notation discussed above.  The algorithm
makes crucial use of side effects in combination with functional
programming. We have therefore decided to implement our algorithms in
the programming language ML. This integration algorithm is
considerably more efficient than the algorithm developed by
Simpson~\cite{kn:Simpson} and implemented by Plume~\cite{kn:Plume}.
This is confirmed by tests performed with our calculator.  Notice
that, although Longley has implemented the algorithm, he could only
test it on constant and identity functions, as he didn't have an
implementation of an exact calculator available in ML.

Other algorithms developed in this project include multiplication by
2, division by 2, addition, multiplication and a certain ``flip''
function. Based on the identities discussed above, the flip function
rewrites a given finite stream, without changing its numerical value,
so that a specified digit is set to $0$. This function is used in the
conversion and integration algorithms.

Finally, in the course of developing our algorithms, we have found a number
of technical results that are interesting in their own right. One of
them concerns the Fibonacci series defined by $F_0=0$, $F_1=1$ and
$F_{i+2}=F_{i}+F_{i+1}$. We have found that for all $j$ and $k$ there
is an~$m$ with
\[ \sum_{i=1}^{2^{k}} \mbox{F}_{3i+j} = 2^{k}m. \]
We have also shown that every number of the form $\frac{1}{2^{k}}$ has
a periodic representation in golden ratio notation. This is surprising because
$\phi$ is an irrational number.

%\begin{small}
\begin{thebibliography}{20}
  
\bibitem{kn:Beeler} M.  Beeler, R.  Gosper, R. Schroppel. {\em
    Hakmem.} Massachusetts Institute of Technology AI Laboratory. MIT
  AI memo 239 (HAKMEM).  Item 101, p.  39-44. 1972.
  
\bibitem{kn:Boehm et al} H. Boehm. R. Cartwright.  {\em Exact Real
    Arithmetic, Formulating Real Numbers as Functions.} In Turner. D.,
  editor, Research Topics in Functional Programming, pages 43-64.
  Addison-Wesley, 1990.
  
\bibitem{kn:Boehm} H. Boehm.  {\em Constructive Real Interpretation of
    Numerical Programs.} SIGPLAN Notices, 22(7):214-221, 1987.

\bibitem{kn:Brouwer} L.E.J. Brouwer
  {\em Besitzt jede reelle {Z}ahl eine {D}ezimalbruchentwicklung?}
  Math Ann, 83:201--210, 1920.
  
\bibitem{kn:Edalat} A.  Edalat, P.J. Potts. {\em A New Representation
    for Exact Real Numbers.} Electronic Notes in Theoretical Computer
  Science, volume 6.
  http://www.elsevier.nl/locate/entcs/volume6.html. 1997.
  
\bibitem{kn:Escardo} Mart\' \i n Escard\'o. {\em Effective and
    sequential definition by cases on the reals via infinite
    signed-digit numerals.} In Electronic Notes in Theoretical
  Computer Science, volume 13.  Available from
  http://www.elsevier.nl/locate/entcs/volume13.html.  1998.
  
\bibitem{kn:DiGianantonio} Pietro Di Gianantonio. {\em A Golden Ratio
    Notation for the Real Numbers.} CWI Technical Report.  Available
  from http://www.dimi.uniud.it/$\sim$pietro/Papers/paper\_arg.html.
  
\bibitem{kn:DiGianantonio2} Pietro Di Gianantonio. {\em A Functional
    Approach to Computability on Real Numbers.} Ph.D. Thesis:
  Universita di Pisa-Genova-Udine. Available from
  http://www.dimi.uniud.it/$\sim$pietro/Papers/paper\_arg.html.
  
\bibitem{kn:Graham} Graham, Knuth, Patashnik. {\em Concrete
    Mathematics: A foundation for Computer Science.} Addison Wesley.
  ISBN 0-201-14236-8.
  
\bibitem{kn:Hall} J. Hall. {\em Numerical Analysis and Optimisation.}
  Course Notes. Department of Mathematics, Edinburgh University.
  1998.
  
\bibitem{kn:Longley} J.R. Longley. {\em When is a functional program
    not a functional program?: a walk through introduction to the
    sequentially realizable functions.} Standard ML source file,
  available from http://www.dcs.ed.ac.uk/home/jrl/, 1998.
  
\bibitem{kn:Longley2} J.R.  Longley. {\em When is a functional program
    not a functional program?}  Paper available from
  http://www.dcs.ed.ac.uk/home/jrl/, March 1999.
  
\bibitem{kn:Menissier-Morian} Val\'erie M\'enissier-Morian. {\em
    Arbitrary precision real arithmetic: design and algorithms.} J.
  Symbolic Computation, 11, 1-000. 1996.
  
\bibitem{kn:Nielsen} A. Nielsen, P.  Kornerup.  {\em MSB-First Digit
    Serial Arithmetic.} Journal of Universal Computer Science, Vol 1,
  no 7, 527-547. 1995.
  
\bibitem{kn:Plume} David Plume. {\em A Calculator for Exact Real
    Number Computation.} BSc Project, School of Computer Science.
  Edinburgh University. Available from
  http://www.dcs.ed.ac.uk/$\sim$dbp/. 1998.
  
\bibitem{kn:Potts} P.J. Potts. {\em Computable Real Arithmetic Using
    Linear Fractional Transformations.}  Electronic Notes in
  Theoretical Computer Science 6, 1997.  Available from
  http://theory.doc.ic.ac.uk/$\sim$pjp/.
  
\bibitem{kn:Potts2} P.J. Potts, A. Edalat. {\em Exact Real Computer
    Arithmetic.} Department of Computing Technical Report DOC 97/9,
  Imperial College, 1997. Available from
  http://theory.doc.ic.ac.uk/$\sim$pjp/.
  
\bibitem{kn:Simpson} A. Simpson. {\em Lazy Functional Algorithms for
    Exact Real Functionals.} Mathematical Foundations of Computer
  Science 1998, Springer LNCS 1450, pp.  456-464, 1998.
  
\bibitem{kn:Sunderhauf} P. S\"{u}nderhauf. {\em Incremental Addition
    in Exact Real Arithmetic.} In R. Harmer et al., editors, Advances
  in Theory and Formal Methods of Computing: Proceedings of the fourth
  Imperial College Workshop, IC Press. Available from
  http://theory.doc.ic.ac.uk/$\sim$ps15/abstracts/incr.html.  December
  1997.
  
\bibitem{kn:Wiedmer} E. Wiedmer.  {\em Computing with Infinite
    Objects.} Theoretical Computer Science 10:133-155.  1980.
  
\end{thebibliography}
%\end{small}                   
\end{document}
