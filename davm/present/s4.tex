\documentclass{csslides}\raggedright

\usepackage{amsmath,amscd,amssymb,stmaryrd,amsfonts}

\newcommand{\Cases}{\operatorname{cases}}
\newcommand{\sscript}[1]{_{\empty_{#1}}}
\newcommand{\id}{\operatorname{id}}
\newcommand{\True}{\operatorname{true}}
\newcommand{\False}{\operatorname{false}}
\newcommand{\parrow}{\rightharpoonup}
\newcommand{\Meaning}[1]{\llbracket #1 \rrbracket}
\newcommand{\norm}{\operatorname{norm}}
\newcommand{\three}{\mathbf{3}}
\newcommand{\thomega}{{\three}^{\omega}}
\newcommand{\twomega}{\{0,1\}^{\omega}}
\newcommand{\myemph}[1]{{\it #1}\/}
\begin{document}

\title{Exact Arithmetic \\~ Using the Golden Ratio \\ [5ex]}
\author{David McGaw}
\date[CS]{}
\maketitle

\begin{slide}{}
  \vfill
  {\bf Aims} \\
  Produce an exact arithmetic calculator using Golden Ratio notation,
  with the following operations:
\begin{itemize}
\item Conversion functions for input and output in decimal.
\item Basic arithmetic functions e.g.  addition, subtraction, multiplication and division.
\item Elementary functions e.g.  trigonometric and logarithmic functions.
\item Basic operations on functions e.g.  definite integration, global maxima and roots.
\end{itemize}
\vfill
\end{slide}

\begin{slide}{}
{\bf Problems with floating-point}

\vfill

{\bf Solving}

\begin{center}
\begin{math}
\begin{array}{ccccc}
-1.41x_{1} & 2x_{2} & & = & 1 \\
x_{1} & -1.41x_{2} & x_{3} & = & 1 \\
 & 2x_{2} & -1.41x_{3}  & = & 1
\end{array}
\end{math}
\end{center}

\vfill

Rounding to 3 significant figures (without using row operations) gives:
\[ x_{1}=0.709, x_{2}=1, x_{3}=0.7.\]

The exact answer (to 3 significant figures) is: 
\[ x_{1}=x_{2}=x_{3}=1.69. \]

\vfill
\end{slide}

\begin{slide}{}
{\bf A Solution} \\
Represent numbers as infinite streams of digits.

\vfill

Problem with infinite streams in {\bf decimal}

\begin{center}
\begin{math}
\begin{array}{rl}
& 3  \\
\times & 0.333\ldots \\ \hline
& ?
\end{array}
\end{math}
\end{center}

\vfill


{\bf Definition}
Finite Character --- A finite number of output digits depends only on
a finite number of input digits.  

In Particular to produce one digit of output we only need to consider
a finite number of input digits.

\vfill
\end{slide}

\begin{slide}{}
{\bf Why?} \\
\begin{picture}(400,200)
\put(125,80){\oval(150,90)[t]}
\put(275,80){\oval(150,90)[t]}
\put(200,100){\oval(60,18)[b]}
\put(200,100){\oval(20,10)[b]}
\put(50,50){\makebox[0in]{\(0=.00\ldots\)}}
\put(200,50){\makebox[0in]{\(0.99\ldots=1\)}}
\put(350,50){\makebox[0in]{\(1.99\ldots=2\)}}
\put(275,140){\(1\)}
\put(125,140){\(0\)}
\put(45,100){\vector(1,0){315}}
\end{picture}
\begin{center}
\begin{math}
\begin{array}{rll}
& 3 \\
\times & 0.3 \\ \hline
& ? & \in [0.9,1.2] \\
& 3 \\
\times & 0.33 \\ \hline
& ? & \in [0.99,1.02]
\end{array}
\end{math}
\end{center}
{\bf We Want} \\
\begin{picture}(400,200)
\put(145,100){\oval(190,70)[t]}
\put(255,100){\oval(190,70)[b]}
\put(45,100){\vector(1,0){315}}
\end{picture}
\end{slide}

\begin{slide}{}
{\bf Golden Ratio} \\
The Golden Ratio $\phi$ is defined s.t. $\phi^{2}=\phi+1$

In Golden Ratio Notation

\[ \llbracket \alpha \rrbracket_s =\sum_{i=1}^{\infty} \frac{\alpha_i}{\phi^{i}} = \frac{\alpha_1}{\phi^{1}} + \frac{\alpha_2}{\phi^{2}} + \ldots + \frac{\alpha_n}{\phi^{n}} +\ldots \]
where $ \alpha = \alpha_1 \alpha_2 \ldots \alpha_n \ldots$ and
$ \alpha_i \in \{0, 1\}.$

Gives the identity
\[ \llbracket \alpha 100 \beta \rrbracket_s = \llbracket \alpha 011 \beta \rrbracket_s \]
where $\alpha \in \{0,1\}^{*}$ and $\beta \in \{0,1\}^{\omega}$.

Since
\[ \begin{array}{rcl}
\phi^2 & = & \phi + 1 \\
1.\phi^2 + 0.\phi^1 + 0.\phi^0 & = & 0.\phi^2 + 1.\phi^1 + 1.\phi^0
\end{array} \]

\end{slide}

\begin{slide}{}
{\bf Choice of language}
\vfill

\begin{center}
{\bf ML Vs Haskell}
\end{center}

{\bf Lazy Evaluation}

Only evaluates an argument when it has to e.g.

fun from n = n :: from (n+1); \\


from 0; \\
Haskell - [0,1,2,3,4,5, ... $<$ctrl-c$>$ \\
ML  - nothing

{\bf Advantages of Haskell}
\begin{itemize}
\item Can implement infinite stream as an infinite list.
\item Can use pattern matching.
\item Readable code.
\end{itemize}

{\bf Advantages of ML}
\begin{itemize}
\item Has exceptions.
\item More efficient?
\end{itemize}
\end{slide}


\begin{slide}{}
The basic operations were designed by Pietro Di Gianantonio.



{\bf I developed  and implemented.}
\begin{itemize}
\item Flip function on a finite list.
\item New algorithms for addition and multiplication by 2.
\item Division by 2.
\item New algorithm for multiplication.
\item Conversion between (finite) Decimal and GR notation.
\item Conversion from GR notation to Signed Binary.
\item Lexicographical Normalisation in Golden Ratio notation.
\end{itemize}
\hfill (continued \ldots)
\end{slide}

\begin{slide}{}
\begin{itemize}
\item Basic Operations.
\item Cases function.
\item Intersection of nested sequences of intervals.
\item Functions for calculating $e^{x}, \sin(x), \cos(x), \log(x)$ and $x^{y}$
\item Definite Integration.
\end{itemize}
\end{slide}

\begin{slide}{}
{\bf Division by 2}
\vfill
\vfill
\vfill
\vfill
\vfill
\begin{picture}(70,60)
\unitlength = .9ex
\put(15,56){$\alpha$}

\put(16,55){\vector(-1,-1){10}} % l
\put(4,39){\shortstack{$0\alpha_{|2}$\\=\\$0:\alpha_{|2}$}}

\put(16,55){\vector(1,-1){10}} % r
\put(25,43){$1\alpha_{|2}$}

\put(27,42){\vector(-2,-1){14}} % rl
\put(11,32){$10\alpha_{|3}$}

\put(13,31){\vector(-1,-1){10}} %rll
\put(0,15){\shortstack{$100\alpha_{|4}$\\=\\$0:11\alpha_{|4}$}}

\put(13,31){\vector(0,-1){10}} %rlc
\put(9,15){\shortstack{$101\alpha_{|4}$\\=\\$0:12\alpha_{|4}$}}

\put(27,42){\vector(0,-1){10}} % rc
\put(25,30){$11\alpha_{|3}$}

\put(27,29){\vector(-2,-3){5}} % rcl
\put(19,15){\shortstack{$110\alpha_{|4}$\\=\\$02:1\alpha_{|4}$}}

\put(27,29){\vector(2,-3){5}} % rcr
\put(29,15){\shortstack{$111\alpha_{|4}$\\=\\$200:\alpha_{|4}$}}

\put(27,42){\vector(2,-1){14}} % rr
\put(39,32){$12\alpha_{|3}$}

\put(41,31){\vector(0,-1){10}} % rrc
\put(39,18){$120\alpha_{|4}$}

\put(41,31){\vector(1,-1){10}} % rrr
\put(50,15){\shortstack{$121\alpha_{|4}$\\=\\$2:10\alpha_{|4}$}}

\put(41,17){\vector(-1,-1){10}} % rrcl
\put(29,1){\shortstack{$1200\alpha_{|5}$\\=\\$200:1\alpha_{|5}$}}

\put(41,17){\vector(1,-1){10}} % rrcr
\put(49,1){\shortstack{$1201\alpha_{|5}$\\=\\$2002:\alpha_{|5}$}}
\end{picture}
\vfill
\end{slide}

\begin{slide}{}
{\bf Integration}

{\bf Modulus of convergence}
\[ \forall \epsilon > 0 \exists k \mbox{ such that} \mid f(x_{|k}) - f(x) \mid < \epsilon \]

\begin{center}
\unitlength=1.8mm
\begin{picture}(70,75)(0,0)
\put(0,5){\vector(1,0){70}} %Initial lines
\put(3,15){\line(1,0){12}}
\put(3,25){\line(1,0){22}}
\put(3,35){\line(1,0){32}}
\put(3,55){\line(1,0){52}}
\put(3,65){\line(1,0){62}}
\put(5,0){\vector(0,1){70}}
\put(35,25){\line(0,1){10}}
\put(65,55){\line(0,1){10}}
\put(0,0){\line(1,1){70}}
\thicklines
\put(5,5){\line(1,0){10}}
\put(15,15){\line(1,0){10}}
\put(25,25){\line(1,0){10}}
\put(55,55){\line(1,0){10}}
\put(15,3){\line(0,1){12}}
\put(25,3){\line(0,1){22}}
\put(35,3){\line(0,1){22}}
\put(55,3){\line(0,1){52}}
\put(65,3){\line(0,1){52}}
\thinlines
\multiput(15,5)(.5,0){20}{\line(0,1){10}}
\multiput(15,5)(0,.5){20}{\line(1,0){10}}
\multiput(25,5)(.5,0){20}{\line(0,1){20}}
\multiput(25,5)(0,.5){40}{\line(1,0){10}}
\multiput(55,5)(.5,0){20}{\line(0,1){50}}
\multiput(55,5)(0,.5){100}{\line(1,0){10}}

\put(40,25){\makebox(0,0)[c]{\ldots}}
\put(2,45){\makebox(0,0)[c]{\vdots}}
\put(2,10){\makebox(0,0)[c]{$\epsilon$}}
\put(2,20){\makebox(0,0)[c]{$\epsilon$}}
\put(2,30){\makebox(0,0)[c]{$\epsilon$}}
\put(2,60){\makebox(0,0)[c]{$\epsilon$}}
\put(2,12){\vector(0,1){3}}
\put(2,8){\vector(0,-1){3}}
\put(2,22){\vector(0,1){3}}
\put(2,18){\vector(0,-1){3}}
\put(2,32){\vector(0,1){3}}
\put(2,28){\vector(0,-1){3}}
\put(2,62){\vector(0,1){3}}
\put(2,58){\vector(0,-1){3}}

\put(71,5){\makebox(0,0)[l]{$x$}}
\put(5,73){\makebox(0,0)[c]{$f(x)$}}

\end{picture}
\end{center}
\end{slide}

\begin{slide}{}
\vfill
\begin{center}
\unitlength=1.8mm
\begin{picture}(80,70)(0,0)
\put(0,10){\vector(1,0){70}} %Initial lines
\put(5,40){\vector(0,1){35}}
\put(71,10){\makebox(0,0)[l]{$x$}}
\put(5,78){\makebox(0,0)[c]{$f(x)$}}
\thicklines
\put(0,40){\line(1,0){65}}
\put(65,5){\line(0,1){35}}
\put(5,5){\line(0,1){35}}
\thinlines
\put(2,57){\vector(0,1){13}}
\put(65,40){\line(0,1){32}}
\put(3,70){\line(1,0){62}}
\put(0,37.5){\line(2,1){70}}
\put(2,55){\makebox(0,0)[c]{$\epsilon$}}
\put(2,57){\vector(0,1){13}}
\put(2,53){\vector(0,-1){13}}
\put(5,2){\makebox(0,0)[c]{$000000\ldots$}}
\put(65,2){\makebox(0,0)[c]{$000011\ldots$}}
\put(0,40){\makebox(0,0)[r]{$011$}}

\multiput(6,10)(1,0){59}{\line(0,1){30}}
\multiput(5,11)(0,1){29}{\line(1,0){60}}
\multiput(5.5,10)(1,0){60}{\line(0,1){30}}
\multiput(5,10.5)(0,1){30}{\line(1,0){60}}

\end{picture}
\end{center}
\vfill
\[\begin{array}{c||l|rl}
  \mbox{input } (x) & 0000 \mid^{k} 00 \ldots & \mbox{but} 
& 0000 \mid^{k} 11 \ldots \\ \hline
  \mbox{output } f(x_{|k}) & 011 \mid_{n} & & 011 \mid_{n} 
\end{array}\]
\vfill
\end{slide}

\begin{slide}{}
\vfill
$ F_0 = 0, F_1 = 1 \mbox{ and } F_{i+2} = F_i + F_{i+1} $

\[ \forall k \in \mathbb{N} \ \exists m \in \mathbb{Z} \mbox{ such that } \sum_{i=1}^{2^{k}} F_{3i} = 2^k m \]

\[ \frac{1}{2^k} = \llbracket \alpha (\beta)^{\omega} \rrbracket_s \]
where $\alpha \in \{0,1\}^{*}$ and $\beta \in \{0,1\}^{\omega}$.

Or $\frac{1}{2^k}$ has a periodic representation in Golden
Ratio notation.  
\vfill
\end{slide}

\begin{slide}{}
\vfill
{\bf Summary}
\begin{itemize}
\item Worked out properties of the golden ratio to develop new algorithms.
\item Proved correctness of algorithms.
\item Implemented algorithms in ML.
\item Comparison with previous work
\begin{itemize}
\item Basic operations were slower than in functions implemented in Haskell.
\item Integration is more efficient than other implementations.
\item New algorithms for basic operations in Golden Ratio notation.
\end{itemize}
\end{itemize}
\vfill
\end{slide}

\end{document}