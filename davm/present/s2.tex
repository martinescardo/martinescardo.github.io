\documentclass{csslides}\raggedright

\usepackage{amsmath,amscd,amssymb,stmaryrd}

\newcommand{\Cases}{\operatorname{cases}}
\newcommand{\sscript}[1]{_{\empty_{#1}}}
\newcommand{\id}{\operatorname{id}}
\newcommand{\True}{\operatorname{true}}
\newcommand{\False}{\operatorname{false}}
\newcommand{\parrow}{\rightharpoonup}
\newcommand{\Meaning}[1]{\llbracket #1 \rrbracket}
\newcommand{\norm}{\operatorname{norm}}
\newcommand{\three}{\mathbf{3}}
\newcommand{\thomega}{{\three}^{\omega}}
\newcommand{\twomega}{\{0,1\}^{\omega}}
\newcommand{\myemph}[1]{{\it #1}\/}
\begin{document}

\title{Exact Arithmetic \\~ Using the Golden Ratio \\ [5ex]}
\author{David McGaw}
\date[CS]{}
% \maketitle


\begin{slide}{}
{\bf Summary of Last Talk}
\begin{itemize}
\item Floating Point Arithmetic can introduce uncontrollable errors into the result of a computation.
\item Using an Infinite Stream of digits we can guarantee the answer within a given range.
\item We cannot use infinite streams of decimals.
\item In GR (Golden Ratio) Notation 100=011.
\item Question: Why Golden Ratio?
\end{itemize}
\end{slide}


\begin{slide}{}
\vfill

\begin{center}
{\bf ML Vs Haskell}
\end{center}

{\bf Lazy Evaluation} \\
Only evaluates an argument when it has to e.g.

fun from n = n :: from (n+1); \\


from 0; \\
Haskell - [0,1,2,3,4,5, ... $<$ctrl-c$>$ \\
ML -  nothing

{\bf Advantages of Haskell}
\begin{itemize}
\item Can implement infinite stream as an infinite list.
\item Can use pattern matching.
\item Readable code.
\end{itemize}

{\bf Advantages of ML}
\begin{itemize}
\item Has exceptions.
\item Slightly more efficient.
\end{itemize}
\end{slide}


\begin{slide}{}
{\bf Definition of Infinite streams in ML}

datatype 'a stream =  \\
cons of 'a * (unit -$>$ 'a stream);


fun force s = s();



Example: \\
{\bf Infinite stream of zeros} \\
Defined in ML as: \\
fun all\_zero() = cons(\(0\), fn () =$>$ all\_zero()); \\

Defined in Haskell as: \\
fun all\_zero() = $0$::all\_zero();
\end{slide}

\begin{slide}{}
{\bf Algorithms Implemented.} \\
(designed by Pietro Di Giantonio) \\
\begin{itemize}
\item Addition. 
\item Negation.
\item Subtraction.
\item Multiplication.
\item Division.
\end{itemize}


{\bf Algorithms Developed and Implemented.}
\begin{itemize}
\item Lexicographical Normalisation.
\item Flip function on a finite list.
\item Conversion from Decimal to GR.
\item Conversion from GR to Decimal (two versions).
\end{itemize}
\end{slide}

\begin{slide}{}
{\bf Lexicographical Normalisation} \\
We would like a function to calculate if $x<y$. 
\[ \mbox{where}\ x<y = \left\{ \begin{array}{ll}
    \mbox{true} & \mbox{if }\ x<y \\
    \mbox{false} & \mbox{otherwise} (x \geq y)
\end{array}
\right. \]

However in infinite stream notation we can only check if $x <_{\perp} y$.
\[ x<_{\perp}y = \left\{ \begin{array}{ll}
    \mbox{true} & \mbox{if} \ x<y \\
    \mbox{false} & \mbox{if} \ x>y \\
    \perp & \mbox{otherwise}
\end{array}
\right. \]
$\perp$ = non termination.


In decimal notation: 
\[ \begin{array}{l}
  x = 0.12345\!\!\overbrace{99999}^{\mbox{Ignored}} \\
  y = 0.1234600000
\end{array}\]

In GR Notation:
\[ \begin{array}{l}
  x = 0.1000000000 \ldots, \llbracket x \rrbracket_{s}=1 / \phi    \\
  y = 0.0111111111 \ldots, \llbracket y \rrbracket_{s}=1 \\
\end{array}\]
\end{slide}


\begin{slide}{}
{\bf Flip Function on a Finite List} \\
We can rewrite a list of the form $ [ 0,\alpha,0,0] $ \\
Where $\alpha = \{ 0,1 \}^{*}$. \\
s.t. a given digit is set to $0$.

Example: \\
$ [0,1,0,1,0,1,0,0] = [0,0,1,1,1,1,1,1] $ \\
\setlength{\unitlength}{1ex}
\begin{picture}(20,4)
\thicklines
\put(3.3,1){\vector(0,1){3}}
\end{picture} \\


{\bf A Use} \\
In GR Notation
\[ \begin{array}{ll}
  & 0 \cdot \alpha_{1} \alpha_{2} \alpha_{3} \ldots \alpha_{n} 0 0 \\
  & \alpha_{0}' \cdot 0 \alpha_{2}' \alpha_{3}' \ldots \alpha_{n}' \alpha_{n+1}' \alpha_{n+2}'\\
\mbox{worst case} & \alpha_{0}' \cdot 0\!\overbrace{11 \ldots 111}^{< 1}
\end{array} \]

We can use this to determine if a finite stream is greater than or equal to 1.
\end{slide}


\begin{slide}{}

{\bf Conversion from Decimal to GR} \\
In decimal 
\[ d = 0 \cdot d_{1} d_{2} d_{3} \ldots d_{n}= d_{1}.10^{-1} + d_{2}.10^{-2} + 
\ldots + d_{n}.10^{-n} \]\\
For $d_{i} \in \{0, \ldots, 9\}$

if $ \llbracket z_{d_{i}}:\alpha_{d_{i}} \rrbracket_{f} = d_{i} $ then 
$\llbracket \mbox{Base} \rrbracket_{f} = 10$ in this case
\[ \begin{array}{lc}
d = & \llbracket z_{d_{1}}:\alpha_{d_{1}} \rrbracket_{f} * (\llbracket \mbox{Base} \rrbracket_{f})^{-1} \\
& + \llbracket z_{d_{2}}:\alpha_{d_{2}} \rrbracket_{f} * (\llbracket \mbox{Base} \rrbracket_{f})^{-2} + \\ 
 &  \ldots + \llbracket z_{d_{n}}:\alpha_{d_{n}} \rrbracket_{f} * (\llbracket \mbox{Base} \rrbracket_{f})^{-n} 
\end{array}\]


{\bf Conversion from GR to Decimal} \\
Converting from decimal to decimal:

\[ \begin{array}{rll}
d = & 0 \cdot d_{1} d_{2} d_{3} \ldots d_{n} & \\
10*d = & d_{1} \cdot d_{2} d_{3} \ldots d_{n} & \mbox{digit}=d_{1} \\
(10*d)-d_{1} = & 0 \cdot d_{2} d_{3} \ldots d_{n} & \\
10*((10*d)-d_{1}) = & d_{2} \cdot d_{3} \ldots d_{n} & \mbox{digit}=d_{2} \\
\vdots & \vdots & \\
\end{array}\]
Similar idea for GR Notation.
\end{slide}

\begin{slide}{}
{\bf Algorithm to Redesign and Implement}\\
\begin{itemize}
\item Definite Integration.
\end{itemize}
{\bf Algorithms Still to Design and Implement}
\begin{itemize}
\item Conversion from GR Notation to Signed Binary.
\item Trigonometric and Logarithmic functions.
\end{itemize}
{\bf Revised Timetable}
\begin{itemize}
\item Now until March - Implementing the above algorithms.
\item March until June - Write up.
\end{itemize}
\end{slide}

\end{document}